\documentclass{../../slides-style}

\slidetitleext{Лекция 9: Качество программного обеспечения}{30.04.2024}{Качество ПО}

\begin{document}

    \begin{frame}[plain]
        \titlepage
    \end{frame}

    \section{Понятие качества ПО}

    \begin{frame}
        \frametitle{Что такое качество ПО?}
        \begin{itemize}
            \item Обывательский подход
            \begin{itemize}
                \item Лёгкость в использовании, производительность, отсутствие ошибок, документация, кроссплатформенность и т.п.
            \end{itemize}
            \item Профессиональный подход
            \begin{itemize}
                \item Соответствие требованиям (Crosby, 1979)
                \item Пригодность к использованию (Juran, Gryna, 1970)
            \end{itemize}
            \item Жизненный подход
            \begin{itemize}
                \item Соответствие всем требованиям, явным и неявным
            \end{itemize}
        \end{itemize}
    \end{frame}

    \begin{frame}
        \frametitle{Стоимость качества}
        \begin{itemize}
            \item Решения о качестве принимаются на этапе работы с требованиями
            \item Обычно заказчик полагает качество максимальным
            \item Стоимость:
            \begin{itemize}
                \item стоимость предупреждения дефектов (prevention cost)
                \item стоимость оценки (appraisal cost)
                \item стоимость внутренних сбоев (internal failure cost)
                \item стоимость внешних сбоев (external failure cost)
            \end{itemize}
        \end{itemize}
    \end{frame}

    \begin{frame}
        \begin{center}
            \includegraphics[width=0.9\textwidth]{quality.png}
        \end{center}
    \end{frame}

    \section{Модель качества ПО}

    \begin{frame}
        \frametitle{Модель качества ПО}
        \begin{itemize}
            \item Характеристики качества~--- отдельные точки зрения пользователя на качество
            \item Атрибуты характеристик качества~--- детализация разных аспектов характеристики
            \item Метрики качества
            \begin{itemize}
                \item Метод измерения атрибута
                \item Шкала измерения значений атрибута
                \item Вес (иногда)
            \end{itemize}
        \end{itemize}
    \end{frame}

    \subsection{Характеристики качества}

    \begin{frame}
        \frametitle{Характеристики качества ПО (ISO 25010:2011)}
        \begin{itemize}
            \item Функциональность
            \item Надежность
            \item Удобство использования
            \item Эффективность
            \item Сопровождаемость
            \item Переносимость
        \end{itemize}
    \end{frame}

    \begin{frame}
        \frametitle{Функциональность}
        \begin{itemize}
            \item Функциональная полнота (suitability)
            \item Правильность (точность) (accuracy)
            \item Функциональная совместимость (интероперабельность) (interoperability)
            \item Защищенность (security)
            \item Соответствие стандартам и правилам (compliance)
        \end{itemize}
    \end{frame}

    \begin{frame}
        \frametitle{Надежность}
        \begin{itemize}
            \item Безотказность (maturity)
            \item Устойчивость к отказам (fault tolerance)
            \item Восстанавливаемость (recoverability)
            \item Пригодноспособность (dependability)
            \begin{itemize}
                \item Готовность к использованию (availability)
                \item Готовностью к непрерывному функционированию (reliability)
                \item Безопасность для окружающей среды (safety)
                \item Секретность и сохранность информации (сonfidentiality)
                \item Устойчивость к самопроизвольному изменению (integrity)
                \item Простота выполнения операций обслуживания (maintainability)
            \end{itemize}
        \end{itemize}
    \end{frame}

    \begin{frame}
        \frametitle{Удобство использования}
        \begin{itemize}
            \item Понимаемость (understandability)
            \item Легкость изучения (learnability)
            \item Удобство работы (operability)
            \begin{itemize}
                \item Оперативность
                \item Согласованность
            \end{itemize}
            \item Привлекательность (attractiveness)
        \end{itemize}
    \end{frame}

    \begin{frame}
        \frametitle{Эффективность}
        \begin{itemize}
            \item Временная эффективность, реактивность (time behaviour)
            \item Эффективность ресурсов (resource utilisation)
        \end{itemize}
    \end{frame}

    \begin{frame}
        \frametitle{Сопровождаемость}
        \begin{itemize}
            \item Анализируемость (analyzability)
            \item Изменяемость (changeability)
            \item Стабильность (stability)
            \item Тестируемость (testability)
        \end{itemize}
    \end{frame}

    \begin{frame}
        \frametitle{Переносимость}
        \begin{itemize}
            \item Адаптивность (adaptability)
            \item Настраиваемость, простота инсталляции (installability)
            \item Сосуществование (coexistence)
            \item Заменяемость (replaceability)
        \end{itemize}
    \end{frame}

    \subsection{Метрики}

    \begin{frame}
        \frametitle{Метрики качества ПО}
        \begin{itemize}
            \item Функциональность: метрики тестирования
            \item Надежность: метрики тестирования, динамические метрики
            \item Удобство использования: метрики эргономики
            \item Эффективность: динамические метрики
            \item Сопровождаемость: метрики кода
            \item Переносимость: метрики кода
        \end{itemize}
    \end{frame}

    \begin{frame}
        \frametitle{Классификация метрик}
        \begin{itemize}
            \item Метрики программного продукта
            \begin{itemize}
                \item Внешние
                \begin{itemize}
                    \item Надежность
                    \item Функциональность
                    \item Сопровождение
                    \item Стоимость
                \end{itemize}
                \item Внутренние
                \begin{itemize}
                    \item Размер
                    \item Сложность
                    \item Стиль
                \end{itemize}
            \end{itemize}
            \item Метрики процесса
            \item Метрики использования
        \end{itemize}
    \end{frame}

    \begin{frame}
        \frametitle{Классификация метрик}
        \begin{itemize}
            \item Метрики программного продукта
            \item Метрики процесса
            \begin{itemize}
                \item Общее время разработки и отдельно время для каждой стадии
                \item Время модификации моделей
                \item Время выполнения работ на процессе
                \item Число найденных ошибок при инспектировании
                \item Стоимость проверки качества
                \item Стоимость процесса разработки
            \end{itemize}
            \item Метрики использования
        \end{itemize}
    \end{frame}

    \begin{frame}
        \frametitle{Классификация метрик}
        \begin{itemize}
            \item Метрики программного продукта
            \item Метрики процесса
            \item Метрики использования
            \begin{itemize}
                \item Точность и полнота реализации задач пользователя
                \item Затраченные ресурсы на эффективное решение задач пользователя
            \end{itemize}
        \end{itemize}
    \end{frame}

    \begin{frame}
        \frametitle{Что можно измерять?}
        \begin{itemize}
            \item Размер
            \begin{itemize}
                \item Число классов, строк в программе, объём памяти, ...
            \end{itemize}
            \item Переиспользуемость кода
            \begin{itemize}
                \item Переиспользуемые классы, наследуемые классы, зависимости, ...
            \end{itemize}
            \item Время
            \begin{itemize}
                \item Отклика, общего функционирования системы, выполнения компонента, ...
            \end{itemize}
            \item Усилия
            \begin{itemize}
                \item Производительность труда, трудоемкость, ...
            \end{itemize}
            \item Ошибки
            \begin{itemize}
                \item Количество ошибок, число отказов, ...
            \end{itemize}
        \end{itemize}
    \end{frame}

    \begin{frame}
        \frametitle{Простые метрики}
        \begin{itemize}
            \item Число строк кода (LOC/KLOC)
            \item Производительность = LOC / Затраты
            \item Удельная стоимость = Затраты / LOC
            \item Качество кода = Число ошибок / LOC
            \item Документированность = Число страниц документации / LOC
        \end{itemize}
    \end{frame}

    \begin{frame}
        \frametitle{Ещё метрики}
        \begin{itemize}
            \item Метрики Холстеда
            \item Метрики С. Чидамбера и К. Кемерера
            \item Метрики Ф. Абреу
            \item Метрики Л. Константейна и Э. Йордана
            \item Метрики Л. Отта и Б. Мехра
            \item Метрики Д. Биемена и Б. Кенга
            \item Метрики М. Лоренца и Д. Кидда
            \item Метрики Р. Байндера
            \item ...
        \end{itemize}
    \end{frame}

    \begin{frame}
        \frametitle{Метрики Холстеда}
        \begin{itemize}
            \item Number of Unique Operators (NUOprtr)
            \item Number of Unique Operands (NUOprnd)
            \item Number of Operators (Noprtr)
            \item Number of Operands (Noprnd)
            \item Halstead Program Volume (HPVol) $= (Noprtr + Noprnd) \times log_2(NUOprtr + NUOprnd)$
            \item Halstead Difficulty (HDiff) $= (\frac{NUOprtr}{2}) \times (\frac{Noprnd}{NUOprnd})$
            \item Halstead Effort (HEff) $= HDiff \times HPVol$
        \end{itemize}
    \end{frame}

    \begin{frame}
        \frametitle{Цикломатическая сложность}
        \begin{columns}
            \begin{column}{0.7\textwidth}
                \begin{itemize}
                    \item C $= E – N + 2P$
                    \item E~--- число ребер
                    \item N~--- число узлов
                    \item P~--- число компонентов связности
                \end{itemize}
            \end{column}
            \begin{column}{0.3\textwidth}
                \begin{center}
                    \includegraphics[height=0.8\textheight]{programGraph.png}
                \end{center}
            \end{column}
        \end{columns}
    \end{frame}

    \begin{frame}
        \frametitle{Метрики С. Чидамбера и К. Кемерера}
        \begin{itemize}
            \item Weighted Methods Per Class (WMC)
            \begin{itemize}
                \item $WMC = \sum_{i=1}^{n}C_i$, где $C_i$~--- как-то посчитанная сложность метода $i$
            \end{itemize}
            \item Depth of Inheritance Tree (DIT)
            \item Number of children (NOC)
            \item Coupling between object classes (СВО)
            \begin{itemize}
                \item Количество вызовов методов или полей
            \end{itemize}
            \item Response For a Class (RFC) $= |\{M\} \cup_{i} \{R_i\}| $
            \begin{itemize}
                \item $\{R_i\}$~--- множество методов, вызываемых методом $i$
                \item $\{M\}$~--- множество всех методов в классе
            \end{itemize}
            \item Lack of Cohesion in Methods (LCOM)
            \begin{itemize}
                \item NotRelated~--- количество пар методов без общих полей/свойств
                \item Related~--- количество пар методов с общими полями/свойствами
                    \begin{equation*}
                        LCOM=\begin{cases}
                            NotRelated - Related, & \text{если}\ NotRelated > Related.\\
                            0,                    & \text{в противном случае}.
                        \end{cases}
                    \end{equation*}
            \end{itemize}
        \end{itemize}
    \end{frame}

    \begin{frame}[fragile]
        \frametitle{Полезные модификации WMC}
        \begin{itemize}
            \item WMC2 $= \sum_{i=1}^{n}$ количество параметров i-го метода
            \item ANAM (Average Number of Arguments per Method) $= \frac{WMC2}{WMC}$
        \end{itemize}
        \begin{minted}{c}
SetInterval(min, max),
SetMethod(method),
SetPrecision(precision),
SetFunctionToIntegrate(function),
Integrate();
        \end{minted}
        \vspace{5mm}
        vs
        \vspace{5mm}
        \begin{minted}{c}
Integrate(function, min, max, method, precision);
        \end{minted}
    \end{frame}

    \begin{frame}
        \frametitle{LCOM: недостатки (1)}
        \begin{center}
            \includegraphics[width=0.7\textwidth]{lcomFail1.png}
        \end{center}
    \end{frame}

    \begin{frame}
        \frametitle{LCOM: недостатки (2)}
        \begin{center}
            \includegraphics[width=0.7\textwidth]{lcomFail2.png}
        \end{center}
    \end{frame}

    \begin{frame}
        \frametitle{Модификация LCOM*}
        
        $$LCOM^* = \frac{\Biggl(\frac{1}{a}\sum\limits_{j=1}^{a}m(A_j)\Biggr) - m}{1 - m}$$

        \begin{itemize}
            \item $m$~--- количество методов класса
            \item $a$~--- количество атрибутов класса
            \item $m(A_j)$~--- количество методов, которые имеют доступ к атрибуту $A$
        \end{itemize}
    \end{frame}

    \begin{frame}
        \frametitle{Метрики Лоренца и Кидда}
        \begin{columns}
            \begin{column}{0.65\textwidth}
                \begin{itemize}
                    \item Метрики, ориентированные на классы
                    \begin{itemize}
                        \item Class Size (CS, <= 20)
                        \item Number of Operations Overridden by a Subclass (NOO, <= 3)
                        \item Number of Operations Added by a Subclass (NOA, <= 4)
                        \item Specialization Index (SI, <= 0.15)
                            $SI = (NOO \times \text{уровень}) / M_\text{общ.}$
                    \end{itemize}
                    \item Метрики, ориентированные на операции
                    \begin{itemize}
                        \item Average Operation Size ($OS_{avg}$, <=9)
                        \item Operation Complexity (OC)
                        \item Average Number of Parameters per operation ($NP_{avg}$)
                    \end{itemize}
                \end{itemize}
            \end{column}
            \begin{column}{0.35\textwidth}
                \begin{center}
                    \includegraphics[width=\textwidth]{siCalculation.png}
                \end{center}
            \end{column}
        \end{columns}
    \end{frame}

    \begin{frame}
        \frametitle{Набор метрик Фернандо Абреу (MOOD)}
        \begin{itemize}
            \item Фактор закрытости метода (MHF)
            \item Фактор закрытости атрибута (AHF)
            \item Фактор наследования метода (MIF)
            \item Фактор наследования атрибута (AIF)
            \item Фактор полиморфизма (POF)
            \item Фактор сопряжения (СOF)
        \end{itemize}
    \end{frame}

    \begin{frame}
        \frametitle{Фактор закрытости метода (MHF)}
        $$MHF = \frac{\sum\limits_{i=1}^{TC}M_h(C_i)}{\sum\limits_{i=1}^{TC}M_a(C_i)}$$
        \begin{itemize}
            \item $M_h(C_i)$~--- количество private-методов в классе $C_i$
            \item $M_a(C_i)$~--- общее количество методов в классе $C_i$ (без унаследованных)
        \end{itemize}
    \end{frame}

    \begin{frame}
        \frametitle{Фактор закрытости свойства (AHF)}
        $$AHF = \frac{\sum\limits_{i=1}^{TC}A_h(C_i)}{\sum\limits_{i=1}^{TC}A_a(C_i)}$$
        \begin{itemize}
            \item $A_h(C_i)$~--- количество private-атрибутов в классе $C_i$
            \item $A_a(C_i)$~--- общее количество атрибутов в классе $C_i$
        \end{itemize}
    \end{frame}

    \begin{frame}
        \frametitle{Фактор наследования метода (MIF)}
        $$MIF = \frac{\sum\limits_{i=1}^{TC}M_i(C_i)}{\sum\limits_{i=1}^{TC}M_a(C_i)}$$
        \begin{itemize}
            \item $M_i(C_i)$~--- количество унаследованных и не переопределенных методов в классе $C_i$
            \item $M_a(C_i)$~--- общее количество методов в классе $C_i$
        \end{itemize}
    \end{frame}

    \begin{frame}
        \frametitle{Фактор наследования свойства (AIF)}
        $$AIF = \frac{\sum\limits_{i=1}^{TC}A_i(C_i)}{\sum\limits_{i=1}^{TC}A_a(C_i)}$$
        \begin{itemize}
            \item $A_i(C_i)$~--- количество унаследованных и не переопределенных атрибутов в классе $C_i$
            \item $A_a(C_i)$~--- общее количество атрибутов в классе $C_i$
        \end{itemize}
    \end{frame}

    \begin{frame}
        \frametitle{Фактор полиморфизма (POF)}
        $$POF = \frac{\sum\limits_{i=1}^{TC}M_o(C_i)}{\sum\limits_{i=1}^{TC}M_n(C_i) \times DC(C_i)}$$
        \begin{itemize}
            \item $M_o(C_i)$~--- количество унаследованных и переопределенных методов в $C_i$
            \item $M_n(C_i)$~--- количество новых (не унаследованных и переопределенных) методов в $C_i$
            \item $DC(C_i)$~--- количество потомков класса $C_i$
        \end{itemize}
    \end{frame}

    \begin{frame}
        \frametitle{Фактор сопряжения (COF)}
        $$COF = \frac{\sum\limits_{i=1}^{TC}\biggl(\sum\limits_{j=1}^{TC}is\_client(C_i, C_j)\biggl)}{TC^2 - TC}$$

        \begin{equation*}
            is\_client(C_c, C_s) = \begin{cases}
                1, & \text{если}\ C_c => C_s \cap C_c \neq C_s \\
                0, & \text{в противном случае}.
            \end{cases}
        \end{equation*}

        \begin{itemize}
            \item $C_c$ => $C_s$~--- класс-клиент содержит по меньшей мере одну не унаследованную ссылку на атрибут или метод класса-поставщика
        \end{itemize}
    \end{frame}

    \begin{frame}
        \frametitle{Метрики для тестирования}
        \begin{itemize}
            \item Недостаток связности в методах
            \item Процент публичных и защищенных методов
            \item Публичный доступ к атрибутам
            \item Количество корневых классов
            \item Количество детей, Высота дерева наследования
            \item Процентное количество не переопределенных запросов
            \item Процентное количество динамических запросов
            \item Скачок класса, Скачок системы
        \end{itemize}
    \end{frame}

    \begin{frame}
        \begin{center}
            \includegraphics[width=0.7\textwidth]{wtfsPerMinute.png}
        \end{center}
    \end{frame}

    \begin{frame}
        \frametitle{Аудит программного кода}
        \begin{itemize}
            \item Сбор информации, накопление знаний, формирование эталонов
            \item Ручной
            \begin{itemize}
                \item Экспертный
                \item Расчётный
            \end{itemize}
            \item Автоматический
            \begin{itemize}
                \item \url{https://plugins.jetbrains.com/plugin/93-metricsreloaded}
                \item \url{http://metrics.sourceforge.net/}
                \item \url{https://www.codacy.com/}
            \end{itemize}
        \end{itemize}
    \end{frame}

    \section{CMM}

    \begin{frame}
        \frametitle{Capability Maturity Model Integration (CMMI)}
        \begin{itemize}
            \item Комплексная модель производительности и зрелости компании
            \item Пять уровней зрелости
            \item 22 области усовершенствования
            \begin{itemize}
                \item Управление процессами
                \item Управление проектами
                \item Инженерные области
                \item Служебные области
            \end{itemize}
            \item Цели: общие и специфические
            \item Best Practices
        \end{itemize}
    \end{frame}

    \begin{frame}
        \begin{center}
            \includegraphics[height=0.95\textheight]{cmmi.png}
        \end{center}
    \end{frame}

\end{document}