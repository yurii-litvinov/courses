\documentclass[xetex,mathserif,serif]{beamer}
\usepackage{polyglossia}
\setdefaultlanguage[babelshorthands=true]{russian}
\usepackage{minted}
\usepackage{tabu}

\useoutertheme{infolines}

\usepackage{fontspec}
\setmainfont{FreeSans}
\newfontfamily{\russianfonttt}{FreeSans}

\definecolor{links}{HTML}{2A1B81}
\hypersetup{colorlinks,linkcolor=,urlcolor=links}

\tabulinesep=0.8mm

\title{Продвинутый CI}
\author{Юрий Литвинов}
\date{21.02.2020г}

\begin{document}
	
	\frame{\titlepage}

	\begin{frame}
		\frametitle{Непрерывная интеграция, для тех, кто всё пропустил}
		Зачем?
		\begin{itemize}
			\item Чтобы собиралось и работало не только у вас на компе
			\item Чтобы было эталонное окружение и эталонный процесс сборки, который можно повторить кому угодно
			\item Чтобы узнавать об ошибках сразу после коммита
		\end{itemize}
		Как?
		\begin{itemize}
			\item Облачные CI-системы: AppVeyor, Travis, CircleCI, ...
			\item Локальные CI-системы: Jenkins, TeamCity, TFS, ...
		\end{itemize}
	\end{frame}

	\begin{frame}
		\frametitle{Более подробно, как}
		\framesubtitle{AppVeyor}
		\begin{itemize}
			\item Логинимся на \url{https://www.appveyor.com/} по своему аккаунту на GitHub
			\item Добавляем свой репозиторий через New Project -> GitHub -> Add
			\item Добавляем в рабочую копию (прямо у себя на компе) appveyor.yml
			\begin{itemize}
				\item Файл должен называться именно так, вплоть до регистра букв
				\item Обычно его кладут в мастер, потом вмердживают в каждую ветку с домашкой
			\end{itemize}
			\item Пишем туда описание процесса сборки
			\item Коммитим-пушим
			\item Наслаждаемся билдом
		\end{itemize}
	\end{frame}

	\begin{frame}[fragile]
		\frametitle{Что писать}
		\framesubtitle{AppVeyor, .NET Core}
		\url{https://www.appveyor.com/docs/appveyor-yml/}
		\begin{minted}{yaml}
image: Visual Studio 2017 

before_build: 
    - nuget restore myCoolHomework/Homework.sln

build: 
    project: myCoolHomework/Homework.sln

test_script: 
    - dotnet test myCoolHomework/Homework.sln
		\end{minted}
		Отступы и минусы критически важны.
	\end{frame}

	\begin{frame}
		\frametitle{При этом}
		\begin{itemize}
			\item Нужно добавить Reference на
			\begin{itemize}
				\item Microsoft.NET.Test.Sdk
				\item NUnit3TestAdapter
				\item NUnit
				\item FsUnit
			\end{itemize}
			\item Или другие библиотеки, которыми пользуетесь, но не забыть SDK и раннер.
			\item Как вообще создать проект:
			\begin{itemize}
				\item \url{https://docs.microsoft.com/en-us/dotnet/core/testing/unit-testing-fsharp-with-nunit}
			\end{itemize}
		\end{itemize}
	\end{frame}

	\begin{frame}[fragile]
		\frametitle{Что писать}
		\framesubtitle{AppVeyor, .NET Framework}
		\begin{minted}{yaml}
image: Visual Studio 2017 

before_build: 
    - nuget restore myCoolHomework/Homework.sln

build: 
    project: myCoolHomework/Homework.sln

test_script: 
    - nunit3-console myCoolHomework/Homework.sln \ 
        --result=myresults.xml;format=AppVeyor
		\end{minted}
	\end{frame}

	\begin{frame}[fragile]
		\frametitle{Что писать}
		\framesubtitle{Travis, .NET Core}
		\begin{minted}{yaml}
language: csharp

mono: none
dotnet: 2.1

before_build: 
    - nuget restore myCoolHomework/Homework.sln

build: 
    - dotnet build myCoolHomework/Homework.sln

test_script: 
    - dotnet test myCoolHomework/Homework.sln
		\end{minted}
	\end{frame}

	\begin{frame}[fragile]
		\frametitle{Что писать}
		\framesubtitle{Travis, .NET Framework}
		\begin{small}
			\begin{minted}{yaml}
language: csharp

install:
  - nuget restore myCoolHomework/Homework.sln
  - nuget install NUnit.Console -Version 3.9.0 -OutputDirectory testrunner

script:
  - msbuild /p:Configuration=Release myCoolHomework/Homework.sln
  - mono ./testrunner/NUnit.ConsoleRunner.3.9.0/tools/nunit3-console.exe \
        .myCoolHomework/Homework.sln
			\end{minted}
		\end{small}
	\end{frame}

	\begin{frame}[fragile]
		\frametitle{А как же другие домашки?}
		\framesubtitle{AppVeyor, .NET Core}
		\begin{minted}{yaml}
image: Visual Studio 2017 

environment:
    matrix:
        - SOLUTION: myCoolHomework/Homework1.sln
        - SOLUTION: myCoolHomework2/Homework2.sln

before_build: 
    - nuget restore %SOLUTION%

build: 
    project: %SOLUTION%

test_script: 
    - dotnet test %SOLUTION%
		\end{minted}
	\end{frame}

	\begin{frame}[fragile]
		\frametitle{Автоматизация}
		\framesubtitle{AppVeyor, .NET Core}
		\begin{minted}{yaml}
image: Visual Studio 2017 

build_script: 
    - For /R %%I in (*.sln) do dotnet test %%I

test: off
		\end{minted}
	\end{frame}

\end{document}