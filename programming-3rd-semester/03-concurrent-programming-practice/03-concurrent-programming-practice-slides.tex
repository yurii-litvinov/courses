\documentclass{../../slides-style}

\slidetitle[Практика]{Многопоточное программирование}{21.09.2022}

\begin{document}

    \begin{frame}[plain]
        \titlepage
    \end{frame}

    \section{Домашка}

    \begin{frame}
        \frametitle{По домашке про матрицы}
        \begin{itemize}
            \item Замерять производительность, запустив код один раз --- неправильно
            \begin{itemize}
                \item Масса случайных факторов, от планировщика до занятости ОС другими задачами
            \end{itemize}
            \item Все замеры времени \textbf{всегда} требуют матстата!
            \begin{itemize}
                \item Время работы --- случайная величина с нормальным (как правило) распределением
                \item Замеряя время, мы строим выборку этой величины
                \item По выборке можно оценить матожидание и дисперсию
                \item Именно матожидание и дисперсия есть результаты замера!
            \end{itemize}
        \end{itemize}
    \end{frame}

    \section{Задача}

    \begin{frame}
        \frametitle{Напоминание про условия взаимной блокировки}
        \begin{enumerate}
            \item имеется разделяемый ресурс, к которому потоки хотят получить доступ, но пользоваться им может только один поток
            \item таких ресурсов несколько и поток, захватив один, хочет получить доступ к другим, которые в этот момент захвачены другими потоками
            \item нельзя отнять захваченный ресурс у потока
            \item потоки ждут друг друга <<по кругу>>
        \end{enumerate}
        Блокировка возможна, тиолько если выполнены сразу все эти условия.
    \end{frame}

    \begin{frame}
        \frametitle{Задача, ``Обедающие философы''}
        \begin{columns}
            \begin{column}{0.5\textwidth}
                \begin{itemize}
                    \item Есть N тарелок спагетти, N вилок и N философов
                    \item Философ может думать и есть
                    \item Чтобы есть, философу нужны две вилки
                    \item Пример --- транзакция, переводящая деньги со счёта на счёт
                \end{itemize}
            \end{column}
            \begin{column}{0.5\textwidth}
                \includegraphics[width=0.9\textwidth]{diningPhilosophers.png}
                \attribution{A. Tanenbaum, Modern Operating Systems}
            \end{column}
        \end{columns}
    \end{frame}

    \begin{frame}
        \frametitle{Что надо сделать}
        \begin{itemize}
            \item Смоделировать ситуацию обедающих философов
            \begin{itemize}
                \item Придумать красивую объектно-ориентированную модель
            \end{itemize}
            \item Выводить на экран состояния философов
            \item Считаем, что философы думают и едят случайное, но небольшое количество времени
            \item Реализация должна гарантировать отсутствие взаимоблокировок
            \item Нужно уметь корректно останавливать процесс и распускать философов по домам
        \end{itemize}
    \end{frame}

\end{document}
