\documentclass{../../slides-style}

\slidetitle[Низкий уровень]{Работа с сетью}{30.09.2025}

\begin{document}

    \begin{frame}[plain]
        \titlepage
    \end{frame}

    \section{Архитектура сети}

    \begin{frame}
        \frametitle{Архитектура глобальной сети}
        \begin{center}
            \includegraphics[width=0.9\textwidth]{internetArchitecture.png}
            \attribution{Э. Таненбаум}
        \end{center}
    \end{frame}

    \begin{frame}
        \frametitle{Уровневая архитектура}
        \framesubtitle{Модель OSI}
        \begin{center}
            \includegraphics[width=0.6\textwidth]{osiStack.png}
            \attribution{Э. Таненбаум}
        \end{center}
    \end{frame}

    \begin{frame}
        \frametitle{Модель TCP/IP}
        \begin{center}
            \includegraphics[width=0.9\textwidth]{tcpIpStack.png}
            \attribution{Э. Таненбаум}
        \end{center}
    \end{frame}

    \begin{frame}
        \frametitle{Физический уровень}
        \begin{itemize}
            \item Физические параметры канала (электрические, электромагнитные, ...)
            \item Ethernet (витая пара), USB, xDSL, Bluetooth, IEEE 802.11 (WiFi), оптические сети, спутниковая связь, мобильные сети (GSM, EDGE, LTE) и т.д.
            \begin{itemize}
                \item RFC 1149 ``IP over Avian Carriers'' (\url{https://tools.ietf.org/html/rfc1149})
            \end{itemize}
            \item Отвечает только за передачу сигнала в рамках среды распространения между двумя точками
            \item Вопросы кодирования битов уровнями сигнала, синхронизации, помехоустойчивости, мультиплексирования
            \item Передаёт биты или блоки битов
        \end{itemize}
    \end{frame}

    \begin{frame}
        \frametitle{Канальный уровень}
        \begin{itemize}
            \item Общение напрямую соединённых устройств сети
            \item PPP (Point to Point Protocol)
            \item Понятия MAC и LLC (Logical Link Control)
            \begin{itemize}
                \item MAC-адрес: D8-FB-5E-E5-55-67
            \end{itemize}
            \item Вопросы коррекции ошибок физического уровня (коды Хэмминга, Рида-Соломона, свёрточные коды и прочая алгебра с теорией чисел), повтора передачи пропавших данных, управления скоростью передачи
            \item Передаёт фрэймы (или кадры)
        \end{itemize}
    \end{frame}

    \begin{frame}
        \frametitle{Сетевой уровень}
        \begin{itemize}
            \item Сеть из нескольких устройств
            \item Вопросы поиска оптимального маршрута внутри сети (роутинга), передачи по принципиально разным сетям (например, один пакет по оптоволокну, второй --- через спутник)
            \item IP (Internet Protocol)
            \item Понятие IP-адреса (IPv4, IPv6)
            \item Передаёт пакеты
        \end{itemize}
    \end{frame}

    \begin{frame}
        \frametitle{Транспортный уровень}
        \begin{itemize}
            \item Соединение двух устройств через сеть
            \item Вопросы надёжности доставки, разделения-сборки сообщения, правильного порядка сообщений, подтверждения и повторной отправки
            \item Протоколы TCP (Transmission Control Protocol), UDP (User Datagram Protocol)
            \begin{itemize}
                \item TCP --- протокол, гарантирующий доставку данных в правильном порядке, без потерь и порчи, если это вообще возможно
                \begin{itemize}
                    \item Передача файлов, текстовых данных (включая веб-страницы), веб-сервисы
                \end{itemize}
                \item UDP --- протокол, позволяющий отправлять ``датаграммы'' без гарантий их доставки или доставки в правильном порядке, но в разы быстрее TCP
                \begin{itemize}
                    \item Стриминг фильмов, музыки, компьютерные игры
                \end{itemize}
            \end{itemize}
        \end{itemize}
    \end{frame}

    \begin{frame}
        \frametitle{Сеансовый уровень}
        \begin{itemize}
            \item Установление, поддержание и закрытие соединения
            \item Протокол TCP
        \end{itemize}
    \end{frame}

    \begin{frame}
        \frametitle{Уровень представления}
        \begin{itemize}
            \item Кодировка и представление передаваемых данных
            \begin{itemize}
                \item Шифрование
                \item Сериализация/десериализация
            \end{itemize}
        \end{itemize}
    \end{frame}

    \begin{frame}
        \frametitle{Прикладной уровень}
        \begin{itemize}
            \item Общение конкретных приложений
            \item Протоколы HTTP, FTP, SMTP и т.д.
            \item Протоколы поверх HTTP: REST, SOAP и т.д.
        \end{itemize}
    \end{frame}

    \section{Технические детали}

    \begin{frame}
        \frametitle{IP-адреса}
        \begin{itemize}
            \item IPv4: 192.168.0.1 (4 байта)
            \begin{itemize}
                \item Уникален в рамках подсети (не глобально уникальный)
                \item 192.168.x.x, 172.16-31.x.x, 10.x.x.x --- адреса, зарезервированные для локальных подсетей
                \item 127.0.0.1 (точнее, 127.x.x.x) --- loopback (локальный адрес самого компа), часто используется для отладки
                \item Маска подсети --- битовая маска, определяющая кусок IP-адреса
            \end{itemize}
            \item IPv6: fe80::488f:1f6:9030:46c7\%10 (16 байт)
            \item Пример: \url{https://miminet.ru/web_network_shared?guid=385ccc51-9a6e-4b9a-8e90-fbf27ae73186}
        \end{itemize}
    \end{frame}

    \begin{frame}
        \frametitle{Формат пакета IPv4}
        \begin{center}
            \includegraphics[width=0.9\textwidth]{ipv4.png}
            \attribution{Э. Таненбаум}
        \end{center}
    \end{frame}

    \begin{frame}
        \frametitle{Порты и сокеты}
        \begin{itemize}
            \item Порт --- число от 1 до 65535, привязан к сетевому интерфейсу
            \item Ресурс, управляемый ОС
            \item Типичные порты
            \begin{itemize}
                \item 22 --- SSH
                \item 25 --- SMTP
                \item 80 --- HTTP
                \item 443 --- HTTPS
                \item 666 --- Doom
                \item Первые 1024 порта зарезервированы
            \end{itemize}
            \item Ненужные порты обычно закрыты на уровне ОС (фаерволл), чтобы было труднее взломать компьютер --- поэтому ваше первое сетевое приложение, скорее всего, не заработает
            \item Сокет --- программный интерфейс к порту
            \item Сетевой стек --- важная часть операционной системы, сокеты --- способ для прикладного программиста с ним работать
        \end{itemize}
    \end{frame}

    \begin{frame}
        \frametitle{DNS, NAT}
        \begin{itemize}
            \item DNS --- сопоставление непонятным IP-адресам читаемых доменных имён
            \begin{itemize}
                \item Более-менее глобальный сервис
                \item DNS-запрос по доменному имени (google.com) возвращает IP-адрес (64.233.164.113), только после этого возможен ``настоящий'' запрос
                \item Есть локальные DNS-сервера, есть общеизвестные (например, 8.8.8.8, Google Public DNS)
                \item localhost --- всегда (более-менее) раскрывается в 127.0.0.1
            \end{itemize}
            \item NAT --- Network Address Translation, механизм, позволяющий компьютерам с локальными IP получать ответы из Интернет (только если они инициировали запросы)
            \item Ports forwarding --- механизм, позволяющий компьютерам за NAT принимать входящие запросы
            \item Прокси --- программа, которая пересылает запросы (и может делать с ними что-нибудь)
        \end{itemize}
    \end{frame}

    \begin{frame}
        \frametitle{Как работает NAT}
        \framesubtitle{Или ещё одна причина, почему ваше первое сетевое приложение не заработает}
        \begin{center}
            \includegraphics[width=0.9\textwidth]{nat.png}
            \attribution{Э. Таненбаум}
        \end{center}
        \begin{itemize}
            \item Пример: \url{https://miminet.ru/web_network?guid=447a8424-3f84-4860-a85d-dfa2a1eefd1f}
        \end{itemize}
    \end{frame}

    \begin{frame}
        \frametitle{Полезные консольные команды}
        \begin{itemize}
            \item ping --- проверка соединения с указанным IP или доменным именем, показывает время отклика узла
            \begin{itemize}
                \item Удалённый компьютер имеет право не отвечать
            \end{itemize}
            \item tracert (traceroute) --- показывает все узлы, через которые шёл запрос со временами их отклика (если они хотят откликнуться)%
            \begin{itemize}
                \item Хороший способ диагностировать проблемы с интернетом
            \end{itemize}
            \item ipconfig под Windows, ifconfig под Linux --- узнать всё про локальный сетевой интерфейс (IP-адреса, MAC-адреса, используемые DNS и т.д.)
            \begin{itemize}
                \item Наиболее полезен ipconfig /all
            \end{itemize}
        \end{itemize}
    \end{frame}

    \begin{frame}[fragile]
        \frametitle{Полезные консольные команды (2)}
        \begin{itemize}
            \item netcat, nc --- позволяет опросить указанный порт или наоборот, прикинуться сервером, работающим по данному порту, очень полезна при отладке сетевых приложений
            \begin{itemize}
                \item Под Windows не входит в стандартную поставку, надо ставить отдельно
            \end{itemize}
            \item telnet --- открывает TCP-соединение с заданным хостом на заданный порт
            \begin{itemize}
                \item Пример:
                \begin{scriptsize}
                    \begin{minted}{text}
telnet smtp.gmail.com 25
220 smtp.gmail.com ESMTP m71-v6sm2246896lje.84 - gsmtp
HELP
214 2.0.0  https://www.google.com/search?btnI&q=RFC+5321 m71-v6sm2246896lje.84 
    - gsmtp
                    \end{minted}
                \end{scriptsize}
                \item Выйти --- Ctrl + `]', quit
                \item Под Windows не входит в стандартную поставку, надо ставить отдельно
            \end{itemize}
        \end{itemize}
    \end{frame}

    \section{Работа с сетью в .NET}

    \begin{frame}
        \frametitle{Работа с сетью в .NET}
        \begin{itemize}
            \item Пространство имён System.Net
            \item Классы TcpListener, TcpClient, UdpClient --- управляют стеком протоколов, предоставляют сокеты или потоки байтов
            \item Класс Socket --- абстракция сетевого соединения (сокета)
            \item Чаще всего в реальной жизни обработка запросов на сервере асинхронна --- каждый клиент обслуживается своей задачей в пуле потоков
            \item Dns --- класс, отвечающий за работу с DNS-службой
            \item IPEndPoint --- абстракция адреса (IP-адрес + порт)
        \end{itemize}
    \end{frame}

    \begin{frame}[fragile]
        \frametitle{Минимальный пример, сервер}
        \begin{footnotesize}
            \begin{minted}{csharp}
static void Main(string[] args)
{
    const int port = 8888;
    var listener = new TcpListener(IPAddress.Any, port);
    listener.Start();
    Console.WriteLine($"Listening on port {port}...");
    using (var socket = listener.AcceptSocket())
    {
        var stream = new NetworkStream(socket);
        var streamReader = new StreamReader(stream);
        var data = streamReader.ReadToEnd();
        Console.WriteLine($"Received: {data}");
    }
    listener.Stop();
}
            \end{minted}
        \end{footnotesize}
    \end{frame}

    \begin{frame}[fragile]
        \frametitle{Минимальный пример, клиент}
        \begin{footnotesize}
            \begin{minted}{csharp}
static void Main(string[] args)
{
    const int port = 8888;
    using (var client = new TcpClient("localhost", port))
    {
        Console.WriteLine($"Sending to port {port}...");
        var stream = client.GetStream();
        var writer = new StreamWriter(stream);
        writer.Write("Hello, world!");
        writer.Flush();
    }
}
            \end{minted}
        \end{footnotesize}
    \end{frame}

    \begin{frame}[fragile]
        \frametitle{Канал работает в обе стороны}
        \framesubtitle{Сервер}
        \begin{scriptsize}
            \begin{minted}{csharp}
static void Main(string[] args)
{
    const int port = 8888;
    var listener = new TcpListener(IPAddress.Any, port);
    listener.Start();
    Console.WriteLine($"Listening on port {port}...");
    using (var socket = listener.AcceptSocket())
    {
        var stream = new NetworkStream(socket);
        var reader = new StreamReader(stream);
        var data = reader.ReadLine();
        Console.WriteLine($"Received: {data}");

        Console.WriteLine($"Sending \"Hi!\"");
        var writer = new StreamWriter(stream);
        writer.Write("Hi!");
        writer.Flush();
    }
    listener.Stop();
}
            \end{minted}
        \end{scriptsize}
    \end{frame}

    \begin{frame}[fragile]
        \frametitle{Канал работает в обе стороны}
        \framesubtitle{Клиент}
        \begin{footnotesize}
            \begin{minted}{csharp}
static void Main(string[] args)
{
    const int port = 8888;
    using (var client = new TcpClient("localhost", port))
    {
        Console.WriteLine($"Sending \"Hello!\" to port {port}...");
        var stream = client.GetStream();
        var writer = new StreamWriter(stream);
        writer.WriteLine("Hello!");
        writer.Flush();

        var reader = new StreamReader(stream);
        var data = reader.ReadToEnd();
        Console.WriteLine($"Received: {data}");
    }
}
            \end{minted}
        \end{footnotesize}
    \end{frame}

    \begin{frame}[fragile]
        \frametitle{Немного асинхронности}
        \framesubtitle{Сервер}
        \begin{scriptsize}
            \begin{minted}{csharp}
static async Task Main(string[] args)
{
    const int port = 8888;
    var listener = new TcpListener(IPAddress.Any, port);
    listener.Start();
    Console.WriteLine($"Listening on port {port}...");
    using (var socket = await listener.AcceptSocketAsync())
    {
        var stream = new NetworkStream(socket);
        var reader = new StreamReader(stream);
        var data = await reader.ReadLineAsync();
        Console.WriteLine($"Received: {data}");

        Console.WriteLine($"Sending \"Hi!\"");
        var writer = new StreamWriter(stream);
        writer.AutoFlush = true;
        await writer.WriteAsync("Hi!");
    }
    listener.Stop();
}
            \end{minted}
        \end{scriptsize}
    \end{frame}

    \begin{frame}[fragile]
        \frametitle{Или, более типично}
        \framesubtitle{Сервер}
        \begin{scriptsize}
            \begin{minted}{csharp}
static async Task Main(string[] args) {
    const int port = 8888;
    var listener = new TcpListener(IPAddress.Any, port);
    listener.Start();
    Console.WriteLine($"Listening on port {port}...");
    while (true) {
        var socket = await listener.AcceptSocketAsync();
        Task.Run(async () => {
            var stream = new NetworkStream(socket);
            var reader = new StreamReader(stream);
            var data = await reader.ReadLineAsync();
            Console.WriteLine($"Received: {data}");

            Console.WriteLine($"Sending \"Hi!\"");
            var writer = new StreamWriter(stream);
            await writer.WriteAsync("Hi!");
            await writer.FlushAsync();

            socket.Close();
        });
    }
}
            \end{minted}
        \end{scriptsize}
    \end{frame}

    \begin{frame}[fragile]
        \frametitle{Теперь можно писать и читать одновременно}
        \framesubtitle{Полнодуплексное соединение, на примере сервера}
        \begin{scriptsize}
            \begin{minted}{csharp}
private static async Task Main(string[] args) {
    ...
    while (true) {
        var client = await listener.AcceptTcpClientAsync();
        Writer(client.GetStream());
        Reader(client.GetStream());
    }
}

private static void Writer(NetworkStream stream) {
    Task.Run(async () => {
        ...
    });
}

private static void Reader(NetworkStream stream) {
    Task.Run(async () => {
        ...
    });
}
            \end{minted}
        \end{scriptsize}
    \end{frame}

    \begin{frame}[fragile]
        \frametitle{Например}
        \begin{footnotesize}
            \begin{minted}{csharp}
private static void Writer(NetworkStream stream)
{
    Task.Run(async () =>
    {
        var writer = new StreamWriter(stream) { AutoFlush = true };
        while (true)
        {
            Console.WriteLine(">");
            var data = Console.ReadLine();
            await writer.WriteAsync(data + "\n");
        }
    });
}
            \end{minted}
        \end{footnotesize}
    \end{frame}

    \begin{frame}[fragile]
        \frametitle{UdpClient}
        \framesubtitle{Сервер}
        \begin{footnotesize}
            \begin{minted}{csharp}
static async Task Main(string[] args)
{
    const int port = 8888;
    var udpClient = new UdpClient(port);
    Console.WriteLine($"Listening on port {port}...");
    var received = await udpClient.ReceiveAsync();
    var data = Encoding.UTF8.GetString(received.Buffer);
    Console.WriteLine($"Received: {data}");
}
            \end{minted}
        \end{footnotesize}
    \end{frame}

    \begin{frame}[fragile]
        \frametitle{UdpClient}
        \framesubtitle{Клиент}
        \begin{footnotesize}
            \begin{minted}{csharp}
static async Task Main(string[] args)
{
    const int port = 8888;
    var udpClient = new UdpClient();

    Console.WriteLine($"Sending \"Hello!\" to port {port}...");
    var data = Encoding.UTF8.GetBytes("Hello!");
    await udpClient.SendAsync(data, data.Length, "localhost", port);
}
            \end{minted}
        \end{footnotesize}
        Не следует посылать UDP-датаграммы более 508 байт размером
    \end{frame}

\end{document}
