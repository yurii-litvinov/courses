\documentclass{../../slides-style}

\slidetitle{Рефлексия}{24.10.2024}

\begin{document}

    \begin{frame}[plain]
        \titlepage
    \end{frame}

    \section{Рефлексия}

    \begin{frame}
        \frametitle{Рефлексия}
        \begin{itemize}
            \item Позволяет во время выполнения получать информацию о типах
            \begin{itemize}
                \item И главное, создавать объекты этих типов и вызывать их методы
            \end{itemize}
            \item Зачем:
            \begin{itemize}
                \item Плагины
                \item Анализаторы кода
                \item Тестовые системы
                \item Библиотеки сериализации
                \item Библиотеки для работы с базами данных
                \item Inversion of Control-контейнеры
                \item ...
            \end{itemize}
            \item Проблемы:
            \begin{itemize}
                \item Медленно
                \item Нет помощи от системы типов
                \item Плохо работает с Ahead-of-time-компиляцией
            \end{itemize}
        \end{itemize}
    \end{frame}

    \section{Сборки}

    \begin{frame}
        \frametitle{Рефлексия в .NET}
        \begin{itemize}
            \item Пространство имён System.Reflection
            \item Байт-код хранит всю информацию о типах 
            \begin{itemize}
                \item И даже параметрах-типах у генериков, в отличие от Java
            \end{itemize}
            \item Самая крупная штука, которой оперирует рефлексия --- \textbf{сборка}
            \begin{itemize}
                \item .dll или .exe, единица развёртывания программы
                \begin{itemize}
                    \item Для .NET Framework это только \emph{почти} всегда так, детали см. в CLR via C\#
                \end{itemize}
            \end{itemize}
            \item Сборка хранит метаинформацию:
            \begin{itemize}
                \item Таблицы модулей, типов, методов, полей, параметров, свойств и событий
            \end{itemize}
            \item На всё это можно посмотреть в ILDasm (поставляется с VS), ILSpy, DotPeek
        \end{itemize}
    \end{frame}

    \begin{frame}[fragile]
        \frametitle{Загрузка сборки}
        \begin{small}
            \begin{minted}{csharp}
public class Assembly {
    public static Assembly Load(AssemblyName assemblyRef);
    public static Assembly Load(String assemblyString);
    public static Assembly Load(byte[] rawAssembly)
    public static Assembly LoadFrom(String path);
    ...
}
            \end{minted}
            например,
            \begin{minted}{csharp}
var a = Assembly.LoadFrom(@"http://example.com/ExampleAssembly.dll");
            \end{minted}
            Такая сборка должна быть ещё не загружена. Выгружать сборки нельзя.
        \end{small}
    \end{frame}

    \begin{frame}[fragile]
        \frametitle{Загрузка только метаданных сборки}
        \begin{itemize}
            \item MetadataLoadContext --- словарь, отображающий имя сборки в загруженные метаданные
            \item Работает, даже когда сборку нельзя загрузить по-настоящему
            \begin{itemize}
                \item Например, для другой целевой платформы
            \end{itemize}
            \item Живёт в отдельном NuGet-пакете System.Reflection.MetadataLoadContex
            \item MetadataAssemblyResolver --- абстракция алгоритма поиска нужной сборки
            \begin{itemize}
                \item PathAssemblyResolver --- ищет сборки в переданных путях
                \item Пути должны включать не только целевую сборку, но и все её зависимости (как минимум, \mintinline{csharp}|typeof(object).Assembly.Location|)
            \end{itemize}
        \end{itemize}
    \end{frame}

    \begin{frame}[fragile]
        \frametitle{Сильные и слабые имена сборок}
        \begin{itemize}
            \item Сильные сборки подписаны асимметричным шифром
            \begin{itemize}
                \item Публичная часть ключа внедряется в саму сборку
                \item От сборки считается SHA-1-хеш, шифруется приватным ключом и внедряется в сборку
                \item CLR при загрузке сборки раньше считал от неё SHA-1-хеш и проверял, что он совпал с подписанным
            \end{itemize}
            \item Пример сильного имени:
            \begin{scriptsize}
                \begin{minted}{csharp}
"MyTypes, Version=1.0.8123.0, Culture=neutral, PublicKeyToken=b77a5c561934e089"
                \end{minted}
            \end{scriptsize}
            \item PublicKeyToken --- короткий хеш публичного ключа
            \item Сборка с сильным именем может ссылаться только на сборки с сильными именами
            \item В современном .NET не очень актуально
            \begin{itemize}
                \item В силу уклона в контейнеризацию и \enquote{xcopy deployment}
            \end{itemize}
        \end{itemize}
    \end{frame}

    \begin{frame}[fragile]
        \frametitle{Пример}
        \framesubtitle{Распечатать имена всех типов в сборке}
        \begin{minted}{csharp}
using System.Reflection;

var assembly = Assembly.Load("Example");

foreach (Type t in assembly.GetTypes())
{
    Console.WriteLine(t.FullName);
}
        \end{minted}
        \begin{itemize}
            \item Уже загруженную сборку так не загрузить (ошибки не будет, но список типов будет пустым)
            \item Бывает GetTypes, DefinedTypes, GetExportedTypes
        \end{itemize}
    \end{frame}

    \section{Типы}

    \begin{frame}[fragile]
        \frametitle{Создание экземпляра объекта}
        \begin{itemize}
            \item System.Activator.CreateInstance --- можно передавать тип или строку с именем типа
            \begin{itemize}
                \item Версии со строкой возвращают System.Runtime.Remoting.ObjectHandle, надо вызвать Unwrap()
            \end{itemize}
            \item System.Activator.CreateInstanceFrom --- вызывает LoadFrom для сборки
            \item System.Reflection.ConstructorInfo.Invoke --- просто вызов конструктора (несколько дольше писать, чем предыдущие варианты)
            \item Рефлексия ничего не знает о синонимах
            \begin{itemize}
                \item То есть int везде называется System.Int32
            \end{itemize}
        \end{itemize}
        Пример:
        \begin{small}
            \begin{minted}{csharp}
var zero = Activator.CreateInstance("mscorlib.dll", "System.Int32").Unwrap();
            \end{minted}
        \end{small}
    \end{frame}

    \begin{frame}[fragile]
        \frametitle{Создание экземпляра типа-генерика}
        \begin{minted}{csharp}
Type openType = typeof(Dictionary<,>);
Type closedType = openType.MakeGenericType(
        typeof(string), typeof(int));
object o = Activator.CreateInstance(closedType);
Console.WriteLine(o.GetType());

internal sealed class Dictionary<TKey, TValue> { }
        \end{minted}
    \end{frame}

    \begin{frame}
        \frametitle{Пример: как сделать свою плагинную систему}
        \begin{itemize}
            \item Сделать отдельную сборку с описанием интерфейса плагина и типов данных, которые он использует
            \begin{itemize}
                \item Менять её будет очень проблематично
            \end{itemize}
            \item Сделать \enquote{ядро системы} --- отдельную сборку, ссылающуюся на сборку с интерфейсом плагина
            \item Делать набор плагинов, ссылающихся на сборку с интерфейсом плагина и реализующих его
        \end{itemize}
    \end{frame}

    \begin{frame}[fragile]
        \frametitle{Пример: интерфейс плагина}
        \begin{minted}{csharp}
namespace MyCoolSystem.SDK;

public interface IAddIn {
    string DoSomething(int x);
}
        \end{minted}
    \end{frame}

    \begin{frame}[fragile]
        \frametitle{Пример: плагины}
        \begin{minted}{csharp}
using MyCoolSystem.SDK;

public sealed class AddInA : IAddIn {
    public String DoSomething(int x)
        => "AddInA: " + x.ToString();
}

public sealed class AddInB : IAddIn {
    public String DoSomething(int x)
        => "AddInB: " + (x * 2).ToString();
}
        \end{minted}
    \end{frame}

    \begin{frame}[fragile]
        \frametitle{Пример: ядро системы}
        \begin{scriptsize}
            \begin{minted}{csharp}
using System.Reflection;

string addInDir = Path.GetDirectoryName(Assembly.GetEntryAssembly()?.Location) 
    ?? throw new Exception("Unable to load assembly");

var addInAssemblies = Directory.EnumerateFiles(addInDir, "*.dll");
var addInTypes =
    addInAssemblies.Select(Assembly.Load)
        .SelectMany(a => a.ExportedTypes)
        .Where(t => t.IsClass
                && typeof(IAddIn).GetTypeInfo().IsAssignableFrom(t.GetTypeInfo()));

foreach (Type t in addInTypes)
{
    var addIn = Activator.CreateInstance(t) as IAddIn;
    Console.WriteLine(addIn?.DoSomething(5));
}
            \end{minted}
        \end{scriptsize}
    \end{frame}

    \section{Поля, методы и т.д.}

    \begin{frame}
        \frametitle{Информация о типах}
        \begin{center}
            \begin{tiny}
                \begin{forest}
                    for tree={rectangle,draw,l sep=1cm,s sep=3mm,edge=open triangle 60-}
                    [Object
                        [Reflection.MemberInfo
                            [Reflection.TypeInfo]
                            [Reflection.FieldInfo]
                            [Reflection.MethodBase
                                [Reflection.ConstructorInfo]
                                [Reflection.MethodInfo]
                            ]
                            [Reflection.PropertyInfo]
                            [Reflection.EventInfo]
                        ]
                    ]
                \end{forest}
            \end{tiny}
        \end{center}
    \end{frame}

    \begin{frame}[fragile]
        \frametitle{Пример: распечатать информацию о полях и методах}
        \begin{scriptsize}
            \begin{minted}{csharp}
using System.Reflection;

Assembly[] assemblies = AppDomain.CurrentDomain.GetAssemblies();
foreach (Assembly a in assemblies) {
    Console.WriteLine($"Assembly: {a}");
    foreach (Type t in a.ExportedTypes) {
        Console.WriteLine($"  Type: {t}");
        foreach (MemberInfo mi in t.GetTypeInfo().DeclaredMembers) {
            var typeName = mi switch {
                FieldInfo _ => "FieldInfo",
                MethodInfo _ => "MethodInfo",
                ConstructorInfo _ => "ConstructorInfo",
                _ => ""
            };
            Console.WriteLine($"    {typeName}: {mi}");
        }
    }
}
            \end{minted}
        \end{scriptsize}
    \end{frame}

    \begin{frame}
        \frametitle{Полезные свойства MemberInfo}
        \begin{itemize}
            \item Name (string) --- имя члена класса
            \item DeclaringType (Type) --- тип
            \item Module (Module) --- модуль, в котором он объявлен
            \item CustomAttributes (IEnumerable<CustomAttributeData>) --- коллекция атрибутов, соответствующих этому члену класса
            \begin{itemize}
                \item Пример --- модульные тесты
            \end{itemize}
        \end{itemize}
    \end{frame}

    \begin{frame}
        \frametitle{Как что-нибудь сделать с MemberInfo}
        \begin{itemize}
            \item GetValue и SetValue для FieldInfo и PropertyInfo
            \item Invoke для ConstructorInfo и MethodInfo
            \item AddEventHandler и RemoveEventHandler для EventInfo
        \end{itemize}
    \end{frame}

    \begin{frame}[fragile]
        \frametitle{Пример: создать объект и вызвать его метод}
        \begin{scriptsize}
            \begin{minted}{csharp}
using System.Reflection;

Type t = typeof(SomeType);
Type ctorArgument = Type.GetType("System.Int32") ?? throw new Exception("Failed to load Int32");
ConstructorInfo ctor = t.GetTypeInfo().DeclaredConstructors.First(
        c => c.GetParameters()[0].ParameterType == ctorArgument);
object obj = ctor.Invoke([12]);
MethodInfo mi = obj.GetType().GetTypeInfo().GetDeclaredMethod("DoSomething") 
    ?? throw new Exception("Failed to find method");

int result = (int)(mi.Invoke(obj, [3]) ?? throw new Exception("Method returned null"));
Console.WriteLine($"result = {result}");

internal sealed class SomeType
{
    public SomeType(int test) { }
    private int DoSomething(int x) => x * 2;
}
            \end{minted}
        \end{scriptsize}
    \end{frame}

    \begin{frame}[fragile]{UnsafeAccessorAttribute}
        \begin{itemize}
            \item Атрибут, генерирующий код доступа к закрытому члену класса
            \item Появился в .NET 8
            \item \enquote{Рефлексия для бедных}, но в десятки раз быстрее рефлексии
        \end{itemize}
        Пример:
        \begin{scriptsize}
            \begin{minted}{csharp}
using System.Runtime.CompilerServices;

static void CallStaticPrivateMethod() {
    StaticPrivateMethod(null);

    [UnsafeAccessor(UnsafeAccessorKind.StaticMethod, Name = nameof(StaticPrivateMethod))]
    extern static void StaticPrivateMethod(Class c);
}
static void GetSetStaticPrivateField() {
    ref int f = ref GetSetStaticPrivateField(null);

    [UnsafeAccessor(UnsafeAccessorKind.StaticField, Name = "StaticPrivateField")]
    extern static ref int GetSetStaticPrivateField(Class c);
}
public class Class {
    static void StaticPrivateMethod() { }
    static int StaticPrivateField;
}
            \end{minted}
        \end{scriptsize}
    \end{frame}

    \section{Атрибуты}

    \begin{frame}
        \frametitle{Атрибуты}
        \begin{itemize}
            \item Способ добавить произвольную информацию к коду во время компиляции
            \item Эта информация может быть использована потом во время компиляции или во время выполнения
            \begin{itemize}
                \item Типичный пример --- атрибуты юнит-тестов (Test, ExpectedException, ...)
            \end{itemize}
            \item Могут быть применены к сборке, типу, полю, методу, параметру метода, возвращаемому значению, свойству, событию, параметру-типу
            \item Могут иметь параметры
            \item На самом деле, экземпляры классов-наследников System.Attribute
        \end{itemize}
    \end{frame}

    \begin{frame}[fragile]
        \frametitle{Объявление своего атрибута}
        \begin{footnotesize}
            \begin{minted}{csharp}
public enum Animal
{
    Dog = 1,
    Cat,
    Bird,
}

public class AnimalTypeAttribute : Attribute
{
    public AnimalTypeAttribute(Animal pet)
        => this.Pet = pet;

    public Animal Pet { get; set; }
}
            \end{minted}
        \end{footnotesize}
        \attribution{MSDN}
    \end{frame}

    \begin{frame}[fragile]
        \frametitle{Использование атрибута}
        \begin{minted}{csharp}
class AnimalTypeTestClass
{
    [AnimalType(Animal.Dog)]
    public void DogMethod() { }

    [AnimalType(Animal.Cat)]
    public void CatMethod() { }

    [AnimalType(Animal.Bird)]
    public void BirdMethod() { }
}
        \end{minted}
    \end{frame}

    \begin{frame}[fragile]
        \frametitle{Получение атрибута рефлексией}
        \begin{footnotesize}
            \begin{minted}{csharp}
using System.Reflection;

var testClass = new AnimalTypeTestClass();
Type type = testClass.GetType();

foreach (MethodInfo mInfo in type.GetMethods())
{
    foreach (Attribute attr in Attribute.GetCustomAttributes(mInfo))
    {
        // Check for the AnimalType attribute.
        if (attr.GetType() == typeof(AnimalTypeAttribute))
            Console.WriteLine(
                $"Method {mInfo.Name} has a pet " +
                $"{((AnimalTypeAttribute)attr).Pet} attribute.");
    }
}
            \end{minted}
        \end{footnotesize}
    \end{frame}

    \begin{frame}[fragile]
        \frametitle{Ограничение области применения атрибута}
        \begin{minted}{csharp}
namespace System;

[AttributeUsage(AttributeTargets.Enum, Inherited = false)]
public class FlagsAttribute : System.Attribute {
    public FlagsAttribute() {
    }
}
        \end{minted}
        Атрибуты у атрибутов!
    \end{frame}

    \section{dynamic}

    \begin{frame}[fragile]
        \frametitle{Ключевое слово dynamic}
        \begin{scriptsize}
            \begin{minted}{csharp}
static class DynamicDemo
{
    public static void Main()
    {
        dynamic value;
        for (int demo = 0; demo < 2; demo++)
        {
            value = (demo == 0) ? 5 : "A";
            value = value + value;
            M(value);
        }
    }

    private static void M(int n) { Console.WriteLine("M(int): " + n); }
    private static void M(string s) { Console.WriteLine("M(string): " + s); }
}
            \end{minted}
        \end{scriptsize}
    \end{frame}

    \section{Динамические сборки}

    \begin{frame}[fragile]
        \frametitle{Динамические сборки}
        \begin{itemize}
            \item Динамическая сборка --- сборка, существующая только в памяти
            \item Генерируется в рантайме средствами из System.Reflection.Emit
            \item Может быть сохранена на диск
            \begin{itemize}
                \item Всегда хотели написать свой язык и компилятор, но боитесь машинно-зависимых оптимизаций?
            \end{itemize}
            \item Для .NET неотличима от обычной сборки
        \end{itemize}
    \end{frame}

    \begin{frame}[fragile]
        \frametitle{Генерация кода \enquote{на лету}}
        \begin{scriptsize}
            \begin{minted}{csharp}
public static void Main() {
    AssemblyName assemblyName = new AssemblyName {Name = "HelloEmit"};
    AppDomain appDomain = AppDomain.CurrentDomain;
    AssemblyBuilder assemblyBuilder = appDomain.DefineDynamicAssembly(
        assemblyName, AssemblyBuilderAccess.Save);
    ModuleBuilder moduleBuilder = 
        assemblyBuilder.DefineDynamicModule(assemblyName.Name, "Hello.exe");
    TypeBuilder typeBuilder = moduleBuilder.DefineType("Test.MainClass",
        TypeAttributes.Public | TypeAttributes.Class);
    MethodBuilder methodBuilder = typeBuilder.DefineMethod("Main",
        MethodAttributes.Public | MethodAttributes.Static,
        typeof(int), new[] { typeof(string[]) });

    ILGenerator ilGenerator = methodBuilder.GetILGenerator();
    ilGenerator.Emit(OpCodes.Ldstr, "Hello, World!");
    ilGenerator.Emit(OpCodes.Call,
        typeof(Console).GetMethod("WriteLine", new[] { typeof(string) }));
    ilGenerator.Emit(OpCodes.Ldc_I4_0);
    ilGenerator.Emit(OpCodes.Ret);

    typeBuilder.CreateType();
    assemblyBuilder.SetEntryPoint(methodBuilder, PEFileKinds.ConsoleApplication);
    assemblyBuilder.Save("Hello.exe");
}
            \end{minted}
        \end{scriptsize}
    \end{frame}

\end{document}
