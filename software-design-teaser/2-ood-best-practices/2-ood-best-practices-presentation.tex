\documentclass[xetex,mathserif,serif]{beamer}
\usepackage{polyglossia}
\setdefaultlanguage[babelshorthands=true]{russian}
\usepackage{minted}

\useoutertheme{infolines}

\setmainfont{FreeSans}
\newfontfamily{\russianfonttt}{FreeSans}

\title{Принципы объектно-ориентированного проектирования}
\author[Юрий Литвинов]{Юрий Литвинов \newline \textcolor{gray}{\small\texttt{yurii.litvinov@gmail.com}}}
\date{03.04.2017г}

\begin{document}
	
	\frame{\titlepage}
	
	\section{Введение}

	\begin{frame}
		\frametitle{Модульность}
		\begin{itemize}
			\item Разделение системы на компоненты
			\item Потенциально позволяет создавать сколь угодно сложные системы
		\end{itemize}
		\vskip 1cm
		\begin{center}
			\includegraphics[width=0.5\textwidth]{modulesCost.png}
		\end{center}
	\end{frame}

	\begin{frame}
		\frametitle{Информационная закрытость}
		\begin{itemize}
			\item Со­держание модулей должно быть скрыто друг от друга
			\begin{itemize}
				\item Все модули независимы
				\item Обмениваются только информацией, необходимой для работы
				\item Доступ к операциям и структурам данных модуля ограничен
			\end{itemize}
			\item Обеспечивается возможность разработки модулей различными независимыми коллективами
			\item Обеспечивается лёгкая модификация системы
		\end{itemize}
	\end{frame}

	\begin{frame}
		\frametitle{Подходы к декомпозиции}
		\begin{itemize}
			\item Восходящее проектирование
			\item Нисходящее проектирование
			\begin{itemize}
				\item Постепенная реализация модулей
				\item Строгое задание интерфейсов
				\item Активное использование ``заглушек''
				\item Модули
				\begin{itemize}
					\item Четкая декомпозиция
					\item Минимизация
					\item Один модуль --- одна функциональность
					\item Отсутствие побочных эффектов
					\item Независимость от других модулей
					\item Принцип сокрытия данных
				\end{itemize}
			\end{itemize}
		\end{itemize}
	\end{frame}

	\begin{frame}
		\frametitle{Объекты}
		\begin{itemize}
			\item Objects may contain data, in the form of fields, often known as attributes; and code, in the form of procedures, often known as methods --- \textbf{\href{https://en.wikipedia.org/wiki/Object-oriented\_programming}{Wikipedia}}
			\item An object stores its state in fields and exposes its behavior through methods --- \textbf{\href{https://docs.oracle.com/javase/tutorial/java/concepts/object.html}{Oracle}}
			\item Each object looks quite a bit like a little computer --- it has a state, and it has operations that you can ask it to perform --- \textbf{\href{http://amzn.to/1PBmQpm}{Thinking in Java}}
			\item An object is some memory that holds a value of some type --- \textbf{\href{http://amzn.to/1XyGCtk}{The C++ Programming Language}}
			\item An object is the equivalent of the quanta from which the universe is constructed --- \textbf{\href{http://amzn.to/266oJr4}{Object Thinking}}
		\end{itemize}
	\end{frame}

	\section{Некоторые принципы ОО-проектирования}

	\begin{frame}
		\frametitle{Определение объектов реального мира}
		\begin{itemize}
			\item Определение объектов и их атрибутов
			\item Определение действий, которые могут быть выполнены над каждым объектом
			\item Определение связей между объектами
			\item Определение интерфейса каждого объекта
		\end{itemize}
		\begin{center}
			\includegraphics[width=0.4\textwidth]{billDomainModel.png}
		\end{center}
	\end{frame}

	\begin{frame}
		\frametitle{Согласованные абстракции}
		\begin{itemize}
			\item Выделение существенных характеристик объекта и игнорирование несущественных
			\item Определение его концептуальных границы с точки зрения наблюдателя
			\begin{itemize}
				\item Определение интерфейсов
			\end{itemize}
			\item Управление сложностью через фиксацию внешнего поведения
			\item Необходимы разные уровни абстракции
		\end{itemize}
		\begin{center}
			\includegraphics[width=0.55\textwidth]{abstraction.png}
		\end{center}
	\end{frame}

	\begin{frame}
		\frametitle{Инкапсуляция деталей реализации}
		\begin{itemize}
			\item Отделение друг от друга внутреннего устройства и внешнего поведения
			\item Изолирование контрактов интерфейса от реализации
			\item Управление сложностью через сокрытие деталей реализации
		\end{itemize}
		\vskip 1.5cm
		\begin{center}
			\includegraphics[width=0.55\textwidth]{incapsulation.png}
		\end{center}
	\end{frame}

	\begin{frame}
		\frametitle{Сокрытие ``лишней'' информации}
		\begin{columns}
			\begin{column}{0.65\textwidth}
				\begin{itemize}
					\item Изоляция ``личной'' информации
					\begin{itemize}
						\item секреты, которые скрывают сложность
						\item секреты, которые скрывают источники изменений
					\end{itemize}
					\item Барьеры, препятствующие сокрытию
					\begin{itemize}
						\item избыточное распространение информации
						\item поля класса как глобальные данные
						\item снижение производительности
					\end{itemize}
				\end{itemize}
			\end{column}
			\begin{column}{0.3\textwidth}
				\vskip 3cm
				\begin{flushright}
					\includegraphics[width=\textwidth]{complexityHiding.png}
				\end{flushright}
			\end{column}
		\end{columns}
	\end{frame}

	\begin{frame}
		\frametitle{Изоляция возможных изменений}
		\begin{itemize}
			\item Определите элементы, изменение которых кажется вероятным
			\item Отделите элементы, изменение которых кажется вероятным
			\item Изолируйте элементы, изменение которых кажется вероятным
			\item Источники изменений
			\begin{itemize}
				\item Бизнес-правила
				\item Зависимости от оборудования
				\item Ввод-вывод
				\item Нестандартные возможности языка
				\item Сложные аспекты проектирования и конструирования
				\item Переменные статуса
				\item Размеры структур данных
				\item ...
			\end{itemize}
		\end{itemize}
	\end{frame}

	\begin{frame}
		\frametitle{Сопряжение и связность}
		\begin{itemize}
			\item \textbf{Сопряжение (Coupling)} --- мера того, насколько взаимозависимы разные модули в программе
			\item \textbf{Связность (Cohesion)} --- степень, в которой задачи, выполняемые одним модулем, связаны друг с другом
			\item Цель: слабое сопряжение и сильная связность
		\end{itemize}
	\end{frame}

	\begin{frame}
		\frametitle{Дополнительные принципы}
		\begin{itemize}
			\item Формализуйте контракты классов
			\item Проектируйте систему для тестирования
			\item Рисуйте диаграммы
		\end{itemize}
	\end{frame}

	\begin{frame}
		\frametitle{Принципы SOLID}
		\begin{itemize}
			\item Single responsibility principle
			\item Open/closed principle
			\item Liskov substitution principle
			\item Interface segregation principle
			\item Dependency inversion principle
		\end{itemize}
	\end{frame}

	\begin{frame}
		\frametitle{Single responsibility principle}
		\begin{itemize}
			\item Каждый объект должен иметь одну обязанность
			\item Эта обязанность должна быть полностью инкапсулирована в класс
		\end{itemize}
		\begin{flushright}
			\includegraphics[width=0.25\textwidth]{singleResponsibility.png}
		\end{flushright}
	\end{frame}

	\begin{frame}
		\frametitle{Open/closed principle}
		\begin{itemize}
			\item программные сущности (классы, модули, функции и т. п.) должны быть открыты для расширения, но закрыты для изменения
			\begin{itemize}
				\item переиспользование через наследование
				\item неизменные интерфейсы
			\end{itemize}
		\end{itemize}
		\begin{flushright}
			\includegraphics[width=0.5\textwidth]{openClosedPrinciple.png}
		\end{flushright}
	\end{frame}

	\begin{frame}
		\frametitle{Liskov substitution principle}
		\begin{itemize}
			\item Функции, которые используют базовый тип, должны иметь возможность использовать подтипы базового типа, не зная об этом
		\end{itemize}
		\begin{flushright}
			\includegraphics[width=0.4\textwidth]{liskovSubstitutionPrinciple.png}
		\end{flushright}
	\end{frame}

	\begin{frame}
		\frametitle{Interface segregation principle}
		\begin{itemize}
			\item Клиенты не должны зависеть от методов, которые они не используют
			\begin{itemize}
				\item слишком ``толстые'' интерфейсы необходимо разделять на более мелкие и специфические
			\end{itemize}
		\end{itemize}
		\begin{flushright}
			\includegraphics[width=0.5\textwidth]{interfaceSegregationPrinciple.png}
		\end{flushright}
	\end{frame}

	\begin{frame}
		\frametitle{Dependency inversion principle}
		\begin{itemize}
			\item Модули верхних уровней не должны зависеть от модулей нижних уровней. Оба типа модулей должны зависеть от абстракций
			\item Абстракции не должны зависеть от деталей. Детали должны зависеть от абстракций
		\end{itemize}
		\begin{flushright}
			\includegraphics[width=0.5\textwidth]{dependencyInversionPrinciple.png}
		\end{flushright}
	\end{frame}

	\begin{frame}
		\frametitle{Закон Деметры}
		\begin{itemize}
			\item ``Не разговаривай с незнакомцами!''
			\item Объект A не должен иметь возможность получить непосредственный доступ к объекту C, если у объекта A есть доступ к объекту B, и у объекта B есть доступ к объекту C
			\begin{itemize}
				\item \mintinline{java}|book.pages.last.text|
				\item \mintinline{java}|book.pages().last().text()|
				\item \mintinline{java}|book.lastPageText()|
			\end{itemize}
		\end{itemize}
	\end{frame}

	\section{Технические подробности}

	\begin{frame}
		\frametitle{Абстрактные типы данных}
		\begin{itemize}
			\item \mintinline{java}|currentFont.size = 16| --- плохо
			\item \mintinline{java}|currentFont.size = PointsToPixels(12)| --- чуть лучше
			\item \mintinline{java}|currentFont.sizeInPixels = PointsToPixels(12)| --- ещё чуть лучше
			\item \mintinline{java}|currentFont.setSizeInPoints(sizeInPoints)| \newline
					\mintinline{java}|currentFont.setSizeInPixels(sizeInPixels)| --- совсем хорошо
		\end{itemize}
	\end{frame}

	\begin{frame}[fragile]
		\frametitle{Пример плохой абстракции}
		\begin{minted}{java}
public class Program {
   public void initializeCommandStack() { ... }
   public void pushCommand(Command command) { ... }
   public Command popCommand() { ... }
   public void shutdownCommandStack() { ... }
   public void initializeReportFormatting() { ... }
   public void formatReport(Report report) { ... }
   public void printReport(Report report) { ... }
   public void initializeGlobalData() { ... }
   public void shutdownGlobalData() { ... }
}
		\end{minted}
\end{frame}

	\begin{frame}[fragile]
		\frametitle{Пример хорошей абстракции}
		\begin{footnotesize}
			\begin{minted}{java}
public class Employee {
   public Employee(
           FullName name,
           String address,
           String workPhone,
           String homePhone,
           TaxId taxIdNumber,
           JobClassification jobClass
   ) { ... }

   public FullName getName() { ... }
   public String getAddress() { ... }
   public String getWorkPhone() { ... }
   public String getHomePhone() { ... }
   public TaxId getTaxIdNumber() { ... }
   public JobClassification getJobClassification() { ... }
}
			\end{minted}
		\end{footnotesize}
\end{frame}

	\begin{frame}[fragile]
		\frametitle{Уровень абстракции (плохо)}
		\begin{minted}{java}
public class EmployeeRoster implements MyList<Employee> {
   public void addEmployee(Employee employee) { ... }
   public void removeEmployee(Employee employee) { ... }
   public Employee nextItemInList() { ... }
   public Employee firstItem() { ... }
   public Employee lastItem() { ... }
}
		\end{minted}
\end{frame}

	\begin{frame}[fragile]
		\frametitle{Уровень абстракции (хорошо)}
		\begin{minted}{java}
public class EmployeeRoster {
   public void addEmployee(Employee employee) { ... }
   public void removeEmployee(Employee employee) { ... }
   public Employee nextEmployee() { ... }
   public Employee firstEmployee() { ... }
   public Employee lastEmployee() { ... }
}
		\end{minted}
\end{frame}

	\begin{frame}
		\frametitle{Общие рекомендации}
		\begin{itemize}
			\item Про каждый класс знайте, реализацией какой абстракции он является
			\item Учитывайте противоположные методы (add/remove, on/off, ...)
			\item Соблюдайте принцип единственности ответственности
			\begin{itemize}
				\item Может потребоваться разделить класс на несколько разных классов просто потому, что методы по смыслу слабо связаны
			\end{itemize}
			\item По возможности делайте некорректные состояния невыразимыми в системе типов
			\begin{itemize}
				\item Комментарии в духе ``не пользуйтесь объектом, не вызвав  init()'' можно заменить конструктором
			\end{itemize}
			\item При рефакторинге надо следить, чтобы интерфейсы не деградировали
		\end{itemize}
	\end{frame}

	\begin{frame}[fragile]
		\frametitle{Инкапсуляция}
		\begin{itemize}
			\item Принцип минимизации доступности методов
			\item Паблик-полей не бывает:
		\end{itemize}
		\begin{columns}
			\begin{column}{0.25\textwidth}
				\begin{minted}{java}
class Point {
   public float x;
   public float y;
   public float z;
}
				\end{minted}
			\end{column}
			\begin{column}{0.1\textwidth}
				vs
			\end{column}
			\begin{column}{0.5\textwidth}
				\begin{minted}{java}
class Point {
   private float x;
   private float y;
   private float z;
   public float getX() { ... }
   public float getY() { ... }
   public float getZ() { ... }
   public void setX(float x) { ... }
   public void setY(float y) { ... }
   public void setZ(float z) { ... }
}
				\end{minted}
			\end{column}
		\end{columns}
\end{frame}

	\begin{frame}
		\frametitle{Ещё рекомендации}
		\begin{itemize}
			\item Класс не должен ничего знать о своих клиентах
			\item Лёгкость чтения кода важнее, чем удобство его написания
			\item Опасайтесь семантических нарушений инкапсуляции
			\begin{itemize}
				\item ``Не будем вызывать ConnectToDB(), потому что GetRow() сам его вызовет, если соединение не установлено'' --- это программирование \textit{сквозь} интерфейс
			\end{itemize}
			\item Protected- и package- полей тоже не бывает
			\begin{itemize}
				\item На самом деле, у класса два интерфейса --- для внешних объектов и для потомков (может быть отдельно третий, для классов внутри пакета, но это может быть плохо)
			\end{itemize}
		\end{itemize}
	\end{frame}

	\begin{frame}
		\frametitle{Наследование}
		\begin{itemize}
			\item Включение лучше
			\begin{itemize}
				\item Переконфигурируемо во время выполнения
				\item Более гибко
				\item Иногда более естественно
			\end{itemize}
			\item Наследование --- отношение ``является'', закрытого наследования не бывает
			\begin{itemize}
				\item Наследование --- это наследование интерфейса (полиморфизм подтипов, subtyping)
			\end{itemize}
			\item Хороший тон --- явно запрещать наследование (final- или sealed-классы)
			\item Не вводите новых методов с такими же именами, как у родителя
			\item Code smells:
			\begin{itemize}
				\item Базовый класс, у которого только один потомок
				\item Пустые переопределения
				\item Очень много уровней в иерархии наследования
			\end{itemize}
		\end{itemize}
	\end{frame}

	\begin{frame}[fragile]
		\frametitle{Пример}
		\begin{footnotesize}
			\begin{columns}
				\begin{column}{0.35\textwidth}
					\begin{minted}{java}
class Operation {
   private char sign = '+';
   private int left;
   private int right;
   public int eval()
   {
       switch (sign) {
           case '+': return left + right;
       }
       throw new RuntimeException();
   }
}
					\end{minted}
				\end{column}
				\begin{column}{0.1\textwidth}
					vs
				\end{column}
				\begin{column}{0.45\textwidth}
					\begin{minted}{java}
abstract class Operation {
   private int left;
   private int right;
   protected int getLeft() { return left; }
   protected int getRight() { return right; }
   abstract public int eval();
}

class Plus extends Operation {
   @Override public int eval() { 
        return getLeft() + getRight(); 
   }
}
					\end{minted}
				\end{column}
			\end{columns}
		\end{footnotesize}
\end{frame}

	\begin{frame}
		\frametitle{Конструкторы}
		\begin{itemize}
			\item Инициализируйте все поля, которые надо инициализировать
			\begin{itemize}
				\item После конструктора должны выполняться все инварианты
			\end{itemize}
			\item НЕ вызывайте виртуальные методы из конструктора
			\item private-конструкторы для объектов, которые не должны быть созданы (или одиночек)
			\item Deep copy предпочтительнее Shallow copy
			\begin{itemize}
				\item Хотя второе может быть эффективнее
			\end{itemize}
		\end{itemize}
	\end{frame}

	\begin{frame}
		\frametitle{Когда создавать классы}
		\begin{itemize}
			\item Объекты предметной области
			\item Абстрактные объекты
			\item Изоляция сложности
			\item Сокрытие деталей реализации
			\item Изоляция изменчивости
			\item Упаковка родственных операций
			\begin{itemize}
				\item Статические классы вполне ок
			\end{itemize}
		\end{itemize}
	\end{frame}

\end{document}