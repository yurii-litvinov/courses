\documentclass{../../slides-style}

\slidetitle{Непрерывная интеграция, оформление репозитория}{09.07.2025}

\begin{document}

    \begin{frame}[plain]
        \titlepage
    \end{frame}

    \section{Continuous Integration}

    \begin{frame}
        \frametitle{Continuous Integration}
        Непрерывная интеграция --- практика слияния всех изменений по нескольку раз в день, сборки их в известном окружении и запуска модульных тестов.
        \begin{itemize}
            \item Автоматическая сборка
            \begin{itemize}
                \item Всё, что нужно для сборки, есть в репозитории, может быть получено на чистую (ну, практически) машину и собрано одной консольной командой
            \end{itemize}
            \item Большое количество юнит-тестов, запускаемых автоматически
            \item Выделенная машина, слушающая репозиторий и выполняющая сборку
            \begin{itemize}
                \item Чаще всего каждая сборка запускается на заранее настроенной виртуальной машине или в Docker-контейнере
            \end{itemize}
        \end{itemize}
    \end{frame}

    \begin{frame}
        \frametitle{Continuous Integration}
        \begin{itemize}
            \item Извещение всех разработчиков о статусе
            \begin{itemize}
                \item Если сборка не прошла, разработка приостанавливается до её починки
            \end{itemize}
            \item Автоматическое выкладывание
            \item Пока сборка не прошла, задача не считается сделанной
            \begin{itemize}
                \item Короткие сборки (<10 мин.)
                \item deployment pipeline
                \begin{itemize}
                    \item Отдельная машина для сборки, для коротких тестов, для длинных тестов, для выкладывания
                \end{itemize}
            \end{itemize}
        \end{itemize}
    \end{frame}

    \section{GitHub Actions}

    \begin{frame}
        \frametitle{GitHub Actions}
        \begin{itemize}
            \item Бесплатная система облачной сборки для проектов на GitHub
            \item \url{https://docs.github.com/en/actions}
            \item Как настроить:
            \begin{itemize}
                \item В репозитории на GitHub Settings -> Actions -> Allow all actions
                \item Создаём в корне репозитория папку .github/workflows/
                \item В нём создаём файл <имя действия>.yml (например, ci.yml)
                \item Описываем процесс сборки согласно \url{https://docs.github.com/en/actions/learn-github-actions/workflow-syntax-for-github-actions}
                \begin{itemize}
                    \item Пример и описание линуксовой сборки: \url{https://www.incredibuild.com/blog/using-github-actions-with-your-c-project}
                \end{itemize}
                \item Коммитим-пушим
                \item Смотрим статус коммита и пуллреквеста
            \end{itemize}
        \end{itemize}
    \end{frame}

    \begin{frame}
        \frametitle{Что получится}
        \begin{center}
            \includegraphics[width=0.9\textwidth]{githubActionsBuildStatus.png}
        \end{center}
        И появятся иконки статуса рядом с коммитами и пуллреквестами
    \end{frame}

    \begin{frame}
        \frametitle{GitHub Actions, Workflow и Job}
        \begin{center}
            \includegraphics[width=0.7\textwidth]{githubActionsWorkflow}
        \end{center}
        \begin{itemize}
            \item Step --- это либо скрипт, либо \emph{Action}
            \item Action --- произвольный код (по сути, отдельное приложение), выполняющийся как шаг Job-а
            \begin{itemize}
                \item Переиспользуемый строительный блок
                \item Можно переиспользовать Workflow-ы
            \end{itemize}
        \end{itemize}
    \end{frame}

    \begin{frame}[fragile]
        \frametitle{Типичный Workflow для сборки}
        \begin{minted}{yaml}
name: Build
on: [push, pull_request]
jobs:
    build:
        runs-on: ubuntu-latest
        steps:
            - uses: actions/checkout@v4
            - uses: actions/setup-dotnet@v4
                with:
                    dotnet-version: '9.x'
            - name: Build and run tests
                run: dotnet test
        \end{minted}
    \end{frame}

    \begin{frame}[fragile]
        \frametitle{Матрица сборки}
        \begin{minted}{yaml}
name: Build
on: [push, pull_request]
jobs:
    build:
        runs-on: ${{ matrix.os }}
        strategy:
            matrix:
                os: [ubuntu-latest, windows-latest]
        steps:
            - uses: actions/checkout@v4
            - uses: actions/setup-dotnet@v4
                with:
                    dotnet-version: '9.x'
            - name: Build and run tests
                run: dotnet test
        \end{minted}
    \end{frame}

    \begin{frame}[fragile]
        \frametitle{Переменные окружения}
        \begin{scriptsize}
            \begin{minted}{yaml}
env:
  DAY_OF_WEEK: Monday

jobs:
  greeting_job:
    runs-on: ubuntu-latest
    env:
      Greeting: Hello
    steps:
      - name: "Say Hello Mona it's Monday"
        if: ${{ env.DAY_OF_WEEK == 'Monday' }}
        run: echo "$Greeting $First_Name. Today is $DAY_OF_WEEK!"
        env:
          First_Name: Mona
            \end{minted}
        \end{scriptsize}
    \end{frame}

    \begin{frame}
        \frametitle{Что ещё?}
        \begin{itemize}
            \item Секреты
            \begin{itemize}
                \item \mintinline{yaml}|super_secret: ${{ secrets.SUPERSECRET }}|
            \end{itemize}
            \item Кеширование промежуточных результатов
            \item Автоматическое развёртывание
            \begin{itemize}
                \item В том числе, автодеплой документации на github-pages
            \end{itemize}
            \item Проверка стиля кодирования, статический анализ кода и т.п.
            \begin{itemize}
                \item Может быть интересно для Python-разработчиков
            \end{itemize}
            \item Можно иметь несколько Workflow-ов в одном репозитории
        \end{itemize}
    \end{frame}

    \section{Оформление репозитория}

    \begin{frame}
        \frametitle{Оформление репозитория (1)}
        \begin{itemize}
            \item README.md --- самая важная часть любого репозитория
            \begin{itemize}
                \item Плашки CI и анализаторов
                \item Общее описание проекта
                \item Пример использования (с картинками, если уместно)
                \item Как собрать и запустить
                \item Если проект большой, то куда писать баги и как поучаствовать в разработке
            \end{itemize}
            \item Настроена секция About, указаны темы (topics)
            \item Адекватные сообщения к коммитам
            \begin{itemize}
                \item См. \url{https://www.conventionalcommits.org/en/v1.0.0/}
            \end{itemize}
            \item Настроена защита веток (это в Settings на GitHub)
            \item Настроен Dependabot, CodeQL
            \item Выкладываются релизы (раздел Releases репозитория)
        \end{itemize}
    \end{frame}

    \begin{frame}
        \frametitle{Оформление репозитория (2)}
        \begin{itemize}
            \item .gitignore (и .gitattributes, если используете кириллицу и не хотите эльфийские руны в диффах)
            \begin{itemize}
                \item Никаких результатов сборки в репо быть не должно
            \end{itemize}
            \item CI обязательно
            \item Модульные тесты в CI --- обязательно
            \item Линтер --- обязательно
            \item Внешние анализаторы типа Codacy или CodeCov --- опционально, но чем больше --- тем лучше
            \item Техническая документация по коду --- прямо в README, в вики или в комментариях в коде
            \begin{itemize}
                \item Настроена автоматическая сборка и деплой документации на GitHub Pages
            \end{itemize}
            \item Лицензия --- обязательно
        \end{itemize}
    \end{frame}

    \begin{frame}
        \frametitle{Оформление пуллреквеста}
        \begin{itemize}
            \item Короткое, но понятное и отражающее суть дела название
            \item Описание пуллреквеста
            \begin{itemize}
                \item Список предлагаемых изменений --- если применимо, со ссылками на закрытые issues, в формате \# номер
                \item Описание работы предлагаемой функциональности:
                \begin{itemize}
                    \item Если фронтенд, с gif-кой с демо
                    \item Если оптимизация, то замеры производительности (статистически корректные)
                    \item Если бэкенд, то описание новых методов API и сценариев использования и т.д.
                    \item Даже если делаете пуллреквест к лучшим друзьям в репо
                \end{itemize}
                \item Техническое описание изменений
                \begin{itemize}
                    \item Можно диаграмму классов UML и текст, можно просто текстом
                \end{itemize}
            \end{itemize}
            \item Модульные тесты на новый код
            \item CI проходит, новые тесты проходят
            \item Проверена лицензионная совместимость
            \item Стиль кодирования и коментариев к коммитам соответствуют принятым
            \begin{itemize}
                \item Если надо склеить коммиты в один, склеены в один
            \end{itemize}
        \end{itemize}
    \end{frame}

    \section{Лицензии}

    \begin{frame}
        \frametitle{Лицензия}
        \begin{itemize}
            \item Open source-кодом можно пользоваться, только если автор явно это разрешил, так что просто код на GitHub --- не совсем open source
            \item Бывают исключительные и личные неимущественные права
            \begin{itemize}
                \item Личные неимущественные права неотчуждаемы
                \item Исключительные права можно передать
                \item Права появляются в момент создания произведения и принадлежат автору
                \begin{itemize}
                    \item Если произведение создано по служебному заданию --- работодателю
                    \item Знак копирайта служит только для информирования, регистрация прав не требуется
                \end{itemize}
                \item Соавторы владеют произведением в равной степени
            \end{itemize}
            \item Идея не охраняется, охраняется её физическое выражение
        \end{itemize}
    \end{frame}

    \begin{frame}
        \frametitle{Open source-лицензии}
        \begin{itemize}
            \item Лицензия --- способ передачи части прав на произведение
            \item Пример --- ``Do what the **** you want to public license''
            \begin{itemize}
                \item ``Want to'' может включать в себя патентование произведения и подачу в суд на автора за нарушение патента, поэтому обычно лицензии более длинны и унылы
                \item В России и Европе программы не патентуют, в США --- да
            \end{itemize}
            \item Каждый нормальный open source-проект должен иметь лицензию
        \end{itemize}
    \end{frame}

    \begin{frame}
        \frametitle{Open source-лицензии}
        \begin{itemize}
            \item Часто используемые open source-лицензии:
            \begin{itemize}
                \item GPL, LGPL (GPL вирусная, поэтому использовать её, внезапно, плохая практика)
                \item MIT License
                \item Apache License 2.0 (может применяться пофайлово)
                \item BSD License (в разных вариантах)
                \item The Unlicense --- явная передача произведения в Public Domain
                \item Семейство лицензий Creative Commons --- не для софта, но хорошо подходит для ресурсов (картинок, текстов и т.д.)
            \end{itemize}
        \end{itemize}
    \end{frame}

\end{document}
