\documentclass{../../slides-style}

\slidetitle{Файлы и память}{14.02.2026}

\begin{document}

    \begin{frame}[plain]
        \titlepage
    \end{frame}

    \begin{frame}
        \frametitle{Формальные вопросы}
        \begin{outline}
            \1 Занятия по субботам на второй и третьей паре в 3389
                \2 В расписании аж четыре пары, по две для каждой подгруппы
            \1 Курс на HwProj: \url{https://hwproj.ru/courses/50071}
            \1 Будут три контрольные, на 4-й паре
            \1 Будут \enquote{Летучки}
            \1 В конце курса будут доклады
            \1 \emph{Конец курса в начале апреля!}
                \2 Зачёт будет прямо посреди семестра!
        \end{outline}
    \end{frame}

    \begin{frame}
        \frametitle{Критерии оценивания}
        \begin{outline}
            \1 Баллы:
                \2 За домашки (их будет мало, но объёмные)
                \2 За контрольные --- выбирается две лучшие попытки из трёх
                    \3 Плюс переписывания на зачёте, пересдаче, комиссии
            \1 Итоговый балл за домашки: $MAX(0, (\frac{n}{N}\ –\ 0.6)) * 2.5 * 100$
                \2 Примерно 50 баллов максимум за обязательные домашние работы
            \1 Летучки дают дополнительные баллы к домашке (максимум где-то три балла за каждую), которые не считаются в максимум
            \1 Доклады также дают дополнительные баллы (примерно 5), но они достанутся только наиболее обречённым
            \1 Есть дедлайны (минус балл к максимуму за каждую неделю просрочки, но не больше половины баллов)
            \1 В качестве итогового берётся \textbf{минимум} из баллов за домашние работы и контрольные
        \end{outline}
    \end{frame}

    \begin{frame}
        \frametitle{Примерные баллы}
        \begin{tabu} {| X[0.9 l p] | X[1 l p] | X[1 l p] | }
            \tabucline-
            Балл за домашку         & Балл за контрольные   & Оценка ECTS  \\
            \tabucline-
            \everyrow{\tabucline-}
            48-50                   & 18-20                 & A            \\
            46-47                   & 16-17                 & B            \\
            44-45                   & 14-15                 & C            \\
            43-44                   & 12-13                 & D            \\
            40-42                   & 10-11                 & E            \\
            0-39                    & 0-9                   & на пересдачу \\
        \end{tabu}
    \end{frame}

    \begin{frame}
        \frametitle{Что будет в курсе}
        \begin{outline}
            \1 Алгоритмы и структуры данных
                \2 Деревья, деревья поиска, самобалансирующиеся деревья
                \2 Графы
                \2 Формальные языки, автоматы и лексический анализ
                \2 Немного больше про сортировки
            \1 Системное программирование
                \2 Немного про то, что, судя по зачёту, не очень зашло на \enquote{Информатике}: файлы, память, функции ОС и как этим пользоваться в программах
                \2 Профиляторы, Perf
        \end{outline}
        Сначала будет просто, потом тяжело, потом снова просто
    \end{frame}

\end{document}

