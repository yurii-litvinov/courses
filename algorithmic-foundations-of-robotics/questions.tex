\documentclass[a5paper]{article}
\usepackage[a5paper, top=8mm, bottom=8mm, left=8mm, right=8mm]{geometry}

\usepackage{polyglossia}
\setdefaultlanguage{russian}

\usepackage{fontspec}
\setmainfont{Liberation Serif}
\newfontfamily{\cyrillicfont}{Liberation Serif}

\usepackage[style=russian]{csquotes}

\sloppy
\pagestyle{plain}

\begin{document}

\thispagestyle{empty}

\section*{Вопросы к экзамену \enquote{Алгоритмические основы робототехники}}

\begin{flushright}\begin{small}Юрий Литвинов\\\small{y.litvinov@spbu.ru}\end{small}\end{flushright}

\begin{enumerate}
    \item Понятие робототехники как междисциплинарного направления научных исследований и инженерных разработок. Примеры \enquote{роботов}, технологий и алгоритмов. 
    \item Типы механизмов движения, достоинства и недостатки. Летающие, колёсные, гусеничные, ползающие,  ходячие  и прочие всякие нестационарные роботы.
    \item Степень мобильности, степень управляемости и степень маневренности робота, степени свободы, голономность. Управление движением.  
    \item Колёсные роботы, виды колёс, конфигурации колёс, достоинства и недостатки.
    \item Кинематическая модель трёхколёсного робота с дифференциальным управлением. Кинематика разных видов колёс.
    \item Типы сенсоров (пассивные, активные, proprioceptive, exteroceptive). Виды сенсоров, физические принципы их работы, принципиальные достоинства и недостатки.
    \item Точность измерения и ошибка (систематическая, случайная), представления погрешности измерений, распространение ошибки.
    \item Основные алгоритмы технического зрения: depth from focus, оптический поток, стереозрение, structure from motion, ZLoG.
    \item Выделение геометрических примитивов.
    \item Общая схема локализации, трудности, матмодель погрешности одометрической локализации.
    \item Поведенческая модель как альтернатива локализации, представление гипотезы о позиции робота.
    \item Другие принципы локализации (по ориентирам, по маякам, включая спутниковые)
    \item Представление карты (непрерывное, дискретное с точной декомпозицией, с фиксированной и адаптивной сеткой, топологическое представление). 
    \item Автономное построение карты. Алгоритмы планирования пути. 
    \item Алгоритмы обхода препятствий.
    \item Робототехнические платформы, аппаратное обеспечение, шины.
\end{enumerate}

\end{document}
