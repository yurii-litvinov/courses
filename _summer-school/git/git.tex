\documentclass[xetex,mathserif,serif]{beamer}
\usepackage{polyglossia}
\setdefaultlanguage[babelshorthands=true]{russian}
\usepackage{minted}
\usepackage{tabu}

\useoutertheme{infolines}

\usepackage{fontspec}
\setmainfont{FreeSans}
\newfontfamily{\russianfonttt}{FreeSans}

\definecolor{links}{HTML}{2A1B81}
\hypersetup{colorlinks,linkcolor=,urlcolor=links}

\usepackage{forest}
\usetikzlibrary{arrows}

\tabulinesep=0.7mm

\newcommand{\attribution}[1] {
    \vspace{-5mm}\begin{flushright}\begin{scriptsize}\textcolor{gray}{\textcopyright\, #1}\end{scriptsize}\end{flushright}
}

\title{Системы контроля версий, git}
\author[Юрий Литвинов]{Юрий Литвинов \newline \textcolor{gray}{\small\texttt{yurii.litvinov@gmail.com}}}

\date{15.07.2021г}

\begin{document}
    
    \frame{\titlepage}
    
    \begin{frame}
        \frametitle{Мотивация}
        \begin{itemize}
            \item Откат изменений
            \item Управление версиями
            \item Централизованное хранение кода
            \item Командная разработка
        \end{itemize}
    \end{frame}

    \begin{frame}
        \frametitle{Локальные копии}
        \begin{center}
            \includegraphics[width=0.6\textwidth]{localCopies.png}
            \attribution{https://git-scm.com/book/ru}
        \end{center}
    \end{frame}

    \begin{frame}
        \frametitle{Централизованные VCS}
        \begin{center}
            \includegraphics[width=0.6\textwidth]{centralizedVcs.png}
            \attribution{https://git-scm.com/book/ru}
        \end{center}
    \end{frame}

    \begin{frame}
        \frametitle{Распределенные VCS}
        \begin{center}
            \includegraphics[width=0.4\textwidth]{distributedVcs.png}
            \attribution{https://git-scm.com/book/ru}
        \end{center}
    \end{frame}

    \begin{frame}
        \frametitle{Управление версиями}
        \begin{center}
            \includegraphics[width=0.6\textwidth]{deltaVersioning.png}

            \vspace{5mm}
            \includegraphics[width=0.6\textwidth]{snapshotVersioning.png}
            \attribution{https://git-scm.com/book/ru}
        \end{center}
    \end{frame}

    \begin{frame}
        \frametitle{Дельта}
        \begin{center}
            \includegraphics[width=0.8\textwidth]{delta.png}
        \end{center}
    \end{frame}

    \begin{frame}
        \frametitle{Жизненный цикл файла}
        \begin{center}
            \includegraphics[width=0.8\textwidth]{fileLifeCycle.png}
            \attribution{https://git-scm.com/book/ru}
        \end{center}
    \end{frame}

    \begin{frame}
        \frametitle{Основные команды}
        \begin{itemize}
            \item git add --- добавить новый файл под управление git или добавить изменение к коммиту
            \item git status --- показать список изменённых/добавленных/удалённых файлов
            \item git diff --- показать изменения по каждому файлу
            \item git commit --- зафиксировать изменения, создав новый коммит
            \item git rm --- удалить файл и удалить его из репозитория
            \item git log --- просмотреть список коммитов
            \item git checkout --- откатить изменения в файле или перейти на другую ветку
        \end{itemize}
    \end{frame}

    \begin{frame}[fragile]
        \frametitle{Как всё устроено}
        \begin{minted}{text}
$ git add README test.rb LICENSE
$ git commit -m 'initial commit of my project'
        \end{minted}
        \begin{center}
            \includegraphics[width=0.8\textwidth]{blobs.png}
            \attribution{https://git-scm.com/book/ru}
        \end{center}
    \end{frame}

    \begin{frame}
        \frametitle{Коммит и его родители}
        \begin{center}
            \includegraphics[width=0.8\textwidth]{commits.png}
            \attribution{https://git-scm.com/book/ru}
        \end{center}
    \end{frame}

    \begin{frame}
        \frametitle{Ветки}
        \begin{center}
            \includegraphics[width=0.8\textwidth]{branches.png}
            \attribution{https://git-scm.com/book/ru}
        \end{center}
    \end{frame}

    \begin{frame}[fragile]
        \frametitle{Создание ветки}
        \begin{minted}{text}
$ git branch testing
        \end{minted}
        \begin{center}
            \includegraphics[width=0.8\textwidth]{creatingBranch.png}
            \attribution{https://git-scm.com/book/ru}
        \end{center}
    \end{frame}

    \begin{frame}[fragile]
        \frametitle{Переключение ветки}
        \begin{minted}{text}
$ git checkout testing
        \end{minted}
        \begin{center}
            \includegraphics[width=0.8\textwidth]{checkout.png}
            \attribution{https://git-scm.com/book/ru}
        \end{center}
    \end{frame}

    \begin{frame}[fragile]
        \frametitle{Новый коммит}
        \begin{minted}{text}
<Что-то поделали с файлами в рабочей копии>
$ git add <изменения, которые хотим коммитить>
$ git commit -m 'made a change'
        \end{minted}
        \begin{center}
            \includegraphics[width=0.8\textwidth]{newCommit.png}
            \attribution{https://git-scm.com/book/ru}
        \end{center}
    \end{frame}

    \begin{frame}[fragile]
        \frametitle{Переключимся на master}
        \begin{minted}{text}
$ git checkout master
        \end{minted}
        \begin{center}
            \includegraphics[width=0.8\textwidth]{checkoutToMaster.png}
            \attribution{https://git-scm.com/book/ru}
        \end{center}
    \end{frame}

    \begin{frame}[fragile]
        \frametitle{Сделаем новый коммит там}
        \begin{minted}{text}
<Что-то поделали с файлами в рабочей копии>
$ git add <изменения, которые хотим коммитить>
$ git commit -m 'made other changes'
        \end{minted}
        \begin{center}
            \includegraphics[width=0.8\textwidth]{newCommitToMaster.png}
            \attribution{https://git-scm.com/book/ru}
        \end{center}
    \end{frame}

    \begin{frame}[fragile]
        \frametitle{Слияние веток}
        \begin{minted}{text}
$ git checkout master
Switched to branch 'master'
$ git merge testing
Merge made by the 'recursive' strategy.
index.html |    1 +
1 file changed, 1 insertion(+)
        \end{minted}
        \begin{center}
            \includegraphics[width=0.8\textwidth]{merge.png}
            \attribution{https://git-scm.com/book/ru}
        \end{center}
    \end{frame}

    \begin{frame}
        \frametitle{Результат}
        \begin{center}
            \includegraphics[width=0.8\textwidth]{mergeResult.png}
            \attribution{https://git-scm.com/book/ru}
        \end{center}
    \end{frame}

    \begin{frame}
        \frametitle{Конфликты}
        \begin{center}
            \includegraphics[width=0.95\textwidth]{conflicts.png}
        \end{center}
    \end{frame}

    \begin{frame}
        \frametitle{Конфликты в коде}
        \begin{center}
            \includegraphics[width=0.5\textwidth]{conflictsInCode.png}
        \end{center}
    \end{frame}

    \begin{frame}
        \frametitle{Удалённые репозитории}
        \begin{columns}
            \begin{column}{0.4\textwidth}
                \begin{itemize}
                    \item git clone
                    \item git remote
                    \item git push
                    \item git fetch
                    \item git pull
                \end{itemize}
            \end{column}
            \begin{column}{0.6\textwidth}
                \begin{center}
                    \includegraphics[width=0.95\textwidth]{remoteRepos.png}
                    \attribution{https://www.git-tower.com/learn/git/ebook/en}
                \end{center}
            \end{column}
        \end{columns}
    \end{frame}

    \begin{frame}
        \frametitle{Процесс работы}
        \begin{footnotesize}
            \begin{itemize}
                \item Программист хочет сделать новую фичу
                \item Отводит себе ветку от мэйна/мастера
                \item Реализует там фичу
                \item Тестит и рефакторит её, когда считает, что она готова, делает пуллреквест
                \item Пока пуллреквест ревьюят, программист делает новую фичу (опять-таки, отведя новую ветку от мастера)
                \item По пуллреквесту появляются замечания, программист переключается на ветку пуллреквеста и правит там замечания
                \item Когда поправил, коммитит и пушит исправления, они автоматом добавляются в пуллреквест
                \item Просит ревьюеров, чтобы они посмотрели фиксы
                \item Переключается обратно на свою рабочую ветку и продолжает писать код, возможно, делая ещё пуллреквесты 
                \item Цикл повторяется до тех пор, пока пуллреквест не принимают
                \item Программист удаляет ветку с фичей, когда она замерджена
            \end{itemize}
        \end{footnotesize}
    \end{frame}

    \begin{frame}
        \frametitle{Что надо выкладывать}
        \begin{itemize}
            \item Исходные файлы
            \item Проектные файлы:
            \begin{itemize}
                \item Visual Studio: .vcxproj, .csproj, .sln
                \item CLion: всё содержимое папки .idea, кроме workspace.xml и tasks.xml
            \end{itemize}
            \item Текстовые файлы и прочие ресурсы, которые используются в тестах или во время работы программы
            \item Скрипты сборки
        \end{itemize}
    \end{frame}

    \begin{frame}
        \frametitle{Что не надо выкладывать}
        \begin{itemize}
            \item Бинарные файлы: .exe, .dll, бинарники под линуксом
            \begin{itemize}
                \item Включите себе отображение расширений файлов
            \end{itemize}
            \item Промежуточные результаты компиляции: .o, .obj, папки bin, obj...
            \item Скрытую папку .vs в Visual Studio
            \begin{itemize}
                \item Включите себе отображение скрытых файлов
            \end{itemize}
        \end{itemize}
        .gitignore:
        \begin{itemize}
            \item \url{https://git-scm.com/docs/gitignore}
            \item \url{https://github.com/github/gitignore}
        \end{itemize}
    \end{frame}

    \begin{frame}
        \frametitle{Хорошие практики}
        \begin{itemize}
            \item Аккуратно заполняем Имя и Email
            \begin{itemize}
                \item Желательно, чтобы они совпадали с именем аккаунта и почтой, с которой регались на GitHub
            \end{itemize}
            \item Коммитим только то, что нужно, чтобы получить в чистую папку и собрать проект
            \item Всегда пишем адекватные комментарии к коммитам
            \item Коммитим как можно чаще
            \item Один коммит --- одна функциональность
            \begin{itemize}
                \item Сделали что-то, хоть немного напоминающее осмысленное -> коммит
            \end{itemize}
        \end{itemize}
    \end{frame}

    \begin{frame}
        \frametitle{Хорошие практики (2)}
        \begin{itemize}
            \item Коммит не должен содержать в себе файлы, не относящиеся к изменениям
            \begin{itemize}
                \item .gitignore
            \end{itemize}
            \item Коммит не должен добавлять/убирать пустые строки, менять пробелы на табы и т.д., если это не суть коммита
            \item Стиль исходного кода и отступов должен совпадать с текстом вокруг
        \end{itemize}
        \begin{center}
            \includegraphics[width=0.4\textwidth]{inCaseOfFire.png}
        \end{center}
    \end{frame}

    \begin{frame}[fragile]
        \frametitle{Ещё полезные команды}
        \begin{itemize}
            \item \verb|git add -p| --- интерактивное добавление изменений к коммиту, позволяет коммитить только часть файла
            \item \verb|git commit --amend| --- исправить последний коммит
            \begin{itemize}
                \item \verb|git commit --amend -m "an updated commit message"|
                \item Применять \textbf{только до} git push
            \end{itemize}
            \item \verb|git reset --hard| --- откатить все изменения в рабочей копии до последнего коммита
            \begin{itemize}
                \item Обязательно проверить git status, что не откатите лишнего
            \end{itemize}
            \item \verb|git reset --hard <хеш коммита>| --- откатить все изменения в текущей ветке до указанного коммита, забыть все коммиты, что были после
            \begin{itemize}
                \item И случайно грохнуть всю домашку перед зачётом
            \end{itemize}
        \end{itemize}
    \end{frame}
    
    \begin{frame}
        \frametitle{Полезные ссылки}
        \begin{itemize}
            \item Консольный клиент: \url{https://git-scm.com/downloads}
            \item TortoiseGit: \url{https://tortoisegit.org/}
            \item Удобная консоль под винду: \url{http://cmder.net/}
            \item Удобный консольный файловый менеджер под винду: \url{https://www.farmanager.com/}
            \item Книжка с картинками: \url{http://git-scm.com/book/ru/} (must read)
            \item Рекомендации по процессу от гитхаба: \url{https://guides.github.com/introduction/flow/index.html} (для общего развития)
            \item Инструкция по пользованию гитом от Тимофея Брыксина: \url{https://docs.google.com/document/d/1URPcqZDMwlHDW9KoMbjoPwLJGswlMimq0-FD1_68rOY}
        \end{itemize}
    \end{frame}

\end{document}

