\begin{course}{Архитектура и проектирование систем}{sd}

\hours{I курс, сентябрь--декабрь.}{4~зачётные единицы, 2~часа лекций, 2~часа практик в~неделю.}{зачёт, экзамен.}

\myabstract{
Этот курс --- про проектирование крупных программных продуктов и всё, что с ним связано.
%
Рассмотрим, что такое вообще архитектура, какую роль в команде играет архитектор, рассмотрим разные инструменты и техники, используемые в архитектуре: визуальные языки, паттерны (и антипаттерны) проектирования, основные архитектурные стили, предметно-ориентированное проектирование. Будет уделено внимание особенностям проектирования распределённых приложений.
%
В практической части курса попробуем руками попроектировать небольшие системы, а потом реализовать и немного попопровождать их. Будет довольно много домашних заданий на проектирование и программирование, и работы в аудитории. Также на практиках будет и теория, как бы странно это ни звучало --- рассмотрим примеры архитектур известных приложений, обсудим <<тактику>> проектирования.
};

\rightblock{
  \item Владение хотя бы одним объектно-ориентированным языком.
  \item Умение пользоваться системами контроля версий, сборки.
  \item Желателен (но не обязателен) практический опыт разработчки чего-то больше пары тысяч строк по объёму.
}{
  \item В~командных проектах.
  \item В~работе на должности архитектора и даже разработчика программного обеспечения.
}{
  \item Архитектура и её роль в жизненном цикле программы.
  \item Объектно-ориентированное проектирование.
  \item Моделирование, язык~UML.
  \item Паттерны проектирования.
  \item Архитектурные стили.
  \item Предметно-ориентированное проектирование.
  \item Проектирование распределённых приложений.
  \
}{
  \item проектировать самые разные приложения;
  \item писать идеологически правильный код на объектно-ориентированных языках;
  \item оформлять архитектурную документацию, рисовать диаграммы;
  \item создавать веб-сервисы и распределённые приложения.
}
\end{course}
