\documentclass[xetex,mathserif,serif]{beamer}
\usepackage{polyglossia}
\setdefaultlanguage[babelshorthands=true]{russian}
\usepackage{minted}
\usepackage{tabu}

\useoutertheme{infolines}

\usepackage{fontspec}
\setmainfont{FreeSans}
\newfontfamily{\russianfonttt}{FreeSans}

\definecolor{links}{HTML}{2A1B81}
\hypersetup{colorlinks,linkcolor=,urlcolor=links}

\tabulinesep=0.7mm

\newcommand{\attribution}[1] {
	\vspace{-5mm}\begin{flushright}\begin{scriptsize}\textcolor{gray}{\textcopyright\, #1}\end{scriptsize}\end{flushright}
}

\title{Практика по рисованию диаграмм}
\author[Юрий Литвинов]{Юрий Литвинов \newline \textcolor{gray}{\small\texttt{yurii.litvinov@gmail.com}}}

\date{11.03.2020г}

\begin{document}
	
	\frame{\titlepage}

	\begin{frame}
		\frametitle{Диаграммы активностей}
		\framesubtitle{Activity diagrams}
		\begin{center}
			\includegraphics[width=0.415\textwidth]{activityDiagram.png}
		\end{center}
	\end{frame}

	\begin{frame}
		\frametitle{Диаграммы активностей, разделы}
		\framesubtitle{Swimlanes}
		\begin{center}
			\includegraphics[width=0.55\textwidth]{activitySwimlanes.png}
		\end{center}
	\end{frame}

	\begin{frame}
		\frametitle{Задача 1}
		Нарисовать диаграмму активностей, моделирующую бизнес-процесс проведения ``промежуточной аттестации'' в учебном отделе.
		\begin{itemize}
			\item Зачёты и зачётная неделя
			\item Сессия
			\item Результаты --- отчисление, перевод в следующий семестр, назначение стипендии
			\item Использовать разделы для представления разных заинтересованных сторон
		\end{itemize}
	\end{frame}

	\begin{frame}
		\frametitle{Диаграммы последовательностей}
		\framesubtitle{Sequence diagrams}
		\begin{center}
			\includegraphics[width=0.6\textwidth]{sequenceDiagram.png}
		\end{center}
	\end{frame}

	\begin{frame}
		\frametitle{Диаграммы последовательностей, создание и удаление объектов}
		\begin{center}
			\includegraphics[width=0.65\textwidth]{sequenceLifeCycle.png}
		\end{center}
	\end{frame}

	\begin{frame}
		\frametitle{Диаграммы последовательностей, фреймы}
		\begin{center}
			\includegraphics[width=0.8\textwidth]{sequenceFrames.png}
		\end{center}
	\end{frame}

	\begin{frame}
		\frametitle{Задача 2}
		Нарисовать диаграмму последовательностей --- типичный сценарий взаимодействия пользователя и HwProj при посылке решения
		\begin{itemize}
			\item Начиная с авторизации и до окончания взаимодействия
			\item HwProj умеет общаться с GitHub-ом, чтобы проверить статус пуллреквеста
		\end{itemize}
	\end{frame}

	\begin{frame}
		\frametitle{Диаграмма развёртывания UML}
		\begin{columns}
			\begin{column}{0.5\textwidth}
				\begin{itemize}
					\item Показывает отображение компонентов и физических артефактов на реальные (или виртуальные) устройства
					\item Бывает полезна на начальных этапах проектирования, даже до диаграмм компонентов
				\end{itemize}
			\end{column}
			\begin{column}{0.5\textwidth}
				\begin{center}
					\includegraphics[width=\textwidth]{deploymentDiagram.png}
					\attribution{М. Фаулер, UML. Основы}
				\end{center}
			\end{column}
		\end{columns}
	\end{frame}

	\begin{frame}
		\frametitle{Задача 3}
		Нарисовать диаграмму развёртывания для приложения, описанного в RFP про автомобильный завод
		\begin{itemize}
			\item \url{https://goo.gl/MiyH8c}
		\end{itemize}
	\end{frame}

	\begin{frame}
		\frametitle{Временные диаграммы}
		\begin{columns}
			\begin{column}{0.5\textwidth}
				\begin{center}
					\includegraphics[width=0.9\textwidth]{timingDiagrams.png}
				\end{center}
			\end{column}
			\begin{column}{0.5\textwidth}
				\begin{center}
					\includegraphics[width=0.8\textwidth]{timingDiagramsAlternate.png}
					\attribution{М. Фаулер, UML. Основы}
				\end{center}
			\end{column}
		\end{columns}
	\end{frame}

	\begin{frame}
		\frametitle{Задача 4}
		Нарисовать временную диаграмму любого сценария работы микроволновки
		\begin{itemize}
			\item В VP это может быть не совсем тривиально: \url{https://www.visual-paradigm.com/support/documents/vpuserguide/94/2586/6715_drawingtimin.html}
		\end{itemize}
	\end{frame}

	\begin{frame}
		\frametitle{Домашнее задание: Магазин книг}
		Выполнить анализ предметной области и построить модель в виде диаграммы классов для интернет-магазина книг по следующему ТЗ:
		\begin{itemize}
			\item \url{https://goo.gl/94LyFc}
		\end{itemize}

		Обратите внимание, что это должна быть модель предметной области, детали реализации наподобие способа хранения информации в базе данных не важны.

		Будет оцениваться точность следования ТЗ, соответствие модели сущностям предметной области (в том числе, неявным) и, естественно, пунктуальность в следовании синтаксису UML.
	\end{frame}

\end{document}
