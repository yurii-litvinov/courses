\documentclass[xetex,mathserif,serif]{beamer}
\usepackage{polyglossia}
\setdefaultlanguage[babelshorthands=true]{russian}
\usepackage{minted}
\usepackage{tabu}
\usepackage[11pt]{moresize}

\useoutertheme{infolines}

\usepackage{fontspec}
\setmainfont{FreeSans}
\newfontfamily{\russianfonttt}{FreeSans}

\usepackage{textpos}
\setlength{\TPHorizModule}{1cm}
\setlength{\TPVertModule}{1cm}

\definecolor{links}{HTML}{2A1B81}
\hypersetup{colorlinks,linkcolor=,urlcolor=links}

\tabulinesep=0.7mm

\title{Распределённые приложения, RabbitMQ}
\author[Юрий Литвинов]{Юрий Литвинов \newline \textcolor{gray}{\small\texttt{yurii.litvinov@gmail.com}}}

\date{13.05.2019г}

\begin{document}

	\frame{\titlepage}

	\section{RabbitMQ}

	\begin{frame}
		\frametitle{RabbitMQ}
		\begin{itemize}
			\item Сервер и клиенты системы надёжной передачи сообщений
			\begin{itemize}
				\item Сообщение посылается на сервер и хранится там, пока его не заберут
				\item Продвинутые возможности по маршрутизации сообщений
			\end{itemize}
			\item Реализует протокол AMQP (Advanced Message Queuing Protocol), но может использовать и другие протоколы
			\item Сервер написан на Erlang, клиентские библиотеки доступны для практически чего угодно
		\end{itemize}
		\begin{textblock}{3}(8,0)
			\includegraphics[width=\textwidth]{rabbitmqLogo.png}
		\end{textblock}
	\end{frame}

	\begin{frame}[fragile]
		\frametitle{Пример, отправитель}
		\begin{ssmall}
			\begin{minted}{java}
public class Sender {
    private static final String QUEUE_NAME = "MyQueue";

    public static void main(String[] args) throws IOException, TimeoutException {
        var factory = new ConnectionFactory();
        factory.setHost("localhost");
        Connection connection = factory.newConnection();
        Channel channel = connection.createChannel();

        channel.queueDeclare(QUEUE_NAME, false, false, false, null);
        String message = "Hello World!";
        channel.basicPublish("", QUEUE_NAME, null, message.getBytes());
        System.out.println(" [x] Sent '" + message + "'");

        channel.close();
        connection.close();
    }
}
			\end{minted}
		\end{ssmall}
	\end{frame}

	\begin{frame}[fragile]
		\frametitle{Пример, получатель}
		\begin{ssmall}
			\begin{minted}{java}
public class Receiver {
    private static final String QUEUE_NAME = "MyQueue";

    public static void main(String[] args) throws IOException, TimeoutException {
        var factory = new ConnectionFactory();
        factory.setHost("localhost");
        Connection connection = factory.newConnection();
        Channel channel = connection.createChannel();

        channel.queueDeclare(QUEUE_NAME, false, false, false, null);
        System.out.println(" [*] Waiting for messages. To exit press CTRL+C");

        Consumer consumer = new DefaultConsumer(channel) {
            @Override
            public void handleDelivery(String consumerTag, Envelope envelope,
                                       AMQP.BasicProperties properties, byte[] body)
                    throws IOException {
                String message = new String(body, "UTF-8");
                System.out.println(" [x] Received '" + message + "'");
            }
        };
        channel.basicConsume(QUEUE_NAME, true, consumer);
    }
}
			\end{minted}
		\end{ssmall}
	\end{frame}

	\begin{frame}
		\frametitle{Как всё собрать и запустить}
		\begin{itemize}
			\item Поставить рантайм Erlang
			\item Поставить сервер RabbitMQ
			\begin{itemize}
				\item \url{https://www.rabbitmq.com/download.html}
			\end{itemize}
			\item Добавить зависимость от клиента RabbitMQ в проект
			\begin{itemize}
				\item \mintinline{groovy}|compile group: 'com.rabbitmq', name: 'amqp-client', version: '5.7.0'|
			\end{itemize}
			\item Пролистать Getting Started 
			\begin{itemize}
				\item \url{https://www.rabbitmq.com/tutorials/tutorial-one-java.html}
			\end{itemize}
		\end{itemize}
	\end{frame}

	\section{Задача}

	\begin{frame}
		\frametitle{Задача на пару}
		Переделать сетевой чат на RabbitMQ
		\begin{itemize}
			\item Сервер для обмена сообщениями, о котором договариваются клиенты
			\begin{itemize}
				\item Центральный сервер, задаваемый при запуске (127.0.0.1 по умолчанию)
			\end{itemize}
			\item Нет списка контактов, есть именованные каналы, на которые можно подписываться и постить туда
			\begin{itemize}
				\item Должна быть кнопка подписки на канал и поле для ввода имени канала
				\item Сообщения в разные каналы должны быть в разных вкладках
				\item Подписка на несуществующий канал должна его создавать
			\end{itemize}
			\item Нет истории, получать только те сообщения, что были опубликованы с момента подключения
			\begin{itemize}
				\item Может помочь \url{https://www.rabbitmq.com/tutorials/tutorial-three-java.html}
			\end{itemize}
		\end{itemize}
	\end{frame}

\end{document}