\documentclass[a5paper]{article}
\usepackage[a5paper, top=8mm, bottom=8mm, left=8mm, right=8mm]{geometry}

\usepackage{polyglossia}
\setdefaultlanguage[babelshorthands=true]{russian}

\usepackage{fontspec}
\setmainfont{FreeSerif}
\newfontfamily{\russianfonttt}[Scale=0.7]{DejaVuSansMono}

\usepackage[font=scriptsize]{caption}

\usepackage{amsmath}
\usepackage{amssymb,amsfonts,textcomp}
\usepackage{color}
\usepackage{array}
\usepackage{hhline}
\usepackage{cite}

\usepackage[hang,multiple]{footmisc}
\renewcommand{\footnotelayout}{\raggedright}

\PassOptionsToPackage{hyphens}{url}\usepackage[xetex,linktocpage=true,plainpages=false,pdfpagelabels=false]{hyperref}
\hypersetup{colorlinks=true, linkcolor=blue, citecolor=blue, filecolor=blue, urlcolor=blue, pdftitle=1, pdfauthor=, pdfsubject=, pdfkeywords=}

\usepackage{tabu}

\usepackage{graphicx}
\usepackage{indentfirst}
\usepackage{multirow}
\usepackage{subfig}
\usepackage{footnote}
\usepackage{minted}

\sloppy
\pagestyle{plain}

\title{IO в Java}
\author{Юрий Литвинов\\\small{yurii.litvinov@gmail.com}}

\date{23.01.2019}

\begin{document}

\maketitle
\thispagestyle{empty}

\section{Введение}

В Java ввод-вывод довольно развит, есть даже несколько пакетов стандартной библиотеки, реализующих функциональность файлового ввода вывода и делающих в принципе одно и то же несколько по-разному. Сегодня речь пойдёт про исторически первый и самый простой способ читать/писать данные, пакет java.io.

Основа системы ввода вывода --- понятие ``Поток'', точнее, отдельно ``Поток ввода'' и ``Поток вывода''. Для того, чтобы говорить потокам, откуда читать и куда писать, применяются либо просто строковые пути до файлов, либо файловые дескрипторы (как в C++) --- класс File. File же представляет кроссплатформенное API для работы с файловой системой --- именно там есть методы, проверяющие наличие файла, его атрибуты, методы создания, удаления, перемещения и т.д. File же, контринтуитивно, отвечает за работу с папками --- листинг, создание и удаление.

Потоки в стандартной библиотеке представляются несколькими интерфейсами (на самом деле, абстрактными классами). InputStream, OutputStream представляют байтовые потоки --- они позволяют осуществлять ввод-вывод на уровне байтов или массивов байт. Reader, Writer представляют символьные потоки, они читают/пишут char-ы в определённой кодировке.

Конкретные реализации этих абстракций можно разбить на несколько категорий:

\begin{itemize}
	\item File* (FileInputStream, FileOutputStream, FileReader, FileWriter) --- работа с файлами
	\item Buffered* --- оптимизация с помощью буферизации
	\item Data* ---работа с данными
	\item ByteArray* --- потоки в оперативной памяти
\end{itemize}

Интересно, что раз это Java, тут всё архитектурно красиво, поэтому потоки могут вкладываться друг в друга, предоставляя ``слои'' функциональности (паттерн ``Декоратор'', если по-научному). Например, файловые потоки умеют читать только посимвольно, но если мы ``наденем'' на FileReader буферизующий поток, то, во-первых, получим прирост производительности раз в 10 за счёт буферизации чтения, во-вторых, получим возможность читать построчно:

\begin{minted}{java}
new BufferedReader(new FileReader("path/to/file"));
\end{minted}

Причём, поскольку BufferedReader-у всё равно, что за поток в него вложен, лишь бы это был Reader, то его можно ``надеть'' на любую реализацию Reader-а --- из файла, из памяти, по сети и т.д.

Как обычно, при начале работы потоки открывают тот файл, с которым работают, и ожидают, что в конце работы файл будет закрыт. Но тут проблема --- в Java нет деструкторов, и то, что в C++ делал деструктор сам, в Java требуется делать явно (ну, почти), вызывая у потока метод close(). Либо попросив компилятор вызвать close() автоматически, с помощью синтаксиса try-with-resources, о котором чуть позже. try-with-resources --- самое близкое к деструкторам C++, и единственный идеологически правильный способ закрывать потоки, потому что с close() надо думать, не пропустим ли мы вызов close() из-за случайного исключения. Как и обычно, потоки надо закрывать обязательно, и чем раньше, тем лучше.

\section{Потоки, основные операции}

На самом деле, сейчас будет краткий пересказ JavaDoc-ов стандартной библиотеки, и при любом вопросе желательно обратиться к первоисточнику (\url{https://docs.oracle.com/en/java/javase/11/docs/api/index.html}). Тем не менее, наиболее важные штуки приведены тут. Что можно делать с потоками ввода:

\begin{itemize}
	\item \mintinline{java}|int read()| --- чтение элемента. Возвращает \textit{двухбайтовый} беззнаковый char (число от 0 до 65535), либо -1, если достигнут конец потока.
	\item \mintinline{java}|read(T[] v)|, \mintinline{java}|read(T[] v, off, len)| --- чтение элементов в массив. Возвращает число прочитанных элементов, либо -1, если ничего не прочиталось. Обратите внимание, в Java массивы знают свой размер, поэтому переполнения буфера опасаться не стоит.
	\item \mintinline{java}|skip(n)| --- пропуск n элементов. Возвращает, сколько символов реально пропущено.
	\item \mintinline{java}|available()| --- сколько элементов доступно.
	\item \mintinline{java}|mark(limit)| --- пометка текущей позиции, чтобы потом можно было на неё вернуться reset-ом. limit --- сколько символов мы можем считать, сохраняя при этом возможность вернуться. Если из потока к моменту вызова reset считано больше символов, reset имеет право не отработать (это чтобы не было так, что из-за забытого mark-а мы буферизовали 20 гигабайт сетевого трафика). не все реализации стримов поддерживают mark.
	\item \mintinline{java}|reset()| --- возврат к помеченной позиции.
	\item \mintinline{java}|close()| --- закрытие потока. Закрывает также вложенные потоки (хотя это надо уточнять в документации про каждый конкретный класс). Этот метод реализует метод из интерфейса Closeable, который, в свою очередь, реализует интерфейс AutoCloseable. Про AutoCloseable знает компилятор, так что для любого класса, реализующего AutoCloseable, работает try-with-resources --- генерируется вызов close(), который отработает и при корректном завершении работы, и при исключении.
\end{itemize}

Вот что можно делать с потоками вывода:

\begin{itemize}
	\item \mintinline{java}|write(int v)| --- запись элемента (двухбайтового символа).
	\item \mintinline{java}|write(T[] v)| --- запись массива элементов.
	\item \mintinline{java}|write(T[] v, off, len)| --- запись части массива (начиная с индекса off, len ячеек).
	\item \mintinline{java}|flush()| --- запись буфера. Если поток работает поверх ещё какой-нибудь реализации потока вывода, делает flush() и ей. Метод объявлен в интерфейсе Flushable.
	\item \mintinline{java}|close()| --- закрытие потока. Тут действуют те же соображения про AutoCloseable, что и с потоками чтения.
\end{itemize}

Например, как скопировать данные из одного потока в другой:

\begin{minted}{java}
void copy(InputStream is, OutputStream os) throws IOException
{
    var b = new byte[1024];
    int c = 0;
    while ((c = is.read(b)) > 0) {
        os.write(b, 0, c);
    }
}
\end{minted}

Или как считать байты из файла и побайтово сделать что-нибудь:

\begin{minted}{java}
FileInputStream in = null;

try {
   in = new FileInputStream("test.txt");
   int c;

   while ((c = in.read()) != -1) {
       doSomething(c);
   }
} finally {
   if (in != null) {
       in.close();
   }
}
\end{minted}

\section{try-with-resources}

Обратите внимание на страшную конструкцию с finally, чтобы если исключение было брошено, файл всё равно закрылся. Как раз для этого и придуман try-with-resources, и начиная с Java 7 правильнее (и короче) писать так:

\begin{minted}{java}
try (var in = new FileInputStream("test.txt")) {
   int c;

   while ((c = in.read()) != -1) {
       doSomething(c);
   }
}
\end{minted}

Если ресурсов, с которыми надо работать, много, они указываются через точку с запятой:

\begin{minted}{java}
void copy(String src, String dst) throws IOException
{
    try (var in = new FileInputStream(src);
            var out = new FileOutputStream(dst)) {
        var b = new byte[1024];
        int c = 0;
        while ((c = in.read(b)) > 0) {
            out.write(b, 0, c);
        }
    }
}
\end{minted}

\section{Исключения}

Основные исключения, бросаемые операциями ввода вывода, таковы. IOException --- это корень иерархии исключений подсистемы ввода-вывода, если вы хотите ловить ситуации, когда просто с вводом-выводом что-то не так, имеет смысл ловить именно его. EOFException --- когда достигнут конец потока, но вы пытаетесь сделать что-то, что ожидает данные в потоке. Обратите внимание, что read() это исключение не бросает, а просто возвращает -1. FileNotFoundException кидается, когда запрошенный файл не найден, оказался папкой, у вас нет на него прав, он вообще только для чтения и т.д. UnsupportedEncodingException бросается методами, которые принимают кодировку, если передать им кодировку, которая неизвестна Java-машине, на которой они работают.

\section{Преобразование потоков}

Как уже упоминалось ранее, потоки можно комбинировать, ``вкладывая'' один в другой. Вот самые распространённые случаи такого комбинирования:

\begin{itemize}
	\item При чтении возможно преобразование байтового потока в символьный, с указанием кодировки. Это делает класс InputStreamReader, с конструктором \mintinline{java}|InputStreamReader(InputStream, encoding?)|, принимающим опциональный параметр с кодировкой. Если кодировка не указана, используется кодировка из текущей локали.
	\item Наоборот, из символьного в байтовый поток преобразует OutputStreamWriter, с конструктором \mintinline{java}|OutputStreamWriter(OutputStream, encoding?)|. Тут с кодировками дела обстоят так же, как и у InputStreamReader-а
\end{itemize}

Небольшой пример, перекодирование файла из кодировки 1251 в кодировку 866:

\begin{minted}{java}
Reader reader = new InputStreamReader(
    new FileInputStream("input.txt"), "Cp1251");
Writer writer = new OutputStreamWriter(
    new FileOutputStream("output.txt"), "Cp866");
int c = 0;
while ((c = reader.read()) >= 0) {
    writer.write(c);
}

reader.close();
writer.close();
\end{minted}

Ну или более правильно с точки зрения исключений:

\begin{minted}{java}
try (var reader = new InputStreamReader(
            new FileInputStream("input.txt"), "Cp1251");
        var writer = new OutputStreamWriter(
            new FileOutputStream("output.txt"), "Cp866")) {
    int c = 0;
    while ((c = reader.read()) >= 0) {
        writer.write(c);
    }
}
\end{minted}

Ещё один полезный случай комбинирования потоков --- это буферизация. BufferedReader читает данные сразу пачкой из ``вложенного'' в него потока и складывает в буфер. Настоятельно рекомендуется к использованию при чтении из файлов. Кроме того, BufferedReader позволяет читать по строкам. Пример:

\begin{minted}{java}
static String readFirstLineFromFile(String path)
       throws IOException {
   try (var br = new BufferedReader(new FileReader(path))) {
       return br.readLine();
   }
}
\end{minted}

Ну или вот пример чтения пользовательского ввода из консоли:
\begin{minted}{java}
try (BufferedReader in = new BufferedReader(
    new InputStreamReader(System.in))) 
{
   String input = in.readLine();
   ...
} catch (Exception e) {
   ...
}
\end{minted}

\section{Потоки для работы с данными в памяти}

Механизм работы с  потоками ввода вывода бывает полезен не только при собственно вводе-выводе, но и как средство удобно преобразовывать данные прямо в памяти. Также классы, симулирующие ввод-вывод на данных в памяти, бывают очень полезны, если у вас есть API, принимающее потоки, а вы не хотите что-то писать в файл только для того, чтобы тут же считать это что-то из файла. Вот основные такие классы.

\begin{itemize}
	\item ByteArrayInputStream --- чтение из массива байт. Это самый настоящий InputStream, использующий массив байт в памяти как источник данных.
	\item CharArrayReader --- чтение из массива символов. Это самый настоящий Reader, делающий то же самое.
	\item StringReader --- чтение из строки, если массив символов --- это не удобно.
	\item ByteArrayOutputStream --- запись в массив байт. Используем его как обычный OutputStream, а потом, как закончили, вызываем метод toByteArray() и получаем массив, который записался.
	\item CharArrayWriter --- запись в массив символов. В общем-то, то же самое, но Writer. Для получения данных используются методы toString() или toCharArray().
	\item StringWriter – запись в StringBuffer. Для получения результатов используется либо toString(), либо toStringBuffer(). StringBuffer --- это по сути потокобезопасный мутабельный String.
\end{itemize}

\section{Высокоуровневая работа с вводом-выводом}

Работать с байтами и символами такое себе удовольствие, если вам нужно сохранять/загружать сложные структуры данных, читать пользовательский ввод и т.д. Поэтому придуманы более высокоуровневые вещи, которые знают про элементарные типы и умеют их сохранять/загружать. На самом деле, в Java ещё есть встроенный механизм сериализации, через интерфейс Serializable и рефлексию, но про него предлагается почитать самостоятельно.

Первая высокоуровневая штука --- это класс Scanner, который можно нацепить на поток ввода и заставить его возвращать значения элементарных типов, которые он сам будет парсить:

\begin{minted}{java}
try (var s = new Scanner(
       new BufferedReader(new FileReader("test.txt")))) {
   while (s.hasNextDouble()) {
       System.out.println(s.nextDouble());
   }
}
\end{minted}

Ещё есть классы DataOutputStream и DataInputStream (реализации интерфейсов DataOutput и DataInput соответственно). DataOutputStream также реализует OutputStream, DataInputStream --- InputStream, как не трудно догадаться. Они обеспечивают кроссплатформенную запись/чтение элементарных типов.

Вот основные методы, DataOutputStream:

\begin{itemize}
	\item \mintinline{java}|writeBoolean()|, \mintinline{java}|writeByte()|, \mintinline{java}|writeChar()| и т.д. --- запись примитивных типов
	\item \mintinline{java}|writeUTF()| --- запись строки в кодировке UTF-8
\end{itemize}

И DataInputStream:
\begin{itemize}
	\item \mintinline{java}|T readT()| --- чтение примитивных типов
	\item \mintinline{java}|readUTF()| --- чтение строки в кодировке UTF-8
\end{itemize}

\end{document}
