\documentclass[xetex,mathserif,serif]{beamer}
\usepackage{polyglossia}
\setdefaultlanguage[babelshorthands=true]{russian}
\usepackage{minted}
\usepackage{tabu}

\useoutertheme{infolines}

\usepackage{fontspec}
\setmainfont{FreeSans}
\newfontfamily{\russianfonttt}{FreeSans}

\usepackage{textpos}
\setlength{\TPHorizModule}{1cm}
\setlength{\TPVertModule}{1cm}

\definecolor{links}{HTML}{2A1B81}
\hypersetup{colorlinks,linkcolor=,urlcolor=links}

\tabulinesep=1.2mm

\title{Effective Java}
\author[Юрий Литвинов]{Юрий Литвинов\\\small{\textcolor{gray}{yurii.litvinov@gmail.com}}}
\date{20.02.2019г}

\newcommand{\attribution}[1] {
\vspace{-5mm}\begin{flushright}\begin{scriptsize}\textcolor{gray}{\textcopyright\, #1}\end{scriptsize}\end{flushright}
}

\begin{document}

	\frame{\titlepage}

	\section{Контрольная}

	\begin{frame}
		\frametitle{О контрольной}
		\begin{itemize}
			\item Ожидалось решение, использующее Object для хранения данных, разбор случаев и кучу unchecked cast-ов
			\begin{itemize}
				\item Естественно, заглушенных с помощью \mintinline{java}|//noinspection unchecked|
				\item Было одно решение, где использовалась иерархия классов-обёрток с одним полем, получилось лучше, чем ожидалось
			\end{itemize}
			\item Надо было реализовать size, get, set, add и remove по индексу
			\begin{itemize}
				\item Остальное реализовано в AbstractList достаточно эффективно
			\end{itemize}
			\item addAll в конструкторе, принимающем коллекцию --- нехорошо
			\item System.arraycopy
			\item Магические константы
			\item Беда с сырыми типами и unchecked cast-ами не по делу
		\end{itemize}
	\end{frame}

	\begin{frame}[fragile]
		\frametitle{О домашке}
		Заклинание, заставляющее красиво отображеть состояние тестов в CI:
		\begin{minted}{groovy}
test {
    useJUnitPlatform()
    testLogging {
        events 'PASSED', 'FAILED', 'SKIPPED'
    }
}
		\end{minted}
	\end{frame}

	\section{Книжка}

	\begin{frame}
		\frametitle{Книжка, по которой рассказ}
		Joshua Bloch, Effective Java, 3rd Edition. Addison-Wesley Professional, 2017. 412PP.
		\begin{center}
			\includegraphics[width=0.25\textwidth]{bookCover.png}
		\end{center}
	\end{frame}

	\section{Защитное программирование}

	\begin{frame}
		\frametitle{49. Проверяем достоверность параметров}
		\begin{itemize}
			\item Часто на аргументы методов накладываются ограничения
			\begin{itemize}
				\item Нужно их документировать
				\begin{itemize}
					\item \mintinline{java}|@throws|
				\end{itemize}
				\item А метод начинать с проверки и быстро завершать, иначе
				\begin{itemize}
					\item Половина метода может выполниться, оставив некорректное состояние
					\item Метод тихо вернёт некорректный результат
				\end{itemize}
			\end{itemize}
			\item Особенно если аргументы откладываются для обработки в дальнейшем
			\begin{itemize}
				\item Частый случай --- конструкторы
			\end{itemize}
			\item IllegalArgumentException, IndexOutOfBoundsException, NullPointerException...
			\begin{itemize}
				\item Или преобразование исключения в понятное пользователю
			\end{itemize}
			\item java.util.Objects.checkFromIndexSize, checkFromToIndex​, checkIndex
		\end{itemize}
	\end{frame}

	\begin{frame}[fragile]
		\frametitle{public-методы}
		\begin{small}
			\begin{minted}{java}
/*
* Возвращает объект Biglnteger, значением которого является модуль 
* данного числа по основанию m. Этот метод отличается от метода 
* remainder тем, что всегда возвращает неотрицательное значение 
* Biglnteger.
*
* @раrаm m модуль, должен быть положительным числом
* @return this mod m
* @throws ArithmeticException, if m <= 0
*/
public Biglnteger mod(BigInteger m) {
    if (m.signum() <= 0) {
        throw new ArithmeticException("Modulus not positive");
    }
    ... // Вычисления
}
			\end{minted}
		\end{small}
	\end{frame}

	\begin{frame}[fragile]
		\frametitle{private-методы}
		\begin{small}
			\begin{minted}{java}
// Закрытая вспомогательная функция для рекурсивной сортировки
private static void sort(long a[], int offset, int length) {
    assert a != null;
    assert offset >= 0 && offset <= a.length;
    assert length >= 0 && length <= a.length - offset;
    // Вычисления
}
			\end{minted}
		\end{small}
	\end{frame}

	\begin{frame}
		\frametitle{50. Резервные копии могут быть полезны}
		\begin{itemize}
			\item JVM даёт уверенность в безопасности работы с памятью
			\begin{itemize}
				\item В отличие от C++, инварианты класса будут выполняться, что бы ни происходило в остальной системе
			\end{itemize}
			\item Однако всё равно надо оберегать свои инварианты!
			\begin{itemize}
				\item Программировать надо так, будто все клиенты вашего класса делают (практически) всё возможное, чтобы нарушить его инварианты
			\end{itemize}
		\end{itemize}
	\end{frame}

	\begin{frame}[fragile]
		\begin{small}
			\begin{minted}{java}
// Неправильный класс «неизменяемого» периода времени
public final class Period {
   private final Date start;
   private final Date end;
   /**
    @param start - начало периода.
    @param end - конец периода; не должен предшествовать началу.
    @throws IllegalArgumentException, если start позже, чем end.
    @throws NullPointerException, если start или end равны null.
   */
   public Period(Date start, Date end) {
       if (start.compareTo(end) > 0)
           throw new IllegalArgumentException(start + " after " + end);
       this.start = start;
       this.end = end;
   }
   public Date start() { return start; }
   public Date end() { return end; }
   // Остальное опущено
}
			\end{minted}
		\end{small}
	\end{frame}

	\begin{frame}[fragile]
		\begin{minted}{java}
// Атака на содержимое экземпляра Period
Date start = new Date();
Date end = new Date();
Period p = new Period(start, end);
end.setYear(78); // Изменяет содержимое объекта p!
		\end{minted}
	\end{frame}

	\begin{frame}[fragile]
		\frametitle{Defensive copies}
		\begin{minted}{java}
// Исправленный конструктор: для представленных параметров
// создает резервные копии
public Period(Date start, Date end) {
   this.start = new Date(start.getTime());
   this.end = new Date(end.getTime());
   if (this.start.compareTo(this.end) > 0) {
       throw new IllegalArgumentException(start + " after " + end);
   }
}
		\end{minted}
		\begin{itemize}
			\item Проверяем уже созданные копии, а не исходные параметры (ибо многопоточность)
			\item Не используем clone(), потому что его могут переопределить и подсунуть нам подкласс
		\end{itemize}
	\end{frame}

	\begin{frame}[fragile]
		\frametitle{Атака №2}
		\begin{minted}{java}
Date start = new Date();
Date end = new Date();
Period p = new Period(start, end);
p.end().setYear(78); // Изменяет внутренние данные p!
		\end{minted}
		\vspace{3mm}
		\pause
		Решение:
		\begin{minted}{java}
public Date start() {
   return (Date) start.clone();
}

public Date end() {
   return (Date) end.clone();
}
		\end{minted}
	\end{frame}

	\begin{frame}
		\frametitle{Мораль}
		\begin{itemize}
			\item Для каждого аргумента сеттера или конструктора, который сохраняем
			\begin{itemize}
				\item А является ли передаваемый объект изменяемым?
				\item Будет ли наш объект устойчив к его изменениям?
				\begin{itemize}
					\item Например, ключ в Map
					\item Массивы ненулевой длины всегда изменяемы
				\end{itemize}
			\end{itemize}
			\item Или можно хранить что-то другое
			\begin{itemize}
				\item Например, Instant, LocalDateTime, ZonedDateTime или даже Date.getTime() вместо Date
			\end{itemize}
			\item Не забыть отразить всё это в документации к классу!
			\item А также помнить про производительность
		\end{itemize}
	\end{frame}

	\section{Сигнатуры методов}

	\begin{frame}
		\frametitle{51. Советы по проектированию сигнатур методов}
		\begin{itemize}
			\item Тщательно выбирайте названия методов
			\begin{itemize}
				\item С учётом всех соглашений и стайлгайдов
			\end{itemize}
			\item Методов не должно быть слишком много
			\begin{itemize}
				\item Одна операция --- один метод
				\item Не надо гнаться за удобством сразу, пусть время покажет
			\end{itemize}
			\item Длинный перечень параметров --- плохо
			\begin{itemize}
				\item 4 --- уже много (особенно если они похожих типов)
				\item Что делать?
				\begin{itemize}
					\item Разбить метод на несколько ортогональных (например, метод subList в List, возвращающий view на часть списка)
					\item Создание вспомогательных классов, выделение параметров в сущность
					\item Шаблон Builder
				\end{itemize}
			\end{itemize}
		\end{itemize}
	\end{frame}

	\begin{frame}
		\frametitle{Ещё советы}
		\begin{itemize}
			\item Интерфейсы вместо конкретных классов параметров
			\begin{itemize}
				\item Более обобщённый код
			\end{itemize}
			\item Двухэлементные перечисления вместо Boolean
			\begin{itemize}
				\item \mintinline{java}|Thermometer.newInstance(true);|
				\item \mintinline{java}|public enum TemperatureScale { FAHRENHEIT, CELSIUS }|
				
					\mintinline{java}|Thermometer.newInstance(TemperatureScale.CELSIUS);|
				\item Потом можно и Кельвины какие-нибудь добавить при желании
				\item В enum можно добавить полезных методов
			\end{itemize}
		\end{itemize}
	\end{frame}

	\begin{frame}[fragile]
		\frametitle{52. Перегружаем методы осторожно}
		\begin{small}
			\begin{minted}{java}
// Что выведет данная программа?
public class CollectionClassifier {
   public static String classify(Set<?> s) { return "Set"; }
   public static String classify(List<?> l) { return "List"; }
   public static String classify(Collection<?> c) { 
      return "Unknown Collection"; }

   public static void main(String[] args) {
       Collection<?>[] collections = {
               new HashSet<String>(),
               new ArrayList<BigInteger>(),
               new HashMap<String, String>().values()
       };
       for (Collection<?> c : collections)
           System.out.println(classify(c));
   }
}
			\end{minted}
		\end{small}
		\pause
		\begin{textblock}{3}(9,-4)
			Выбор варианта перегрузки \\
			осуществляется на стадии \\
			компиляции!
		\end{textblock}
	\end{frame}

	\begin{frame}[fragile]
		\frametitle{А вот с переопределением всё иначе}
		\begin{small}
			\begin{minted}{java}
class Wine {
   String name() { return "wine"; }
}
class SparklingWine extends Wine {
   @Override String name() { return "sparkling wine"; }
}
class Champagne extends SparklingWine {
   @Override String name() { return "champagne"; }
}
public class Overriding {
   public static void main(String[] args) {
       Wine[] wines = { new Wine(), new SparklingWine(), new Champagne() };
       for (Wine wine : wines) System.out.println(wine.name());
   }
}
			\end{minted}
		\end{small}
	\end{frame}

	\begin{frame}
		\frametitle{Мораль}
		\begin{itemize}
			\item Не следует писать код, непонятный среднему программисту
			\begin{itemize}
				\item Особенно в рамках API
				\item Стоит избегать запутанных вариантов перегрузки
				\item Всегда можно назвать методы по-разному
			\end{itemize}
			\item Одинаковое количество параметров перегруженных методов --- плохо
			\begin{itemize}
				\item Надо будет следить за приводимостью типов
				\item Методы с varargs лучше не перегружать вообще
			\end{itemize}
			\item А как же конструкторы?
			\begin{itemize}
				\item Статические фабричные методы
			\end{itemize}
			\item Особая боль --- автоупаковщики
		\end{itemize}
	\end{frame}

	\begin{frame}[fragile]
		\frametitle{Ещё один пример}
		\begin{columns}
			\begin{column}{0.7\textwidth}
				\begin{minted}{java}
public class SetList {
   public static void main(String[] args) {
       Set<Integer> set = new TreeSet<Integer>();
       List<Integer> list = new ArrayList<Integer>();
       for (int i = -3; i < 3; i++) {
           set.add(i);
           list.add(i);
       }
       for (int i = 0; i < 3; i++) {
           set.remove(i);
           list.remove(i);
       }
       System.out.println(set + " " + list);
   }
}
				\end{minted}
			\end{column}
			\begin{column}{0.3\textwidth}
				\pause
				Вывод:
				\begin{minted}{text}
[-3, -2, -1] [-2, 0, 2]
				\end{minted}
				\vspace{5mm}
				Перегрузка -- зло.
			\end{column}
		\end{columns}
	\end{frame}

	\begin{frame}[fragile]
		\frametitle{53. varargs --- опасная штука}
		\begin{footnotesize}
			\begin{minted}{java}
// Простое использование varargs
static int sum(int... args) {
   int sum = 0;
   for (int arg : args)
       sum += arg;
   return sum;
}
			\end{minted}
			\pause
			\begin{minted}{java}
// Неверное использование varargs для передачи одного или более аргументов!
static int min(int... args) {
   if (args.length == 0)
       throw new IllegalArgumentException("Too few arguments");
   int min = args[0];
   for (int i = 1; i < args.length; i++)
       if (args[i] < min)
           min = args[i];
   return min;
}
			\end{minted}
		\end{footnotesize}
	\end{frame}

	\begin{frame}[fragile]
		\frametitle{Более правильный вариант}
		\begin{minted}{java}
static int min(int firstArg, int... remainingArgs) {
   int min = firstArg:
   for (int arg : remainingArgs)
       if (arg < min)
           min = arg;
   return min;
}
		\end{minted}
	\end{frame}

	\begin{frame}[fragile]
		\frametitle{varargs методы}
		\begin{itemize}
			\item Метод с final массивом можно заменить на varargs метод
			\begin{itemize}
				\item Но это не значит, что так надо делать
			\end{itemize}
			\item Идиома печати массива (java 1.5-):
				\begin{minted}{java}
System.out.println(Arrays.asList(myArray));
				\end{minted}
			\item Но с примитивными типами всё равно получим не то:
				\begin{minted}{java}
public static void main(String[] args) {
   int[] digits = { 3, 1, 4, 1, 5, 9, 2, 6, 5, 4 };
   System.out.println(Arrays.asList(digits));
}

// Правильный способ напечатать массив
System.out.рrintln(Arrays.toString(myArray));
				\end{minted}
		\end{itemize}
	\end{frame}

	\begin{frame}[fragile]
		\frametitle{Производительность}
		\begin{itemize}
			\item Каждый вызов приводит к размещению и инициализации массива
			\item Если критична производительность --- делаем несколько перегрузок:
				\begin{minted}{java}
public void foo()  {}
public void foo(int a1)  {}
public void foo(int a1, int a2)  {}
public void foo(int a1, int a2, int a3)  {}
public void foo(int a1, int a2, int a3, int... rest)  {}
				\end{minted}
		\end{itemize}
	\end{frame}

	\begin{frame}[fragile]
		\frametitle{54. Возвращаем массив нулевой длины вместо null}
		\begin{small}
			\begin{columns}
				\begin{column}{0.5\textwidth}
					\begin{minted}{java}
private List cheesesInStock = ...;
public List<Cheese> getCheeses() {
    return cheesesInStock.isEmpty() 
        ? null
        : new ArrayList<>(cheesesInStock);
}
					\end{minted}
				\end{column}
				\begin{column}{0.5\textwidth}
					\begin{itemize}
						\item Можно возвращать закэшированный пустой список
						\item И вообще, предварительная оптимизация это зло
					\end{itemize}
				\end{column}
			\end{columns}
			\vspace{5mm}
			\begin{minted}{java}
List<Cheese> cheeses = shop.getCheeses();
if (cheeses != null && cheeses.contains(Cheese.STILTON)) {
   System.out.println("Jolly good, just the thing.");
}
			\end{minted}
		\end{small}
	\end{frame}

	\begin{frame}[fragile]
		\frametitle{Выгрузка элементов из коллекции в массив}
		\begin{small}
			\begin{minted}{java}
// Правильный способ вывести массив из коллекции
private final List cheesesInStock = ...;
private static final Cheese[] EMPTY_CHEESE_ARRAY = new Cheese[0]:
public Cheese[] getCheeses() {
   return (Cheese[]) cheesesInStock.toArray(EMPTY_CHEESE_ARRAY);
}

// Правильный способ возврата копии коллекции.
public List<Cheese> getCheeseList() {
   if (cheesesInStock.isEmpty())  
      return Collections.emptyList(); // Always returns same list
   else 
      return new ArrayList<Cheese>(cheesesInStock);
}
			\end{minted}
		\end{small}
	\end{frame}

	\begin{frame}
		\frametitle{56. Для всех открытых элементов API пишите доки!}
		\begin{itemize}
			\item А ещё стоит документировать и другие элементы для тех, кто будет это всё сопровождать
			\item Документация для метода должна лаконично описывать контракт между этим методом и его клиентами
			\begin{itemize}
				\item Что делает, а не как
				\begin{itemize}
					\item Исключения --- методы в классах, предназначенных для наследования
				\end{itemize}
				\item Предусловия и постусловия
				\begin{itemize}
					\item \mintinline{java}|@throws| для описания предусловий
					\item Или в \mintinline{java}|@param|
				\end{itemize}
				\item Побочные эффекты
				\item Потокобезопасность
			\end{itemize}
		\end{itemize}
	\end{frame}

	\begin{frame}[fragile]
		\begin{footnotesize}
			\begin{minted}{java}
/**
* Возвращает элемент, который занимает заданную позицию в данном списке.
*
* <р>Этот метод <i>He</i> дает гарантии, что будет выполняться в постоянное
* время. В некоторых реализациях он будет выполняться во время,
* пропорциональное положению его элементов.
*
* @param index индекс элемента, который нужно возвратить;
*    индекс должен быть меньше размера списка и неотрицательным
* @return элемент, занимающий в списке указанную позицию
* @throws IndexOutOfBoundsException, если индекс лежит вне диапазона
*     ({@code index < 0 || index >= this.size()})
*/
E get(int index);
			\end{minted}
		\end{footnotesize}
	\end{frame}

	\begin{frame}
		\frametitle{Подробности}
		\begin{itemize}
			\item Методы и конструкторы --- глагольная конструкция
			\begin{itemize}
				\item ``Создаёт пустой список'', ``Возвращает количество элементов''
			\end{itemize}
			\item Классы, интерфейсы, поля --- именная конструкция
			\begin{itemize}
				\item ``Значение, наиболее близкое к пи'', ``Задача для однократного исполнения''
			\end{itemize}
			\item Для генериков не забываем комментировать параметры типа
			\item Для перечислений документируем константы, тип и все открытые методы
			\begin{itemize}
				\item Можно в одну строку, если комментарий краткий
			\end{itemize}
			\item Не забываем документировать аннотации для аннотируемых типов
			\item Для пакетов создаём package-info.java
			\item Наследование комментариев
		\end{itemize}
	\end{frame}

	\section{Исключения}

	\begin{frame}
		\frametitle{70. Проверяемые vs непроверяемые исключения}
		\begin{itemize}
			\item Проверяемые исключения
			\begin{itemize}
				\item Явное предупреждение клиентскому коду
				\item Вызывающая сторона в состоянии обработать и восстановиться
				\begin{itemize}
					\item Вспомогательные методы и данные в объекте исключения
				\end{itemize}
			\end{itemize}
			\item Непроверяемые исключения
			\begin{itemize}
				\item RuntimeExceptions --- индикация программных ошибок
				\item Errors -- невозможность дальнейшего выполнения
			\end{itemize}
			\item Throwable
			\begin{itemize}
				\item Не рекомендуется наследовать явно
			\end{itemize}
		\end{itemize}
	\end{frame}

	\begin{frame}[fragile]
		\frametitle{71. Избегайте ненужного использования проверяемых исключений}
		\begin{itemize}
			\item Невозможно предотвратить условие для исключительной ситуации
			\item Программист может как-то полезно её обработать
		\end{itemize}
		\begin{minted}{java}
} catch (TheCheckedException е) {
    throw new AssertionError(); // Этого случиться не может!
}
		\end{minted}
		\vspace{3mm}
		\begin{minted}{java}
} catch (TheCheckedException е) {
    // nothing to do
}
		\end{minted}
	\end{frame}

	\begin{frame}[fragile]
		\frametitle{Вариант решения: разделение на два метода}
		\begin{columns}
			\begin{column}{0.4\textwidth}
				\begin{itemize}
					\item Прямо как в Iterator
					\item Beware!
					\begin{itemize}
						\item Многопоточность
						\item Изменение состояния извне
					\end{itemize}
				\end{itemize}
			\end{column}
			\begin{column}{0.6\textwidth}
				\begin{footnotesize}
					\begin{minted}{java}
// Вызов с проверяемым исключением
try {
    obj.action(args);
} catch (TheCheckedException e) {
    // Обработать исключительную ситуацию
}
					\end{minted}
					\vspace{3mm}
					\begin{minted}{java}
// Вызов с использованием метода проверки
// состояния и непроверяемого исключения
if (obj.actionPermitted(args)) {
    obj.action(args);
} else {
    // Обработать исключительную ситуацию
}
					\end{minted}
				\end{footnotesize}
			\end{column}
		\end{columns}
	\end{frame}

	\begin{frame}[fragile]
		\frametitle{73. Выбрасывайте исключения, соответствующие абстракции}
		\begin{itemize}
			\item Трансляция исключений
			\begin{itemize}
				\item Перехватывайте исключения, бросайте дальше что-то более адекватное
				\item Exception chaining
			\end{itemize}
		\end{itemize}
		\begin{footnotesize}
			\begin{minted}{java}
/**
* Возвращает элемент, находящийся в указанной позиции в этом списке.
* @throws IndexOutOfBoundsException, если индекс находится
* за пределами диапазона (index < 0 || index >= size()).
*/
public E get(int index) {
    ListIterator<E> i = listIterator(index);
    try {
        return i.next();
    } catch (NoSuchElementException e) {
        throw new IndexOutOfBoundsException("Index: " + index);
    }
}
			\end{minted}
		\end{footnotesize}
	\end{frame}

	\begin{frame}
		\frametitle{74. Документируйте все выбрасываемые исключения}
		\begin{itemize}
			\item Объявляйте все проверяемые исключения при помощи \mintinline{java}|throws| и \mintinline{java}|@throws|
			\begin{itemize}
				\item Не экономьте при помощи суперкласса (\mintinline{java}|throws Throwable|)
				\begin{itemize}
					\item Однако бывают исключения в публичных интерфейсах
				\end{itemize}
			\end{itemize}
			\item Документируйте все непроверяемые исключения при помощи \mintinline{java}|@throws|
			\begin{itemize}
				\item В сигнатуре метода не нужно
				\item Не всегда получается это сделать в полной мере
			\end{itemize}
			\item Документирование на уровне класса в целом
			\begin{itemize}
				\item ``Все методы выбрасывают NPE, если им передают аргументом null''
			\end{itemize}
		\end{itemize}
	\end{frame}

	\begin{frame}
		\frametitle{77. Не игнорируйте исключения}
		\begin{itemize}
			\item Пустой блок \mintinline{java}|catch| лишает смысла механизм исключений
			\item Как минимум -- комментарий, почему исключение игнорируется
			\item Логирование исключений
			\item Fail fast
		\end{itemize}
	\end{frame}

\end{document}
