\documentclass[xetex,mathserif,serif]{beamer}
\usepackage{polyglossia}
\setdefaultlanguage[babelshorthands=true]{russian}
\usepackage{minted}
\usepackage{tabu}
\usepackage{moresize}

\useoutertheme{infolines}

\usepackage{fontspec}
\setmainfont{FreeSans}
\newfontfamily{\russianfonttt}{FreeSans}

\definecolor{links}{HTML}{2A1B81}
\hypersetup{colorlinks,linkcolor=,urlcolor=links}

\tabulinesep=1.2mm

\title{Функциональное программирование на языке F\#}
\subtitle{Введение}
\author{Юрий Литвинов}
\date{14.02.2020г}

\begin{document}
	
	\frame{\titlepage}
	
	\section{Введение}
	
	\begin{frame}
		\frametitle{О чём этот курс}
		\begin{itemize}
			\item Теория и практика функционального программирования
			\begin{itemize}
				\item $\lambda$-исчисление
				\item Базовые принципы ФП (программирование без состояний, функции высших порядков, каррирование~и~т.д.)
				\item Типы в функциональном программировании (немутабельные коллекции,
					генерики, автообобщение~и~т.д.)
				\item Паттерны функционального программирования (CPS, монады, point-free)
			\end{itemize}
			\item Программирование на F\# 
			\begin{itemize}
				\item ООП в F\#
				\item Асинхронное и многопоточное программирование в F\#
				\item Может, компиляторы, анализ данных, машинное обучение\dots
			\end{itemize}
		\end{itemize}
	\end{frame}

	\begin{frame}
		\frametitle{Отчётность}
		\begin{itemize}
			\item Домашка (много несложной)
			\item Одна контрольная в середине семестра
			\item Учебная практика
			\item Доклад (-1 домашка)
		\end{itemize}
	\end{frame}
	
	\begin{frame}
		\frametitle{О практиках}
		\begin{itemize}
			\item Объём --- 5-7 страниц содержательного текста
			\begin{itemize}
				\item Титульный лист (\url{http://math.spbu.ru/rus/study/alumni\_info.html})
				\item Оглавление
				\item Введение в предметную область, постановка задачи
				\item Обзор литературы и существующих решений
				\item Описание предлагаемого решения, сравнение с существующими
				\item Заключение
				\item Список источников (ГОСТ Р 7.0.5--2008)
			\end{itemize}
			\item Конференции
			\begin{itemize}
				\item {}<<Современные технологии в теории и практике программирования>> --- дедлайн 18 марта
			\end{itemize}
		\end{itemize}
	\end{frame}

	\begin{frame}
		\frametitle{Где брать темы}
		\begin{itemize}
			\item Продолжать начатое
			\item Стажировки JetBrains
			\item Студпроекты Теркома
			\item Придумать самим 
			\begin{itemize}
				\item Политически немудро, но может быть интересно
			\end{itemize}
			\item Взять что-нибудь у кого-нибудь поблизости
			\begin{itemize}
				\item Робототехника
				\item Формальные методы
				\item Machine Learning
			\end{itemize}
		\end{itemize}
	\end{frame}

	\section{Введение в функциональное программирование}
	
	\begin{frame}
		\frametitle{Императивное программирование}
		Программа как последовательность \textbf{операторов}, изменяющих \textbf{состояние} вычислителя.

		Для конечных программ есть \textbf{начальное состояние}, \textbf{конечное состояние} и последовательность переходов:
		$$\sigma = \sigma_1 \rightarrow \sigma_2 \rightarrow ... \rightarrow \sigma_n = \sigma'$$
		
		Основные понятия:
		\begin{itemize}
			\item Переменная
			\item Присваивание
			\item Поток управления
			\begin{itemize}
				\item Последовательное исполнение
				\item Ветвления
				\item Циклы
			\end{itemize}
		\end{itemize}
	\end{frame}
	
	\begin{frame}
		\frametitle{Функциональное программирование}
		Программа как вычисление значения \textbf{выражения} в математическом смысле на некоторых входных данных.
		$$\sigma' = f(\sigma)$$
	
		\begin{itemize}
			\item Нет состояния $\Rightarrow$ нет переменных
			\item Нет переменных $\Rightarrow$ нет циклов
			\item Нет явной спецификации потока управления
		\end{itemize}
		Порядок вычислений не важен, потому что нет состояния, результат вычисления зависит только от входных данных.
	\end{frame}
	
	\begin{frame}[fragile]
		\frametitle{Сравним}
		\begin{alertblock}{C++}
			\begin{minted}{cpp}
int factorial(int n) {
    int result = 1;
    for (int i = 1; i <= n; ++i) {
        result *= i;
    }
    return result;
}
			\end{minted}
		\end{alertblock}
		\begin{exampleblock}{F\#}
			\begin{minted}{fsharp}
let rec factorial x =
    if x = 1 then 1 else x * factorial (x - 1)
            \end{minted}
		\end{exampleblock}
\end{frame}

	\begin{frame}[fragile]
		\frametitle{Как с этим жить}
		\begin{itemize}
			\item Состояние и переменные <<эмулируются>> параметрами функций
			\item Циклы <<эмулируются>> рекурсией
			\item Последовательность вычислений --- рекурсия + параметры
		\end{itemize}
		\begin{exampleblock}{F\#}
			\begin{minted}{fsharp}
let rec sumFirst3 ls acc i =
    if i = 3 then 
         acc 
    else 
        sumFirst3 
            (List.tail ls) 
            (acc + ls.Head) 
            (i + 1)
            \end{minted}
		\end{exampleblock}
\end{frame}

	\begin{frame}
		\frametitle{Зачем}
		\begin{itemize}
			\item Строгая математическая основа
			\item Семантика программ более естественна
			\begin{itemize}
				\item Применима математическая интуиция
			\end{itemize}
			\item Программы проще для анализа
			\begin{itemize}
				\item Автоматический вывод типов
				\item Оптимизации
			\end{itemize}
			\item Более декларативно
			\begin{itemize}
				\item Ленивость
				\item Распараллеливание
			\end{itemize}
			\item Модульность и переиспользуемость
			\item Программы более выразительны
		\end{itemize}
	\end{frame}
	
	\begin{frame}[fragile]
		\frametitle{Пример: функции высших порядков}
		\begin{exampleblock}{F\#}
			\begin{minted}{fsharp}
let sumFirst3 ls = 
    Seq.fold 
        (fun x acc -> acc + x) 
        0 
        (Seq.take 3 ls)
            \end{minted}
		\end{exampleblock}
		\begin{exampleblock}{F\#}
			\begin{minted}{fsharp}
let sumFirst3 ls = ls |> Seq.take 3 |> Seq.fold (+) 0
            \end{minted}
		\end{exampleblock}
		\begin{exampleblock}{F\#}
			\begin{minted}{fsharp}
let sumFirst3 = Seq.take 3 >> Seq.fold (+) 0
            \end{minted}
		\end{exampleblock}
\end{frame}

	\begin{frame}[fragile]
		\frametitle{Ещё пример}
		\framesubtitle{Возвести в квадрат и сложить все чётные числа в списке}
		\begin{exampleblock}{F\#}
			\begin{minted}{fsharp}
let calculate = 
    Seq.filter (fun x -> x % 2 = 0) 
    >> Seq.map (fun x -> x * x) 
    >> Seq.reduce (+)
            \end{minted}
		\end{exampleblock}
\end{frame}

	\begin{frame}
		\frametitle{Почему тогда все не пишут функционально}
		\begin{itemize}
			\item Чистые функции не могут оказывать влияние на внешний мир. Ввод-вывод, работа с данными,
					вообще выполнение каких-либо действий не укладывается в функциональную модель.
			\item Сложно анализировать производительность, иногда функциональные программы проигрывают
					в производительности императивным. <<Железо>>, грубо говоря, представляет собой 
					реализацию машины Тьюринга, тогда как функциональные программы определяются над
					$\lambda$-исчислением.
			\item Требуется математический склад ума и вообще желание думать.
		\end{itemize}
	\end{frame}

	\section{F\#}

	\begin{frame}
		\frametitle{F\#}
		\begin{itemize}
			\item Типизированный функциональный язык для платформы .NET
			\item НЕ чисто функциональный (можно императивный стиль и ООП)
			\item Первый раз представлен публике в 2005 г., актуальная версия --- 4.7 (23 сентября 2019 года)
			\item Создавался под влиянием OCaml (практически диалект OCaml под .NET)
			\item Использует .NET CLI
			\item Компилируемый и интерпретируемый
			\item Иногда используется в промышленности, в отличие от многих чисто функциональных языков
		\end{itemize}
	\end{frame}

	\begin{frame}
		\frametitle{Что скачать и поставить}
		\begin{itemize}
			\item Под Windows --- Visual Studio, из коробки
			\item Под Linux
			\begin{itemize}
				\item Rider (студентам бесплатно)
				\item Mono + MonoDevelop + F\# Language Binding, из репозиториев
				\item .NET Core + Visual Studio Code + Ionide
			\end{itemize}
			\item Прямо в браузере: \url{https://dotnetfiddle.net/}
		\end{itemize}
	\end{frame}
	
	\begin{frame}[fragile]
		\frametitle{Пример программы}
		\begin{minted}{fsharp}
printfn "%s" "Hello, world!"
		\end{minted}
		Сравните с
		\begin{minted}{csharp}
namespace HelloWorld
{
    class Program
    {
        static void Main(string[] args)
        {
            System.Console.WriteLine("Hello, world!");
        }
    }
}
		\end{minted}
	\end{frame}

	\section{Let-определения}

	\begin{frame}[fragile]
		\frametitle{let-определение}
		\begin{minted}{fsharp}
let x = 1
let x = 2
printfn "%d" x
		\end{minted}
		можно читать как
		\begin{minted}{fsharp}
let x = 1 in let x = 2 in printfn "%d" x
		\end{minted}
		и понимать как подстановку $\lambda$-терма
	\end{frame}

	\begin{frame}[fragile]
		\frametitle{let-определение, функции}
		\begin{minted}{fsharp}
let powerOfFour x = 
    let xSquared = x * x
    xSquared * xSquared
		\end{minted}
		\begin{itemize}
			\item Позиционный синтаксис
			\begin{itemize}
				\item Отступы строго пробелами
				\item Не надо ";"
			\end{itemize}
			\item Нет особых синтаксических различий между переменной и функцией
			\item Не надо писать типы
			\item Не надо писать \textit{return}
		\end{itemize}
	\end{frame}

	\begin{frame}[fragile]
		\frametitle{Вложенные let-определения}
		\begin{minted}{fsharp}
let powerOfFourPlusTwoTimesSix n =
    let n3 =
        let n1 = n * n
        let n2 = n1 * n1
        n2 + 2
    let n4 = n3 * 6
    n4
		\end{minted}
		\begin{itemize}
			\item \textit{n3} --- не функция!
			\item Компилятор отличает значения и функции по наличию аргументов
			\item Значение вычисляется, когда до \textit{let} <<доходит управление>>, 
					функция --- когда её вызовут. Хотя, конечно, функция --- тоже значение.
		\end{itemize}
	\end{frame}

	\section{Типы}

	\begin{frame}[fragile]
		\frametitle{Типы}
		\begin{minted}{fsharp}
let rec f x =
    if x = 1 then 
        1 
    else 
        x * f (x - 1)
		\end{minted}

		\begin{alertblock}{F\# Interactive}
			\begin{minted}{fsharp}
val f : x:int -> int
			\end{minted}
		\end{alertblock}
		Каждое значение имеет тип, известный во время компиляции
	\end{frame}

	\begin{frame}
		\frametitle{Элементарные типы}
		\begin{itemize}
			\item \textit{int}
			\item \textit{double}
			\item \textit{bool}
			\item \textit{string}
			\item ... (.NET)
			\item \textit{unit} --- тип из одного значения, (). Аналог void.
		\end{itemize}
	\end{frame}
	
	\begin{frame}[fragile]
		\frametitle{Кортежи (tuples)}
		\begin{minted}{fsharp}
let site1 = ("scholar.google.com", 10)
let site2 = ("citeseerx.ist.psu.edu", 5)
let site3 = ("scopus.com", 4)
let sites = (site1, site2, site3)

let url, relevance = site1
let site1, site2, site3 = sites
		\end{minted}
	\end{frame}

	\begin{frame}[fragile]
		\frametitle{Value Tuples}
		\begin{minted}{fsharp}
let origin = struct (0, 0)

let displace struct (x, y) struct (dx, dy)
    = struct (x + dx, y + dy)

displace origin struct (1, 1)
		\end{minted}
	\end{frame}

	\begin{frame}[fragile]
		\frametitle{Лямбды}
		\begin{minted}{fsharp}
let primes = [2; 3; 5; 7]
let primeCubes = List.map (fun n -> n * n * n) primes
		\end{minted}
		\begin{alertblock}{F\# Interactive}
			\begin{minted}{fsharp}
> primeCubes;;
val it : int list = [8; 27; 125; 343]
			\end{minted}
		\end{alertblock}
		\begin{minted}{fsharp}
let f = fun x -> x * x
let n = f 4
		\end{minted}
	\end{frame}


	\begin{frame}
		\frametitle{Списки}
		\begin{small}
			\begin{tabu} {| X[0.9 l p] | X[1 l p] | X[1 l p] |}
				\tabucline-
				Синтаксис                               & Описание                  & Пример                                      \\
				\tabucline-
				\everyrow{\tabucline-}
				$[]$                                    & Пустой список             & $[]$                                        \\
				$[expr; ...; expr]$                     & Список с элементами       & $[1; 2; 3]$                                 \\
				$expr :: list$                          & cons, добавление в голову & $1 :: [2; 3]$                               \\
				$[expr\ ..\ expr]$                      & Промежуток целых чисел    & $[1 .. 10]$                                 \\
				$[for\ x\ in\ list\ \rightarrow\ expr]$ & Генерированный список     & $[for\ x\ in\ 1..99\ \rightarrow\ x\ *\ x]$ \\
				$list\ @\ list$                         & Конкатенация              & $[1; 2]\ @\ [3; 4]$                         \\
			\end{tabu}
		\end{small}
	\end{frame}

	\begin{frame}[fragile]
		\frametitle{Примеры работы со списками}
		\begin{minted}{fsharp}
let oddPrimes = [3; 5; 7; 11]
let morePrimes = [13; 17]
let primes = 2 :: (oddPrimes @ morePrimes)
		\end{minted}
		\begin{minted}{fsharp}
let printFirst primes =
    match primes with
    | h :: t -> printfn "First prime in the list is %d" h
    | [] -> printfn "No primes found in the list"
		\end{minted}
	\end{frame}

	\begin{frame}[fragile]
		\frametitle{Устройство списков}
		\begin{center}
			\includegraphics[width=0.4\textwidth]{lists.png}
		\end{center}
		\begin{minted}{fsharp}
let list3 = [3; 4]
let list1 = 2 :: list3
let list2 = 1 :: list3
		\end{minted}
		\begin{itemize}
			\item Списки немутабельны
			\item Cons-ячейки, указывающие друг на друга
			\item cons за константное время, @ --- за линейное
		\end{itemize}
	\end{frame}

	\begin{frame}
		\frametitle{Операции над списками}
		\framesubtitle{Модуль Microsoft.FSharp.Collections.List}
		\begin{small}
			\begin{tabu} {| X[0.5 l p] | X[1 l p] | X[1 l p] | X[0.5 l p] |}
				\tabucline-
				Функция                & Описание                            & Пример                                              & Результат            \\
				\tabucline-
				\everyrow{\tabucline-}
				List.length            & Длина списка                        & $List.length\ [1;2;3]$                              & $3$                  \\
				List.nth               & n-ый элемент списка                 & $List.nth\ [1; 2; 3]\ 1$                            & $2$                  \\
				List.init              & Генерирует список                   & $List.init\ 3 (fun\ i\ \rightarrow\ i * i)$         & $[0; 1; 4]$          \\
				List.head              & Голова списка                       & $List.head\ [1; 2; 3]$                              & $1$                  \\
				List.tail              & Хвост списка                        & $List.tail\ [1; 2; 3]$                              & $[2; 3]$             \\
				List.map               & Применяет функцию ко всем элементам & $List.map\ (fun\ i\ \rightarrow\ i * i)\ [1; 2; 3]$ & $[1; 4; 9]$          \\
				List.filter            & Отбирает нужные элементы            & $List.filter\ (fun\ x\ \rightarrow\ x\ \%\ 2 <> 0)\ [1; 2; 3]$ & $[1; 3]$  \\
				List.fold              & "Свёртка"  & $List.fold\ (fun\ x\ acc\ \rightarrow\ acc * x)\ 1\ [1; 2; 3]$               & $6$                  \\
			\end{tabu}
		\end{small}
	\end{frame}
	
	\begin{frame}[fragile]
		\frametitle{Тип Option}
		Либо \textit{Some что-то}, либо \textit{None}, представляет возможное отсутствие значения.
		\begin{minted}{fsharp}
let people = [ ("Adam", None); ("Eve" , None);
    ("Cain", Some("Adam","Eve"));
    ("Abel", Some("Adam","Eve")) ]
		\end{minted}
		\begin{minted}{fsharp}
let showParents (name, parents) =
    match parents with
    | Some(dad, mum) -> 
        printfn "%s, father %s, mother %s" name dad mum
    | None -> printfn "%s has no parents!" name
		\end{minted}
	\end{frame}

	\section{Функции}

	\begin{frame}[fragile]
		\frametitle{Рекурсия}
		\begin{minted}{fsharp}
let rec length l =
    match l with
    | [] -> 0
    | h :: t -> 1 + length t

let rec even n = (n = 0u) || odd(n - 1u)
and odd n = (n <> 0u) && even(n - 1u)
		\end{minted}
	\end{frame}

	\begin{frame}[fragile]
		\frametitle{Каррирование, частичное применение}
		\begin{minted}{fsharp}
let shift (dx, dy) (px, py) = (px + dx, py + dy)
let shiftRight = shift (1, 0)
let shiftUp = shift (0, 1)
let shiftLeft = shift (-1, 0)
let shiftDown = shift (0, -1)
		\end{minted}
		\begin{alertblock}{F\# Interactive}
			\begin{minted}{fsharp}
> shiftDown (1, 1);;
val it : int * int = (1, 0)
			\end{minted}
		\end{alertblock}
	\end{frame}

	\begin{frame}[fragile]
		\frametitle{Зачем --- функции высших порядков}
		\begin{minted}{fsharp}
let lists = [[1; 2]; [1]; [1; 2; 3]; [1; 2]; [1]]
let lengths = List.map List.length lists
		\end{minted}
		или
		\begin{minted}{fsharp}
let lists = [[1; 2]; [1]; [1; 2; 3]; [1; 2]; [1]]
let squares = List.map (List.map (fun x -> x * x)) lists
		\end{minted}
		\vspace{3mm}
		Функции стандартной библиотеки стараются принимать список последним, для каррирования
	\end{frame}

	\begin{frame}[fragile]
		\frametitle{Оператор $|>$}
		\framesubtitle{Pipe forward}
		\begin{minted}{fsharp}
let (|>) x f = f x
		\end{minted}

		\begin{minted}{fsharp}
let sumFirst3 ls = ls |> Seq.take 3 |> Seq.fold (+) 0
		\end{minted}
		вместо
		\begin{minted}{fsharp}
let sumFirst3 ls= Seq.fold (+) 0 (Seq.take 3 ls)
		\end{minted}
	\end{frame}

	\begin{frame}[fragile]
		\frametitle{Оператор $>>$}
		\framesubtitle{Композиция}
		\begin{minted}{fsharp}
let (>>) f g x = g (f x)
		\end{minted}
		\begin{minted}{fsharp}
let sumFirst3 = Seq.take 3 >> Seq.fold (+) 0
let result = sumFirst3 [1; 2; 3; 4; 5]
		\end{minted}
	\end{frame}

	\begin{frame}[fragile]
		\frametitle{Операторы $<|$ и $<<$}
		\framesubtitle{Pipe-backward и обратная композиция}
		\begin{minted}{fsharp}
let (<|) f x = f x
let (<<) f g x = f (g x)
		\end{minted}
		Зачем? Чтобы не ставить скобки:
		\begin{minted}{fsharp}
printfn "Result = %d" <| factorial 5
		\end{minted}
	\end{frame}

	\section{.NET}

	\begin{frame}[fragile]
		\frametitle{Использование библиотек .NET}
		\begin{scriptsize}
			\begin{minted}{fsharp}
open System.Windows.Forms

let form = new Form(Visible = false, TopMost = true, Text = "Welcome to F#")
let textB = new RichTextBox(Dock = DockStyle.Fill, Text = "Some text")
form.Controls.Add(textB)

open System.IO
open System.Net

/// Get the contents of the URL via a web request
let http(url: string) =
    let req = System.Net.WebRequest.Create(url)
    let resp = req.GetResponse()
    let stream = resp.GetResponseStream()
    let reader = new StreamReader(stream)
    let html = reader.ReadToEnd()
    resp.Close()
    html

textB.Text <- http("http://www.google.com")

form.ShowDialog () |> ignore
			\end{minted}
		\end{scriptsize}
	\end{frame}

	\section{Сопоставление шаблонов}
	
	\begin{frame}[fragile]
		\frametitle{Сопоставление шаблонов}
		\begin{minted}{fsharp}
let urlFilter url agent =
    match (url, agent) with
    | "http://www.google.com", 99 -> true
    | "http://www.yandex.ru" , _ -> false
    | _, 86 -> true
    | _ -> false
		\end{minted}

		\begin{minted}{fsharp}
let sign x =
    match x with
    | _ when x < 0 -> -1
    | _ when x > 0 -> 1
    | _ -> 0
		\end{minted}
	\end{frame}

	\begin{frame}[fragile]
		\frametitle{F\# --- не Prolog}
		Не получится писать так:
		\begin{minted}{fsharp}
let isSame pair =
    match pair with
    | (a, a) -> true
    | _ -> false
		\end{minted}
		Нужно так:
		\begin{minted}{fsharp}
let isSame pair =
    match pair with
    | (a, b) when a = b -> true
    | _ -> false
		\end{minted}
	\end{frame}

	\begin{frame}
		\frametitle{Какие шаблоны бывают}
		\begin{small}
			\begin{tabu} {| X[0.9 l p] | X[1 l p] | X[1 l p] |}
				\tabucline-
				Синтаксис                               & Описание                  & Пример                  \\
				\tabucline-
				\everyrow{\tabucline-}
				$(pat, \ldots, pat)$                    & Кортеж                    & $(1, 2, (``3``, x))$    \\
				$[pat; \ldots; pat]$                    & Список                    & $[x; y; 3]$             \\
				$pat :: pat$                            & cons                      & $h :: t$                \\
				$pat\ |\ pat$                           & "Или"                     & $[x]\ |\ [``X``\ ;\ x]$ \\
				$pat\ \&\ pat$                          & "И"                       & $[p] \& [(x, y)]$       \\
				$pat\ as\ id$                           & Именованный шаблон        & $[x]\ as\ inp$          \\
				$id$                                    & Переменная                & $x$                     \\
				$\_$                                    & Wildcard (что угодно)     & $\_$                    \\
				литерал                                 & Константа                 & $239, DayOfWeek.Monday$ \\
				$:?\ type$                              & Проверка на тип           & $:?\ string$            \\
			\end{tabu}
		\end{small}
	\end{frame}

	\section{Хвостовая рекурсия}

	\begin{frame}[fragile]
		\frametitle{Хвостовая рекурсия, проблема}
		\framesubtitle{Императивный вариант}
		\begin{minted}{fsharp}
open System.Collections.Generic

let createMutableList () =
    let l = new List<int>()
    for i = 0 to 100000 do
        l.Add(i)
    l
		\end{minted}
	\end{frame}

	\begin{frame}[fragile]
		\frametitle{Хвостовая рекурсия, проблема}
		\framesubtitle{Рекурсивный вариант, казалось бы}
		\begin{minted}{fsharp}
let createImmutableList () =
    let rec createList i max =
        if i = max then
            []
        else
            i :: createList (i + 1) max
    createList 0 100000
		\end{minted}
	\end{frame}

	\begin{frame}[fragile]
		\frametitle{Факториал без хвостовой рекурсии}
		\begin{minted}{fsharp}
let rec factorial x =
    if x <= 1
    then 1 
    else x * factorial (x - 1)
		\end{minted}

		\begin{minted}{fsharp}
let rec factorial x =
    if x <= 1
    then
        1
    else
        let resultOfRecusion = factorial (x - 1)
        let result = x * resultOfRecusion
        result
		\end{minted}
	\end{frame}

	\begin{frame}[fragile]
		\frametitle{Факториал с хвостовой рекурсией}
		\begin{minted}{fsharp}
let factorial x =
    let rec tailRecursiveFactorial x acc =
        if x <= 1 then
            acc
        else
            tailRecursiveFactorial (x - 1) (acc * x)
    tailRecursiveFactorial x 1
		\end{minted}
	\end{frame}
	
	\begin{frame}[fragile]
		\frametitle{После декомпиляции в C\#}
		\begin{alertblock}{C\#}
			\begin{minted}{csharp}
public static int tailRecursiveFactorial(int x, int acc)
{
    while (true)
    {
        if (x <= 1)
        {
            return acc;
        }
        acc *= x;
        x--;
    }
}
			\end{minted}
		\end{alertblock}
	\end{frame}

	\begin{frame}[fragile]
		\frametitle{Паттерн ``Аккумулятор''}
		\begin{minted}{fsharp}
let rec map f list =
    match list with
    | [] -> []
    | hd :: tl -> (f hd) :: (map f tl)

let map f list =
    let rec mapTR f list acc =
        match list with
        | [] -> acc
        | hd :: tl -> mapTR f tl (f hd :: acc)
    mapTR f (List.rev list) []
		\end{minted}
	\end{frame}

\end{document}
