\documentclass[xetex,mathserif,serif]{beamer}
\usepackage{polyglossia}
\setdefaultlanguage[babelshorthands=true]{russian}
\usepackage{minted}
\usepackage{tabu}
\usepackage{algorithm2e}

\useoutertheme{infolines}

\usepackage{fontspec}
\setmainfont{FreeSans}
\newfontfamily{\russianfonttt}{FreeSans}

\usepackage{textpos}
\setlength{\TPHorizModule}{1cm}
\setlength{\TPVertModule}{1cm}

\setbeamertemplate{blocks}[rounded][shadow=false]

\setbeamercolor*{block title example}{fg=green!50!black,bg=green!15}
\setbeamercolor*{block body example}{fg=black,bg=green!5}

\tabulinesep=0.5mm

\hypersetup{colorlinks=true, linkcolor=blue, citecolor=blue, filecolor=blue, urlcolor=blue, pdftitle=1, pdfauthor=, pdfsubject=, pdfkeywords=}

\newcommand{\attribution}[1] {
    \begin{flushright}\begin{scriptsize}\textcolor{gray}{#1}\end{scriptsize}\end{flushright}
}

\title{Синтаксический анализ на F\#}
\subtitle{Часть 2: FParsec vs FsLex/FsYacc}
\author{Юрий Литвинов}
\date{08.05.2020г}

\begin{document}
    
    \frame{\titlepage}

    \section{Введение}
    
    \begin{frame}
        \frametitle{Арифметический интерпретатор}
        Задача: разработать интерпретатор арифметических выражений
        \begin{itemize}
            \item Должны поддерживаться 
            \begin{itemize}
                \item +, -, *, / (с приоритетами операций)
                \item Унарный минус
                \item Скобки
                \item Целые числа
            \end{itemize}
            \item По входной строке надо явно построить AST
            \item По построенному AST вычислить выражение
        \end{itemize}
    \end{frame}

    \begin{frame}[fragile]
        \frametitle{Грамматика}
        \begin{minted}{bnf}
E ::= E + E
    | E - E
    | E * E
    | E / E
    | -E
    | (E)
    | NUMBER

NUMBER ::= [0..9]+
        \end{minted}
    \end{frame}

    \section{FParsec}

    \begin{frame}[fragile]
        \frametitle{Подготовительная работа}
        \begin{itemize}
            \item Создаём проект
            \item Добавляем ссылку на FParsec в проект
            \item Убеждаемся, что всё работает
        \end{itemize}
        \begin{minted}{fsharp}
open FParsec

[<EntryPoint>]
let main argv =
    let result = "1.23" |> (run pfloat)
    printfn "%A" result
    0 
        \end{minted}
    \end{frame}

    \begin{frame}[fragile]
        \frametitle{Представление AST}
        \begin{minted}{fsharp}
type Expression =
    | Plus of Expression * Expression
    | Minus of Expression * Expression
    | Multiplication of Expression * Expression
    | Division of Expression * Expression
    | Negation of Expression
    | Number of int
        \end{minted}
    \end{frame}

    \begin{frame}[fragile]
        \frametitle{Начнём с Number}
        \begin{minted}{fsharp}
let number = digit

let testInput = "9"
printfn "%A" (testInput |> run number)

let testInput = "12"
printfn "%A" (testInput |> run number)
        \end{minted}

        \begin{exampleblock}{F\# Interactive}
            \begin{minted}{fsharp}
Success: '9'
Success: '1'
            \end{minted}
        \end{exampleblock}
    \end{frame}

    \begin{frame}[fragile]
        \frametitle{Позитивное замыкание, правильный разбор числа}
        \begin{minted}{fsharp}
let number = many1 digit

let testInput = "9"
printfn "%A" (testInput |> run number)

let testInput = "12"
printfn "%A" (testInput |> run number)
        \end{minted}

        \begin{exampleblock}{F\# Interactive}
            \begin{minted}{fsharp}
Success: ['9']
Success: ['1'; '2']
            \end{minted}
        \end{exampleblock}
    \end{frame}

    \begin{frame}[fragile]
        \frametitle{Делаем узел дерева}
        \begin{minted}{fsharp}
let number = 
    many1 digit 
    |>> (List.fold (fun acc x -> acc * 10 + int (x.ToString())) 0 >> Number)

let testInput = "9"
printfn "%A" (testInput |> run number)

let testInput = "12"
printfn "%A" (testInput |> run number)
        \end{minted}

        \begin{exampleblock}{F\# Interactive}
            \begin{minted}{fsharp}
Success: Number 9
Success: Number 12
            \end{minted}
        \end{exampleblock}
    \end{frame}

    \begin{frame}[fragile]
        \frametitle{Рекурсивные правила}
        \begin{minted}{fsharp}
let expression, expressionRef = createParserForwardedToRef()
let negation = pchar '-' >>. expression |>> Negation
expressionRef := choice [negation; number]

let testInput = "-9"
printfn "%A" (testInput |> run expression)

let testInput = "--12"
printfn "%A" (testInput |> run expression)
        \end{minted}

        \begin{exampleblock}{F\# Interactive}
            \begin{minted}{fsharp}
Success: Negation (Number 9)
Success: Negation (Negation (Number 12))
            \end{minted}
        \end{exampleblock}
    \end{frame}

    \begin{frame}[fragile]
        \frametitle{Победим пробелы}
        \begin{minted}{fsharp}
let expression, expressionRef = createParserForwardedToRef()
let negation = 
    pchar '-' .>> spaces >>. expression .>> spaces |>> Negation
expressionRef := choice [negation; number]

let testInput = "- 9"
printfn "%A" (testInput |> run expression)
        \end{minted}

        \begin{exampleblock}{F\# Interactive}
            \begin{minted}{fsharp}
Success: Negation (Number 9)
            \end{minted}
        \end{exampleblock}
    \end{frame}

    \begin{frame}[fragile]
        \frametitle{Сложение!}
        \framesubtitle{Наивный подход-1}
        \begin{minted}{fsharp}
let expression, expressionRef = createParserForwardedToRef()
let negation = 
    pchar '-' .>> spaces >>. expression .>> spaces |>> Negation
let plus = expression .>> pchar '+' .>>. expression |>> Plus

expressionRef := choice [negation; number; plus]

let testInput = "1 + 2"
printfn "%A" (testInput |> run expression)
        \end{minted}

        \begin{exampleblock}{F\# Interactive}
            \begin{minted}{fsharp}
Success: Number 1
            \end{minted}
        \end{exampleblock}
    \end{frame}

    \begin{frame}[fragile]
        \frametitle{Сложение!}
        \framesubtitle{Наивный подход-2}
        \begin{minted}{fsharp}
let expression, expressionRef = createParserForwardedToRef()
let negation = 
    pchar '-' .>> spaces >>. expression .>> spaces |>> Negation
let plus = expression .>> pchar '+' .>>. expression |>> Plus

expressionRef := choice [negation; plus; number]

let testInput = "1 + 2"
printfn "%A" (testInput |> run expression)
        \end{minted}

        \begin{exampleblock}{F\# Interactive}
            \begin{minted}{fsharp}
Stack overflow.
            \end{minted}
        \end{exampleblock}
        \begin{textblock}{2}(9,-3.5)
            \includegraphics[width=\textwidth]{hmm.png}
        \end{textblock}
    \end{frame}

    \begin{frame}[fragile]
        \frametitle{Факторизуем грамматику}
        \begin{minted}{bnf}
E ::= PRIMARY E'

E' ::= + PRIMARY E'
    | - PRIMARY E'
    | * PRIMARY E'
    | / PRIMARY E'
    | e

PRIMARY ::= -E
    | (E)
    | NUMBER

NUMBER ::= [0..9]+
        \end{minted}
    \end{frame}

    \begin{frame}[fragile]
        \frametitle{Перепишем парсер}
        \begin{minted}{fsharp}
let expression, expressionRef = createParserForwardedToRef()
let negation = 
    pchar '-' .>> spaces >>. expression .>> spaces |>> Negation
let brackets =
     pchar '(' .>> spaces >>. expression .>> spaces .>> pchar ')' .>> spaces 

let primary = 
    negation
    <|> brackets
    <|> number

let expression', expression'Ref  = createParserForwardedToRef()
expression'Ref := pchar '+' >>. primary .>>. expression' |>> ???
        \end{minted}
        AST строить неудобно!
    \end{frame}

    \begin{frame}[fragile]
        \frametitle{Введём промежуточное представление дерева}
        \framesubtitle{Parse tree}
        \begin{minted}{fsharp}
type Primary =
    | Negation of E
    | Brackets of E
    | Number of int
and E = 
    | E of Primary * E'
and E' =
    | Plus of Primary * E'
    | Minus of Primary * E'
    | Multiplication of Primary * E'
    | Division of Primary * E'
    | Epsilon
        \end{minted}
    \end{frame}

    \begin{frame}[fragile]
        \frametitle{Теперь уже перепишем парсер (1)}
        \framesubtitle{Чтобы он строил Parse tree}
        \begin{minted}{fsharp}
let e, eRef = createParserForwardedToRef()

let negation = pchar '-' .>> spaces >>. e .>> spaces |>> Negation
let brackets = 
    pchar '(' .>> spaces >>. e .>> spaces .>> pchar ')' .>> spaces
    |>> Brackets

let primary = 
    negation
    <|> brackets
    <|> number

        \end{minted}
    \end{frame}

    \begin{frame}[fragile]
        \frametitle{Теперь уже перепишем парсер (2)}
        \framesubtitle{E' и всё вместе}
        \begin{small}
            \begin{minted}{fsharp}
let e', e'Ref  = createParserForwardedToRef()
e'Ref := 
    (pchar '+' >>. spaces >>. primary .>> spaces .>>. e' |>> Plus)
    <|> (pchar '-' >>. spaces >>. primary .>> spaces .>>. e' |>> Minus)
    <|> (pchar '*' >>. spaces >>. primary .>> spaces .>>. e' |>> Multiplication)
    <|> (pchar '/' >>. spaces >>. primary .>> spaces .>>. e' |>> Division)
    <|> preturn Epsilon

eRef := primary .>> spaces .>>. e' |>> E

let testInput = "1 + 2"
printfn "%A" (testInput |> run e)
            \end{minted}

            \begin{exampleblock}{F\# Interactive}
                \begin{minted}{fsharp}
Success: E (Number 1,Plus (Number 2,Epsilon))
                \end{minted}
            \end{exampleblock}
        \end{small}
    \end{frame}

    \begin{frame}[fragile]
        \frametitle{Небольшой рефакторинг}
        \begin{minted}{fsharp}
let (!) parser = parser .>> spaces

let e', e'Ref  = createParserForwardedToRef()
e'Ref := 
    (!(pchar '+') >>. !primary .>>. !e' |>> Plus)
    <|> (!(pchar '-') >>. !primary .>>. !e' |>> Minus)
    <|> (!(pchar '*') >>. !primary .>>. !e' |>> Multiplication)
    <|> (!(pchar '/') >>. !primary .>>. !e' |>> Division)
    <|> preturn Epsilon

eRef := !primary .>>. !e' .>> eof |>> E
        \end{minted}
    \end{frame}

    \begin{frame}[fragile]
        \frametitle{Приоритет операций, проблема}
        \begin{small}
            \begin{minted}{fsharp}
let testInput = "1 + 2 * 3"
printfn "%A" (testInput |> run e)

let testInput = "1 * 2 + 3"
printfn "%A" (testInput |> run e)
            \end{minted}

            \begin{exampleblock}{F\# Interactive}
                \begin{minted}{fsharp}
Success: E (Number 1,Plus (Number 2,Multiplication (Number 3,Epsilon)))
Success: E (Number 1,Multiplication (Number 2,Plus (Number 3,Epsilon)))
                \end{minted}
            \end{exampleblock}
            Алгоритм сортировочной станции? Нет! У нас есть вся мощь формальных языков и библиотека парсер-комбинаторов
        \end{small}
    \end{frame}

    \begin{frame}[fragile]
        \frametitle{Ещё раз подправим грамматику}
        \begin{small}
            \begin{minted}{bnf}
E ::= TERM E'

E' ::= + TERM E'
    | - TERM E'
    | e

TERM ::= FACTOR TERM'

TERM' = 
    | * FACTOR TERM'
    | / FACTOR TERM'
    | e

FACTOR ::= -E
    | (E)
    | NUMBER

NUMBER ::= [0..9]+
            \end{minted}
        \end{small}
    \end{frame}

    \begin{frame}[fragile]
        \frametitle{Приведём Parse Tree в соответствие}
        \begin{small}
            \begin{minted}{fsharp}
type E =
    | E of Term * E'
and E' =
    | Plus of Term * E'
    | Minus of Term * E'
    | Epsilon
and Term = 
    | Term of Factor * Term'
and Term' =
    | Multiplication of Factor * Term'
    | Division of Factor * Term'
    | Epsilon
and Factor =
    | Negation of E
    | Brackets of E
    | Number of int
            \end{minted}
        \end{small}
    \end{frame}

    \begin{frame}[fragile]
        \frametitle{И сам парсер}
        \begin{scriptsize}
            \begin{minted}{fsharp}
let e, eRef = createParserForwardedToRef()

let factor = !(pchar '-') >>. !e |>> Negation
             <|> (!(pchar '(') >>. !e .>> !(pchar ')') |>> Brackets)
             <|> number

let term', term'Ref = createParserForwardedToRef()

term'Ref := !(pchar '*') >>. !factor .>>. !term' |>> Multiplication
            <|> (!(pchar '/') >>. !factor .>>. !term' |>> Division)
            <|> preturn Epsilon

let term = !factor .>>. !term' |>> Term

let e', e'Ref  = createParserForwardedToRef()

e'Ref :=
    !(pchar '+') >>. !term .>>. !e' |>> Plus
    <|> (!(pchar '-') >>. !term .>>. !e' |>> Minus)
    <|> preturn E'.Epsilon

eRef := !term .>>. !e' |>> E
            \end{minted}
        \end{scriptsize}
    \end{frame}

    \begin{frame}[fragile]
        \frametitle{Теперь}
        \begin{minted}{fsharp}
let testInput = "1 + 2 * 3"
printfn "%A" (testInput |> run e)

let testInput = "1 * 2 + 3"
printfn "%A" (testInput |> run e)
        \end{minted}

        \begin{exampleblock}{F\# Interactive}
            \begin{minted}{fsharp}
Success: E (Term (Number 1,Epsilon),
    Plus (Term (Number 2,Multiplication (Number 3,Epsilon)),Epsilon))
Success: E (Term (Number 1,Multiplication (Number 2,Epsilon)),
    Plus (Term (Number 3,Epsilon),Epsilon))
            \end{minted}
        \end{exampleblock}
        Сложнее, но тут уже получилась некоторая структура
    \end{frame}

    \begin{frame}[fragile]
        \frametitle{Построим AST по Parse Tree}
        \framesubtitle{Сначала Factor}
        \begin{minted}{fsharp}
let rec buildAST expr =
    let buildFactor = function
    | Negation(e) -> Expression.Negation(buildAST e)
    | Brackets(e) -> buildAST e
    | Number(x) -> Expression.Number(x)

    ()
        \end{minted}
    \end{frame}

    \begin{frame}[fragile]
        \frametitle{Построим AST по Parse Tree}
        \framesubtitle{Теперь термы}
        \begin{minted}{fsharp}
let rec buildTerm' acc = function
| Multiplication(factor, rest) -> 
    buildTerm' (Expression.Multiplication(acc, buildFactor factor)) rest
| Division(factor, rest) -> 
    buildTerm' (Expression.Division(acc, buildFactor factor)) rest
| Epsilon -> acc

let buildTerm (Term(factor, rest)) = buildTerm' (buildFactor factor) rest
        \end{minted}
    \end{frame}

    \begin{frame}[fragile]
        \frametitle{Построим AST по Parse Tree}
        \framesubtitle{А теперь и всё выражение}
        \begin{minted}{fsharp}
let rec buildE' acc = function
| Plus(factor, rest) -> 
    buildE' (Expression.Plus(acc, buildTerm factor)) rest
| Minus(factor, rest) -> 
    buildE' (Expression.Minus(acc, buildTerm factor)) rest
| E'.Epsilon -> acc

let buildE (E(term, rest)) = buildE' (buildTerm term) rest

buildE expr
        \end{minted}
    \end{frame}

    \begin{frame}[fragile]
        \frametitle{Потестим}
        \begin{minted}{fsharp}
let testInput = "1 * 2 + 3"

let result = testInput |> run e
printfn "%A" result

match result with
| Success(result, _, _) -> printfn "%A" <| buildAST result
| _ -> printfn "%A" result
        \end{minted}

        \begin{exampleblock}{F\# Interactive}
            \begin{minted}{fsharp}
Success: E (Term (Number 1,Multiplication (Number 2,Epsilon)),
    Plus (Term (Number 3,Epsilon),Epsilon))
Plus (Multiplication (Number 1,Number 2),Number 3)
            \end{minted}
        \end{exampleblock}
    \end{frame}

    \begin{frame}
        \frametitle{Что дальше}
        \begin{small}
            \begin{itemize}
                \item А считать выражение по такому дереву мы уже умеем
                \item Что в итоге получилось: \url{https://gist.github.com/yurii-litvinov/3b8b9e9328e06ac49d15481ba2cb3684}
                \item Что ещё умеет FParsec: \url{https://www.quanttec.com/fparsec/tutorial.html}
                \item Полное описание API библиотеки: \url{https://www.quanttec.com/fparsec/reference/}
                \item Монады! 
                
                \url{https://www.quanttec.com/fparsec/users-guide/where-is-the-monad.html}
                \item Как на самом деле парсить арифметические выражения: 
                
                \url{https://www.quanttec.com/fparsec/reference/operatorprecedenceparser.html}
            \end{itemize}
        \end{small}
    \end{frame}

\end{document}
