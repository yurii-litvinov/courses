\documentclass[xetex,mathserif,serif]{beamer}
\usepackage{polyglossia}
\setdefaultlanguage[babelshorthands=true]{russian}
\usepackage{minted}
\usepackage{tabu}
\usepackage{algorithm2e}

\useoutertheme{infolines}

\usepackage{fontspec}
\setmainfont{FreeSans}
\newfontfamily{\russianfonttt}{FreeSans}

\setbeamertemplate{blocks}[rounded][shadow=false]

\setbeamercolor*{block title alerted}{fg=red!50!black,bg=red!20}
\setbeamercolor*{block body alerted}{fg=black,bg=red!3}

\tabulinesep=0.5mm

\hypersetup{colorlinks=true, linkcolor=blue, citecolor=blue, filecolor=blue, urlcolor=blue, pdftitle=1, pdfauthor=, pdfsubject=, pdfkeywords=}

\newcommand{\attribution}[1] {
    \begin{flushright}\begin{scriptsize}\textcolor{gray}{#1}\end{scriptsize}\end{flushright}
}

\title{Синтаксический анализ на F\#}
\subtitle{Часть 2: FParsec vs FsLex/FsYacc}
\author{Юрий Литвинов}
\date{08.05.2020г}

\begin{document}
    
    \frame{\titlepage}

    \subsection{FParsec}

    \begin{frame}[fragile]
        \frametitle{Пример}
        \begin{minted}{fsharp}
open FParsec

[<EntryPoint>]
let main argv =

    let result = "1.23" |> (run pfloat)

    match result with
    | Success(result, _, _) -> printfn "%f" result
    | Failure(e, _, _) -> printfn "%s" e

    0
        \end{minted}
    \end{frame}

    \begin{frame}[fragile]
        \frametitle{Комбинаторы конкатенации}
        \begin{minted}{fsharp}
val (>>.): Parser<'a,'u> -> Parser<'b,'u> -> Parser<'b,'u>
val (.>>): Parser<'a,'u> -> Parser<'b,'u> -> Parser<'a,'u>

let str s = pstring s
let floatBetweenBrackets = str "[" >>. pfloat .>> str "]"

val pstring: string -> Parser<string,'u>
        \end{minted}
    \end{frame}

    \begin{frame}[fragile]
        \frametitle{Что получилось}
        \begin{small}
            \begin{alertblock}{F\# Interactive}
                \begin{minted}{text}
> test floatBetweenBrackets "[1.0]";;
Success: 1.0

> test floatBetweenBrackets "[]";;
Failure: Error in Ln: 1 Col: 2
[]
 ^
Expecting: floating-point number

> test floatBetweenBrackets "[1.0";;
Failure: Error in Ln: 1 Col: 5
[1.0
    ^
Note: The error occurred at the end of the input stream.
Expecting: ']'
                \end{minted}
            \end{alertblock}
        \end{small}
    \end{frame}

    \begin{frame}[fragile]
        \frametitle{Свои комбинаторы}
        \begin{minted}{fsharp}
let betweenStrings s1 s2 p = str s1 >>. p .>> str s2

let floatBetweenBrackets = pfloat |> betweenStrings "[" "]"
let floatBetweenDoubleBrackets = pfloat |> betweenStrings "[[" "]]"
        \end{minted}

        или

        \begin{minted}{fsharp}
let between pBegin pEnd p  = pBegin >>. p .>> pEnd
let betweenStrings s1 s2 p = p |> between (str s1) (str s2)
        \end{minted}

        between --- библиотечный комбинатор, его определять не надо
    \end{frame}

    \begin{frame}[fragile]
        \frametitle{Разбор списков}
        Грамматика: ("[" float "]")*

        Парсер:
        \begin{alertblock}{F\# Interactive}
            \begin{minted}{text}
> test (many floatBetweenBrackets) "";;
Success: []
> test (many floatBetweenBrackets) "[1.0]";;
Success: [1.0]
> test (many floatBetweenBrackets) "[2][3][4]";;
Success: [2.0; 3.0; 4.0]

> test (many floatBetweenBrackets) "[1][2.0E]";;
Failure: Error in Ln: 1 Col: 9
[1][2.0E]
        ^
Expecting: decimal digit
            \end{minted}
        \end{alertblock}
    \end{frame}

    \begin{frame}[fragile]
        \frametitle{Один или больше элементов}
        Грамматика: ("[" float "]")+

        Парсер:
        \begin{alertblock}{F\# Interactive}
            \begin{minted}{text}
> test (many1 floatBetweenBrackets) "(1)";;
Failure: Error in Ln: 1 Col: 1
(1)
^
Expecting: '['
            \end{minted}
        \end{alertblock}
    \end{frame}

    \begin{frame}[fragile]
        \frametitle{Обработка ошибок}
        \begin{alertblock}{F\# Interactive}
            \begin{minted}{text}
> test (many1 (floatBetweenBrackets 
                          <?> "float between brackets")) "(1)";;
Failure: Error in Ln: 1 Col: 1
(1)
^
Expecting: float between brackets
            \end{minted}
        \end{alertblock}
    \end{frame}

    \begin{frame}[fragile]
        \frametitle{Ещё пример}
        Грамматика: \verb|"[" (float ("," float)*)? "]"|
        
        Примеры: \verb|"[]", "[1.0]", "[2,3,4]"|

        Парсер:
        \begin{minted}{fsharp}
let floatList = str "[" >>. sepBy pfloat (str ",") .>> str "]"
        \end{minted}

        Что получилось:
        \begin{alertblock}{F\# Interactive}
            \begin{minted}{fsharp}
> test floatList "[]";;
Success: []
> test floatList "[1.0]";;
Success: [1.0]
> test floatList "[4,5,6]";;
Success: [4.0; 5.0; 6.0]
            \end{minted}
        \end{alertblock}
    \end{frame}

    \begin{frame}[fragile]
        \frametitle{Пробелы}
        Проблема:
        \begin{alertblock}{F\# Interactive}
            \begin{minted}{fsharp}
> test floatBetweenBrackets "[1.0, 2.0]";;
Failure: Error in Ln: 1 Col: 5
[1.0, 2.0]
       ^
Expecting: ']'
            \end{minted}
        \end{alertblock}
        Решение:
        \begin{minted}{fsharp}
let ws = spaces
let str_ws s = pstring s .>> ws
let float_ws = pfloat .>> ws
let numberList = str_ws "[" >>. sepBy float_ws (str_ws ",") .>> str_ws "]"
        \end{minted}

    \end{frame}

    \begin{frame}[fragile]
        \frametitle{Что получилось}
        \begin{alertblock}{F\# Interactive}
            \begin{minted}{fsharp}
> test numberList @"[ 1 ,
                          2 ] ";;
Success: [1.0; 2.0]

> test numberList @"[ 1,
                         2; 3]";;

Failure: Error in Ln: 2 Col: 27
                         2; 3]
                          ^
Expecting: ',' or ']'
            \end{minted}
        \end{alertblock}
    \end{frame}

    \begin{frame}[fragile]
        \frametitle{Парсинг строк}
        \begin{alertblock}{F\# Interactive}
            \begin{minted}{fsharp}
> test (many (str "a" <|> str "b")) "abba";;
Success: ["a"; "b"; "b"; "a"]

> test (skipStringCI "<float>" >>. pfloat) "<FLOAT>1.0";;
Success: 1.0
            \end{minted}
        \end{alertblock}
        \begin{minted}{fsharp}
let identifier =
    let isIdentifierFirstChar c = isLetter c || c = '_'
    let isIdentifierChar c = isLetter c || isDigit c || c = '_'

    many1Satisfy2L isIdentifierFirstChar isIdentifierChar "identifier"
    .>> ws // skips trailing whitespace
        \end{minted}
        Есть встроенный парсер identifier
    \end{frame}

    \begin{frame}[fragile]
        \frametitle{Строковые литералы}
        \begin{small}
            Грамматика:
            \begin{minted}{text}
stringLiteral: ' " ' (normalChar | escapedChar)* ' " '
normalChar:    any char except '\' and '"'
escapedChar:   '\\' ('\\' | ' " ' | 'n' | 'r' | 't')
            \end{minted}
            Парсер:
            \begin{minted}{fsharp}
let stringLiteral =
    let normalChar = satisfy (fun c -> c <> '\\' && c <> '"')
    let unescape c = match c with
                     | 'n' -> '\n'
                     | 'r' -> '\r'
                     | 't' -> '\t'
                     | c   -> c
    let escapedChar = pstring "\\" >>. (anyOf "\\nrt\"" |>> unescape)
    between (pstring "\"") (pstring "\"")
            (manyChars (normalChar <|> escapedChar))
            \end{minted}
        \end{small}
    \end{frame}

    \begin{frame}[fragile]
        \frametitle{Комбинирование результатов}
        \begin{minted}{fsharp}
val pipe2: Parser<'a,'u> -> Parser<'b,'u> 
                  -> ('a -> b -> 'c) -> Parser<'c,'u>

let product = pipe2 float_ws (str_ws "*" >>. float_ws)
                    (fun x y -> x * y)
        \end{minted}
        Что получилось:
        \begin{alertblock}{F\# Interactive}
            \begin{minted}{text}
> test product "3 * 5";;
Success: 15.0
            \end{minted}
        \end{alertblock}
    \end{frame}

    \begin{frame}[fragile]
        \frametitle{Ещё комбинаторы}
        \begin{minted}{fsharp}
type StringConstant = StringConstant of string * string

let stringConstant = pipe3 identifier (str_ws "=") stringLiteral
                           (fun id _ str -> StringConstant(id, str))
        \end{minted}
        Что получилось:
        \begin{alertblock}{F\# Interactive}
            \begin{minted}{fsharp}
> test stringConstant "myString = \"stringValue\"";;
Success: StringConstant ("myString","stringValue")
            \end{minted}
        \end{alertblock}

        \begin{minted}{fsharp}
fun tuple2 p1 p2 = pipe2 p1 p2 (fun x1 x2 -> (x1, x2))
        \end{minted}

        \begin{alertblock}{F\# Interactive}
            \begin{minted}{fsharp}
> test (float_ws .>>. (str_ws "," >>. float_ws)) "123, 456";;
Success: (123.0, 456.0)
            \end{minted}
        \end{alertblock}
    \end{frame}

    \begin{frame}[fragile]
        \frametitle{Разбор альтернатив}
        \begin{small}
            \begin{minted}{fsharp}
val (<|>): Parser<'a,'u> -> Parser<'a,'u> -> Parser<'a, 'u>

let boolean = (stringReturn "true"  true)
                  <|> (stringReturn "false" false)
            \end{minted}
            Что получилось:
            \begin{alertblock}{F\# Interactive}
                \begin{minted}{text}
> test boolean "false";;
Success: false
> test boolean "true";;
Success: true
> test boolean "tru";;
Failure: Error in Ln: 1 Col: 1
tru
^
Expecting: 'false' or 'true'
                \end{minted}
            \end{alertblock}
        \end{small}
    \end{frame}

    \begin{frame}[fragile]
        \frametitle{Важные особенности}
        \begin{small}
            \begin{itemize}
                \item \verb!<|>! применяет правую часть, только если левая пофэйлилась и не использовала ни один символ из входного потока
                \item \verb!<|>! не делает никакой предпросмотр
                \item \verb!<|>! не ищет самую длинную подходящую строку
            \end{itemize}
            Пример:
            \begin{alertblock}{F\# Interactive}
                \begin{minted}{text}
> run (pstring "a" <|> pstring "ab") "ab";;
val it : ParserResult<string,unit> = Success: "a"

> test ((ws >>. str "a") <|> (ws >>. str "b")) " b";;
Failure: Error in Ln: 1 Col: 2
 b
 ^
Expecting: 'a'
                \end{minted}
            \end{alertblock}
        \end{small}
    \end{frame}

    \begin{frame}[fragile]
        \frametitle{Как пофиксить}
        \begin{alertblock}{F\# Interactive}
            \begin{minted}{text}
> test (ws >>. (str "a" <|> str "b")) " b";;
Success: "b"
            \end{minted}
        \end{alertblock}
    \end{frame}

    \begin{frame}[fragile]
        \frametitle{Ещё комбинаторы}
        \begin{minted}{fsharp}
p1 <|> p2 <|> p3
choice [p1; p2; p3]

val attempt: Parser<'a,'u> -> Parser<'a,'u>
        \end{minted}
        \begin{alertblock}{F\# Interactive}
            \begin{minted}{text}
run ((attempt ab) <|> ac) "ac";;
val it : ParserResult<(string * string),unit> = Success: ("a", "c")
            \end{minted}
        \end{alertblock}
        Комбинаторы бэктрекинга:
        \begin{minted}{fsharp}
val (>>?):  Parser<'a,'u> -> Parser<'b,'u> -> Parser<'b,'u>
(.>>?): Parser<'a,'u> -> Parser<'b,'u> -> Parser<'a,'u>
val (.>>.?): Parser<'a,'u> -> Parser<'b,'u> -> Parser<('a * 'b),'u>
        \end{minted}
    \end{frame}

    \subsection{Парсер JSON}

    \begin{frame}[fragile]
        \frametitle{Большой пример: парсер JSON}
        \begin{itemize}
            \item Грамматика: \url{https://www.json.org/json-en.html}
            \item Пример:
                \begin{minted}{json}
{
    "employee":{ "name":"John", "age":30, "city":"New York" }
    "array": [ "John", "Anna", "Peter" ]
}
                \end{minted}
                \attribution{\url{https://www.w3schools.com/js/js_json_datatypes.asp}}
            \item AST:
                \begin{minted}{fsharp}
type Json = JString of string
          | JNumber of float
          | JBool   of bool
          | JNull
          | JList   of Json list
          | JObject of Map<string, Json>
                \end{minted}
        \end{itemize}
    \end{frame}

    \begin{frame}[fragile]
        \frametitle{Элементарные типы}
        \begin{minted}{fsharp}
let jnull  = stringReturn "null" JNull
let jtrue  = stringReturn "true"  (JBool true)
let jfalse = stringReturn "false" (JBool false)
let jnumber = pfloat |>> JNumber
        \end{minted}
    \end{frame}

    \begin{frame}[fragile]
        \frametitle{Строки}
        \begin{tiny}
            \begin{minted}{fsharp}
let str s = pstring s

let stringLiteral =
    let escape =  anyOf "\"\\/bfnrt"
                  |>> function
                      | 'b' -> "\b"
                      | 'f' -> "\u000C"
                      | 'n' -> "\n"
                      | 'r' -> "\r"
                      | 't' -> "\t"
                      | c   -> string c // every other char is mapped to itself

    let unicodeEscape =
        /// converts a hex char ([0-9a-fA-F]) to its integer number (0-15)
        let hex2int c = (int c &&& 15) + (int c >>> 6)*9

        str "u" >>. pipe4 hex hex hex hex (fun h3 h2 h1 h0 ->
            (hex2int h3)*4096 + (hex2int h2)*256 + (hex2int h1)*16 + hex2int h0
            |> char |> string
        )

    let escapedCharSnippet = str "\\" >>. (escape <|> unicodeEscape)
    let normalCharSnippet  = manySatisfy (fun c -> c <> '"' && c <> '\\')

    between (str "\"") (str "\"")
            (stringsSepBy normalCharSnippet escapedCharSnippet)

let jstring = stringLiteral |>> JString
            \end{minted}
        \end{tiny}
    \end{frame}

    \begin{frame}[fragile]
        \frametitle{Рекурсия}
        \begin{minted}{fsharp}
let jvalue, jvalueRef = createParserForwardedToRef<Json, unit>()
        \end{minted}
        \begin{itemize}
            \item jvalue --- парсер, который просто вызывает парсер из ref-ячейки jvalueRef
            \item Изначально там парсер, который ничего не делает
            \item Поскольку ref-ячейка мутабельна, присвоим туда настоящий парсер позже
        \end{itemize}
    \end{frame}

    \begin{frame}[fragile]
        \frametitle{Списки и объекты}
        \begin{minted}{fsharp}
let ws = spaces
let listBetweenStrings sOpen sClose pElement f =
    between (str sOpen) (str sClose)
            (ws >>. sepBy (pElement .>> ws) (str "," >>. ws) |>> f)
let jlist   = listBetweenStrings "[" "]" jvalue JList

let keyValue = stringLiteral .>>. (ws >>. str ":" >>. ws >>. jvalue)
let jobject = listBetweenStrings "{" "}" keyValue (Map.ofList >> JObject)
        \end{minted}
    \end{frame}

    \begin{frame}[fragile]
        \frametitle{Соберём всё воедино}
        \begin{minted}{fsharp}
do jvalueRef := choice [jobject
                        jlist
                        jstring
                        jnumber
                        jtrue
                        jfalse
                        jnull]

let json = ws >>. jvalue .>> ws .>> eof
        \end{minted}
    \end{frame}

    \subsection{Заключение}

    \begin{frame}
        \frametitle{Заключение}
        \begin{small}
            \begin{itemize}
                \item Подробности: \url{http://www.quanttec.com/fparsec/tutorial.html}
                \item Ещё большие подробности: \url{http://www.quanttec.com/fparsec/users-guide/}
                \item И ещё большие подробности: \url{http://www.quanttec.com/fparsec/reference/}
                \item Осталось за бортом:
                    \begin{itemize}
                        \item FsLex/FsYacc --- неидиоматичный, но более ``взрослый'' генератор парсеров
                            \begin{itemize}
                                \item \url{https://fsprojects.github.io/FsLexYacc/}
                            \end{itemize}
                        \item ANTLR --- стандарт де-факто в серьёзном синтаксическом анализе
                            \begin{itemize}
                                \item \url{https://www.antlr.org/}
                                \item Не поддерживает F\#, но с C\# всё ок
                            \end{itemize}
                        \item YaccConstructor --- мощная библиотека и генератор парсеров, для исследовательских целей
                            \begin{itemize}
                                \item \url{https://github.com/YaccConstructor/YaccConstructor}
                                \item Написан на F\#
                                \item Разрабатывается на матмехе
                            \end{itemize}
                    \end{itemize}
            \end{itemize}
        \end{small}
    \end{frame}

\end{document}
