\documentclass[a5paper]{article}
\usepackage[a5paper, top=8mm, bottom=8mm, left=8mm, right=8mm]{geometry}

\usepackage{polyglossia}
\setdefaultlanguage[babelshorthands=true]{russian}

\usepackage{fontspec}
\setmainfont{FreeSerif}
\newfontfamily{\russianfonttt}[Scale=0.7]{DejaVuSansMono}

\usepackage[font=scriptsize]{caption}

\usepackage{amsmath}
\usepackage{amssymb,amsfonts,textcomp}
\usepackage{color}
\usepackage{array}
\usepackage{hhline}
\usepackage{cite}

\usepackage[hang,multiple]{footmisc}
\renewcommand{\footnotelayout}{\raggedright}

\PassOptionsToPackage{hyphens}{url}\usepackage[xetex,linktocpage=true,plainpages=false,pdfpagelabels=false]{hyperref}
\hypersetup{colorlinks=true, linkcolor=blue, citecolor=blue, filecolor=blue, urlcolor=blue, pdftitle=1, pdfauthor=, pdfsubject=, pdfkeywords=}

\usepackage{tabu}

\usepackage{graphicx}
\usepackage{indentfirst}
\usepackage{multirow}
\usepackage{subfig}
\usepackage{footnote}
\usepackage{minted}
\usepackage{xcolor}

\newcommand{\attribution}[1] {
    \vspace{-5mm}\begin{flushright}\begin{scriptsize}\textcolor{gray}{\textcopyright\, #1}\end{scriptsize}\end{flushright}
}

\sloppy
\pagestyle{plain}

\title{Практика 3: моделирование структуры}
\author{Юрий Литвинов\\\small{yurii.litvinov@gmail.com}}

\date{21.02.2022}

\begin{document}

\maketitle
\thispagestyle{empty}

\section{Задание на практику}

Проанализировать запрос \url{https://bit.ly/defects-rfp}, подумать над тем, как бы вы стали делать такую систему, и построить по нему:

\begin{enumerate}
    \item диаграмму компонентов требуемой системы, как вы её видите;
    \item диаграммы классов --- по одной на каждый компонент;
    \begin{itemize}
        \item надо только основные классы, технические детали не интересны;
        \item не забудьте про связь с другими компонентами --- рисуйте классы из соседних компонентов с квалифицированными именами и без атрибутов/операций.
    \end{itemize}
\end{enumerate}

Сдавать нужно в свои репозитории отдельным пуллреквестом, либо исходник диаграммы, либо .md-файл со ссылкой на проект в какой-либо облачной рисовалке (в этом случае не забудьте расшарить).

\end{document}