\documentclass{../cscslides}

\slidetitle{Практика 6: Архитектурная документация}{21.03.2022}

\begin{document}
    
    \frame{\titlepage}

    \section{Design Document}

    \begin{frame}
        \frametitle{Design Document, что это и зачем}
        \begin{itemize}
            \item Основной продукт работы архитектора
            \item Представляет и объясняет основные принятые архитектурные решения
            \begin{itemize}
                \item НЕ набор UML-диаграмм
            \end{itemize}
            \item Решение проблемы ``Architecture By Implication''
            \item Наличие хорошего диздока может сократить затраты на кодирование 
            \begin{itemize}
                \item в разы
            \end{itemize}
        \end{itemize}
    \end{frame}

    \begin{frame}
        \frametitle{Design Document, что это и зачем}
        \begin{itemize}
            \item В индустриальной практике он часто неформально, но обязательно присутствует --- как правило, это набор вики-страниц, иногда встречаются формальные документы
            \item Чеклист, позволяющий проверить, что обо всём подумали и всё служит какой-то осмысленной цели
            \item Не путать с Game Design Document (хотя он служит тем же целям и очень похож по структуре)
        \end{itemize}
    \end{frame}

    \begin{frame}
        \frametitle{Как писать}
        \begin{itemize}
            \item Достаточно подробно, чтобы при программировании не требовалось принимать важных архитектурных решений
            \item Разные \textit{точки зрения}, предназначенные для разных аудиторий
            \begin{itemize}
                \item Даже для одной целевой аудитории используется несколько точек зрения, например, статическая структура, поведение, схема БД и требования
            \end{itemize}
            \item Рекомендуется использовать диаграммы для иллюстрации архитектуры
        \end{itemize}
    \end{frame}

    \begin{frame}
        \frametitle{Как писать}
        \begin{itemize}
            \item Должен документировать не только принятые решения, но и:
            \begin{itemize}
                \item Альтернативы
                \begin{itemize}
                    \item Чётко формулировать, что в итоге решили
                \end{itemize}
                \item Преимущества принятого решения
                \item Риски
                \item Связь с требованиями
            \end{itemize}
            \item Должны быть \textit{полнота} и \textit{консистентность}
            \item Стандарты IEEE 1016-2009 и ISO/IEC/IEEE 42010:2011 (он же ГОСТ Р 57100-2016)
        \end{itemize}
    \end{frame}

    \section{IEEE 42010:2011}

    \begin{frame}
        \frametitle{Подробнее, IEEE 42010:2011}
        \framesubtitle{Architecture description}
        \begin{itemize}
            \item Стандартизует не формат документа, а концепции и общие требования
            \item Делегирует конкретные архитектурные описания архитектурным фреймворкам
            \item В каком-то смысле ``метастандарт'' --- декларирует требования к фреймворкам
            \item Тем не менее, содержит ряд полезных мыслей
        \end{itemize}
    \end{frame}

    \begin{frame}
        \frametitle{IEEE 42010:2011, контекст системы}
        \begin{center}
            \includegraphics[width=0.9\textwidth]{ieee42010ContextBlack.png}
        \end{center}
    \end{frame}

    \begin{frame}
        \frametitle{IEEE 42010:2011, архитектурное описание}
        \begin{center}
            \includegraphics[width=0.7\textwidth]{ieee42010ArchitectureDescriptionBlack.png}
        \end{center}
    \end{frame}

    \begin{frame}
        \frametitle{IEEE 42010:2011, архитектурный фреймворк}
        \begin{center}
            \includegraphics[width=0.9\textwidth]{ieee42010ArchitectureFrameworkBlack.png}
        \end{center}
    \end{frame}

    \begin{frame}
        \frametitle{IEEE 42010:2011, архитектурное решение}
        \begin{center}
            \includegraphics[width=0.9\textwidth]{ieee42010DecisionBlack.png}
        \end{center}
    \end{frame}

    \begin{frame}
        \frametitle{IEEE 42010:2011, требования к документации}
        \begin{itemize}
            \item Общая информация о документе и о системе
            \item Стейкхолдеры и их интересы
                \begin{itemize}
                    \item пользователи, операторы, приобретатели, владельцы, поставщики, разработчики, строители, сопровождающие
                    \item назначение, соответствие архитектуры решаемым задачам, выполнимость разработки и развёртывания, риски и влияние системы на стейкхолдеров, способность к эволюции
                \end{itemize}
            \item Определение Viewpoint-ов
            \item Архитектурные виды
            \item Отношения между элементами архитектуры
            \item Обоснование архитектуры
        \end{itemize}
    \end{frame}
    
    \section{IEEE 1016-2009}

    \begin{frame}
        \frametitle{IEEE 1016-2009}
        \begin{itemize}
            \item Не действует с 2020 года
            \item Схож по содержанию с IEEE 42010:2011
            \item Более конкретный
                \begin{itemize}
                    \item Определяет структуру документа
                    \item Определяет архитектурные виды (то есть может считаться примером архитектурного фреймворка)
                    \item Чем нам и интересен
                \end{itemize}
        \end{itemize}
    \end{frame}

    \begin{frame}
        \frametitle{IEEE 1016-2009, содержание документа}
        \begin{itemize}
            \item Различная служебная информация
            \item Общие сведения о системе (несколько абзацев)
            \begin{itemize}
                \item Назначение
                \item Границы системы (Scope)
                \item Контекст, в котором существует система
            \end{itemize}
            \item Architectural drivers
        \end{itemize}
    \end{frame}

    \begin{frame}
        \frametitle{Architectural drivers}
        \begin{itemize}
            \item Architectural drivers --- ключевые требования, определяющие архитектуру
            \item Бывают:
            \begin{itemize}
                \item Технические ограничения
                \item Бизнес-ограничения
                \item Качественные характеристики системы
                \begin{itemize}
                    \item Сопровождаемость, расширяемость и т.д.
                    \item Масштабируемость, производительность
                    \item Безопасность
                \end{itemize}
                \item Ключевые функциональные требования
            \end{itemize}
        \end{itemize}
    \end{frame}

    \begin{frame}
        \frametitle{IEEE 1016-2009, содержание документа (2)}
        \begin{itemize}
            \item Views (каждый из которых --- экземпляр Viewpoint-а)
            \begin{itemize}
                \item Требования, роли и случаи использования
                \item Структура системы
                \item Поведение системы
                \item Структура данных
                \item ...
            \end{itemize}
            \item Причины принятых решений, за/против
            \begin{itemize}
                \item Эта информация обычно приводится и во viewpoint-ах, тут summary
            \end{itemize}
        \end{itemize}
    \end{frame}

    \begin{frame}
        \frametitle{IEEE 1016-2009, концепции документа}
        \begin{center}
            \includegraphics[width=0.8\textwidth]{ieee1016ConceptsBlack.png}
        \end{center}
    \end{frame}

    \begin{frame}
        \frametitle{IEEE 1016-2009, элементы архитектуры}
        \begin{center}
            \includegraphics[width=0.8\textwidth]{ieee1016ArchitectureElementsBlack.png}
        \end{center}
    \end{frame}

    \begin{frame}
        \frametitle{IEEE 1016-2009, точки зрения}
        \begin{itemize}
            \item Всего выделено 12 точек зрения
            \begin{itemize}
                \item Контекст
                \item Композиция
                \item Логическая структура
                \item Зависимости
                \item Информационная структура
                \item Использование шаблонов
                \item Интерфейсы
                \item Структура системы
                \item Взаимодействия
                \item Динамика состояний
                \item Алгоритмы
                \item Ресурсы
            \end{itemize}
            \item Все точки зрения в документе не обязательны
            \begin{itemize}
                \item Тем не менее, есть требование полноты
            \end{itemize}
            \item Есть ещё overlays --- виды с дополнительной информацией
        \end{itemize}
    \end{frame}

    \begin{frame}
        \frametitle{Контекст системы}
        \begin{itemize}
            \item Назначение --- описывает, что система должна делать, фиксирует окружение системы. Состоит из сервисов и акторов, которые могут быть связаны информационными потоками. Система представляет собой ``чёрный ящик''
            \begin{itemize}
                \item Может быть определён Deployment overlay
                \begin{itemize}
                    \item Может быть отдельным видом, если аппаратное обеспечение --- часть разработки
                \end{itemize}
            \end{itemize}
            \item Соображения --- функциональные требования, роли, границы системы
            \begin{itemize}
                \item Корень иерархии уточняющих дизайн системы видов, стартовая точка при проектировании системы
            \end{itemize}
            \item Типичные языки --- диаграмма активностей UML, IDEF0 (SADT)
        \end{itemize}
    \end{frame}

    \begin{frame}
        \frametitle{Композиция}
        \begin{itemize}
            \item Назначение --- на самом деле, ``декомпозиция'', описывает крупные части системы и их предназначение
            \item Соображения --- локализация и распределение функциональности системы по её структурным элементам, impact analysis, переиспользование (в том числе, покупка компонентов), оценка, планирование, управление проектом, инструментальная поддержка (репозитории, трекер и т.д.)
            \item Типичные языки --- диаграммы компонентов UML, IDEF0
        \end{itemize}
    \end{frame}

    \begin{frame}
        \frametitle{Логическая структура}
        \begin{itemize}
            \item Назначение --- структура системы в терминах классов, интерфейсов и отношений между ними
            \begin{itemize}
                \item Используются также примеры экземпляров классов для пояснения решений
            \end{itemize}
            \item Соображения --- разработка и переиспользование
            \begin{itemize}
                \item Разделение на то, что можно взять и приспособить, и то, что придётся написать
            \end{itemize}
            \item Типичные языки --- диаграммы классов UML, диаграммы объектов UML
        \end{itemize}
    \end{frame}

    \begin{frame}
        \frametitle{Зависимости}
        \begin{itemize}
            \item Назначение --- определяет связи по данным между элементами
            \begin{itemize}
                \item Разделяемая между элементами информация, порядок выполнения и т.д.
            \end{itemize}
            \item Соображения --- анализ изменений, идентификация узких мест производительности, планирование, интеграционное тестирование
            \item Типичные языки --- диаграммы компонентов UML, диаграммы пакетов UML
        \end{itemize}
    \end{frame}

    \begin{frame}
        \frametitle{Информационная структура}
        \begin{itemize}
            \item Назначение --- определяет персистентные данные в системе
            \item Соображения --- информация, которую требуется хранить, схема БД, доступ к данным
            \item Типичные языки --- диаграммы классов UML, IDEF1x, ER, ORM
        \end{itemize}
    \end{frame}

    \begin{frame}
        \frametitle{Использование шаблонов}
        \begin{itemize}
            \item Назначение --- документирование использования локальных паттернов проектирования
            \item Соображения --- переиспользование на уровне идей и архитектурных стилей
            \item Типичные языки --- диаграммы классов UML, диаграммы пакетов UML, диаграммы коллабораций UML
        \end{itemize}
    \end{frame}

    \begin{frame}
        \frametitle{Интерфейсы}
        \begin{itemize}
            \item Назначение --- специфицирует информацию о внешних и внутренних интерфейсах, не прописанную явно в требованиях
            \begin{itemize}
                \item Пользовательский интерфейс рассматривается отдельным видом в рамках этой точки зрения
            \end{itemize}
            \item Соображения --- договорённости о конкретных схемах взаимодействия компонентов, позволяющие разрабатывать и тестировать их независимо
            \item Типичные языки --- IDL, диаграммы компонентов UML, макеты пользовательского интерфейса, неформальные описания сценариев использования
        \end{itemize}
    \end{frame}

    \begin{frame}
        \frametitle{Структура системы}
        \begin{itemize}
            \item Назначение --- рекурсивное описание внутренней структуры компонентов системы
            \item Соображения --- структура системы, переиспользование
            \item Типичные языки --- диаграммы композитных структур UML, диаграммы классов UML, диаграммы пакетов UML
        \end{itemize}
    \end{frame}

    \begin{frame}
        \frametitle{Взаимодействия}
        \begin{itemize}
            \item Назначение --- описывает взаимодействие между сущностями: почему когда, как и на каком уровне выполняется взаимодействие
            \item Соображения --- распределение ответственностей между участниками взаимодействия, определение протоколов взаимодействия
            \item Типичные языки --- диаграммы композитных структур UML, диаграммы взаимодействия UML, диаграммы последовательностей UML
        \end{itemize}
    \end{frame}

    \begin{frame}
        \frametitle{Динамика состояний}
        \begin{itemize}
            \item Назначение --- описание состояний и правил переходов между состояниями в реактивных системах
            \item Соображения --- поведение системы, включая внутренние состояния, события и логику переходов
            \item Типичные языки --- диаграммы конечных автоматов UML, диаграммы Харела, сети Петри
        \end{itemize}
    \end{frame}

    \begin{frame}
        \frametitle{Алгоритмы}
        \begin{itemize}
            \item Назначение --- описывает в деталях поведение каждой сущности, логику работы методов
            \item Соображения --- анализ эффективности работы программы, реализация, юнит-тестирование
            \item Типичные языки --- диаграммы активностей UML, псевдокод, настоящие языки программирования
        \end{itemize}
    \end{frame}

    \begin{frame}
        \frametitle{Ресурсы}
        \begin{itemize}
            \item Назначение --- описывает использование внешних ресурсов (как правило, аппаратных или третьесторонних сервисов)
            \item Соображения --- эффективность работы программы, доступность и эффективность использования ресурсов
            \item Типичные языки --- диаграммы развёртывания UML, диаграммы классов UML, OCL
        \end{itemize}
    \end{frame}

    \begin{frame}
        \frametitle{Примеры}
        \begin{itemize}
            \item Формальные документы:
            \begin{itemize}
                \item \url{http://robotics.ee.uwa.edu.au/courses/design/examples/example_design.pdf}
                \item \url{https://arxiv.org/ftp/arxiv/papers/1005/1005.0595.pdf}
            \end{itemize}
            \item Неформальные документы:
            \begin{itemize}
                \item \url{https://github.com/dotnet/efcore/blob/master/docs/security.md}
                \item \url{https://github.com/fsharp/fslang-suggestions/}
            \end{itemize}
        \end{itemize}
    \end{frame}

    \section{Домашнее задание}

    \begin{frame}
        \frametitle{Домашнее задание: Roguelike}
        \begin{itemize}
            \item Жанр компьютерных игр, назван в честь игры Rogue, 1980 года выхода
            \item Характеризуется:
            \begin{itemize}
                \item Простой тайловой или консольной графикой
                \item Активным использованием случайной генерации
                \item Перманентной смертью персонажа и невозможностью загрузить предыдущее сохранение
                \item Чрезвычайно развитым набором игровых правил
                \item Высокой свободой действий персонажа (``игры-песочницы'')
            \end{itemize}
            \item Примеры:
            \begin{itemize}
                \item \url{https://en.wikipedia.org/wiki/NetHack}
                \item \url{https://en.wikipedia.org/wiki/Angband_(video_game)}
                \item \url{https://en.wikipedia.org/wiki/Ancient_Domains_of_Mystery}
            \end{itemize}
        \end{itemize}
    \end{frame}

    \begin{frame}
        \frametitle{Функциональные требования}
        \begin{itemize}
            \item Персонаж игрока, способный перемещаться по карте, управляемый с клавиатуры
            \begin{itemize}
                \item Карта обычно генерируется, но для некоторых уровней грузится из файла
                \item Характеристики --- здоровье, сила атаки и т.д.
            \end{itemize}
            \item У персонажа есть инвентарь, состоящий из вещей, которые он носит с собой
            \begin{itemize}
                \item Вещи из инвентаря можно надеть и снять, надетые вещи влияют на характеристики персонажа
                \item Вещи изначально находятся на карте, их можно поднять, чтобы добавить в инвентарь
                \item Снятые вещи находятся в инвентаре, их можно надеть в дальнейшем
            \end{itemize}
            \item Консольная графика
        \end{itemize}
    \end{frame}

    \begin{frame}
        \frametitle{Домашнее задание}
        Разделиться на команды по два-три человека и написать архитектурное описание Roguelike
        \begin{itemize}
            \item Общие сведения о системе
            \item Architectural drivers
            \item Роли и случаи использования
            \begin{itemize}
                \item Описание типичного пользователя
            \end{itemize}
            \item Композиция (диаграмма компонентов)
            \item Логическая структура (диаграмма классов)
            \item Взаимодействия и состояния (диаграммы последовательностей и конечных автоматов)
        \end{itemize}
    \end{frame}

\end{document}