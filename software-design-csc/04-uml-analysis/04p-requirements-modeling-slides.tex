\documentclass[xetex,mathserif,serif]{beamer}
\usepackage{polyglossia}
\setdefaultlanguage[babelshorthands=true]{russian}
\usepackage{minted}
\usepackage{tabu}

\useoutertheme{infolines}

\usepackage{fontspec}
\setmainfont{FreeSans}
\newfontfamily{\russianfonttt}{FreeSans}

\definecolor{links}{HTML}{2A1B81}
\hypersetup{colorlinks,linkcolor=,urlcolor=links}

\tabulinesep=0.7mm

\newcommand{\attribution}[1] {
    \vspace{-5mm}\begin{flushright}\begin{scriptsize}\textcolor{gray}{\textcopyright\, #1}\end{scriptsize}\end{flushright}
}

\title{Практика 4: Моделирование требований}
\author[Юрий Литвинов]{Юрий Литвинов \newline \textcolor{gray}{\small\texttt{yurii.litvinov@gmail.com}}}

\date{28.02.2022}

\begin{document}
    
    \frame{\titlepage}

    \section{Задачи на практику}

    \begin{frame}
        \frametitle{Задачи на практику}
        Вспомнить запрос \url{https://bit.ly/defects-rfp}, построить по нему:
        \begin{enumerate}
            \item диаграмму случаев использования, описывающую пользователей и случаи использования разрабатываемого приложения
            \item диаграмму активностей для основного бизнес-процесса, поддерживаемого приложением --- регистрации и ремонта дефекта
            % \item BPMN-диаграмму для всего бизнес-процесса завода, включая внешних его участников
            % Если работа будет командной, включить предыдущую строку, но с комментарием, где рисовать BPMN-диаграммы.
        \end{enumerate}
        Сдавать в свои репозитории отдельным пуллреквестом.
    \end{frame}

\end{document}
