\documentclass{../../slides-style}

\slidetitle{Коллекции стандартной библиотеки, LINQ}{22.04.2025}

\begin{document}

    \begin{frame}[plain]
        \titlepage
    \end{frame}

    \section{Контейнеры стандартной библиотеки .NET}

    % По https://learn.microsoft.com/en-us/dotnet/standard/collections/

    \begin{frame}{Контейнеры стандартной библиотеки}
        \begin{itemize}
            \item System.Array --- массив, встроен в язык
            \item System.Collections --- негенериковые коллекции
            \item System.Collections.Generic --- генериковые коллекции
            \item System.Collections.Concurrent --- потокобезопасные коллекции
            \item System.Collections.Immutable --- немутабельные коллекции
        \end{itemize}
    \end{frame}

    \begin{frame}{Архитектурные соглашения в стандартной библиотеке}
        \begin{itemize}
            \item Любая коллекция возвращает энумератор
            \begin{itemize}
                \item Есть \mintinline{csharp}{IEnumerable} и \mintinline{csharp}{IEnumerable<T>}, второй наследуется от первого
                \item С \mintinline{csharp}{IEnumerable<T>} работает LINQ
            \end{itemize}
            \item Все коллекции реализуют CopyTo --- копирование в массив
            \item Есть Capacity и Count, но не у всех коллекций
            \item Все индексированные коллекции начинают индексы с нуля
            \item Все коллекции из System.Collections и System.Collections.Generic не потокобезопасны
            \item Все коллекции вида \enquote{ключ-значение} ожидают, что ключи не будут меняться
            \begin{itemize}
                \item Но не могут это проверить
            \end{itemize}
        \end{itemize}
    \end{frame}

    \begin{frame}{Какие коллекции вообще бывают}
        \begin{small}
            \begin{itemize}
                \item Dictionary<TKey,TValue> --- словарь на хеш-таблицах
                \item HashSet<T> --- множество на хеш-таблицах, умеет операции
                \item LinkedList<T> --- двусвязный список
                \begin{itemize}
                    \item LinkedListNode<T>
                \end{itemize}
                \item List<T> --- список на массивах
                \item PriorityQueue<TElement,TPriority> --- очередь на куче
                \begin{itemize}
                    \item \enquote{array-backed quaternary min-heap}
                \end{itemize}
                \item Queue<T> --- очередь на круговых массивах
                \item SortedDictionary<TKey,TValue> --- словарь на красно-чёрных деревьях
                \item SortedList<TKey,TValue> --- последовательность пар (или словарь) на сортированных списках с двоичным поиском
                \item SortedSet<T> --- множество на красно-чёрных деревьях
                \item Stack<T> --- стек на массивах
            \end{itemize}
        \end{small}
    \end{frame}

    \begin{frame}{Алгоритмическая сложность}
        \begin{footnotesize}
            \begin{tabu} {| X[1 l p] | X[1 l p] | X[1 l p] |}
                \tabucline-
                Операция                 & Амортизированная трудоёмкость  & Худший случай  \\
                \tabucline-
                \everyrow{\tabucline-}
                Stack<T>.Push            & O(1)                           & O(n)           \\
                Queue<T>.Enqueue         & O(1)                           & O(n)           \\
                List<T>.Add              & O(1)                           & O(n)           \\
                List<T>.Item[Int32]      & O(1)                           & O(1)           \\
                List<T>.Enumerator       & O(n)                           & O(n)           \\
                HashSet<T>.Add, поиск    & O(1)                           & O(n)           \\
                SortedSet<T>.Add         & O(log n)                       & O(n)           \\
                Dictionary<T>.Add        & O(1)                           & O(n)           \\
                поиск в Dictionary<T>    & O(1)                           & O(1) или O(n)  \\
                SortedDictionary<T>.Add  & O(log n)                       & O(n log n)     \\
            \end{tabu}
        \end{footnotesize}
    \end{frame}

    \begin{frame}{Вспомогательные интерфейсы и методы}
        \begin{itemize}
            \item Object.GetHashCode используется всеми коллекциями на хеш-таблицах
            \begin{itemize}
                \item Обязан возвращать равные значения для равных (в терминах Equals) объектов
                \item Хеш-код нельзя хранить между запусками приложения или передавать
            \end{itemize}
            \item IEqualityComparer<T> --- интерфейс компаратора со встроенным GetHashCode \enquote{вне} объекта
            \begin{itemize}
                \item Коллекции на хеш-таблицах могут принимать его в конструктор, тогда \enquote{родной} GetHashCode игнорируется
                \item EqualityComparer<T> и его метод Create для быстрой реализации
            \end{itemize}
            \item IComparer<T> --- то же для сортированных коллекций
            \begin{itemize}
                \item Сравнения уважают культуру
                \begin{itemize}
                    \item См. CultureInfo и InvariantCulture
                \end{itemize}
            \end{itemize}
        \end{itemize}
    \end{frame}

    \begin{frame}{Вспомогательные интерфейсы и методы (2)}
        \begin{itemize}
            \item IEquatable<T> --- определяет равенство \enquote{изнутри} элемента
            \begin{itemize}
                \item Используется в Contains, IndexOf и т.п.
                \item Если нет, используется Object.Equals
            \end{itemize}
            \item IComparable<T> --- то же для сравнения объектов
            \begin{itemize}
                \item Есть негенериковая версия, IComparable
                \item Если не реализован, компаратор надо передавать явно
            \end{itemize}
            \item IReadOnlyCollection<T> и его потомки --- часть ICollection<T> без методов изменения
            \begin{itemize}
                \item Никто не мешает коллекцию, которую передали по этому интерфейсу, менять извне
                \item AsReadOnly<T> в CollectionExtensions
            \end{itemize}
        \end{itemize}
    \end{frame}

    \begin{frame}{Как писать свои коллекции}
        \begin{itemize}
            \item Не делайте этого, всё написано до вас
            \item Collection<T> --- базовый класс для реализации своих архитектурно правильных коллекций
            \begin{itemize}
                \item Переопределением protected-методов InsertItem, RemoveItem, ClearItems, SetItem
                \item Внутри по умолчанию List<T>, можно использовать \enquote{из коробки}
            \end{itemize}
        \end{itemize}
    \end{frame}

    \section{LINQ}

    \begin{frame}{Language Integrated Query}
        \begin{itemize}
            \item LINQ --- набор методов-расширений для операций над разными структурами данных
            \item Прежде всего для работы с базами данных, но хорошо работает и с коллекциями (LINQ to Objects)
            \begin{itemize}
                \item Предназначался прежде всего, чтобы запросы к БД не надо было писать строками в коде
            \end{itemize}
            \item Имеет свой набор ключевых слов, делающих запрос похожим на SQL
            \item Архитектурно интересно устроен
            \begin{itemize}
                \item Запрос не исполняется, пока не нужен результат (ленивость)
                \item Запрос может транслироваться перед исполнением (например, в SQL)
            \end{itemize}
            \item Живёт в System.Linq
            \item В Java ближайший аналог (но не такой навороченный) --- Stream API
        \end{itemize}
    \end{frame}

    \begin{frame}[fragile]{Пример}
        \begin{minted}{csharp}
string sentence = "the quick brown fox jumps over the lazy dog";
string[] words = sentence.Split(' ');

var query = words.
    GroupBy(w => w.Length, w => w.ToUpper()).
    Select(g => new { Length = g.Key, Words = g }).
    OrderBy(o => o.Length);

foreach (var obj in query)
{
    Console.WriteLine($"Words of length {obj.Length}:");
    foreach (string word in obj.Words)
        Console.WriteLine(word);
}
        \end{minted}
        \attribution{\url{https://learn.microsoft.com/en-us/dotnet/csharp/linq/standard-query-operators/}}
    \end{frame}

    \begin{frame}[fragile]{Или то же самое на встроенном языке запросов}
        \begin{minted}{csharp}
string sentence = "the quick brown fox jumps over the lazy dog";
string[] words = sentence.Split(' ');

var query = from word in words
    group word.ToUpper() by word.Length into gr
    orderby gr.Key
    select new { Length = gr.Key, Words = gr };

foreach (var obj in query)
{
    Console.WriteLine($"Words of length {obj.Length}:");
    foreach (string word in obj.Words)
        Console.WriteLine(word);
}
        \end{minted}
        \attribution{\url{https://learn.microsoft.com/en-us/dotnet/csharp/linq/standard-query-operators/}}
    \end{frame}

    \begin{frame}{Выборка и преобразование}
        \begin{footnotesize}
            \begin{tabu} {| X[1 l p] | X[3 l p] |}
                \tabucline-
                Операция                 & Описание  \\
                \tabucline-
                \everyrow{\tabucline-}
                Where       & Возвращает коллекцию, элементы которой удовлетворяют предикату (filter) \\
                Select      & Возвращает коллекцию, полученную применением лямбда-функции к каждому элементу исходной (map) \\
                SelectMany  & Возвращает коллекцию, полученную применением лямбда-функции, возвращающей коллекцию, к каждому элементу исходной, и склеивает результирующие коллекции в одну (collect) \\
                Zip         & Превращает две последовательности в последовательность пар (zip) \\
            \end{tabu}
        \end{footnotesize}
    \end{frame}

    \begin{frame}[fragile]{SelectMany, пример}
        \begin{minted}{csharp}
List<string> phrases = ["an apple a day", "the quick brown fox"];

var query = phrases.SelectMany(phrases => phrases.Split(' '));

foreach (string s in query)
{
    Console.WriteLine(s);
}
        \end{minted}
        \attribution{\ssmall\url{https://learn.microsoft.com/en-us/dotnet/csharp/linq/standard-query-operators/projection-operations}}
    \end{frame}

    \begin{frame}[fragile]{Zip, пример}
        \begin{minted}{csharp}
IEnumerable<int> numbers = [1, 2, 3, 4, 5, 6, 7];
IEnumerable<char> letters = ['A', 'B', 'C', 'D', 'E', 'F'];
IEnumerable<string> emoji = [ ":)", ":D", ";)", ":(", ":/", "=^_^="];

foreach ((int number, char letter, string em) 
    in numbers.Zip(letters, emoji))
{
    Console.WriteLine(
        $"Number: {number} is zipped with letter: '{letter}' and emoji: {em}");
}
        \end{minted}
        \attribution{\ssmall\url{https://learn.microsoft.com/en-us/dotnet/csharp/linq/standard-query-operators/projection-operations}}
    \end{frame}

    \begin{frame}{Операции над множествами}
        \begin{footnotesize}
            \begin{tabu} {| X[1 l p] | X[3 l p] |}
                \tabucline-
                Операция                 & Описание  \\
                \tabucline-
                \everyrow{\tabucline-}
                Distinct    & Возвращает коллекцию без дубликатов (с точки зрения Equals или переданного компаратора) \\
                DistinctBy  & Distinct, но можно указать лямбда-функцию --- \emph{проекцию}, которая должна возвращать ключ \\
                Except      & Возвращает элементы первой коллекции, которых нет во второй \\
                ExceptBy    & Except, но с проекцией, отображающей элементы первой коллекции во вторую \\
                Intersect   & Возвращает коллекцию элементов, которые встречаются в обеих исходных коллекциях \\
                IntersectBy & Intersect с проекцией \\
                Union       & Возвращает коллекцию элементов, которые встречаются в одной из исходных коллекций \\
                UnionBy     & Union с проекцией \\
            \end{tabu}
        \end{footnotesize}
    \end{frame}

    \begin{frame}[fragile]{DistinctBy, пример}
        \begin{minted}{csharp}
string[] words = ["the", "quick", "brown", "fox", "jumped", 
    "over", "the", "lazy", "dog"];

foreach (var word in words.DistinctBy(p => p.Length))
{
    Console.WriteLine(word);
}
        \end{minted}
        \attribution{\ssmall\url{https://learn.microsoft.com/en-us/dotnet/csharp/linq/standard-query-operators/set-operations}}
    \end{frame}

    \begin{frame}[fragile]{ExceptBy, пример}
        \begin{minted}{csharp}
string[] words = ["the", "quick", "brown", "fox", "jumped", 
    "over", "the", "lazy", "dog"];
int[] forbiddenLengths = [3, 4];

foreach (var word in words.ExceptBy(forbiddenLengths, 
    w => w.Length))
{
    Console.WriteLine(word);
}
        \end{minted}
    \end{frame}

    \begin{frame}[fragile]{IntersectBy, пример}
        \begin{minted}{csharp}
// Student — класс со свойствами FirstName и LastName
// Teacher — с аналогичными свойствами First и Last
// Найдём всех студентов, кто не является учителем
foreach (Student person in
    students.IntersectBy(
        teachers.Select(t => (t.First, t.Last)), 
            s => (s.FirstName, s.LastName)))
{
    Console.WriteLine($"{person.FirstName} {person.LastName}");
}
        \end{minted}
    \end{frame}

    \begin{frame}{Сортировка}
        \begin{footnotesize}
            \begin{tabu} {| X[1 l p] | X[3 l p] |}
                \tabucline-
                Операция                 & Описание  \\
                \tabucline-
                \everyrow{\tabucline-}
                OrderBy            & Сортировка по возрастанию по ключу                                                                                  \\
                OrderByDescending  & Сортировка по убыванию по ключу                                                                                     \\
                ThenBy             & Сортировка по вторичному признаку (возможно, потому что методы сортировки возвращают IOrderedEnumerable<TElement>)  \\
                ThenByDescending   & Сортировка по вторичному признаку по убыванию                                                                       \\
                Reverse            & Развернуть коллекцию                                                                                                \\
            \end{tabu}
        \end{footnotesize}
    \end{frame}

    \begin{frame}[fragile]{ThenBy, пример}
        \begin{minted}{csharp}
string[] fruits = { "grape", "passionfruit", "banana", "mango",
    "orange", "raspberry", "apple", "blueberry" };

// Сортируем по длине и затем равные по длине — по алфавиту 
IEnumerable<string> query =
    fruits.OrderBy(fruit => fruit.Length).ThenBy(fruit => fruit);

foreach (string fruit in query)
{
    Console.WriteLine(fruit);
}
        \end{minted}
    \end{frame}

    \begin{frame}{Квантификаторы}
        \begin{footnotesize}
            \begin{tabu} {| X[1 l p] | X[3 l p] |}
                \tabucline-
                Операция                 & Описание  \\
                \tabucline-
                \everyrow{\tabucline-}
                All       & Возвращает true, если для всех элементов коллекции выполнен предикат           \\
                Any       & Возвращает true, если для хотя бы одного элемента коллекции выполнен предикат  \\
                Contains  & Возвращает true, коллекция содержит указанный элемент                          \\
            \end{tabu}
        \end{footnotesize}
    \end{frame}

    \begin{frame}{Разбиение коллекций}
        \begin{footnotesize}
            \begin{tabu} {| X[1 l p] | X[3 l p] |}
                \tabucline-
                Операция                 & Описание  \\
                \tabucline-
                \everyrow{\tabucline-}
                Skip       & Возвращает коллекцию без первых n элементов                                          \\
                Take       & Возвращает коллекцию из первых n элементов                                           \\
                SkipWhile  & Возвращает коллекцию, пропустив первые элементы, для которых выполняется предикат    \\
                TakeWhile  & Возвращает коллекцию, из элементов до первого, для которого не выполняется предикат  \\
                Chunk      & Делит коллекцию на куски указанного размера                                          \\
            \end{tabu}
        \end{footnotesize}
    \end{frame}

    \begin{frame}[fragile]{TakeWhile, SkipWhile, пример}
        \begin{minted}{csharp}
foreach (int number in Enumerable.Range(0, 8).TakeWhile(n => n < 5))
{
    Console.WriteLine(number);
}

foreach (int number in Enumerable.Range(0, 8).SkipWhile(n => n < 5))
{
    Console.WriteLine(number);
}
        \end{minted}
    \end{frame}

    \begin{frame}[fragile]{Chunk, пример}
        \begin{minted}{csharp}
int chunkNumber = 1;
foreach (int[] chunk in Enumerable.Range(0, 8).Chunk(3))
{
    Console.WriteLine($"Chunk {chunkNumber++}:");
    foreach (int item in chunk)
    {
        Console.WriteLine($"    {item}");
    }

    Console.WriteLine();
}
        \end{minted}
    \end{frame}

    \begin{frame}{Преобразование типов}
        \begin{footnotesize}
            \begin{tabu} {| X[1 l p] | X[3 l p] |}
                \tabucline-
                Операция                 & Описание  \\
                \tabucline-
                \everyrow{\tabucline-}
                Cast          & Преобразует все элементы коллекции к указанному виду                                   \\
                OfType        & Оставляет в коллекции только значения, которые можно преобразовать к указанному типу   \\
                ToArray       & Превращает коллекцию в массив                                                          \\
                ToDictionary  & Превращает коллекцию в словарь, используя данную лямбду для вычисления ключа элемента  \\
                ToList        & Превращает коллекцию в список                                                          \\
            \end{tabu}
        \end{footnotesize}
    \end{frame}

    \begin{frame}[fragile]{OfType, пример}
        \begin{minted}{csharp}
var fruits = new System.Collections.ArrayList(4);
fruits.Add("Mango");
fruits.Add("Orange");
fruits.Add("Apple");
fruits.Add(3.0);
fruits.Add("Banana");

// Apply OfType() to the ArrayList.
IEnumerable<string> query1 = fruits.OfType<string>();

Console.WriteLine("Elements of type 'string' are:");
foreach (string fruit in query1)
{
    Console.WriteLine(fruit);
}
        \end{minted}
    \end{frame}

    \begin{frame}[fragile]{Cast, пример}
        \begin{minted}{csharp}
var fruits = new System.Collections.ArrayList();
fruits.Add("mango");
fruits.Add("apple");
fruits.Add("lemon");

IEnumerable<string> query =
    fruits.Cast<string>()
    .OrderBy(fruit => fruit)
    .Select(fruit => fruit);

foreach (string fruit in query)
{
    Console.WriteLine(fruit);
}
        \end{minted}
        \attribution{\url{https://learn.microsoft.com/en-us/dotnet/api/system.linq.enumerable.cast?view=net-8.0}}
    \end{frame}

    \begin{frame}{Join}
        \begin{footnotesize}
            \begin{tabu} {| X[1 l p] | X[3 l p] |}
                \tabucline-
                Операция                 & Описание  \\
                \tabucline-
                \everyrow{\tabucline-}
                Join          & Zip с проекциями, которые определяют, по каким ключам выполнять JOIN  \\
                GroupJoin     & Join, но с последующим преобразованием результата                     \\
            \end{tabu}
        \end{footnotesize}
    \end{frame}

    \begin{frame}[fragile]{Join, пример}
        \begin{minted}{csharp}
var query = students.Join(departments,
    student => student.DepartmentID, department => department.ID,
    (student, department) => new { 
        Name = $"{student.FirstName} {student.LastName}", 
        DepartmentName = department.Name });

foreach (var item in query)
{
    Console.WriteLine($"{item.Name} - {item.DepartmentName}");
}
        \end{minted}
        \attribution{\url{https://learn.microsoft.com/en-us/dotnet/csharp/linq/standard-query-operators/join-operations}}
    \end{frame}

    \begin{frame}[fragile]{GroupJoin, пример}
        \begin{minted}{csharp}
IEnumerable<IEnumerable<Student>> studentGroups = 
    departments.GroupJoin(students, 
        department => department.ID, 
        student => student.DepartmentID,
    (department, studentGroup) => studentGroup);

foreach (IEnumerable<Student> studentGroup in studentGroups)
{
    Console.WriteLine("Group");
    foreach (Student student in studentGroup)
    {
        Console.WriteLine($"  - {student.FirstName}, {student.LastName}");
    }
}
        \end{minted}
        \attribution{\url{https://learn.microsoft.com/en-us/dotnet/csharp/linq/standard-query-operators/join-operations}}
    \end{frame}

    \begin{frame}{Группировка данных}
        \begin{footnotesize}
            \begin{tabu} {| X[1 l p] | X[3 l p] |}
                \tabucline-
                Операция                 & Описание  \\
                \tabucline-
                \everyrow{\tabucline-}
                GroupBy   & Раскладывает элементы коллекции по группам (IGrouping<TKey,TElement>) по данной функции-проекции  \\
                ToLookup  & То же, что GroupBy, но выкладывает элементы в Lookup<TKey,TElement>                               \\
            \end{tabu}
        \end{footnotesize}
    \end{frame}

    \begin{frame}[fragile]{GroupBy, пример}
        \begin{minted}{csharp}
List<int> numbers = [35, 44, 200, 84, 3987, 4, 199, 329, 446, 208];

IEnumerable<IGrouping<int, int>> query = numbers
    .GroupBy(number => number % 2);

foreach (var group in query)
{
    Console.WriteLine(group.Key == 0 ? 
        "\nEven numbers:" : "\nOdd numbers:");
    foreach (int i in group)
    {
        Console.WriteLine(i);
    }
}
        \end{minted}
        \attribution{\url{https://learn.microsoft.com/en-us/dotnet/csharp/linq/standard-query-operators/grouping-data}}
    \end{frame}

    \begin{frame}{Агрегатные операции}
        \begin{footnotesize}
            \begin{tabu} {| X[1 l p] | X[3 l p] |}
                \tabucline-
                Операция                 & Описание  \\
                \tabucline-
                \everyrow{\tabucline-}
                Aggregate     & Свёртка. Применяет функцию последовательно к каждому элементу коллекции и состоянию, меняя состояние                         \\
                Average       & Считает среднее для числовых коллекций                                                                                       \\
                Min, Max      & Считает минимум/максимум для числовых коллекций или для любых коллекций, если передать лямбду, отображающую элемент в число  \\
                MinBy, MaxBy  & То же, что Min/Max, но с лямбдой-проекцией и опциональным компаратором для произвольного типа ключа                          \\
                Sum           & Считает сумму для числовых коллекций или для любых коллекций, если передать лямбду, отображающую элемент в число             \\
            \end{tabu}
        \end{footnotesize}
    \end{frame}

    \begin{frame}{Типы операций}
        \begin{itemize}
            \item Немедленные (Immediate) --- требуют \emph{материализации} коллекции
            \begin{itemize}
                \item Все агрегатные операции
                \item ToArray, ToList
            \end{itemize}
            \item Отложенные (ленивые, Deferred) --- исполняются только когда нужен результат
            \begin{itemize}
                \item Можно понимать, как энумератор, \enquote{надеваемый} на исходный
                \item Могут возвращать разные результаты в разные моменты времени
            \end{itemize}
            \item Отложенные потоковые --- возвращают результат для каждого элемента исходной коллекции, если надо (например, Select, Where)
            \item Отложенные непотоковые --- требуют просмотра всей коллекции для возврата результата (например, Order, GroupBy)
        \end{itemize}
    \end{frame}

    \section{Подъязык LINQ}

    \begin{frame}[fragile]{Встроенный язык написания запросов}
        \begin{minted}{csharp}
// Specify the data source.
int[] scores = [97, 92, 81, 60];

// Define the query expression.
IEnumerable<int> scoreQuery =
    from score in scores
    where score > 80
    select score;

// Execute the query.
foreach (var i in scoreQuery)
    Console.Write($"{i} ");
        \end{minted}
        \begin{itemize}
            \item Синтаксический сахар над методами, любой запрос можно записать в \enquote{методной} форме
            \item Можно комбинировать ключевые слова и методы
        \end{itemize}
    \end{frame}

    \begin{frame}{Части запроса}
        \begin{itemize}
            \item Начинается с from
            \begin{itemize}
                \item Объявляются источники данных и переменные, по которым на них будут ссылаться в запросе
            \end{itemize}
            \item Заканчивается на select или group
            \item Внутри --- where, orderby, join, let или ещё from
            \item into --- сохранить промежуточные результаты
        \end{itemize}
    \end{frame}

    \begin{frame}[fragile]{Пример}
        \begin{small}
            \begin{minted}{csharp}
// percentileQuery is an IEnumerable<IGrouping<int, Country>>
var percentileQuery =
    from country in countries
    let percentile = (int)country.Population / 10_000_000
    group country by percentile into countryGroup
    where countryGroup.Key >= 20
    orderby countryGroup.Key
    select countryGroup;

// grouping is an IGrouping<int, Country>
foreach (var grouping in percentileQuery)
{
    Console.WriteLine(grouping.Key);
    foreach (var country in grouping)
    {
        Console.WriteLine(country.Name + ":" + country.Population);
    }
}
            \end{minted}
        \end{small}
    \end{frame}

    \begin{frame}{Ключевые слова}
        \begin{ssmall}
            \begin{tabu} {| X[1 l p] | X[7 l p] |}
                \tabucline-
                Ключевое слово  & Описание  \\
                \tabucline-
                \everyrow{\tabucline-}
                from        & Объявление источника данных и \enquote{range variable} (прямого аналога переменной в foreach). Транслируется в имя коллекции и параметры у лямбд  \\
                where       & Транслируется в Where                                                                                                                             \\
                select      & Транслируется в Select                                                                                                                            \\
                group       & Транслируется в GroupBy вместе с by                                                                                                               \\
                into        & Транслируется во временную переменную, хранящую результат join, group или select                                                                  \\
                orderby     & Транслируется в OrderBy или OrderByDescending (по умолчанию OrderBy)                                                                              \\
                join        & Транслируется в Join или GroupJoin                                                                                                                \\
                let         & Объявляет ещё одну \enquote{range variable} для итерирования по промежуточным результатам                                                         \\
                in          & Указывает вторую коллекцию для join                                                                                                               \\
                on          & Указывает условие \enquote{склейки} элементов коллекций в join                                                                                    \\
                equals      & Говорит, что должно быть равно в join ... on                                                                                                      \\
                by          & Указывает проекцию для GroupBy                                                                                                                    \\
                ascending   & Для orderby, указывает порядок сортировки                                                                                                         \\
                descending  & Для orderby, указывает порядок сортировки                                                                                                         \\
            \end{tabu}
        \end{ssmall}
    \end{frame}

    \begin{frame}[fragile]{join, пример}
        \begin{small}
            \begin{minted}{csharp}
var innerJoinQuery =
    from category in categories
    join prod in products on category.ID equals prod.CategoryID
    select new { ProductName = prod.Name, Category = category.Name };
            \end{minted}
        \end{small}
    \end{frame}

    \begin{frame}[fragile]{let и два from, пример}
        \begin{small}
            \begin{minted}{csharp}
string[] strings = [
    "A penny saved is a penny earned.",
    "The early bird catches the worm.",
    "The pen is mightier than the sword." ];

// Split the sentence into an array of words
// and select those whose first letter is a vowel.
var earlyBirdQuery = from sentence in strings
    let words = sentence.Split(' ')
    from word in words
    let w = word.ToLower()
    where w[0] == 'a' || w[0] == 'e' || w[0] == 'i' || w[0] == 'o' || w[0] == 'u'
    select word;

// Execute the query.
foreach (var v in earlyBirdQuery)
    Console.WriteLine($"\"{v}\" starts with a vowel");
            \end{minted}
        \end{small}
    \end{frame}

\end{document}
