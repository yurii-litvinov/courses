\documentclass{../../slides-style}

\slidetitle{Практика по настройке репозитория}{14.03.2025}

\begin{document}

    \begin{frame}[plain]
        \titlepage
    \end{frame}

    \begin{frame}
        \frametitle{Чеклист по оформлению репозитория (1)}
        \begin{enumerate}
            \item Лицензия
            \item Лицензия правильно применяется к репозиторию
            \item Используемые третьесторонние компоненты и материалы совместимы с лицензией
            \item Настроенный CI
            \item Модульные тесты в CI
            \item Линтер в CI
        \end{enumerate}
    \end{frame}

    \begin{frame}
        \frametitle{Чеклист по оформлению репозитория (2)}
        \begin{enumerate}
            \item В репозитории нет результатов сборки, настроен .gitignore
            \item В репозитории нет секретной информации (паролей, ключей и т.п.)
            \item Различные сторонние анализаторы (если уместно)
            \item Для кода на С/С++: использование санитайзеров
            \item README.md, плашки CI и анализаторов
            \item README.md, общее описание проекта
        \end{enumerate}
    \end{frame}

    \begin{frame}
        \frametitle{Чеклист по оформлению репозитория (3)}
        \begin{enumerate}
            \item README.md, пример использования
            \item README.md, инструкция по сборке и запуску
            \item README.md, как помочь проекту (если уместно)
            \item README.md, корректные названия компаний и организаций
            \item Настроена секция About, указаны темы (topics)
            \item Код соответствует принятому в сообществе стилю кодирования
        \end{enumerate}
    \end{frame}

    \begin{frame}
        \frametitle{Чеклист по оформлению репозитория (4)}
        \begin{enumerate}
            \item Имеется техническая документация (в README.md или на вики)
            \item В коде достаточно комментариев
            \item Комментарии к коммитам адекватны, коммиты показывают историю проекта
            \item В главной ветке адекватная история коммитов
            \item Добавление релизов
            \begin{itemize}
                \item Semantic versioning
            \end{itemize}
            \item Настройки репозитория
            \begin{itemize}
                \item Запретить force push в main
                \item Dependabot
                \item CodeQL
            \end{itemize}
        \end{enumerate}
    \end{frame}

\end{document}
