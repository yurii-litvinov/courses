\documentclass{../../slides-style}

\slidetitleext{Объектно-ориентированное программирование на C\#}{11.02.2025}{Введение}

\begin{document}
    
    \begin{frame}[plain]
        \titlepage
    \end{frame}

    \section{Организационное}

    \begin{frame}
        \frametitle{Про что этот курс}
        \begin{itemize}
            \item Объектно-ориентированное программирование и близкие вещи
            \begin{itemize}
                \item Абстракция-инкапсуляция-наследование-полиморфизм
                \item Исключения
                \item Генерики
                \item События
                \item Пользовательские интерфейсы
            \end{itemize}
            \item Технология разработки ПО
            \begin{itemize}
                \item Юнит-тесты
                \item CI
                \item Визуальное моделирование
            \end{itemize}
            \item Язык --- C\#
            \item Не будет новых алгоритмов, будут новые технологии
        \end{itemize}
    \end{frame}

    \begin{frame}
        \frametitle{Организационное}
        \begin{itemize}
            \item Две пары в неделю, как в прошлом семестре --- теория и практика
            \item Отчётность
            \begin{itemize}
                \item Домашки, меньше, но объёмнее
                \begin{itemize}
                    \item Сдавать через GitHub, пуллреквестом в свой репозиторий
                    \item HwProj: \url{https://hwproj.ru/courses/40028}
                    \item Дедлайны в две недели, минус балл в неделю за пропуск (не больше чем на половину баллов)
                \end{itemize}
                \item Контрольная в середине семестра (одна)
                \item Зачётная работа в конце семестра
                \item Доклады ($+6$ баллов)
                \item Работа на практиках за 1-2 балла
            \end{itemize}
        \end{itemize}
    \end{frame}

    \begin{frame}
        \frametitle{Критерии оценивания}
        \begin{itemize}
            \item Шкала оценивания ECTS
            \item Итоговый балл за домашки: $MAX(0, (\frac{n}{N} – 0.6)) * 2.5 * 100$
            \begin{itemize}
                \item Если сделано меньше 60\% --- это 0, если 80\% --- 50 баллов
                \item Будут бонусные баллы за выполнение дополнительных условий
            \end{itemize}
            \item Дедлайнов на исправления не будет, но к 23 мая всё должно быть сдано
            \item Итоговые баллы за контрольную и зачёт: $\frac{n}{N} * 100$, их можно переписывать
            \item В качестве итогового берётся \textbf{минимум} из этих баллов
        \end{itemize}
    \end{frame}

    \begin{frame}
    \frametitle{Литература}
        \begin{itemize}
            \item \textbf{Джепикс, Троелсен} Язык программирования C\# 9 и платформа .NET 5 --- C\# для самых маленьких
            \item \textbf{Джозеф Албахари, Бен Албахари} C\# 9.0. Справочник. Полное описание языка
            \item \textbf{Jeffrey Richter} CLR via C\# --- must read про C\#
            \begin{itemize}
                \item Или её русское издание \foreignquote{english}{CLR via C\#. Программирование на платформе Microsoft.NET Framework 4.5 на языке C\#}
            \end{itemize}
            \item \url{https://ulearn.me/} --- набор открытых онлайн-курсов от \enquote{Контур} и УрГУ
            \begin{itemize}
                \item Очень рекомендую все три курса, если наш покажется сложным --- мы за один семестр проходим почти всё, что они за три
                \item Можно получить E, если сдать \emph{все} задачи оттуда
            \end{itemize}
            \item \url{https://devblogs.microsoft.com/dotnet/} --- официальный блог о .NET
        \end{itemize}
    \end{frame}

    \section{Язык C\#}

    \begin{frame}
        \frametitle{Язык C\#}
        \begin{itemize}
            \item Объектно-ориентированный язык общего назначения с сильной типизацией
            \item Основной язык программирования для платформы .NET
            \item Первая версия --- 2002 год, актуальная --- 14.11.2024, C\# 13
            \item В основном для прикладного ПО
            \item 5-е место в индексе TIOBE на январь 2025 (после Python, C++, Java, C)
            \begin{itemize}
                \item \url{https://www.tiobe.com/tiobe-index/}
            \end{itemize}
            \item Работает под Windows и Linux/Mac OS
            \item Средства разработки
            \begin{itemize}
                \item Microsoft Visual Studio (\url{https://www.visualstudio.com})
                \item Visual Studio Code (\url{https://code.visualstudio.com/}) + C\# DevKit
                \item Rider (\url{https://www.jetbrains.com/rider/})
            \end{itemize}
        \end{itemize}
    \end{frame}

    \begin{frame}
        \frametitle{Common Language Infrastructure}
        \begin{columns}
            \begin{column}{0.5\textwidth}
                \begin{itemize}
                    \item Компиляция не в машинные коды, а в байткод виртуальной машины (Common Intermediate Language, CIL)
                    \item Виртуальная машина и набор библиотек (Common Language Runtime) реализуется для каждой платформы (ОС), на которой хотим запускать байт-код
                    \item Машина интерпретирует байт-код или компилирует его \enquote{на лету} в машинные коды
                \end{itemize}
            \end{column}
            \begin{column}{0.5\textwidth}
                \includegraphics[width=\textwidth]{cli.png}
            \end{column}
        \end{columns}
    \end{frame}

    \begin{frame}
        \frametitle{Зачем так}
        \begin{itemize}
            \item Единый байткод с механизмом оптимизации, интерпретации и генерации машинного кода позволяет экономить время на разработку компиляторов
            \begin{itemize}
                \item Поэтому под .NET можно писать на C\#, F\#, Visual Basic (зачем бы он ни был нужен), куче сторонних языков (например, Delphi)
            \end{itemize}
            \item Совместимость кода, написанного на разных языках, возможность из кода на F\# без усилий использовать библиотеки на C\# и наоборот
            \item Виртуальная машина знает всё о выполнении программы и многое может проверять (обращения к памяти, границы массивов и т.д.)
            \item JIT-компиляция
            \item Такая схема использовалась для Паскаля и Лиспа аж в 1970-х
        \end{itemize}
    \end{frame}

    \begin{frame}
        \frametitle{Сборка мусора}
        \begin{columns}
            \begin{column}{0.6\textwidth}
                \begin{itemize}
                    \item Виртуальная машина сама следит за используемой памятью и освобождает её, когда она перестаёт быть нужной
                    \item Корневое множество (глобальные переменные и стек вызовов), достижимое множество
                    \item Вызывается, \enquote{когда захочет}
                    \item Относительно вычислительно сложно
                    \item Устроено обычно довольно хитро
                    \begin{itemize}
                        \item Поколения, Large Object Heap и т.д.
                    \end{itemize}
                \end{itemize}
            \end{column}
            \begin{column}{0.4\textwidth}
                \includegraphics[width=\textwidth]{markAndSweep.png}
            \end{column}
        \end{columns}
    \end{frame}

    \section{Технические детали C\#}

    \begin{frame}[fragile]
        \frametitle{Технические детали C\#}
        \framesubtitle{Как обычно, сначала Hello, world}
        \begin{minted}{csharp}
System.Console.WriteLine("Goodbye, cruel world!");
        \end{minted}
    \end{frame}

    \begin{frame}[fragile]
        \frametitle{Или \enquote{полная форма}}
        \begin{minted}{csharp}
using System;

namespace HelloWorld
{
    class Program
    {
        static void Main(string[] args)
        {
            Console.WriteLine("Goodbye, cruel world!");
        }
    }
}
        \end{minted}
    \end{frame}

    \begin{frame}[fragile]
        \frametitle{Циклы}
        \begin{minted}{csharp}
for (int i = 0; i < 300; ++i)
{
    Console.WriteLine("Мат-мех не для всех!");
}
        \end{minted}
        или
        \begin{minted}{csharp}
for (var i = 0; i < 300; ++i)
{
    Console.WriteLine("Мат-мех не для всех!");
}
        \end{minted}
    \end{frame}

    \begin{frame}[fragile]
        \frametitle{Функции}
        \begin{minted}{csharp}
int Factorial(int n)
{
    if (n <= 1)
    {
        return 1;
    }

    return n * Factorial(n - 1);
}
        \end{minted}
        или так:
        \begin{minted}{csharp}
int Factorial(int n) 
    => n <= 1 ? 1 : n * Factorial(n - 1);
        \end{minted}
    \end{frame}

    \begin{frame}[fragile]
        \frametitle{Использование}
        \begin{minted}{csharp}
using System;

int Factorial(int n) 
    => n <= 1 ? 1 : n * Factorial(n - 1);

Console.WriteLine(Factorial(4));

        \end{minted}
    \end{frame}

    \begin{frame}[fragile]
        \frametitle{Или \enquote{полная} форма}
        \begin{minted}{csharp}
namespace HelloWorld
{
    class Program
    {
        private static int Factorial(int n) 
        {
            return n <= 1 ? 1 : n * Factorial(n - 1);
        }

        static void Main(string[] args)
        {
            System.Console.WriteLine(Factorial(4));
        }
    }
}
        \end{minted}
    \end{frame}

    \begin{frame}
        \frametitle{Стайлгайд}
        \begin{itemize}
            \item C\# Coding Conventions
            \begin{itemize}
                \item \url{https://learn.microsoft.com/en-us/dotnet/csharp/fundamentals/coding-style/coding-conventions}
            \end{itemize}
            \item \url{https://github.com/DotNetAnalyzers/StyleCopAnalyzers}
            \item \href{https://learn.microsoft.com/ru-ru/previous-versions/visualstudio/visual-studio-2015/code-quality/code-analysis-for-managed-code-overview}{Code Analysis for Managed Code}
            \item Здравый смысл
        \end{itemize}
    \end{frame}

    \begin{frame}
        \frametitle{Элементарные типы}
        \begin{itemize}
            \item Всё --- объект, даже \mintinline{csharp}|int| наследуется от \mintinline{csharp}|Object|
            \item Типы стандартизованы: размер и множество значений одинаковы во всех реализациях
            \item Каждому типу соответствует библиотечный класс
            \begin{itemize}
                \item Например, \mintinline{csharp}|int| --- \mintinline{csharp}|System.Int32|
            \end{itemize}
            \item У каждого типа есть значение по умолчанию
            \begin{itemize}
                \item Им переменные и поля инициализируются при создании
            \end{itemize}
        \end{itemize}
    \end{frame}

    \begin{frame}[fragile]
        \frametitle{Методы у типов}
        \begin{minted}{csharp}
var inputString = Console.ReadLine();
int number = int.Parse(inputString);
        \end{minted}
        --- это то же самое, что
        \begin{minted}{csharp}
var inputString = Console.ReadLine();
int number = Int32.Parse(inputString);
        \end{minted}
    \end{frame}

    \begin{frame}[fragile]
        \frametitle{Массивы}
        \begin{minted}{csharp}
int[] a = new int[10];
        \end{minted}
        или
        \begin{minted}{csharp}
var a = new int[10];
        \end{minted}
        Пример:
        \begin{minted}{csharp}
for (var i = 0; i < a.Length; ++i)
{
    a[i] = i;
}
        \end{minted}
        Двумерные массивы:
        \begin{minted}{csharp}
int[,] numbers = new int[3, 3];
numbers[1, 2] = 2; 
int[,] numbers2 = new int[3, 3] { {2, 3, 2}, {1, 2, 6}, {2, 4, 5} };
        \end{minted}
        Массивы знают свою длину (см. свойство Length и метод GetLength())
    \end{frame}

    \begin{frame}[fragile]
        \frametitle{Перечисления}
        Объявление:
        \begin{minted}{csharp}
enum SomeEnum
{
    red,
    green,
    blue
}
        \end{minted}
        Использование:
        \begin{minted}{csharp}
SomeEnum a = SomeEnum.blue;
        \end{minted}
        (ну или через \mintinline{csharp}|var|: \mintinline{csharp}|var a = SomeEnum.blue;|)
    \end{frame}

    \begin{frame}[fragile]
        \frametitle{Структуры}
        \begin{minted}{csharp}
struct Point
{
    public int x;
    public int y;
}
        \end{minted}
    \end{frame}

\end{document}