\documentclass{../../slides-style}

\slidetitle{Практика 3: Практика по рисованию диаграмм}{19.03.2024}

\begin{document}

    \begin{frame}[plain]
        \titlepage
    \end{frame}

    \begin{frame}
        \frametitle{Диаграмма классов}
        \begin{center}
            \includegraphics[height=0.8\textheight]{umlClassDiagram.png}
            \attribution{М. Фаулер. ``UML. Основы''}
        \end{center}
    \end{frame}

    \begin{frame}
        \frametitle{Частые ошибки}
        \begin{itemize}
            \item Если есть ассоциация, то не надо рисовать поле
            \item Не надо писать над стрелками их смысл
            \begin{itemize}
                \item Должны быть говорящие имена ролей (поля)
                \item Либо комментарии
            \end{itemize}
            \item Не путайте направления генерализаций
            \item Ромбик рисуется у <<хозяина>>
            \item Наследование от класса и реализация интерфейса --- разные вещи
            \item Не надо старательно выписывать каждое поле и метод
            \item Обращайте внимание на синтаксис параметров и возвращаемого значения
            \item Старайтесь использовать множественность вместо классов-контейнеров
        \end{itemize}
    \end{frame}

    \begin{frame}
        \frametitle{Диаграммы компонентов}
        \begin{center}
            \includegraphics[width=0.95\textwidth]{componentDiagrams.png}
        \end{center}
    \end{frame}

    \begin{frame}
        \frametitle{Диаграмма развёртывания UML}
        \begin{columns}
            \begin{column}{0.5\textwidth}
                \begin{itemize}
                    \item Узлы --- реальные (или виртуальные) устройства
                    \item Связи --- каналы данных между узлами (не компонентами)
                    \item Связи подписываются протоколом
                \end{itemize}
            \end{column}
            \begin{column}{0.5\textwidth}
                \begin{center}
                    \includegraphics[width=\textwidth]{deploymentDiagram.png}
                    \attribution{М. Фаулер, UML. Основы}
                \end{center}
            \end{column}
        \end{columns}
    \end{frame}

    \begin{frame}
        \frametitle{Задачи на пару}
        \begin{itemize}
            \item Вспомнить запрос \url{https://goo.gl/MiyH8c}
            \item Задача 1: Построить \textit{модель данных} разрабатываемого приложения в виде диаграммы классов
            \item Задача 2: Нарисовать диаграмму компонентов разрабатываемого приложения
            \begin{itemize}
                \item То есть первое приближение архитектуры
            \end{itemize}
            \item Задача 3: Нарисовать диаграмму развёртывания разрабатываемого приложения
        \end{itemize}
    \end{frame}

\end{document}
