\documentclass{../../slides-style}

\slidetitle{Практика 2: Практика по визуальному моделированию}{17.03.2026}

\begin{document}

    \begin{frame}[plain]
        \titlepage
    \end{frame}

    \begin{frame}
        \frametitle{Напоминание, диаграмма случаев использования UML}
        \begin{columns}
            \begin{column}{0.6\textwidth}
                \begin{itemize}
                    \item Акторы (или актёры, роли) --- внешние сущности, использующие систему
                    \begin{itemize}
                        \item Люди или другие программные системы
                    \end{itemize}
                    \item Случаи использования (прецеденты)  --- цель использования системы актором
                    \begin{itemize}
                        \item Одно-два слова о цели использования
                        \item \textit{Не последовательность действий}
                        \item \textit{Не модель данных}
                    \end{itemize}
                    \item Цель: определиться с количеством и функциональностью АРМ
                \end{itemize}
            \end{column}
            \begin{column}{0.4\textwidth}
                \begin{center}
                    \includegraphics[width=\textwidth]{useCaseDiagram.png}
                    \attribution{М. Фаулер, UML. Основы}
                \end{center}
            \end{column}
        \end{columns}
    \end{frame}

    \begin{frame}
        \frametitle{Напоминание, диаграмма активностей}
        \begin{columns}
            \begin{column}{0.5\textwidth}
                \begin{itemize}
                    \item Модель бизнес-процесса
                    \item Как раз тут --- последовательность действий, но без привязки к конкретной системе
                    \item \url{https://www.uml-diagrams.org/activity-diagrams.html}
                    \item Обратите внимание на синтаксис ветвлений
                    \item Пользуйтесь дорожками
                \end{itemize}
            \end{column}
            \begin{column}{0.5\textwidth}
                \begin{center}
                    \includegraphics[width=0.7\textwidth]{activityDiagram.png}
                    \attribution{М. Фаулер, UML. Основы}
                \end{center}
            \end{column}
        \end{columns}
    \end{frame}

    \begin{frame}
        \frametitle{Задачи на пару}
        \begin{itemize}
            \item Проанализировать запрос \url{https://goo.gl/MiyH8c}
            \item Задача 1: Нарисовать диаграмму случаев использования разрабатываемого приложения
            \item Задача 2: Нарисовать диаграмму активностей для бизнес-процесса предприятия, для которого разрабатывается приложение
            \item Пользоваться одним из упоминавшихся инструментов
            \begin{itemize}
                \item Например, \url{https://app.diagrams.net/}, \url{https://www.visual-paradigm.com/editions/community/}
            \end{itemize}
        \end{itemize}
    \end{frame}

\end{document}
