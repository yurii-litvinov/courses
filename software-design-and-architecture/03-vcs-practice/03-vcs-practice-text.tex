\documentclass{../../text-style}

\texttitle{Практика 1: Проектирование системы контроля версий}

\begin{document}

\maketitle
\thispagestyle{empty}

В командах по 2-3 человека требуется спроектировать систему контроля версий, представляющую из себя консольное приложение и умеющую:

\begin{itemize}
    \item commit с commit message, датой коммита и автором;
    \item работу с ветками: создание и удаление;
    \item checkout по имени ревизии или ветки;
    \item log --- список ревизий вместе с commit message в текущей ветке;
    \item merge --- сливает указанную ветку с текущей.
\end{itemize}

При этом код системы должен позволять себя использовать как библиотеку, но предполагается также наличие консольного интерфейса.

Что надо сделать:

\begin{itemize}
    \item диаграмму компонентов с классами, пока как умеете --- про формальные аспекты рисования диаграмм начнём говорить на следующей неделе;
    \item текстовое описание не требуется, поясняйте непонятные моменты в комментариях на диаграмме;
    \item нельзя подсматривать в Git Book и другую архитектурную документацию систем контроля версий (там всё написано, хотя конкретно git нельзя назвать примером хорошей архитектуры, и у нас про это ещё отдельная пара будет).
\end{itemize}

Решения в духе <<куча классов, хаотично соединённых стрелками>> не пойдут, подумайте над некой высокоуровневой структурой.

Обратите внимание на следующие вещи.

\begin{itemize}
    \item Как представляются файлы, коммиты, ветки, репозиторий?
    \item Как выполняется компрессия и выполняется ли вообще? Насколько просто получить текущую, предыдущую, произвольную версии?
    \item Каков жизненный цикл файла?
    \item Как выполняется работа с файловой системой?
    \item Как выполняется работа с пользователем? Как представляются команды?
\end{itemize}

\end{document}