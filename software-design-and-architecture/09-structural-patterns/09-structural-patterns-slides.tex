\documentclass[xetex,mathserif,serif]{beamer}
\usepackage{polyglossia}
\setdefaultlanguage[babelshorthands=true]{russian}
\usepackage{minted}
\usepackage{tabu}

\useoutertheme{infolines}

\usepackage{fontspec}
\setmainfont{FreeSans}
\newfontfamily{\russianfonttt}{FreeSans}

\usepackage{textpos}
\setlength{\TPHorizModule}{1cm}
\setlength{\TPVertModule}{1cm}

\definecolor{links}{HTML}{2A1B81}
\hypersetup{colorlinks,linkcolor=,urlcolor=links}

\tabulinesep=1.2mm

\title[Структурные шаблоны]{Лекция 6: Структурные шаблоны}
\author[Юрий Литвинов]{Юрий Литвинов\\\small{\textcolor{gray}{yurii.litvinov@gmail.com}}}
\date{14.04.2020г}

\newcommand{\attribution}[1] {
	\vspace{-5mm}\begin{flushright}\begin{scriptsize}\textcolor{gray}{\textcopyright\, #1}\end{scriptsize}\end{flushright}
}

\begin{document}

	\frame{\titlepage}

	\section{Введение}

	\begin{frame}
		\frametitle{Паттерны проектирования}
		\textbf{Шаблон проектирования} --- это повторимая архитектурная конструкция, являющаяся решением некоторой типичной технической проблемы
		\begin{itemize}
			\item Подходит для класса проблем
			\item Обеспечивает переиспользуемость знаний
			\item Позволяет унифицировать терминологию
			\item В удобной для изучения форме
			\item НЕ конкретный рецепт или указания к действию
		\end{itemize}
	\end{frame}

	\begin{frame}
		\frametitle{Паттерны и архитектурные стили}
		\begin{center}
			\includegraphics[width=0.5\textwidth]{architecturalStylesPatternsHighlighted.png}
			\attribution{N. Medvidovic}
		\end{center}
	\end{frame}

	\begin{frame}
		\frametitle{Книжка про паттерны}
		\framesubtitle{Must read!}

		\begin{columns}
			\begin{column}{0.6\textwidth}
				Приемы объектно-ориентированного проектирования. Паттерны проектирования

				Э. Гамма, Р. Хелм, Р. Джонсон, Дж. Влиссидес

				Design Patterns: Elements of Reusable Object-Oriented Software
			\end{column}
			\begin{column}{0.4\textwidth}
				\begin{center}
					\includegraphics[width=0.8\textwidth]{patternBookCover.png}
				\end{center}
			\end{column}
		\end{columns}
	\end{frame}

	\section{Паттерн ``Компоновщик''}

	\begin{frame}
		\frametitle{Паттерн ``Компоновщик''}
		\framesubtitle{Composite}
		\begin{columns}
			\begin{column}{0.5\textwidth}
				\begin{itemize}
					\item Представление иерархии объектов вида часть-целое
					\item Единообразная обработка простых и составных объектов
					\item Простота добавления новых компонентов
					\item Пример:
					\begin{itemize}
						\item Синтаксические деревья
					\end{itemize}
				\end{itemize}
			\end{column}
			\begin{column}{0.5\textwidth}
				\begin{center}
					\includegraphics[width=\textwidth]{composite.png}
				\end{center}
			\end{column}
		\end{columns}
	\end{frame}

	\begin{frame}
		\frametitle{``Компоновщик'' (Composite), детали реализации}
		\begin{itemize}
			\item Ссылка на родителя
			\begin{itemize}
				\item Может быть полезна для простоты обхода
				\item ``Цепочка обязанностей''
				\item Но дополнительный инвариант
				\item Обычно реализуется в Component
			\end{itemize}
			\item Разделяемые поддеревья и листья
			\begin{itemize}
				\item Позволяют сильно экономить память
				\item Проблемы с навигацией к родителям и разделяемым состоянием
				\item Паттерн ``Приспособленец''
			\end{itemize}
			\item Идеологические проблемы с операциями для работы с потомками
			\begin{itemize}
				\item Не имеют смысла для листа
				\begin{itemize}
					\item Можно считать Leaf Composite-ом, у которого всегда 0 потомков
				\end{itemize}
				\item Операции add и remove можно объявить и в Composite, тогда придётся делать cast
				\begin{itemize}
					\item Иначе надо бросать исключения в add и remove
				\end{itemize}
			\end{itemize}
		\end{itemize}
	\end{frame}

	\begin{frame}
		\frametitle{``Компоновщик'', детали реализации (2)}
		\begin{itemize}
			\item Операция getComposite() – более аккуратный аналог cast-а
			\item Порядок потомков может быть важен, может нет
			\item Кеширование информации для обхода или поиска
			\begin{itemize}
				\item Например, кеширование ограничивающих прямоугольников для фрагментов картинки
				\item Инвалидация кеша
			\end{itemize}
			\item Удаление потомков
			\begin{itemize}
				\item Если нет сборки мусора, то лучше в Composite
				\item Следует опасаться разделяемых листьев/поддеревьев
			\end{itemize}
		\end{itemize}
	\end{frame}

	\section{Паттерн ``Декоратор''}

	\begin{frame}
		\frametitle{Усовершенствование UI}
		\begin{itemize}
			\item Хотим сделать рамку вокруг текста и полосы прокрутки, отключаемые по опции
			\item Желательно убирать и добавлять элементы обрамления так, чтобы другие объекты даже не знали, что они есть
			\item Хотим менять во время выполнения --- наследование не подойдёт
			\begin{itemize}
				\item Наш выбор ­--- композиция
				\item Прозрачное обрамление
			\end{itemize}
		\end{itemize}
	\end{frame}

	\begin{frame}
		\frametitle{Моноглиф}
		\begin{columns}
			\begin{column}{0.6\textwidth}
				\begin{itemize}
					\item Абстрактный класс с ровно одним сыном
					\begin{itemize}
						\item Вырожденный случай компоновщика
					\end{itemize}
					\item ``Обрамляет'' сына, добавляя новую функциональность
				\end{itemize}
			\end{column}
			\begin{column}{0.4\textwidth}
				\begin{center}
					\includegraphics[width=0.9\textwidth]{monoglyph.png}
					\attribution{Э. Гамма и др., Приемы объектно-ориентированного проектирования}
				\end{center}
			\end{column}
		\end{columns}
	\end{frame}

	\begin{frame}
		\frametitle{Структура глифов}
		\begin{center}
			\includegraphics[width=0.9\textwidth]{glyphStructure.png}
			\attribution{Э. Гамма и др., Приемы объектно-ориентированного проектирования}
		\end{center}
	\end{frame}

	\begin{frame}
		\frametitle{Паттерн ``Декоратор''}
		\framesubtitle{Decorator}
		\begin{center}
			\includegraphics[width=0.6\textwidth]{decorator.png}
		\end{center}
	\end{frame}

	\begin{frame}
		\frametitle{Декоратор, особенности}
		\begin{itemize}
			\item Динамическое добавление (и удаление) обязанностей объектов
			\begin{itemize}
				\item Большая гибкость, чем у наследования
			\end{itemize}
			\item Позволяет избежать перегруженных функциональностью базовых классов
			\item Много мелких объектов
		\end{itemize}
	\end{frame}

	\begin{frame}
		\frametitle{``Декоратор'' (Decorator), детали реализации}
		\begin{columns}
			\begin{column}{0.6\textwidth}
				\begin{itemize}
					\item Интерфейс декоратора должен соответствовать интерфейсу декорируемого объекта
					\begin{itemize}
						\item Иначе получится ``Адаптер''
					\end{itemize}
					\item Если конкретный декоратор один, абстрактный класс можно не делать
				\end{itemize}
			\end{column}
			\begin{column}{0.4\textwidth}
				\begin{center}
					\includegraphics[width=\textwidth]{decorator.png}
				\end{center}
			\end{column}
		\end{columns}
		\begin{itemize}
			\item Component должен быть по возможности небольшим (в идеале, интерфейсом)
			\begin{itemize}
				\item Иначе лучше паттерн ``Стратегия''
				\item Или самодельный аналог, например, список ``расширений'', которые вызываются декорируемым объектом вручную перед операцией или после неё
			\end{itemize}
		\end{itemize}
	\end{frame}

	\section{Паттерн ``Стратегия''}

	\begin{frame}
		\frametitle{Форматирование текста}
		\begin{itemize}
			\item Задача --- разбиение текста на строки, колонки и т.д.
			\item Высокоуровневые параметры форматирования
			\begin{itemize}
				\item Ширина полей, размер отступа, межстрочный интервал и т.д.
			\end{itemize}
			\item Компромисс между качеством и скоростью работы
			\item Инкапсуляция алгоритма
		\end{itemize}
	\end{frame}

	\begin{frame}
		\frametitle{Compositor и Composition}
		\begin{center}
			\includegraphics[width=0.7\textwidth]{compositor.png}
			\attribution{Э. Гамма и др., Приемы объектно-ориентированного проектирования}
		\end{center}
	\end{frame}

	\begin{frame}
		\frametitle{Паттерн ``Стратегия''}
		\framesubtitle{Strategy}
		\begin{itemize}
			\item Назначение --- инкапсуляция алгоритма в объект
			\item Самое важное --- спроектировать интерфейсы стратегии и контекста
			\begin{itemize}
				\item Так, чтобы не менять их для каждой стратегии
			\end{itemize}
			\item Применяется, если
			\begin{itemize}
				\item Имеется много родственных классов с разным поведением
				\item Нужно иметь несколько вариантов алгоритма
				\item В алгоритме есть данные, про которые клиенту знать не надо
				\item В коде много условных операторов
			\end{itemize}
		\end{itemize}
		\begin{center}
			\includegraphics[width=0.6\textwidth]{strategy.png}
		\end{center}
	\end{frame}

	\begin{frame}
		\frametitle{``Стратегия'' (Strategy), детали реализации}
		\begin{center}
			\includegraphics[width=0.6\textwidth]{strategy.png}
		\end{center}
		\begin{itemize}
			\item Передача контекста вычислений в стратегию
			\begin{itemize}
				\item Как параметры метода --- уменьшает связность, но некоторые параметры могут быть стратегии не нужны
				\item Передавать сам контекст в качестве аргумента --- в Context интерфейс для доступа к данным
			\end{itemize}
		\end{itemize}
	\end{frame}

	\begin{frame}
		\frametitle{``Стратегия'' (Strategy), детали реализации (2)}
		\begin{itemize}
			\item Стратегия может быть параметром шаблона
			\begin{itemize}
				\item Если не надо её менять на лету
				\item Не надо абстрактного класса и нет оверхеда на вызов виртуальных методов
			\end{itemize}
			\item Стратегия по умолчанию
			\begin{itemize}
				\item Или просто поведение по умолчанию, если стратегия не установлена
			\end{itemize}
			\item Объект-стратегия может быть приспособленцем
		\end{itemize}
	\end{frame}

\end{document}
