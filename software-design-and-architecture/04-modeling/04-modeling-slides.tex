\documentclass[xetex,mathserif,serif]{beamer}
\usepackage{polyglossia}
\setdefaultlanguage[babelshorthands=true]{russian}
\usepackage{minted}
\usepackage{tabu}

\useoutertheme{infolines}

\usepackage{fontspec}
\setmainfont{FreeSans}
\newfontfamily{\russianfonttt}{FreeSans}

\usepackage{textpos}
\setlength{\TPHorizModule}{1cm}
\setlength{\TPVertModule}{1cm}

\setbeamertemplate{blocks}[rounded][shadow=false]

\setbeamercolor*{block title alerted}{fg=red!50!black,bg=red!20}
\setbeamercolor*{block body alerted}{fg=black,bg=red!10}

\definecolor{links}{HTML}{2A1B81}
\hypersetup{colorlinks,linkcolor=,urlcolor=links}

\tabulinesep=1.2mm

\title[Моделирование]{Лекция 3: Моделирование как инструмент архитектуры}
\author[Юрий Литвинов]{Юрий Литвинов\\\small{\textcolor{gray}{yurii.litvinov@gmail.com}}}
\date{03.03.2020г}

\newcommand{\attribution}[1] {
	\vspace{-5mm}\begin{flushright}\begin{scriptsize}\textcolor{gray}{\textcopyright\, #1}\end{scriptsize}\end{flushright}
}

\begin{document}

	\frame{\titlepage}

	\section{Модели}

	\begin{frame}
		\frametitle{Моделирование}
		\begin{itemize}
			\item \textbf{Модель} --- упрощённое подобие объекта или явления
			\item Нужны для изучения некоторых их свойств, абстрагируясь от сложности ``настоящего'' объекта или явления
			\item Модели используются повсеместно
			\begin{itemize}
				\item Математические модели
				\item Модели как реальные объекты
				\item Модели в разработке ПО
			\end{itemize}
		\end{itemize}
	\end{frame}

	\begin{frame}
		\frametitle{Общие свойства моделей}
		\begin{itemize}
			\item Содержат меньше информации, чем реальность
			\item Существуют для определённой цели
			\item Модели субъективны, что позволяет отделить существенные свойства от несущественных
			\item Моделирование ПО:
			\begin{itemize}
				\item Модели предназначены прежде всего для управления сложностью
				\item Позволяют понять, проанализировать и протестировать систему до её реализации
			\end{itemize}
		\end{itemize}

		\center{\textbf{All models are wrong, some are useful}}
	\end{frame}

	\begin{frame}
		\frametitle{Как выбрать, что моделировать?}
		\begin{itemize}
			\item При моделировании надо определиться с:
			\begin{itemize}
				\item Какие архитектурные решения нуждаются в моделировании
				\item На каком уровне детализации
				\item Насколько формально
			\end{itemize}
			\item Необходимо учитывать соотношение трудозатрат и выгоды
			\begin{itemize}
				\item Стоимость создания \textit{и поддержания} модели не должна быть больше преимуществ от её использования
			\end{itemize}
		\end{itemize}
		\begin{center}
			\includegraphics[width=0.4\textwidth]{concerns.png}
			\attribution{N. Medvidovic}
		\end{center}
	\end{frame}

	\section{Виды моделей}

	\begin{frame}
		\frametitle{Модели бывают разные}
		\begin{itemize}
			\item Используемые нотации и способы моделирования зависят от целей моделирования
			\begin{itemize}
				\item От неформальных набросков до исполнимых моделей
			\end{itemize}
		\end{itemize}
		\begin{center}
			\includegraphics[width=0.9\textwidth]{sketchesVsFormalNotations.png}
			\attribution{N. Medvidovic}
		\end{center}
	\end{frame}

	\begin{frame}
		\frametitle{Виды моделей}
		\framesubtitle{Естественные языки}
		\begin{columns}
			\begin{column}{0.5\textwidth}
				\begin{itemize}
					\item Обычный текст --- вполне себе инструмент моделирования
					\item Очень выразителен, не требует специальных знаний, максимально гибок
					\item Неоднозначен, неформален, не строг, слишком многословен, бесполезен для автоматической обработки
				\end{itemize}
			\end{column}
			\begin{column}{0.5\textwidth}
				\begin{center}
					\includegraphics[width=\textwidth]{naturalLanguage.png}
					\attribution{N. Medvidovic}
				\end{center}
			\end{column}
		\end{columns}
	\end{frame}

	\begin{frame}
		\frametitle{Неформальные графические модели}
		\begin{columns}
			\begin{column}{0.5\textwidth}
				\begin{itemize}
					\item Диаграммы, рисуемые в PowerPoint, InkScape и подобном
					\item Могут быть красивыми, как правило, простые, очень гибкая нотация
					\item Неформальны, неоднозначны, не строги
					\begin{itemize}
						\item Но часто воспринимаются наоборот
					\end{itemize}
					\item Практически бесполезны для автоматической обработки
				\end{itemize}
			\end{column}
			\begin{column}{0.5\textwidth}
				\begin{center}
					\includegraphics[width=0.9\textwidth]{informalModel.png}
					\attribution{N. Medvidovic}
				\end{center}
			\end{column}
		\end{columns}
	\end{frame}

	\begin{frame}
		\frametitle{UML и SysML}
		\begin{columns}
			\begin{column}{0.4\textwidth}
				\begin{small}
					\begin{itemize}
						\item Несколько слабо связанных нотаций (``диаграмм'')
						\item Поддерживают много точек зрения, общеприняты, широкая поддержка инструментами
						\item Нет строгой семантики, сложно обеспечить консистентность, сложно расширять
					\end{itemize}
				\end{small}
			\end{column}
			\begin{column}{0.6\textwidth}
				\begin{center}
					\includegraphics[width=\textwidth]{uml.png}
					\attribution{N. Medvidovic}
				\end{center}
			\end{column}
		\end{columns}
	\end{frame}

	\begin{frame}
		\frametitle{AADL и другие текстовые формальные языки}
		\begin{columns}
			\begin{column}{0.4\textwidth}
				\begin{small}
					\begin{itemize}
						\item Хороши для моделирования встроенных систем и систем реального времени
						\item Описывают одновременно ``железо'' и ``софт'', продвинутые инструменты анализа
						\item Слишком многословны и детальны, сложны в изучении и использовании
					\end{itemize}
				\end{small}
			\end{column}
			\begin{column}{0.6\textwidth}
				\begin{center}
					\includegraphics[width=0.85\textwidth]{aadl.png}
					\attribution{N. Medvidovic}
				\end{center}
			\end{column}
		\end{columns}
	\end{frame}

	\section{UML}

	\begin{frame}
		\frametitle{Вернёмся к визуальным моделям}
		\begin{itemize}
			\item \textbf{Метафора визуализации} --- договорённость о том, как будут представляться сущности языка
			\item \textbf{Точка зрения моделирования} --- какой аспект системы и для кого моделируется
			\item Бывают одноразовые модели, документация и графические исходники
			\begin{itemize}
				\item \textbf{Семантический разрыв} --- неспособность модели полностью специфицировать систему
			\end{itemize}
		\end{itemize}
		\begin{center}
			\includegraphics[width=0.5\textwidth]{hashTable.png}
		\end{center}
	\end{frame}

	\begin{frame}
		\frametitle{Unified Modeling Language}
		\begin{itemize}
			\item Семейство графических нотаций
			\begin{itemize}
				\item 14 видов диаграмм
			\end{itemize}
			\item Общая метамодель
			\item Стандарт под управлением Object Management Group
			\begin{itemize}
				\item UML 1.1 --- 1997 год
				\item UML 2.0 --- 2005 год
				\item UML 2.5.1 --- декабрь 2017 года
			\end{itemize}
			\item Прежде всего, для проектирования ПО
			\begin{itemize}
				\item После UML 2.0 стали появляться нотации и для инженеров
			\end{itemize}
			\item Расширяем
			\begin{itemize}
				\item Профили --- механизм легковесного расширения
				\item Метамоделирование
			\end{itemize}
		\end{itemize}
	\end{frame}

	\begin{frame}
		\frametitle{История}
		\begin{center}
			\includegraphics[width=\textwidth]{umlHistory.png}
		\end{center}
	\end{frame}

	\begin{frame}
		\frametitle{Виды диаграмм}
		\begin{center}
			\includegraphics[width=\textwidth]{umlDiagrams.png}
		\end{center}
	\end{frame}

	\section{Диаграмма классов UML}

	\begin{frame}
		\frametitle{Диаграмма классов}
		\begin{center}
			\includegraphics[height=0.8\textheight]{umlClassDiagram.png}
			\attribution{М. Фаулер. ``UML. Основы''}
		\end{center}
	\end{frame}

	\begin{frame}[fragile]
		\frametitle{Как это связано с кодом}
		\begin{columns}
			\begin{column}{0.5\textwidth}
				\begin{footnotesize}
					\begin{minted}{java}
public class OrderLine {
    private int quantity;
    private Product product;
    public int getQuantity() {
        return quantity;
    }
    public void setQuantity(int quantity) {
        this.quantity = quantity;
    }
    public Money getPrice() {
        return product.getPrice().multiply(quantity);
    }
}
					\end{minted}
				\end{footnotesize}
			\end{column}
			\begin{column}{0.5\textwidth}
				\begin{center}
					\includegraphics[width=0.5\textwidth]{orderLine.png}
				\end{center}
			\end{column}
		\end{columns}
	\end{frame}

	\begin{frame}[fragile]
		\frametitle{Двунаправленные ассоциации}
		\begin{columns}
			\begin{column}{0.5\textwidth}
				\begin{scriptsize}
					\begin{center}
						\includegraphics[width=0.9\textwidth]{twoWayAssociations.png}
					\end{center}
	
					\begin{minted}{csharp}
class Car {
    public Person Owner {
        get { return _owner; }
        set {
            if (_owner != null) 
                _owner.friendCars().Remove(this);
            _owner = value;
            if (_owner != null) 
                _owner.friendCars().Add(this);
        }
    }
    private Person _owner;
}
					\end{minted}
				\end{scriptsize}
				\vspace{2mm}
			\end{column}
			\begin{column}{0.5\textwidth}
				\begin{scriptsize}
					\begin{center}
						\includegraphics[width=0.9\textwidth]{personOwnsCar.png}
					\end{center}
	
					\begin{minted}{csharp}
class Person {
    public IList Cars {
        get { return ArrayList.ReadOnly(_cars); }
    }
    public void AddCar(Car arg) {
        arg.Owner = this;
    }
    private IList _cars = new ArrayList();
    internal IList friendCars() {
        // должен быть использован 
        // только Car.Owner
        return _cars;
    }
}
					\end{minted}
				\end{scriptsize}
			\end{column}
		\end{columns}
	\end{frame}

	\begin{frame}
		\frametitle{Агрегация и композиция}
		Агрегация:
		\begin{center}
			\includegraphics[height=0.1\textheight]{aggregation.png}
		\end{center}
		\vspace{5mm}
		Композиция:
		\begin{center}
			\includegraphics[height=0.1\textheight]{composition.png}
		\end{center}
		\attribution{М. Фаулер. ``UML. Основы''}
	\end{frame}

	\section{Диаграммы пакетов}

	\begin{frame}
		\frametitle{Диаграммы пакетов}
		\begin{center}
			\includegraphics[width=0.8\textwidth]{packageDiagrams.png}
			\attribution{М. Фаулер. ``UML. Основы''}
		\end{center}
	\end{frame}

	\section{Диаграммы объектов}

	\begin{frame}
		\frametitle{Диаграммы объектов}
		\begin{center}
			\includegraphics[width=0.9\textwidth]{objectDiagrams.png}
			\attribution{М. Фаулер. ``UML. Основы''}
		\end{center}
	\end{frame}

	\section{Диаграммы компонентов}
	
	\begin{frame}
		\frametitle{Диаграммы компонентов}
		\begin{center}
			\includegraphics[width=0.95\textwidth]{componentDiagrams.png}
		\end{center}
	\end{frame}

	\begin{frame}
		\frametitle{Более подробно}
		\begin{center}
			\includegraphics[width=0.95\textwidth]{componentDiagramsOverview.png}
			\attribution{\url{http://www.uml-diagrams.org}}
		\end{center}
	\end{frame}

	\begin{frame}
		\frametitle{Диаграмма случаев использования UML}
		\framesubtitle{Диаграмма прецедентов}
		\begin{columns}
			\begin{column}{0.5\textwidth}
				\begin{itemize}
					\item Ивар Якобсон, 1992 год
					\item Акторы (или актёры, роли) --- внешние сущности, использующие систему
					\begin{itemize}
						\item Люди или другие программные системы
					\end{itemize}
					\item Случаи использования (прецеденты)  --- цель использования системы актором
					\begin{itemize}
						\item Раскрываются в набор сценариев, описываемых чаще текстом
					\end{itemize}
				\end{itemize}
			\end{column}
			\begin{column}{0.5\textwidth}
				\begin{center}
					\includegraphics[width=\textwidth]{useCaseDiagram.png}
					\attribution{М. Фаулер, UML. Основы}
				\end{center}
			\end{column}
		\end{columns}
	\end{frame}

	\begin{frame}
		\frametitle{Сценарий использования, типичная структура}
		\begin{itemize}
			\item Заголовок (цель основного актора)
			\item Заинтересованые лица, акторы, основной актор
			\item Предусловия
			\item Триггеры (активаторы)
			\item Основной порядок событий
			\item Альтернативные пути и расширения
			\item Постусловия
		\end{itemize}
	\end{frame}

	\begin{frame}
		\begin{center}
			\includegraphics[width=0.9\textwidth]{useCaseExample.png}
			\attribution{R.M. Roth et al., System Analysis and Design}
		\end{center}
	\end{frame}

	\section{Диаграмма активностей UML}

	\begin{frame}
		\frametitle{Диаграмма активностей UML}
		\framesubtitle{Диаграммы деятельности}
		\begin{columns}
			\begin{column}{0.5\textwidth}
				\begin{itemize}
					\item Используются для моделирования бизнес-процессов, тоже на первых этапах
					\begin{itemize}
						\item Может быть визуализацией сценария использования
					\end{itemize}
					\item Иногда --- для моделирования алгоритма
					\item Расширенные блок-схемы
					\item Семантика на основе сетей Петри
				\end{itemize}
			\end{column}
			\begin{column}{0.5\textwidth}
				\begin{center}
					\includegraphics[width=0.7\textwidth]{activityDiagram.png}
					\attribution{М. Фаулер, UML. Основы}
				\end{center}
			\end{column}
		\end{columns}
	\end{frame}

	\section{Диаграмма развёртывания}
	
	\begin{frame}
		\frametitle{Диаграмма развёртывания UML}
		\begin{columns}
			\begin{column}{0.5\textwidth}
				\begin{itemize}
					\item Показывает отображение компонентов и физических артефактов на реальные (или виртуальные) устройства
					\item Бывает полезна на начальных этапах проектирования, даже до диаграмм компонентов
				\end{itemize}
			\end{column}
			\begin{column}{0.5\textwidth}
				\begin{center}
					\includegraphics[width=\textwidth]{deploymentDiagram.png}
					\attribution{М. Фаулер, UML. Основы}
				\end{center}
			\end{column}
		\end{columns}
	\end{frame}

	\section{Диаграммы ``Сущность-связь''}

	\begin{frame}
		\frametitle{Диаграммы ``Сущность-связь''}
		\begin{columns}
			\begin{column}{0.5\textwidth}
				\begin{itemize}
					\item Описывают концептуальную модель предметной области
					\item Идеальны для моделирования схем реляционных баз данных
					\item 1976 год, Питер Чен
				\end{itemize}
			\end{column}
			\begin{column}{0.5\textwidth}
				\begin{center}
					\includegraphics[width=\textwidth]{erChenNotation.png}
					\attribution{https://ru.wikipedia.org}
				\end{center}
			\end{column}
		\end{columns}
	\end{frame}

	\begin{frame}
		\frametitle{Нотация ``Вороньей лапки''}
		\begin{center}
			\includegraphics[width=\textwidth]{erCrowsFoot.png}
		\end{center}
	\end{frame}

	\section{Диаграммы конечных авоматов}

	\begin{frame}
		\frametitle{Диаграммы конечных автоматов}
		\framesubtitle{Диаграммы состояний}
		\begin{columns}
			\begin{column}{0.5\textwidth}
				\begin{itemize}
					\item Состояния объекта как часть жизненного цикла
					\item Моделирование реактивных объектов
					\begin{itemize}
						\item Например, сетевое соединение
						\item Или знакомый пример с торговым автоматом
					\end{itemize}
					\item Имеют исполнимую семантику
					\item Д. Харел, 1987
				\end{itemize}
			\end{column}
			\begin{column}{0.5\textwidth}
				\begin{center}
					\includegraphics[width=0.7\textwidth]{stateTransitionExample.png}
				\end{center}
			\end{column}
		\end{columns}
	\end{frame}

	\begin{frame}
		\frametitle{Диаграммы конечных автоматов, особенности}
		Активности:
		\begin{columns}
			\begin{column}{0.5\textwidth}
				\begin{center}
					\includegraphics[width=\textwidth]{stateTransitionInternalEventExample.png}
				\end{center}
			\end{column}
			\begin{column}{0.5\textwidth}
				\begin{center}
					\includegraphics[width=0.5\textwidth]{stateTransitionInternalEvents.png}
				\end{center}
			\end{column}
		\end{columns}

		\begin{columns}
			\begin{column}{0.5\textwidth}
				Вложенные состояния:
				\begin{center}
					\includegraphics[width=\textwidth]{stateTransitionNestedStates.png}
					\attribution{М. Фаулер, UML. Основы}
				\end{center}
			\end{column}
			\begin{column}{0.5\textwidth}
				Параллельные состояния, псевдосостояние истории:
				\begin{center}
					\includegraphics[width=0.7\textwidth]{stateTransitionParallelStates.png}
				\end{center}
			\end{column}
		\end{columns}
	\end{frame}

	\begin{frame}[fragile]
		\frametitle{Генерация кода}
		\begin{columns}
			\begin{column}{0.5\textwidth}
				\begin{tiny}
					\begin{minted}{java}
public void handleEvent(PanelEvent anEvent) {
    switch (currentState) {
        case PanelState.Open:
            switch (anEvent) {
                case PanelEvent.SafeClosed:
                    currentState = PanelState.Wait;
            }
            break;
        case PanelState.Wait:
            switch (anEvent) {
                case PanelEvent.CandleRemoved:
                    if (isDoorOpen) {
                        revealLock();
                        currentState = PanelState.Lock;
                    }
            }
            break;
        case PanelState.Lock:
            switch (anEvent) {
                case PanelEvent.KeyTurned:
                    if (isCandleIn) {
                        openSafe();
                        currentState = PanelState.Open;
                    } else {
                        releaseKillerRabbit();
                        currentState = PanelState.Final;
                    }
            }
            break;
    }
}
					\end{minted}
				\end{tiny}
			\end{column}
			\begin{column}{0.5\textwidth}
				\begin{center}
					\includegraphics[width=\textwidth]{stateTransitionSyntax.png}
					\attribution{М. Фаулер, UML. Основы}
				\end{center}
			\end{column}
		\end{columns}
	\end{frame}

	\begin{frame}
		\frametitle{Таблица состояний}
		\begin{center}
			\includegraphics[width=0.4\textwidth]{stateTransitionSyntax.png}
		\end{center}

		\begin{center}
			\includegraphics[width=0.5\textwidth]{stateTransitionStateTable.png}
			\attribution{М. Фаулер, UML. Основы}
		\end{center}
	\end{frame}

	\begin{frame}
		\frametitle{Паттерн ``Состояние''}
		\begin{center}
			\includegraphics[width=0.4\textwidth]{stateTransitionSyntax.png}
		\end{center}

		\begin{center}
			\includegraphics[width=0.5\textwidth]{stateTransitionStatePattern.png}
			\attribution{М. Фаулер, UML. Основы}
		\end{center}
	\end{frame}

	\section{Диаграммы последовательностей}

	\begin{frame}
		\frametitle{Диаграммы последовательностей}
		\begin{columns}
			\begin{column}{0.5\textwidth}
				\begin{itemize}
					\item Применяются для визуализации взаимодействия между объектами
					\begin{itemize}
						\item Особо удобно для асинхронных вызовов
						\item Телекоммуникационные протоколы
					\end{itemize}
					\item Могут применяться на этапе анализа предметной области
					\item Могут применяться для составления плана тестирования
					\item И даже для визуализации логов работающей системы
				\end{itemize}
			\end{column}
			\begin{column}{0.5\textwidth}
				\begin{center}
					\includegraphics[width=0.9\textwidth]{sequenceDiagramSyntax.png}
					\attribution{М. Фаулер, UML. Основы}
				\end{center}
			\end{column}
		\end{columns}
	\end{frame}

	\section{Коммуникационные диаграммы}

	\begin{frame}
		\frametitle{Коммуникационные диаграммы}
		\begin{columns}
			\begin{column}{0.5\textwidth}
				\begin{itemize}
					\item Применяются для визуализации взаимодействия между объектами
					\begin{itemize}
						\item Более легковесный аналог диаграмм последовательностей
						\item Тоже отображают один сценарий взаимодействия
					\end{itemize}
				\end{itemize}
			\end{column}
			\begin{column}{0.5\textwidth}
				\begin{center}
					\includegraphics[width=\textwidth]{communicationDiagram.png}
					\attribution{М. Фаулер, UML. Основы}
				\end{center}
			\end{column}
		\end{columns}
	\end{frame}

	\begin{frame}
		\frametitle{Коммуникационные диаграммы, пример}
		\begin{center}
			\includegraphics[width=0.6\textwidth]{communicationDiagramExample.png}
			\attribution{http://www.uml-diagrams.org/}
		\end{center}
	\end{frame}

	\section{Диаграммы составных структур}

	\begin{frame}
		\frametitle{Диаграммы составных структур}
		\begin{columns}
			\begin{column}{0.5\textwidth}
				\begin{itemize}
					\item По сути, продвинутые диаграммы компонентов
					\item Внутри компоненты не другие компоненты, а части (роли)
				\end{itemize}
				\vspace{3mm}
				\begin{center}
					\includegraphics[width=0.9\textwidth]{compositeStructureElement.png}
				\end{center}
			\end{column}
			\begin{column}{0.5\textwidth}
				\begin{center}
					\includegraphics[width=0.8\textwidth]{compositeStructureDiagram.png}
					\attribution{М. Фаулер, UML. Основы}
				\end{center}
			\end{column}
		\end{columns}
	\end{frame}

	\section{Диаграммы коопераций}

	\begin{frame}
		\frametitle{Диаграммы коопераций}
		\begin{columns}
			\begin{column}{0.5\textwidth}
				\begin{itemize}
					\item Показывают взаимодействие между объектами (ролями) в рамках одного сценария использования
				\end{itemize}
				\vspace{3mm}
				\begin{center}
					\includegraphics[width=0.9\textwidth]{cooperationAlternateNotation.png}
				\end{center}
			\end{column}
			\begin{column}{0.5\textwidth}
				\begin{center}
					\includegraphics[width=0.9\textwidth]{cooperationDiagram.png}
					\attribution{М. Фаулер, UML. Основы}
				\end{center}
			\end{column}
		\end{columns}
	\end{frame}

	\section{Временные диаграммы}

	\begin{frame}
		\frametitle{Временные диаграммы}
		\begin{columns}
			\begin{column}{0.5\textwidth}
				\begin{itemize}
					\item Для моделирования временных ограничений в системах реального времени
				\end{itemize}
				\vspace{3mm}
				\begin{center}
					\includegraphics[width=0.9\textwidth]{timingDiagrams.png}
				\end{center}
			\end{column}
			\begin{column}{0.5\textwidth}
				\begin{center}
					\includegraphics[width=0.8\textwidth]{timingDiagramsAlternate.png}
					\attribution{М. Фаулер, UML. Основы}
				\end{center}
			\end{column}
		\end{columns}
	\end{frame}

	\begin{frame}
		\frametitle{Временная диаграмма, пример}
		\begin{center}
			\includegraphics[width=0.7\textwidth]{timingDiagramExample.png}
			\attribution{http://www.uml-diagrams.org/}
		\end{center}
	\end{frame}

	\section{Диаграммы обзора взаимодействия}

	\begin{frame}
		\frametitle{Диаграммы обзора взаимодействия}
		\begin{columns}
			\begin{column}{0.5\textwidth}
				\begin{itemize}
					\item Диаграммы активностей + диаграммы последовательностей
					\item Применяются при наличии взаимодействия со сложной логикой, когда фреймы неудобны
				\end{itemize}
			\end{column}
			\begin{column}{0.5\textwidth}
				\begin{center}
					\includegraphics[width=0.9\textwidth]{interactionOverviewDiagrams.png}
					\attribution{М. Фаулер, UML. Основы}
				\end{center}
			\end{column}
		\end{columns}
	\end{frame}

	\begin{frame}
		\frametitle{Диаграмма обзора взаимодействия, пример}
		\begin{center}
			\includegraphics[width=0.7\textwidth]{interactionOverviewExample.png}
			\attribution{http://www.uml-diagrams.org/}
		\end{center}
	\end{frame}

	\section{Диаграммы потоков данных}

	\begin{frame}
		\frametitle{Диаграммы потоков данных}
		\framesubtitle{DFD}
		\begin{columns}
			\begin{column}{0.4\textwidth}
				\begin{itemize}
					\item Показывают обмен данными в системе
					\item Внешние сущности, процессы внутри системы, потоки данных
				\end{itemize}
			\end{column}
			\begin{column}{0.6\textwidth}
				\begin{center}
					\includegraphics[width=0.9\textwidth]{dfd.png}
				\end{center}
			\end{column}
		\end{columns}
	\end{frame}
	
	\section{Диаграммы IDEF0}
	
	\begin{frame}
		\frametitle{Диаграммы IDEF0}
		\begin{center}
			\includegraphics[width=0.80\textwidth]{idef0.png}
			\attribution{https://habrahabr.ru/post/322832/}
		\end{center}
	\end{frame}

	\section{CASE-инструменты}

	\begin{frame}
		\frametitle{Примеры CASE-инструментов}
		\begin{itemize}
			\item ``Рисовалки''
			\begin{itemize}
				\item Visio
				\item Dia
				\item SmartDraw
				\item LucidChart
				\item Creately
			\end{itemize}
			\item Полноценные CASE-системы
			\begin{itemize}
				\item Enterprise Architect
				\item Rational Software Architect
				\item MagicDraw
				\item Visual Paradigm
				\item GenMyModel
			\end{itemize}
			\item Забавные штуки
			\begin{itemize}
				\item \url{https://www.websequencediagrams.com/}
				\item \url{http://yuml.me/}
				\item \url{http://plantuml.com/}
			\end{itemize}
		\end{itemize}
	\end{frame}

	\section{Заключение}

	\begin{frame}
		\frametitle{Книжка}
		\begin{columns}
			\begin{column}{0.5\textwidth}
				\begin{center}
					\includegraphics[width=0.4\textwidth]{umlBookCover.png}
				\end{center}
			\end{column}
			\begin{column}{0.5\textwidth}
				М. Фаулер, UML. Основы. Краткое руководство по стандартному языку объектного моделирования. СПб., Символ-Плюс, 2011. 192 С.
			\end{column}
		\end{columns}
	\end{frame}

\end{document}
