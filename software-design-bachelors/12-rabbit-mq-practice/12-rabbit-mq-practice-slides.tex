\documentclass[xetex,mathserif,serif]{beamer}
\usepackage{polyglossia}
\setdefaultlanguage[babelshorthands=true]{russian}
\usepackage{minted}
\usepackage{tabu}
\usepackage[11pt]{moresize}

\useoutertheme{infolines}

\usepackage{fontspec}
\setmainfont{FreeSans}
\newfontfamily{\russianfonttt}{FreeSans}

\usepackage{textpos}
\setlength{\TPHorizModule}{1cm}
\setlength{\TPVertModule}{1cm}

\definecolor{links}{HTML}{2A1B81}
\hypersetup{colorlinks,linkcolor=,urlcolor=links}

\tabulinesep=0.7mm

\title{Распределённые приложения, RabbitMQ}
\author[Юрий Литвинов]{Юрий Литвинов \newline \textcolor{gray}{\small\texttt{yurii.litvinov@gmail.com}}}

\date{10.05.2018г}

\begin{document}

	\frame{\titlepage}

	\section{RabbitMQ}

	\begin{frame}
		\frametitle{RabbitMQ}
		\begin{itemize}
			\item Сервер и клиенты системы надёжной передачи сообщений
			\begin{itemize}
				\item Сообщение посылается на сервер и хранится там, пока его не заберут
				\item Продвинутые возможности по маршрутизации сообщений
			\end{itemize}
			\item Реализует протокол AMQP (Advanced Message Queuing Protocol), но может использовать и другие протоколы
			\item Сервер написан на Erlang, клиентские библиотеки доступны для практически чего угодно
		\end{itemize}
		\begin{textblock}{3}(8,0)
			\includegraphics[width=\textwidth]{rabbitmqLogo.png}
		\end{textblock}
	\end{frame}

	\begin{frame}[fragile]
		\frametitle{Пример, отправитель}
		\begin{ssmall}
			\begin{minted}{csharp}
using System;
using RabbitMQ.Client;
using System.Text;

class Send
{
    public static void Main()
    {
        var factory = new ConnectionFactory() { HostName = "localhost" };
        using (var connection = factory.CreateConnection())
        {
            using (var channel = connection.CreateModel())
            {
                channel.QueueDeclare(queue: "hello", durable: false, exclusive: false,
                                     autoDelete: false, arguments: null);

                string message = "Hello World!";
                var body = Encoding.UTF8.GetBytes(message);

                channel.BasicPublish(exchange: "", routingKey: "hello",
                                     basicProperties: null, body: body);
            }
        }
    }
}
			\end{minted}
		\end{ssmall}
	\end{frame}

	\begin{frame}[fragile]
		\frametitle{Пример, получатель}
		\begin{ssmall}
			\begin{minted}{csharp}
using RabbitMQ.Client;
using RabbitMQ.Client.Events;
using System;
using System.Text;

class Receive
{
    public static void Main()
    {
        var factory = new ConnectionFactory() { HostName = "localhost" };
        using (var connection = factory.CreateConnection())
        using (var channel = connection.CreateModel())
        {
            channel.QueueDeclare(queue: "hello", durable: false, exclusive: false, autoDelete: false, arguments: null);

            var consumer = new EventingBasicConsumer(channel);
            consumer.Received += (model, ea) =>
            {
                var body = ea.Body;
                var message = Encoding.UTF8.GetString(body);
                Console.WriteLine(" [x] Received {0}", message);
            };
            channel.BasicConsume(queue: "hello", autoAck: true, consumer: consumer);
        }
    }
}
			\end{minted}
		\end{ssmall}
	\end{frame}

	\section{Задача}

	\begin{frame}
		\frametitle{Задача на пару}
		Переделать сетевой чат на RabbitMQ
		\begin{itemize}
			\item Сервер для обмена сообщениями, о котором договариваются клиенты
			\begin{itemize}
				\item Центральный сервер, задаваемый в настройках клиента
			\end{itemize}
			\item Нет списка контактов, есть именованные каналы, на которые можно подписываться и постить туда
			\item (Если успеем) должна быть нотификация о печати
			\begin{itemize}
				\item Если один клиент печатает, остальные подписчики канала должны видеть у себя в клиенте, кто сейчас печатает
			\end{itemize}
		\end{itemize}
	\end{frame}

\end{document}