\documentclass[a5paper]{homework}

\begin{document}

\makeHomeworkHeading{
    title = {Домашняя работа 11. Roguelike, часть 4},
    publicationDate = {28.03.2022},
    deadline = {11.04.2022},
    score = {10}
}

Наконец, последняя домашняя работа про Roguelike. В этом задании надо применить поведенческие шаблоны для того, чтобы зарефакторить взаимодействие с пользователем и улучшить ИИ мобов:

\begin{itemize}
    \item Паттерн <<Команда>> для реализации взаимодействия с пользователем --- напрямую обращаться из UI к классам бизнес-логики теперь нельзя, все действия должны выполняться строго через команды. Использовать <<умные>> или <<глупые>> команды --- на ваше усмотрение.
    \item Паттерн <<Состояние>> для придания <<стратегичности>> действиям моба. Напомним, что в одном из предыдущих заданий требовалось реализовать стратегии поведения. Сейчас поведение моба должно определяться его состоянием:
    \begin{itemize}
        \item если здоровье моба ниже определённого уровня, он переходит в состояние <<паника>>, в котором его текущая стратегия заменяется на трусливую (если изначально стратегия была трусливой, никаких изменений в поведении моба нет);
        \item при восстановлении здоровья выше этого уровня моб переходит из состояния <<паника>> в состояние <<всё ок>>, где переключается на исходную стратегию;
        \item +3 бонусных балла за реализацию дополнительных стратегий и состояний (например, <<патрулирую>>, <<движусь в место последнего контакта с врагом>>, <<выполняю поиск в районе последнего контакта>>, <<атакую>>).
    \end{itemize}
\end{itemize}

Как обычно, задача сдаётся отдельным пуллреквестом от ветки с предыдущей задачей.

Также, как обычно, надо обновить архитектурную документацию, включив туда описание новой функциональности.

\end{document}
