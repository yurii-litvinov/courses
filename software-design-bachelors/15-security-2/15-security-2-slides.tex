\documentclass[xetex,mathserif,serif]{beamer}
\usepackage{polyglossia}
\setdefaultlanguage[babelshorthands=true]{russian}
\usepackage{minted}
\usepackage{tabu}
\usepackage[11pt]{moresize}

\useoutertheme{infolines}

\usepackage{fontspec}
\setmainfont{FreeSans}
\newfontfamily{\russianfonttt}{FreeSans}

\usepackage{textpos}
\setlength{\TPHorizModule}{1cm}
\setlength{\TPVertModule}{1cm}

\definecolor{links}{HTML}{2A1B81}
\hypersetup{colorlinks,linkcolor=,urlcolor=links}

\tabulinesep=0.7mm

\title{Практика по сетевой безопасности}
\author[Юрий Литвинов]{Юрий Литвинов \newline \textcolor{gray}{\small\texttt{yurii.litvinov@gmail.com}}}

\newcommand{\attribution}[1] {
\vspace{-5mm}\begin{flushright}\begin{scriptsize}\textcolor{gray}{\textcopyright\, #1}\end{scriptsize}\end{flushright}
}

\date{20.05.2020г}

\begin{document}

    \frame{\titlepage}

    \begin{frame}
        \frametitle{Задача 1}
        \begin{footnotesize}
            Прикрутить аутентификацию по X.509-сертификатам к чатам на RabbitMQ с прошлой практики
            \begin{itemize}
                \item Подробная инструкция: \url{https://www.rabbitmq.com/ssl.html}
                \item Аутентификация обоюдная: сервер проверяет клиент, клиент проверяет сервер
                \item Обратите внимание:
                \begin{itemize}
                    \item У JVM своя (довольно удобная вроде) подсистема работы с сертификатами, у .NET тоже
                    \item Может быть полезно \url{https://www.rabbitmq.com/troubleshooting-ssl.html}
                    \item В общем случае генерим самоподписанный CA-сертификат и генерим с его помощью сертификат сервера
                    \item Можно и клиента (или отдельный CA для клиента, но сервер должен ему доверять)
                    \item См. \url{https://www.rabbitmq.com/ssl.html\#peer-verification-trusted-certificates}
                \end{itemize}
            \end{itemize}
        \end{footnotesize}
    \end{frame}

    \begin{frame}
        \frametitle{Задача 2}
        Реализовать на ``голом HTTPS'' консольный клиент Google Drive, умеющий показать список файлов и папок в корне Google Drive пользователя
        \begin{itemize}
            \item Нельзя использовать библиотеку клиента Google Drive
            \item Придётся зарегистрировать приложение на \url{https://console.developers.google.com}
            \item И включить нужный API
            \item Авторизация: \url{https://developers.google.com/drive/api/v3/about-auth}
            \begin{itemize}
                \item Наш случай: \url{https://developers.google.com/identity/protocols/oauth2/native-app}
            \end{itemize}
            \item Надо добавить OAuth Client ID и получить Client ID и Client secret
        \end{itemize}
    \end{frame}

    \begin{frame}
        \frametitle{Задача 2}
        \framesubtitle{Подсказки}
        \begin{itemize}
            \item Запрос access token должен быть обязательно POST-запросом, GET вернёт 404
            \item Копипастить access token из адресной строки браузера good enough
            \begin{itemize}
                \item Да, надо будет программно запустить браузер с правильным URL
            \end{itemize}
            \item Некоторые значения возвращаются в BASE64-кодировке, а ожидаются в plain text
            \item Не надо продлять Access Token, правильно хранить Refresh Token и всё такое
            \item Значения Client Id и Client Secret можно принимать параметрами командной строки
        \end{itemize}
    \end{frame}

    \begin{frame}
        \frametitle{Что делать}
        \begin{itemize}
            \item Любую задачу на выбор, бонусные баллы, если успеете две
            \item Выложить команду и ссылку на репозиторий в \url{https://forms.gle/NpEYPbcrfLe7qy2v7}
            \item За 20 минут до конца собираемся и показываем результаты
            \begin{itemize}
                \item Надо показать, что всё работает --- иначе потом не проверить
            \end{itemize}
            \item Как обычно, посматривайте в Telegram
        \end{itemize}
    \end{frame}

\end{document}