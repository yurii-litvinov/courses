\documentclass{../../slides-style}

\slidetitle{Лекция 3: Моделирование, UML}{27.02.2023}

\begin{document}

    \begin{frame}[plain]
        \titlepage
    \end{frame}

    \section{Модели}

    \begin{frame}
        \frametitle{Моделирование}
        \begin{itemize}
            \item \textbf{Модель} --- упрощённое подобие объекта или явления
            \item Нужны для изучения некоторых их свойств, абстрагируясь от сложности ``настоящего'' объекта или явления
            \item Модели используются повсеместно
            \begin{itemize}
                \item Математические модели
                \item Модели как реальные объекты
                \item Модели в разработке ПО
            \end{itemize}
        \end{itemize}
    \end{frame}

    \begin{frame}
        \frametitle{Общие свойства моделей}
        \begin{itemize}
            \item Содержат меньше информации, чем реальность
            \item Существуют для определённой цели
            \item Модели субъективны, что позволяет отделить существенные свойства от несущественных
            \item Модели ограничены
        \end{itemize}

        \center{\textbf{All models are wrong, some are useful}}
    \end{frame}

    \begin{frame}
        \frametitle{Моделирование ПО}
        \begin{itemize}
            \item Предназначены прежде всего для управления сложностью
            \item Могут моделировать как саму систему, так и окружение
            \item Позволяют понять, проанализировать и протестировать систему до её реализации
        \end{itemize}
        \begin{center}
            \includegraphics[width=0.4\textwidth]{vendingMachine.png}
            \attribution{N. Medvidovic}
        \end{center}
    \end{frame}

    \begin{frame}
        \frametitle{Модели бывают разные}
        \begin{itemize}
            \item Используемые нотации и способы моделирования зависят от целей моделирования
            \begin{itemize}
                \item От неформальных набросков до исполнимых моделей
            \end{itemize}
        \end{itemize}
        \begin{center}
            \includegraphics[width=0.9\textwidth]{sketchesVsFormalNotations.png}
            \attribution{N. Medvidovic}
        \end{center}
    \end{frame}

    \begin{frame}
        \frametitle{Архитектурные модели}
        \begin{itemize}
            \item Архитектура --- это набор основных решений, принятых для данной системы
            \item Архитектурная модель --- это некоторый артефакт, который отражает некоторые или все эти решения
            \item Архитектурное моделирование --- это процесс уточнения и документирования этих решений
            \item Моделирование непосредственно связано с используемой нотацией
            \begin{itemize}
                \item Нотация архитектурного моделирования --- это язык или другое средство описания архитектурных решений
            \end{itemize}
        \end{itemize}
    \end{frame}

    \begin{frame}
        \frametitle{Как выбрать, что моделировать?}
        \begin{itemize}
            \item При моделировании надо определиться с:
            \begin{itemize}
                \item Какие архитектурные решения нуждаются в моделировании
                \item На каком уровне детализации
                \item Насколько формально
            \end{itemize}
            \item Необходимо учитывать соотношение трудозатрат и выгоды
            \begin{itemize}
                \item Стоимость создания \textit{и поддержания} модели не должна быть больше преимуществ от её использования
            \end{itemize}
        \end{itemize}
        \begin{center}
            \includegraphics[width=0.4\textwidth]{concerns.png}
            \attribution{N. Medvidovic}
        \end{center}
    \end{frame}

    \begin{frame}
        \frametitle{Возможные преимущества моделей}
        \begin{itemize}
            \item Инструмент, направляющий и облегчающий проектирование
            \item Средство коммуникации между разработчиками
            \item Наглядный инструмент для общения с заказчиком
            \item Средство документирования и фиксации принятых решений
            \item Исходник для генерации кода?
        \end{itemize}
        \begin{center}
            \includegraphics[width=0.4\textwidth]{sequenceDiagram.png}
        \end{center}
    \end{frame}

    \section{Виды моделей}

    \begin{frame}
        \frametitle{Виды моделей}
        \framesubtitle{Естественные языки}
        \begin{columns}
            \begin{column}{0.5\textwidth}
                \begin{itemize}
                    \item Обычный текст --- вполне себе инструмент моделирования
                    \item Очень выразителен, не требует специальных знаний, максимально гибок
                    \item Неоднозначен, неформален, не строг, слишком многословен, бесполезен для автоматической обработки
                \end{itemize}
            \end{column}
            \begin{column}{0.5\textwidth}
                \begin{center}
                    \includegraphics[width=\textwidth]{naturalLanguage.png}
                    \attribution{N. Medvidovic}
                \end{center}
            \end{column}
        \end{columns}
    \end{frame}

    \begin{frame}
        \frametitle{Неформальные графические модели}
        \begin{columns}
            \begin{column}{0.5\textwidth}
                \begin{itemize}
                    \item Диаграммы, рисуемые в PowerPoint, InkScape и подобном
                    \item Могут быть красивыми, как правило, простые, очень гибкая нотация
                    \item Неформальны, неоднозначны, не строги
                    \begin{itemize}
                        \item Но часто воспринимаются наоборот
                    \end{itemize}
                    \item Практически бесполезны для автоматической обработки
                \end{itemize}
            \end{column}
            \begin{column}{0.5\textwidth}
                \begin{center}
                    \includegraphics[width=0.9\textwidth]{informalModel.png}
                    \attribution{N. Medvidovic}
                \end{center}
            \end{column}
        \end{columns}
    \end{frame}

    \begin{frame}
        \frametitle{UML и SysML}
        \begin{columns}
            \begin{column}{0.4\textwidth}
                \begin{small}
                    \begin{itemize}
                        \item Несколько слабо связанных нотаций (``диаграмм'')
                        \item Поддерживают много точек зрения, общеприняты, широкая поддержка инструментами
                        \item Нет строгой семантики, сложно обеспечить консистентность, сложно расширять
                    \end{itemize}
                \end{small}
            \end{column}
            \begin{column}{0.6\textwidth}
                \begin{center}
                    \includegraphics[width=\textwidth]{uml.png}
                    \attribution{N. Medvidovic}
                \end{center}
            \end{column}
        \end{columns}
    \end{frame}

    \begin{frame}
        \frametitle{AADL\footnote{Architecture Analysis \& Design Language} и другие текстовые формальные языки}
        \begin{columns}
            \begin{column}{0.4\textwidth}
                \begin{small}
                    \begin{itemize}
                        \item Хороши для моделирования встроенных систем и систем реального времени
                        \item Описывают одновременно ``железо'' и ``софт'', продвинутые инструменты анализа
                        \item Слишком многословны и детальны, сложны в изучении и использовании
                    \end{itemize}
                \end{small}
            \end{column}
            \begin{column}{0.6\textwidth}
                \begin{center}
                    \includegraphics[width=0.85\textwidth]{aadl.png}
                    \attribution{N. Medvidovic}
                \end{center}
            \end{column}
        \end{columns}
    \end{frame}

    \section{UML}

    \begin{frame}
        \frametitle{Вернёмся к визуальным моделям}
        \begin{itemize}
            \item \textbf{Метафора визуализации} --- договорённость о том, как будут представляться сущности языка
            \item \textbf{Точка зрения моделирования} --- какой аспект системы и для кого моделируется
            \item Бывают одноразовые модели, документация и графические исходники
            \begin{itemize}
                \item \textbf{Семантический разрыв} --- неспособность модели полностью специфицировать систему
            \end{itemize}
        \end{itemize}
        \begin{center}
            \includegraphics[width=0.5\textwidth]{hashTable.png}
        \end{center}
    \end{frame}

    \begin{frame}
        \frametitle{Unified Modeling Language}
        \begin{itemize}
            \item Семейство графических нотаций
            \begin{itemize}
                \item 14 видов диаграмм
            \end{itemize}
            \item Общая метамодель
            \item Стандарт под управлением Object Management Group
            \begin{itemize}
                \item UML 1.1 --- 1997 год
                \item UML 2.0 --- 2005 год
                \item UML 2.5.1 --- декабрь 2017 года
            \end{itemize}
            \item Прежде всего, для проектирования ПО
            \begin{itemize}
                \item После UML 2.0 стали появляться нотации и для инженеров
            \end{itemize}
            \item Расширяем
            \begin{itemize}
                \item Профили --- механизм легковесного расширения
                \item Метамоделирование
            \end{itemize}
        \end{itemize}
    \end{frame}

    \begin{frame}
        \frametitle{История}
        \begin{center}
            \includegraphics[width=\textwidth]{umlHistory.png}
        \end{center}
    \end{frame}

    \begin{frame}
        \frametitle{Виды диаграмм}
        \begin{center}
            \includegraphics[width=\textwidth]{umlDiagrams.png}
        \end{center}
    \end{frame}

    \section{Диаграмма классов UML}

    \begin{frame}
        \frametitle{Диаграмма классов}
        \begin{center}
            \includegraphics[height=0.8\textheight]{umlClassDiagram.png}
            \attribution{М. Фаулер. ``UML. Основы''}
        \end{center}
    \end{frame}

    \begin{frame}
        \frametitle{Свойства}
        \begin{columns}
            \begin{column}{0.5\textwidth}
                Атрибуты:
                \begin{center}
                    \includegraphics[width=0.5\textwidth]{attributes.png}
                \end{center}
            \end{column}
            \begin{column}{0.5\textwidth}
                Ассоциация-класс:
                \begin{center}
                    \includegraphics[width=0.85\textwidth]{classAssociation.png}
                \end{center}
            \end{column}
        \end{columns}
        \vspace{5mm}
        Ассоциации:
        \begin{center}
            \includegraphics[width=0.5\textwidth]{associations.png}
        \end{center}
        \attribution{М. Фаулер. ``UML. Основы''}
    \end{frame}

    \begin{frame}
        \frametitle{Синтаксис свойств}
        \begin{itemize}
            \item Объявление поля:
            \begin{itemize}
                \item \texttt{видимость имя: тип кратность = значение по умолчанию \{строка свойств\}}
            \end{itemize}
            \item Видимость:
            \begin{itemize}
                \item \texttt{$+$ (public), $-$ (private), \# (protected), \textasciitilde (package)}
            \end{itemize}
            \item Кратность:
            \begin{itemize}
                \item \texttt{1} (ровно 1 объект), \texttt{0..1} (ни одного или один), \texttt{*} (сколько угодно), \texttt{1..*}, \texttt{2..*}
            \end{itemize}
        \end{itemize}
    \end{frame}

    \begin{frame}[fragile]
        \frametitle{Как это связано с кодом}
        \begin{columns}
            \begin{column}{0.5\textwidth}
                \begin{footnotesize}
                    \begin{minted}{java}
public class OrderLine {
    private int quantity;
    private Product product;
    public int getQuantity() {
        return quantity;
    }
    public void setQuantity(int quantity) {
        this.quantity = quantity;
    }
    public Money getPrice() {
        return product.getPrice().multiply(quantity);
    }
}
                    \end{minted}
                \end{footnotesize}
            \end{column}
            \begin{column}{0.5\textwidth}
                \begin{center}
                    \includegraphics[width=0.5\textwidth]{orderLine.png}
                \end{center}
            \end{column}
        \end{columns}
    \end{frame}

    \begin{frame}[fragile]
        \frametitle{Двунаправленные ассоциации}
        \begin{columns}
            \begin{column}{0.5\textwidth}
                \begin{scriptsize}
                    \begin{center}
                        \includegraphics[width=0.9\textwidth]{twoWayAssociations.png}
                    \end{center}
    
                    \begin{minted}{csharp}
class Car {
    public Person Owner {
        get { return _owner; }
        set {
            if (_owner != null) 
                _owner.friendCars().Remove(this);
            _owner = value;
            if (_owner != null) 
                _owner.friendCars().Add(this);
        }
    }
    private Person _owner;
}
                    \end{minted}
                \end{scriptsize}
                \vspace{2mm}
            \end{column}
            \begin{column}{0.5\textwidth}
                \begin{scriptsize}
                    \begin{center}
                        \includegraphics[width=0.9\textwidth]{personOwnsCar.png}
                    \end{center}
    
                    \begin{minted}{csharp}
class Person {
    public IList Cars {
        get { return ArrayList.ReadOnly(_cars); }
    }
    public void AddCar(Car arg) {
        arg.Owner = this;
    }
    private IList _cars = new ArrayList();
    internal IList friendCars() {
        // должен быть использован 
        // только Car.Owner
        return _cars;
    }
}
                    \end{minted}
                \end{scriptsize}
            \end{column}
        \end{columns}
    \end{frame}

    \begin{frame}
        \frametitle{Интерфейсы}
        \begin{center}
            \includegraphics[width=0.6\textwidth]{interfaces.png}
            \attribution{М. Фаулер. ``UML. Основы''}
        \end{center}
    \end{frame}

    \begin{frame}
        \frametitle{Зависимости}
        \begin{columns}
            \begin{column}{0.5\textwidth}
                \begin{itemize}
                    \item call
                    \item create
                    \item instantiate
                    \item derive 
                \end{itemize}
            \end{column}
            \begin{column}{0.5\textwidth}
                \begin{itemize}
                    \item realize
                    \item responsibility
                    \item refine 
                    \item trace
                \end{itemize}
            \end{column}
        \end{columns}
        \vspace{7mm}
        \begin{center}
            \includegraphics[width=0.5\textwidth]{dependencies.png}
            \attribution{М. Фаулер. ``UML. Основы''}
        \end{center}
    \end{frame}

    \begin{frame}
        \frametitle{Агрегация и композиция}
        Агрегация:
        \begin{center}
            \includegraphics[height=0.1\textheight]{aggregation.png}
        \end{center}
        \vspace{5mm}
        Композиция:
        \begin{center}
            \includegraphics[height=0.1\textheight]{composition.png}
        \end{center}
        \attribution{М. Фаулер. ``UML. Основы''}
    \end{frame}

    \begin{frame}
        \frametitle{Агрегация и композиция, пример}
        \begin{center}
            \includegraphics[width=0.7\textwidth]{aggregationAndCompositionExample.png}
            \attribution{М. Фаулер. ``UML. Основы''}
        \end{center}
    \end{frame}

    \begin{frame}
        \frametitle{Шаблоны и перечисления}
        \begin{center}
            \includegraphics[width=0.95\textwidth]{genericsAndEnums.png}
            \attribution{М. Фаулер. ``UML. Основы''}
        \end{center}
    \end{frame}

    \section{Диаграммы пакетов}

    \begin{frame}
        \frametitle{Диаграммы пакетов}
        \begin{center}
            \includegraphics[width=0.8\textwidth]{packageDiagrams.png}
            \attribution{М. Фаулер. ``UML. Основы''}
        \end{center}
    \end{frame}

    \begin{frame}
        \frametitle{Диаграммы пакетов, зависимости}
        \begin{center}
            \includegraphics[width=0.95\textwidth]{packageDependencies.png}
            \attribution{М. Фаулер. ``UML. Основы''}
        \end{center}
    \end{frame}

    \section{Диаграммы объектов}

    \begin{frame}
        \frametitle{Диаграммы объектов}
        \begin{center}
            \includegraphics[width=0.9\textwidth]{objectDiagrams.png}
            \attribution{М. Фаулер. ``UML. Основы''}
        \end{center}
    \end{frame}

    \section{Диаграммы компонентов}
    
    \begin{frame}
        \frametitle{Диаграммы компонентов}
        \begin{center}
            \includegraphics[width=0.95\textwidth]{componentDiagrams.png}
        \end{center}
    \end{frame}

    \begin{frame}
        \frametitle{Более подробно}
        \begin{center}
            \includegraphics[width=0.95\textwidth]{componentDiagramsOverview.png}
            \attribution{\url{http://www.uml-diagrams.org}}
        \end{center}
    \end{frame}

\end{document}
