\documentclass{../../text-style}

\texttitle{Практика 13: REST-сервис}

\begin{document}

\maketitle
\thispagestyle{empty}

\section{Задание на практику}

В командах по два человека спроектировать и реализовать REST-сервис для <<открытой части>> магазина книг, спроектированного двумя неделями ранее\footnote{Напоминаем, ТЗ: \url{https://goo.gl/94LyFc}.}. Проектируем пока только <<открытую>> часть сервиса, не требующую аутентификации и, соответственно, не изменяющую состояния сервиса (так что требуется реализовать только GET-запросы).

Должен быть поддержан запрос информации о книгах, включая отзывы (пользовательские и редакторские) и поиск по книгам. Представьте, что этим API будут пользоваться сторонние сервисы (например, чужие магазины) или клиент, показывающий пользователю интересующие его новинки.

При этом, для простоты, слой доступа к данным должен возвращать захардкоженные данные, работать с реальной СУБД не надо. Бизнес-логика должна быть честно реализована, то есть поиск по критериям должен работать <<честно>>, хоть и вручную (без SQL и подобных штук).

Для тестирования сервиса используйте Swagger --- при этом сервис должен предоставлять метаинформацию о себе в формате OpenAPI. Также GET-запросы должны работать и просто из адресной строки браузера.

В конце практики выложите код в репозиторий --- этот сервис нам потом ещё потребуется.

Язык реализации и используемая библиотека не важны.

На что ещё стоит обратить внимание:

\begin{itemize}
    \item на аккуратное проектирование доменных классов --- книг, отзывов разных видов, поисковых запросов (паттерн <<Спецификация?>>);
    \item на аккуратный дизайн API в соответствии с принципами REST --- только коллекции и ресурсы, никаких методов типа get\_book\_review;
    \item кстати, аккуратный REST API подразумевает фильтрацию на стороне сервера --- самое место для функциональности поиска;
    \item помимо доменных классов, вся <<обвязка>> должна быть по канонам гексагональной архитектуры --- доменная модель отдельно, средства доставки отдельно; в идеале ещё доменные сервисы (для поиска, например) и даже элементы чистой архитектуры, хотя для этой задачи чистая архитектура может быть овердизайном;
    \item если успеете, нужна также пагинация (чтобы возвращался не весь каталог, а порциями по N книг) и HATEOAS (то есть в каждом ответе также URL с тем, что ещё можно сделать с сервисом).
\end{itemize}

\end{document}