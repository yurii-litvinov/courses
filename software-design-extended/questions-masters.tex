\documentclass[a5paper]{article}
\usepackage[a5paper, top=8mm, bottom=8mm, left=8mm, right=8mm]{geometry}

\usepackage{polyglossia}
\setdefaultlanguage{russian}

\usepackage{fontspec}
%\setmainfont{Liberation Serif}
\setmainfont{FreeSerif}
%\newfontfamily{\cyrillicfont}{Liberation Serif}
\newfontfamily{\cyrillicfont}{FreeSerif}

\usepackage[style=russian]{csquotes}

\sloppy
\pagestyle{plain}

\begin{document}

\thispagestyle{empty}

% \section*{Вопросы к зачёту/экзамену \enquote{Архитектура и проектирование информационных систем}}
% \section*{Вопросы к экзамену по курсу \enquote{Принципы разработки и проектирования}}
\section*{Вопросы к экзамену по курсу \enquote{Проектирование программного обеспечения} (вариант для магистров)}

\begin{flushright}\begin{small}Юрий Литвинов\\\small{yurii.v.litvinov@yandex.ru}\end{small}\end{flushright}

\begin{enumerate}
    \item Понятие архитектуры, профессия \enquote{Архитектор}.
    \item Архитектурные виды.
    \item Роль архитектуры в жизненном цикле программного обеспечения. Понятия prescriptive и descriptive-архитектуры.
    \item Понятие декомпозиции. Модульность, связность, сопряжение, сложность.
    \item Понятия класса и объекта, абстракция, инкапсуляция, наследование. 
    \item Принципы выделения объектов.
    \item Принципы SOLID. Закон Деметры.
    \item Моделирование, визуальные модели, виды моделей, метафора визуализации.
    \item Язык UML. Диаграммы классов.
    \item Диаграммы объектов, диаграммы пакетов UML.
    \item Диаграммы компонентов UML, диаграммы развёртывания UML.
    \item Диаграммы случаев использования UML.
    \item Диаграммы IDEF0 (контекстные), характеристик, Feature tree.
    \item Диаграммы активностей UML.
    \item Язык BPMN.
    \item Диаграммы \enquote{Сущность-связь}.
    \item Концептуальное моделирование, диаграммы ORM.
    \item Диаграммы конечных автоматов UML.
    \item Диаграммы последовательностей UML.
    \item Диаграммы коммуникации, составных структур UML.
    \item Диаграммы коопераций, временные диаграммы UML.
    \item Диаграммы обзора взаимодействия, диаграммы потоков данных.
    \item Диаграммы IDEF0, сети Петри.
    \item Паттерн \enquote{Компоновщик}.
    \item Паттерн \enquote{Декоратор}.
    \item Паттерн \enquote{Стратегия}.
    \item Паттерн \enquote{Адаптер}.
    \item Паттерн \enquote{Заместитель}.
    \item Паттерн \enquote{Фасад}.
    \item Паттерн \enquote{Приспособленец}.
    \item Паттерн \enquote{Мост}.
    \item Паттерн \enquote{Фабричный метод}.
    \item Паттерн \enquote{Абстрактная фабрика}.
    \item Паттерн \enquote{Одиночка}.
    \item Паттерны \enquote{Ленивая инициализация} и \enquote{Пул объектов}.
    \item Паттерн \enquote{Прототип}.
    \item Паттерн \enquote{Строитель}.
    \item Паттерн \enquote{Наблюдатель}.
    \item Паттерн \enquote{Шаблонный метод}.
    \item Паттерн \enquote{Посредник}.
    \item Паттерн \enquote{Команда}.
    \item Паттерн \enquote{Цепочка ответственности}.
    \item Паттерн \enquote{Состояние}.
    \item Паттерн \enquote{Посетитель}.
    \item Паттерн \enquote{Хранитель}.
    \item Паттерн \enquote{Интерпретатор}.
    \item Паттерн \enquote{Итератор}.
    \item Понятие архитектурного стиля, трёхзвенная архитектура.
    \item Шаблоны Model-View-Controller, Sense-Compute-Control.
    \item Слоистый стиль, \enquote{Клиент-сервер}.
    \item Гексагональная архитектура, луковая архитектура.
    \item Чистая архитектура.
    \item Пакетная обработка, каналы и фильтры. 
    \item Стиль Blackboard.
    % \item Событийно-ориентированные стили, Publish-Subscribe, событийная шина.
    % \item Peer-to-peer.
    \item Понятие предметно-ориентированного проектирования, единый язык.
    \item Изоляция предметной области в DDD, антипаттерн \enquote{Умный GUI}.
    \item DDD, основные структурные элементы модели предметной области, идентичность объекта в информационных системах.
    \item DDD, паттерн \enquote{Агрегат}.
    \item DDD, паттерны \enquote{Фабрика}, \enquote{Репозиторий}.
    \item Паттерн \enquote{Спецификация}.
    \item Ограниченный контекст, непрерывная интеграция, карта контекстов.
    \item Подходы к интеграции контекстов.
    \item Смысловое ядро, приёмы дистилляции, абстрактное ядро.
    \item Крупномасштабная структура, метафора системы, разбиение по уровням. 
    \item Типичные уровни в производственных и финансовых системах.
    \item Стили \enquote{Уровень знаний}, \enquote{Подключаемые компоненты}.
    \item Понятие распределённой системы, заблуждения при проектировании распределённых систем.
    \item RPC, RMI. Пример: gRPC.
    \item Веб-сервисы, SOAP. WCF.
    % \item Веб-сервисы, REST. ASP.NET Web APIs.
    % \item Очереди сообщений, RabbitMQ. % , Kafka.
    \item Архитектурные стили распределённых приложений: Big Compute, Big Data.
    \item Web-queue-worker, N-звенная архитектура.
    \item Микросервисная архитектура.
    % \item Хорошие практики проектирования REST-интерфейсов.
    % \item Принципы дизайна распределённых приложений: самовосстановление.
    % \item Паттерн \enquote{Circuit Breaker}.
    % \item Принципы дизайна распределённых приложений: избыточность.
    % \item Принципы дизайна распределённых приложений: минимизация координации.
    % \item Command and Query Responsibility Segregation.
    % \item CAP-теорема, модели ACID и BASE.
    % \item Принципы дизайна распределённых приложений: проектирование для обслуживания.
    % \item Docker, Docker Compose.
    % \item Kubernetes.
    % \item Облачная инфраструктура, AWS, Terraform.
    % \item Основные понятия сетевой безопасности, шифрование с симметричным ключом.
    % \item Алгоритм Диффи-Хелмана. Шифрование с открытым ключом.
    % \item Цифровые подписи, сертификаты.
    % \item Аутентификация. OAuth 2.
\end{enumerate}

\end{document}
