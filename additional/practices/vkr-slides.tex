\documentclass[xetex,mathserif,serif]{beamer}
\usepackage{polyglossia}
\setdefaultlanguage[babelshorthands=true]{russian}
\usepackage{minted}
\usepackage{csquotes}
\usepackage{outlines}

\useoutertheme{infolines}

\setmainfont{FreeSans}
\newfontfamily{\russianfonttt}{FreeSans}

\definecolor{links}{HTML}{2A1B81}
\hypersetup{colorlinks,linkcolor=,urlcolor=links}

\beamertemplatenavigationsymbolsempty

\title{ВКР}
\author[Юрий Литвинов]{Юрий Литвинов \newline \textcolor{gray}{\small\texttt{y.litvinov@spbu.ru}}}
\date{16.02.2026}

\begin{document}

    \frame{\titlepage}

    \section{Регламент}

    \begin{frame}{Формальности}
        \begin{outline}
            \1 Курс \enquote{Преддипломная практика}:
                \2 Отзыв научника на \textbf{преддипломную практику}
                \2 Черновик текста ВКР
                    \3 Титульник, на котором написано \enquote{Отчёт по преддипломной практике}
                \1 Непредпредзащита, то есть выступление на этой паре
            \1 Предзащита --- где-то за две недели до защиты, генеральная репетиция
            \1 Защиты --- с 15 мая по 15 июня
            \1 Начало июля --- выдача диплома
        \end{outline}
    \end{frame}

    \begin{frame}{Даты защит прошлого года}
        \begin{outline}
            \1 Кафедра информатики, кафедра ПА --- 24 мая
            \1 Кафедра ИАС --- 28 мая
            \1 Кафедра СП, техпрог --- 7 и 9 июня
            \1 ПИ --- 11 июня
        \end{outline}
    \end{frame}

    \begin{frame}{Документы для защиты}
        \begin{outline}
            \1 Сданная весенняя сессия
                \2 С долгами (любыми) к защите не допускают
                \2 Сессия в конце апреля
            \1 Текст диплома --- за две недели до защиты (строго, но включительно), в Blackboard
            \1 Аннотация --- абзац текста, про что ВКР, в Blackboard (в т.ч. на английском)
            \1 Отзыв научного руководителя --- за пять дней до защиты
            \1 Отзыв рецензента --- за пять дней до защиты
                \2 Они не так строго, потому что формально это не ваше дело
                \2 Формально можно защищаться без отзыва и/или рецензии, но...
            \1 Научник должен быть на защите, рецензент не обязательно
        \end{outline}
    \end{frame}

    \begin{frame}{Кто такой ГЭК}
        \begin{outline}
            \1 ГЭК --- Государственная Экзаменационная Комиссия
            \1 Формируется из ведущих специалистов в отрасли (не менее 50\%) и преподавателей СПбГУ
                \2 У нас это обычно директора или начальники отделов уважаемых компаний
                \2 Есть и молодые специалисты, понимающие в технологиях и не стесняющиеся задавать вопросы
            \1 Обычно 6-7 человек
            \1 ГЭК формируется для направления, то есть бакалавры техпрога могут защищаться только в ГЭК для СВ.5162
        \end{outline}
    \end{frame}

    \begin{frame}{Как проходит защита}
        \framesubtitle{И как ставятся оценки}
        \begin{footnotesize}
            \begin{outline}
                \1 За одно заседание защищается максимум 8 человек, максимум 2 заседания в день (бывает \enquote{два с половиной})
                \1 Порядок защиты фиксируется (когда вас распределяют по датам)
                    \2 \footnotesize{Приказ о допуске к защите --- где-то в начале-середине мая}
                \1 Выступление защищающегося, вопросы от членов ГЭК и аудитории, отзыв научника, рецензия (зачитывается научником, если рецензента нет), вопросы по отзыву/рецензии, ответное слово (если надо)
                \1 ГЭК совещается (по окончании заседания)
                \1 Члены ГЭК выставляют оценки по критериям, каждый независимо
                    \2 \footnotesize{См. \url{https://edu.spbu.ru/gia.html}}
                \1 Каждый член ГЭК ставит итоговую оценку (при этом критерии --- это только рекомендации), оценки всех членов ГЭК усредняются и ставится итоговая
                    \2 \footnotesize{Оценки научника и рецензента непосредственно не учитываются! Они имеют рекомендательное значение для ГЭК.}
                \1 Итоговая оценка заносится в протокол защиты и выставляется на Blackboard
                \1 Защищающихся приглашают, оглашают результаты, поздравляют с присвоением квалификации (или нет)
            \end{outline}
        \end{footnotesize}
    \end{frame}

    \section{Отчёт}

    \begin{frame}{Отчёт, структура}
        \begin{outline}
            \1 Титульный лист
            \1 Оглавление
            \1 Введение в предметную область, постановка задачи
            \1 Обзор литературы и существующих решений
            \1 Описание предлагаемого решения
                \2 Отдельно архитектура и детали реализации
            \1 Апробация и/или эксперименты
            \1 Заключение
            \1 Список литературы
            \1 Не более 40 страниц (60 для магистров)
                \2 Рекомендуется не более 20 (в эпоху БЯМ большой текст --- не признак большой работы)
            \1 Требования к оформлению: \url{https://edu.spbu.ru/gia.html}
        \end{outline}
    \end{frame}

    \begin{frame}{Введение}
        \begin{columns}
            \begin{column}{0.6\textwidth}
                \begin{outline}
                    \1 Известная информация, \enquote{Background}
                    \1 Неизвестная информация, \enquote{Gap}
                        \2 Актуальность темы
                        \2 Практическая значимость
                        \2 Кому конкретно это надо
                    \1 Кратко про ваш подход к решению задачи, почему он приведёт к успеху (\enquote{Гипотеза} и \enquote{Подход})
                    \1 Этот раздел заслуживает особого внимания!
                \end{outline}
            \end{column}
            \begin{column}{0.4\textwidth}
                \begin{center}
                    \includegraphics[width=\textwidth]{introductionCone.png}
                \end{center}
            \end{column}
        \end{columns}
    \end{frame}

    \begin{frame}{Постановка задачи}
        \begin{outline}
            \1 Цель работы
                \2 Одним предложением --- что конкретно надо сделать
            \1 Задачи
                \2 Отчуждаемые
                \2 Специфичные
                \2 Решение которых приведёт к цели
                \2 Выполнить обзор, спроектировать, реализовать, выполнить апробацию/эксперименты
        \end{outline}
    \end{frame}

    \begin{frame}{Обзор}
        \begin{outline}
            \1 Обзор существующих решений
                \2 Цель обзора, критерии отбора материалов
                \2 Критерии сравнения
                \2 Таблица с результатами
                \2 Выводы
            \1 Обзор используемых чужих результатов
                \2 Всё, написанное и придуманное не вами --- в обзор
            \1 Должен соотноситься с темой и целью
        \end{outline}
    \end{frame}

    \begin{frame}{Описание решения}
        \begin{outline}
            \1 Разделы должны соответствовать списку задач
            \1 Аргументированное обоснование принятых решений и отказа от альтернатив
            \1 Выбор инструментария
            \1 Описание архитектуры, алгоритмов и т.п.
            \1 Описание того, над чем \enquote{пришлось подумать больше пяти минут}
        \end{outline}
    \end{frame}

    \begin{frame}{Описание решения (2)}
        \begin{outline}
            \1 Рисунки и диаграммы
                \2 Лучше использовать стандартную нотацию (UML, ER, ...)
                \2 Подписи, единицы измерения
                    \3 Чужие рисунки --- со ссылкой на источник
                \2 Ссылки из текста
                \2 Сквозная нумерация
            \1 Таблицы
                \2 Единицы измерения в заголовке
                \2 Чтобы было всё видно даже в напечатанном варианте
        \end{outline}
    \end{frame}

    \begin{frame}{Апробация/эксперименты}
        \begin{outline}
            \1 В любом случае должна быть
                \2 Для чисто инженерных работ --- апробация на реальных пользователях
            \1 Лучше численный результат, ещё лучше --- если его можно с кем-то сравнивать
                \2 System Usability Scale, Net Promoter Score, Usability Metric for User Experience
            \1 \textbf{Матстат}
            \1 Никаких лишних цифр после запятой
            \1 Эксперимент должен быть согласован с постановкой задачи
            \1 Threats to validity
            \1 Выводы
        \end{outline}
    \end{frame}

    \begin{frame}
        \frametitle{Заключение}
        \begin{outline}
            \1 Перечисление результатов, выносимых на защиту
            \1 Должно быть согласовано с постановкой задачи (вплоть до полного её повторения, но с уточнением по полученным результатам)
            \1 Должно быть согласовано с текстом
                \2 Никаких результатов из ниоткуда
            \1 Ссылка на репозиторий или пара слов про то, почему её нет и что вы можете показать взамен
                \2 ...Было внедрено, отзыв о внедрении прилагается
            \1 Благодарности (прежде всего консультанту)
            \1 Может быть, Future work
            \1 Этот раздел также заслуживает особого внимания!
        \end{outline}
    \end{frame}

    \begin{frame}
        \frametitle{Литература}
        \begin{outline}
            \1 Cсылок примерно как страниц в работе
            \1 Обязательно на каждый пункт ссылаться из текста
            \1 Лучше ссылаться на научные статьи
                \2 Ещё лучше --- на книги, но по предметной области
                \2 Смотрите на индекс Хирша и число цитирований
            \1 Реально прочитанные работы
        \end{outline}
    \end{frame}

    \begin{frame}
        \frametitle{Литература (2)}
        \begin{outline}
            \1 ГОСТ Р 7.0.5-2008
                \2 А.Н. Терехов, Т.А. Брыксин, Ю.В. Литвинов и др., Архитектура среды визуального моделирования QReal. // Системное программирование. Вып. 4. СПб.: Изд-во СПбГУ. 2009, С. 171-196
                \2 Порядок --- алфавитный (по авторам), в порядке упоминания в тексте, в хронологическом порядке (если это важно)
                \2 Ссылки в тексте --- номер в квадратных скобках: \enquote{блаблабла [1]} (с пробелом)
            \1 В литературу --- только, гм, литературу
                \2 Подстраничные сноски для ссылок на сайты, статьи на Хабре и т.д.
                \2 Электронные источники в списке литературы допустимы (надо указывать дату обращения)
        \end{outline}
    \end{frame}

    \section{Презентация}

    \begin{frame}
        \frametitle{Презентация, структура}
        \begin{outline}
            \1 Титульный слайд
            \1 Введение (примерно 1-2 слайда)
            \1 Постановка задачи (1 слайд)
            \1 Обзор (примерно 1 слайд)
            \1 Предлагаемое решение (примерно 1 слайд)
            \1 Апробация/эксперименты (примерно 1 слайд)
            \1 Результаты, выносимые на защиту (1 слайд) --- обязательно, последним слайдом
        \end{outline}
    \end{frame}

    \begin{frame}
        \frametitle{Демо}
        \begin{outline}
            \1 Короткий видеоролик/gif, на 1-2 минуты сверх выделенных на выступление
            \1 Не обязательно, но ГЭК очень любит
            \1 Даже если это библиотека/консольное приложение, тоже ок, покажите консоль
                \2 Только чтобы всё было видно
            \1 Если работа закрытая, даже демонстрировать может быть нельзя, согласуйте с начальством
        \end{outline}
    \end{frame}

    \begin{frame}
        \frametitle{Тактические соображения}
        \begin{outline}
            \1 Укладывайтесь в 7 минут
            \1 Стоит порепетировать самим и перед научником
            \1 По протоколу положены вопросы
                \2 Если не хотите неожиданных, можно намеренно оставить некую недосказанность
                \2 И подготовить скрытые слайды
                \2 Тем не менее, неожиданные вопросы будут!
            \1 Не увлекайтесь техническими подробностями
            \1 Не увлекайтесь их отсутствием
            \1 Избегайте больших формул на слайдах
            \1 Слайды должны быть такими, чтобы вас можно было особо не слушать
                \2 Расшифровка сокращений, визуализация всего, название-авторы статей и т.п.
        \end{outline}
    \end{frame}

    \section{Общие рекомендации}

    \begin{frame}
        \frametitle{Общие рекомендации}
        \begin{outline}
            \1 Никакого заимствования 
                \2 Сдача чужой работы --- отчисление без права восстановления сразу
                \2 Копипаст даже одного предложения без указания источника --- незачёт
                \2 Правильно оформленный копипаст --- нехорошо
            \1 Обязательно показать и текст, и презентацию научнику перед отсылкой рецензенту
                \2 Порепетировать выступление!
        \end{outline}
    \end{frame}

    \begin{frame}
        \frametitle{Оформление кода}
        \begin{outline}
            \1 Рецензент может (и должен!) смотреть на код
            \1 Аккуратные исходники со стайлгайдом и комментариями
            \1 README 
                \2 Общее описание проекта
                \2 Описание процесса сборки
                \2 Описание воспроизведения эксперимента
                \2 У рецензента должно получиться то же, что и у вас, без вашей помощи
            \1 Настроенный и проходящий билд в CI-системе
            \1 Лицензия (какая хотите, мы рекомендуем Apache License 2.0 или MIT)
            \1 Желательно, всё по-английски
        \end{outline}
    \end{frame}

    \begin{frame}
        \frametitle{Полезные ресурсы}
        \begin{outline}
            \1 Сайт кафедры --- \url{https://se.math.spbu.ru/}
                \2 Раздел \enquote{Студентам} --- архив работ
            \1 Титульники --- \url{https://github.com/spbu-se/matmex-diploma-template}
            \1 Презентация --- \url{https://github.com/spbu-se/report_presentation_template}
            \1 Онлайн-редактор TeX --- \url{https://www.overleaf.com/}
            \1 Чеклист по презентациям --- \url{https://goo.gl/UeDRff}
        \end{outline}
    \end{frame}

    \begin{frame}
        \frametitle{FAQ}
        \begin{outline}
            \1 Можно ли поменять научника/тему диплома?
                \2 Да, где-то до 25-го апреля
                \2 По заявлению в учебный отдел за подписью старого и нового научника 
                \2 Форма заявления: \url{https://disk.yandex.ru/i/r-gPcPixqrPYRQ}
            \1 Можно ли поменять рецензента?
                \2 Формально рецензента вам назначают, так что студент инициировать смену рецензента не может
                \2 Неформально, написать мне, и \textit{возможно}, УОП поправит приказ
            \1 Можно ли перенести защиту?
                \2 Нет. Либо вы защищаетесь, либо отчисляетесь по незащите ВКР с правом восстановиться для защиты
                \2 Теоретически можно закрыть справкой две недели до защиты (чтобы не получить неуд за невыкладывание текста) и две недели до резервного дня, тогда защиту перенесут на осень
                \2 Академ --- тоже вариант, но только по уважительной причине
        \end{outline}
    \end{frame}

\end{document}