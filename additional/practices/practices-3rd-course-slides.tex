\documentclass{../../slides-style}

\slidetitle[Требования, рекомендации]{Учебные практики третьего курса}{07.09.2022}

\begin{document}
    
    \begin{frame}[plain]
        \titlepage
    \end{frame}

    \begin{frame}
        \frametitle{Чем отличается от практики второго курса?}
        \begin{itemize}
            \item На два семестра, с промежуточной отчётностью зимой
            \begin{itemize}
                \item Можно две односеместровые практики, но в порядке исключения
                \item Вообще, по-хорошему вы работаете в некоем проекте, раз в семестр отчитываясь об успехах, не делите практики на одно- и двухсеместровые
            \end{itemize}
            \item Большая <<научность>> или <<производственность>>
            \begin{itemize}
                \item Должна быть некая научная или инженерная новизна
                \item Должна быть кому-то полезна
            \end{itemize}
        \end{itemize}
    \end{frame}

    \begin{frame}
        \frametitle{Требования, осень}
        \begin{itemize}
            \item Отчёт на 10-15 страниц
            \begin{itemize}
                \item Введение, постановка задачи, обзор, начало реализации, план апробации/экспериментов
                \item Выполненные задачи, задачи на весну
                \item На СП будет выборочное рецензирование
            \end{itemize}
            \item Доклад
            \begin{itemize}
                \item Порядка 7-9 минут (можно меньше, но нельзя больше)
                \item Рассказать о задаче и текущих успехах
            \end{itemize}
            \item Отзывы научного руководителя и консультанта
            \item Ссылка на код, если он открыт
            \begin{itemize}
                \item Должен быть грамотно оформлен репозиторий
                \item Ссылка на репозиторий \emph{в заключении отчёта}
            \end{itemize}
        \end{itemize}
    \end{frame}

    \begin{frame}
        \frametitle{Требования, весна}
        \begin{itemize}
            \item Отчёт на 20-30 страниц
            \begin{itemize}
                \item Полное описание работы, включая осеннюю часть
                \item Можно переиспользовать текст осенней практики
                \item Тоже будет выборочное рецензирование
            \end{itemize}
            \item Доклад, опять-таки на 7-9 минут
            \begin{itemize}
                \item Предзащита
                \item Защита
            \end{itemize}
            \item Снова отзывы, ссылку на код в заключение
        \end{itemize}
    \end{frame}

    \begin{frame}
        \frametitle{Откуда брать тему и научного руководителя}
        \begin{itemize}
            \item Список тем на сайте кафедры: \url{https://se.math.spbu.ru/diplomas/index.html}
            \begin{itemize}
                \item Обратите внимание, там есть фильтры, но пока нет фильтра по кафедре --- ориентируйтесь на научника
                \item Если тема заинтересовала, но не подходит, можно пообщаться с консультантом
            \end{itemize}
            \item На стажировке, работе
            \item Результат выбора темы \textbf{до конца сентября} записать сюда: \url{https://docs.google.com/spreadsheets/d/1zUMT5TkjDuiwjABzu-83n972tlx0Jg1b_wwZcYFzLQs}
            \begin{itemize}
                \item Обратите внимание на вкладки по кафедрам
                \item В октябре таблица будет закрыта для редактирования, кто не успел --- минус балл
            \end{itemize}
        \end{itemize}
    \end{frame}

    \begin{frame}
        \frametitle{Полезные ресурсы}
        \begin{itemize}
            \item Сайт кафедры --- \url{https://se.math.spbu.ru/}
            \begin{itemize}
                \item Раздел ``Студентам'' --- архив работ
            \end{itemize}
            \item Шаблон отчёта: \url{https://github.com/spbu-se/matmex-diploma-template}
            \item Шаблон презентации: \url{https://github.com/spbu-se/report_presentation_template}
            \item Онлайн-редакторы TeX --- \url{https://papeeria.com/}, \url{https://www.overleaf.com/}
            \item Все объявления и созвоны --- в Teams
        \end{itemize}
    \end{frame}

    \begin{frame}
        \frametitle{Примерный план, осенний семестр}
        \begin{itemize}
            \item Сентябрь --- определиться с научным руководителем и темой
            \item Работа над практикой
            \begin{itemize}
                \item Быстрый мини-обзор
                \item Введение, постановка задачи, научиться убеждать окружающих в актуальности темы
                \item Обзор
                \item Проектирование
                \item Написание текста
            \end{itemize}
            \item (Для СП) Середина декабря --- сдача текста на выборочное рецензирование
            \item Конец декабря --- защиты
        \end{itemize}
    \end{frame}

    \begin{frame}
        \frametitle{Примерный план, весенний семестр}
        \begin{itemize}
            \item Февраль--май --- работа над практикой
            \begin{itemize}
                \item Реализация
                \item Апробация/эксперименты
                \item Написание текста
            \end{itemize}
            \item Начало мая --- предзащиты
            \item (Для СП) Начало--середина мая --- выборочное рецензирование
            \item Конец мая --- защиты
            \item Минимум раз в неделю отчитываться научному руководителю о ходе работы
            \item (Для СП) Весенние отчёты публикуются на сайте кафедры
        \end{itemize}
    \end{frame}

    \begin{frame}
        \frametitle{Общие рекомендации}
        \begin{itemize}
            \item В тексте работы никакого заимствования 
            \begin{itemize}
                \item Сдача чужой работы --- отчисление без права восстановления сразу
                \item Копипаст даже одного предложения без указания источника --- незачёт
                \item Правильно оформленный копипаст --- попросят убрать
            \end{itemize}
            \item Обязательно показать и текст, и презентацию научному руководителю
            \begin{itemize}
                \item Стоит порепетировать выступление
            \end{itemize}
            \item Из презентации должно быть предельно понятно, что и зачем вы делаете (актуальность, сложность работы) и при чём тут ваша кафедра
            \begin{itemize}
                \item Это будет один из основных пунктов дискуссии на защите
            \end{itemize}
            \item Озаботьтесь получением отзывов заранее
            \item Код --- CI, юнит-тесты, README, лицензия
        \end{itemize}
    \end{frame}

    \begin{frame}
        \frametitle{FAQ}
        \begin{small}
            \begin{itemize}
                \item Можно ли писать групповую практику?
                \begin{itemize}
                    \item Да, но отчёт и презентация у каждого свои
                \end{itemize}
                \item Засчитывают ли выступление на семинаре/конференции за защиту?
                \begin{itemize}
                    \item Нет
                \end{itemize}
                \item Можно ли менять тему и научника?
                \begin{itemize}
                    \item Да, но предупредить руководителя практики
                \end{itemize}
                \item Можно ли перезачесть работу, написанную в прошлом году?
                \begin{itemize}
                    \item Да, но тоже предупредить
                \end{itemize}
                \item Если научник/консультант/лаборатория/бомж с улицы ставит мне зачёт, как его получить в зачётку?
                \begin{itemize}
                    \item Никак, учебные практики принимаются комиссией в рамках процедуры независимой оценки качества образования
                \end{itemize}
                \item А ещё курсовая?
                \begin{itemize}
                    \item Нет, вместо курсовых теперь практики
                \end{itemize}
            \end{itemize}
        \end{small}
    \end{frame}

\end{document}