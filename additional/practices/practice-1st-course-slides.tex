\documentclass{../../slides-style}

\slidetitle[Требования, рекомендации]{Учебные практики первого курса}{12.02.2026}

\begin{document}
    
    \begin{frame}[plain]
        \titlepage
    \end{frame}

    \begin{frame}
        \frametitle{Что такое учебная практика}
        \begin{itemize}
            \item Программно-инженерная или научно-исследовательская работа%
            \begin{itemize}
                \item Решение более-менее сложной практической либо научной задачи
                \item Отчёт (небольшой текст)
                \item Код
            \end{itemize}
            \item На первом курсе типовая, но объёмная задача, на втором курсе и дальше тема у каждого своя
        \end{itemize}
    \end{frame}

    \begin{frame}
        \frametitle{Требования}
        \begin{itemize}
            \item Репозиторий с кодом, оформленный по всем правилам
            \item Отчёт о практике
            \begin{itemize}
                \item Порядка 2-3 страниц, шаблон отчёта будет
            \end{itemize}
            \item Отчёт о выездном дне в компании
            \begin{itemize}
                \item Порядка полстраницы --- что там было, что показалось важным
            \end{itemize}
        \end{itemize}
    \end{frame}

    \begin{frame}
        \frametitle{Кто такой научник, консультант и т.п.}
        \begin{itemize}
            \item \textit{Консультант} --- читает и рецензирует код, помогает с техническими проблемами
            \begin{itemize}
                \item Как правило, представитель компании
            \end{itemize}
            \item \textit{Научный руководитель} --- преподаватель (обязательно), следит за ходом работы, помогает с методологическими вопросами и текстом
            \item \textit{Руководитель практики} --- общая организация процесса, сбор и распределение тем, сбор отчётов и отзывов, организация защит, решение организационных проблем
        \end{itemize}
    \end{frame}

    \begin{frame}
        \frametitle{Примерный план работы}
        \begin{itemize}
            \item Февраль --- получить задачу, погрузиться в тему, поискать источники
            \item Март --- работа по существу, выездной день в компании, отчёт по выездному дню
            \item Апрель --- эксперименты, оформление репозитория
            \item Май --- написание отчёта по практике
        \end{itemize}
    \end{frame}

    \begin{frame}
        \frametitle{Полезные ресурсы}
        \begin{itemize}
            \item Онлайн-редакторы TeX, будут полезны для написания текста --- \url{https://papeeria.com/}, \url{https://www.overleaf.com/}
            \item Чеклист по оформлению репозитория: \url{https://github.com/yurii-litvinov/courses/blob/master/additional/repo-checklist/repo-checklist.pdf}
        \end{itemize}
    \end{frame}

    \begin{frame}
        \frametitle{Отчёт, структура}
        \begin{itemize}
            \item Заголовок, с указанием темы практики, ФИО научного руководителя и консультанта
            \item Краткое описание задачи и её актуальности
            \item Краткий обзор литературы и существующих решений
            \item Описание предлагаемого решения
            \item Описание экспериментов
            \item Заключение
            \item Список литературы
        \end{itemize}
    \end{frame}

    \begin{frame}
        \frametitle{Описание решения}
        \begin{itemize}
            \item Аргументированное обоснование принятых решений и отказа от альтернатив
            \item Выбор инструментария
            \item Краткое описание архитектуры, алгоритмов и т.п.
        \end{itemize}
    \end{frame}

    \begin{frame}
        \frametitle{Эксперименты}
        \begin{itemize}
            \item Доказать, почему всё, что вы делали, вообще осмысленно
            \item Эксперименты --- численное доказательство, что ваш результат лучше аналогов
            \begin{itemize}
                \item Замеры производительности, точности и т.д.
                \item Отдельная большая наука, делайте аккуратно!
            \end{itemize}
        \end{itemize}
    \end{frame}

    \begin{frame}
        \frametitle{Заключение}
        \begin{itemize}
            \item Перечисление результатов работы
            \item Должно быть согласовано с постановкой задачи
            \item Должно быть согласовано с текстом
            \begin{itemize}
                \item Никаких результатов из ниоткуда
            \end{itemize}
            \item Обязательно ссылка на репозиторий
        \end{itemize}
    \end{frame}

    \begin{frame}
        \frametitle{Литература}
        \begin{itemize}
            \item Обязательно на каждый пункт ссылаться из текста
            \item Лучше ссылаться на научные статьи
            \item Реально прочитанные работы
            \begin{itemize}
                \item Всё-таки прочитать бывает полезно
            \end{itemize}
        \end{itemize}
    \end{frame}

    \begin{frame}
        \frametitle{Литература (2)}
        \begin{itemize}
            \item ГОСТ Р 7.0.5-2008
            \begin{itemize}
                \item А.Н. Терехов, Т.А. Брыксин, Ю.В. Литвинов и др., Архитектура среды визуального моделирования QReal. // Системное программирование. Вып. 4. СПб.: Изд-во СПбГУ. 2009, С. 171-196
                \item Порядок --- алфавитный (по авторам), в порядке упоминания в тексте, в хронологическом порядке (если это важно)
                \item Ссылки в тексте --- номер в квадратных скобках: ``блаблабла [1]'' (с пробелом)
                \item Электронные источники в списке литературы допустимы (надо указывать дату обращения)
            \end{itemize}
        \end{itemize}
    \end{frame}

    \begin{frame}
        \frametitle{Общие рекомендации}
        \begin{itemize}
            \item Никакого заимствования 
            \begin{itemize}
                \item Сдача чужой работы --- отчисление без права восстановления сразу
                \item Копипаст даже одного предложения без указания источника --- незачёт
                \item Правильно оформленный копипаст --- попросят убрать
            \end{itemize}
            \item Обязательно показать текст научнику
            \item Озаботьтесь получением отзыва заранее
            \item Код --- CI, юнит-тесты, README, лицензия
        \end{itemize}
    \end{frame}

\end{document}