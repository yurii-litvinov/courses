\documentclass{../../slides-style}

\slidetitle[Требования, рекомендации]{Учебные практики первого курса}{16.02.2026}

\begin{document}
    
    \begin{frame}[plain]
        \titlepage
    \end{frame}

    \begin{frame}{Что такое учебная практика}
        \begin{outline}
            \1 Программно-инженерная или научно-исследовательская работа%
                \2 Решение более-менее сложной практической либо научной задачи
                \2 Отчёт (небольшой текст)
                \2 Код
            \1 На первом курсе типовая, но объёмная задача, на втором курсе и дальше тема у каждого своя
            \1 Будет выездной день в YADRO
            \1 Будут два семинара для обсуждения промежуточных результатов
                \2 Надо будет кратко рассказать, что делали, что получилось, что планируете делать дальше
        \end{outline}
    \end{frame}

    \begin{frame}{Требования}
        \begin{outline}
            \1 Репозиторий с кодом, оформленный по всем правилам
            \1 Отчёт о практике
                \2 Порядка 2-3 страниц, шаблон отчёта будет
            \1 Отзыв научного руководителя
            \1 Отчёт о выездном дне в компании
                \2 Порядка полстраницы --- что там было, что показалось важным
        \end{outline}
    \end{frame}

    \begin{frame}{Кто такой научник, консультант и т.п.}
        \begin{outline}
            \1 \textit{Консультант} --- читает и рецензирует код, помогает с техническими проблемами
                \2 Как правило, представитель компании
                \2 Его может не быть, если работаете непосредственно с научным руководителем
            \1 \textit{Научный руководитель} --- преподаватель (обязательно), следит за ходом работы, помогает с методологическими вопросами и текстом
            \1 \textit{Руководитель практики} --- общая организация процесса, сбор и распределение тем, сбор отчётов и отзывов, организация защит, решение организационных проблем
        \end{outline}
    \end{frame}

    \begin{frame}{Примерный план работы}
        \begin{outline}
            \1 Февраль --- получить задачу, погрузиться в тему, поискать источники
            \1 Март --- работа по существу, выездной день в компании, отчёт по выездному дню
            \1 Апрель --- эксперименты, оформление репозитория
            \1 Май --- написание отчёта по практике
        \end{outline}
    \end{frame}

    \begin{frame}{Полезные ресурсы}
        \begin{outline}
            \1 Онлайн-редакторы TeX, будут полезны для написания текста --- \url{https://papeeria.com/}, \url{https://www.overleaf.com/}
            \1 Чеклист по оформлению репозитория: \url{https://github.com/spbu-se/practice-templates/blob/main/repo-checklist/repo-checklist.pdf}
        \end{outline}
    \end{frame}

    \begin{frame}{Отчёт, структура}
        \begin{outline}
            \1 Заголовок, с указанием темы практики, ФИО научного руководителя и консультанта
            \1 Введение по схеме \enquote{контекст работы --- проблема, которую хотим решить --- краткий план решения}
                \2 Должно быть понятно, где может применяться результат, и где могут применяться полученные навыки
            \1 Постановка задачи: \enquote{Цель работы: \dots. Задачи: 1) \dots 2) \dots 3) \dots}
            \1 Краткий обзор литературы и существующих решений
            \1 Описание предлагаемого решения
            \1 Описание экспериментов
            \1 Заключение
            \1 Список литературы
        \end{outline}
    \end{frame}

    \begin{frame}{Описание решения}
        \begin{outline}
            \1 Аргументированное обоснование принятых решений и отказа от альтернатив
            \1 Выбор инструментария
            \1 Краткое описание архитектуры, алгоритмов и т.п.
        \end{outline}
    \end{frame}

    \begin{frame}{Эксперименты}
        \begin{outline}
            \1 Доказать, почему всё, что вы делали, вообще осмысленно
            \1 Эксперименты --- численное доказательство валидности решения
                \2 Замеры производительности, точности и т.д.
                \2 Отдельная большая наука, делайте аккуратно!
        \end{outline}
    \end{frame}

    \begin{frame}{Заключение}
        \begin{outline}
            \1 Перечисление результатов работы
            \1 Должно быть согласовано с постановкой задачи
            \1 Должно быть согласовано с текстом
                \2 Никаких результатов из ниоткуда
            \1 Обязательно ссылка на репозиторий
        \end{outline}
    \end{frame}

    \begin{frame}{Литература}
        \begin{outline}
            \1 Обязательно на каждый пункт ссылаться из текста
            \1 Лучше ссылаться на научные статьи
            \1 Реально прочитанные работы
                \2 Всё-таки прочитать бывает полезно
        \end{outline}
    \end{frame}

    \begin{frame}
        \frametitle{Литература (2)}
        \begin{outline}
            \1 ГОСТ Р 7.0.5-2008
                \2 А.Н. Терехов, Т.А. Брыксин, Ю.В. Литвинов и др., Архитектура среды визуального моделирования QReal. // Системное программирование. Вып. 4. СПб.: Изд-во СПбГУ. 2009, С. 171-196
                \2 Порядок --- алфавитный (по авторам), в порядке упоминания в тексте, в хронологическом порядке (если это важно)
                \2 Ссылки в тексте --- номер в квадратных скобках: \enquote{блаблабла [1]} (с пробелом)
                \2 Электронные источники в списке литературы допустимы (надо указывать дату обращения)
        \end{outline}
    \end{frame}

    \begin{frame}
        \frametitle{Общие рекомендации}
        \begin{outline}
            \1 Никакого заимствования 
                \2 Сдача чужой работы --- отчисление без права восстановления сразу
                \2 Копипаст даже одного предложения без указания источника --- незачёт
                \2 Правильно оформленный копипаст --- попросят убрать
            \1 Обязательно показать текст научнику
            \1 Озаботьтесь получением отзыва заранее
            \1 Код --- CI, юнит-тесты, README, лицензия
        \end{outline}
    \end{frame}

\end{document}