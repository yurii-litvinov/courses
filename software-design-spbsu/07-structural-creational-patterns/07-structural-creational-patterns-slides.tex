\documentclass[xetex,mathserif,serif]{beamer}
\usepackage{polyglossia}
\setdefaultlanguage[babelshorthands=true]{russian}
\usepackage{minted}
\usepackage{tabu}

\useoutertheme{infolines}

\usepackage{fontspec}
\setmainfont{FreeSans}
\newfontfamily{\russianfonttt}{FreeSans}

\usepackage{textpos}
\setlength{\TPHorizModule}{1cm}
\setlength{\TPVertModule}{1cm}

\setbeamertemplate{blocks}[rounded][shadow=false]

\setbeamercolor*{block title alerted}{fg=red!50!black,bg=red!20}
\setbeamercolor*{block body alerted}{fg=black,bg=red!10}

\tabulinesep=1.2mm

\title[Шаблоны]{Лекция 7: Структурные и порождающие шаблоны}
\author[Юрий Литвинов]{Юрий Литвинов\\\small{\textcolor{gray}{yurii.litvinov@gmail.com}}}
\date{02.11.2017г}

\newcommand{\todo}[1] {
	\begin{center}\textcolor{red}{TODO: #1}\end{center}
}

\newcommand{\DownArrow} {
	\hspace{2cm}\begin{LARGE}$\downarrow$\end{LARGE}
}

\newcommand{\attribution}[1] {
	\vspace{-5mm}\begin{flushright}\begin{scriptsize}\textcolor{gray}{\textcopyright\, #1}\end{scriptsize}\end{flushright}
}

\begin{document}

	\frame{\titlepage}

	\section{Паттерн ``Фасад''}

	\begin{frame}
		\frametitle{Паттерн ``Фасад''}
		\framesubtitle{Facade}
		\begin{itemize}
			\item Простой интерфейс к сложной системе
			\item Отделение подсистем от клиента и друг от друга
			\item Многоуровневая архитектура
		\end{itemize}
	\end{frame}

	\begin{frame}
		\frametitle{``Фасад'' (Facade), детали реализации}
		\begin{columns}
			\begin{column}{0.5\textwidth}
				\begin{itemize}
					\item Абстрактный Facade
					\begin{itemize}
						\item Существенно снижает связность клиента с подсистемой
					\end{itemize}
				\end{itemize}
			\end{column}
			\begin{column}{0.5\textwidth}
				\begin{center}
					\includegraphics[width=0.8\textwidth]{facade.png}
				\end{center}
			\end{column}
		\end{columns}
		\begin{itemize}
			\item Открытые и закрытые классы подсистемы
			\begin{itemize}
				\item Пространства имён и пакеты помогают, но требуют дополнительных соглашений
				\begin{itemize}
					\item Пространство имён details
				\end{itemize}
				\item Инкапсуляция целой подсистемы --- это хорошо
			\end{itemize}
		\end{itemize}
	\end{frame}

	\section{Паттерн ``Мост''}

	\begin{frame}
		\frametitle{Паттерн ``Мост'' (Bridge)}
		Отделяет абстракцию от реализации

		Пример:
		\begin{itemize}
			\item Есть система, интерпретирующая программы для роботов
			\item Есть класс \textit{Sensor}, от которого наследуются \textit{SonarSensor}, \textit{LightSensor}, ...
			\item Связь с роботом может выполняться по USB или Bluetooth, а может быть, программа и вовсе исполняется на симуляторе
			\item Интерпретатор хочет работать с сенсорами, не заморачиваясь реализацией механизма связи
			\item Рабоче-крестьянская реализация --- \textit{USBLightSensor}, \textit{BluetoothLightSensor}, \textit{USBSonarSensor}, \textit{BluetoothSonarSensor}, ...
			\item Число классов --- произведение количества сенсоров и типов связи
		\end{itemize}
	\end{frame}

	\begin{frame}
		\frametitle{``Мост'', пример}
		\begin{center}
			\includegraphics[width=0.7\textwidth]{noBridge.png}
			\Huge{$$\downarrow$$}
			\includegraphics[width=0.7\textwidth]{bridge.png}
		\end{center}
	\end{frame}

	\begin{frame}
		\frametitle{``Мост'', общая схема}
		\begin{center}
			\includegraphics[width=0.7\textwidth]{bridgeGeneral.png}
		\end{center}
		\begin{itemize}
			\item \textit{Abstraction} --- определяет интерфейс абстракции, хранит ссылку на реализацию
			\item \textit{RefinedAbstraction} --- расширяет интерфейс абстракции, делает полезную работу, используя реализацию
			\item \textit{Implementor} --- определяет интерфейс реализации, в котором абстракции предоставляются низкоуровневые операции
			\item \textit{ConcreteImplementor} --- предоставляет конкретную реализацию Implementor
		\end{itemize}
	\end{frame}

	\begin{frame}
		\frametitle{Когда применять}
		\begin{itemize}
			\item Когда хочется разделить абстракцию и реализацию, например, когда реализацию можно выбирать во время компиляции или во время выполнения
			\begin{itemize}
				\item ``Стратегия'', ``Прокси''
			\end{itemize}
			\item Когда абстракция и реализация должны расширяться новыми подклассами
			\item Когда хочется разделить одну реализацию между несколькими объектами
			\begin{itemize}
				\item Как copy-on-write в строках
			\end{itemize}
		\end{itemize}
	\end{frame}

	\begin{frame}
		\frametitle{Тонкости реализации}
		Создание правильного Implementor-а
		\begin{itemize}
			\item Самой абстракцией в конструкторе, в зависимости от переданных параметров
			\begin{itemize}
				\item Как вариант --- выбор реализации по умолчанию и замена её по ходу работы
			\end{itemize}
			\item Принимать реализацию извне (как параметр конструктора, или, реже, как значение в сеттер)
			\item Фабрика/фабричный метод
			\begin{itemize}
				\item Позволяет спрятать платформозависимые реализации, чтобы не зависеть от них всех при сборке
			\end{itemize}
		\end{itemize}
	\end{frame}

	\begin{frame}
		\frametitle{Pointer To Implementation (PImpl)}
		Вырожденный мост для C++, когда ``абстракция'' имеет ровно одну реализацию, часто полностью дублирующую её интерфейс

		Зачем: чтобы клиенты класса не зависели при сборке от его реализации

		\begin{itemize}
			\item Позитивно сказывается на времени компиляции программ на C++
			\item Позволяет менять реализацию независимо
			\begin{itemize}
				\item Сохраняя бинарную совместимость
			\end{itemize}
		\end{itemize}

		Как: предварительное объявление класса-реализации, полное определение --- в .cpp-файле вместе с методами абстракции

		Часто используется в реализации библиотек (например, Qt)
	\end{frame}

	\section{Паттерн ``Приспособленец''}

	\begin{frame}
		\frametitle{Паттерн ``Приспособленец'' (Flyweight)}
		Предназначается для эффективной поддержки множества мелких объектов

		Пример:

		\begin{itemize}
			\item Есть текстовый редактор
			\item Хочется работать с каждым символом как с объектом
			\begin{itemize}
				\item Единообразие алгоритмов форматирования и внутренней структуры документа
				\item Более красивая и ООПшная реализация
				\begin{itemize}
					\item Паттерн ``Компоновщик'', структура ``Символ'' $\rightarrow$ ``Строка'' $\rightarrow$ ``Страница''
				\end{itemize}
			\end{itemize}
			\item Наивная реализация привела бы к чрезмерной расточительности по времени работы и по памяти, потому что документы с миллионами символов не редкость
		\end{itemize}
	\end{frame}

	\begin{frame}
		\frametitle{``Приспособленец'', пример}
		\begin{center}
			\includegraphics[width=0.38\textwidth]{noFlyweight.png}
			\raisebox{0.1\textheight}{\quad\Huge{$\rightarrow$}\quad}
			\includegraphics[width=0.38\textwidth]{flyweightExample.png}
		\end{center}
	\end{frame}

	\begin{frame}
		\frametitle{``Приспособленец'', общая схема}
		\begin{center}
			\includegraphics[width=0.7\textwidth]{flyweight.png}
		\end{center}
		\begin{footnotesize}
			\begin{itemize}
				\item \textit{Flyweight} --- определяет интерфейс, через который приспособленцы могут получать внешнее состояние
				\item \textit{ConcreteFlyweight} --- реализует интерфейс Flyweight и может иметь внутреннее состояние, не зависит от контекста
				\item \textit{UnsharedConcreteFlyweight} --- неразделяемый ``приспособленец'', хранящий всё состояние в себе, бывает нужен, чтобы собирать иерархические структуры из Flyweight-ов (``Компоновщик'')
				\item \textit{FlyweightFactory} --- содержит пул приспособленцев, создаёт их и управляет их жизнью
			\end{itemize}
		\end{footnotesize}
	\end{frame}

	\begin{frame}
		\frametitle{``Приспособленец'', диаграмма объектов}
		\begin{center}
			\includegraphics[width=0.7\textwidth]{flyweightObjects.png}
		\end{center}
		\begin{itemize}
			\item Клиенты могут быть разных типов
			\item Клиенты могут разделять приспособленцев
			\begin{itemize}
				\item Один клиент может иметь несколько ссылок на одного приспособленца
			\end{itemize}
			\item Во время выполнения клиенты имеют право не знать про фабрику
		\end{itemize}
	\end{frame}

	\begin{frame}
		\frametitle{Когда применять}
		\begin{itemize}
			\item Когда в приложении используется много мелких объектов
			\item Они допускают разделение состояния на внутреннее и внешнее
			\begin{itemize}
				\item Желательно, чтобы внешнее состояние было вычислимо
			\end{itemize}
			\item Идентичность объектов не важна
			\begin{itemize}
				\item Используется семантика Value Type
			\end{itemize}
			\item Главное, когда от такого разделения можно получить ощутимый выигрыш
		\end{itemize}
	\end{frame}

	\begin{frame}
		\frametitle{Тонкости реализации}
		\begin{itemize}
			\item Внешнее состояние --- по сути, отдельный объект, поэтому если различных внешних состояний столько же, сколько приспособленцев, смысла нет
			\begin{itemize}
				\item Один объект-состояние покрывает сразу несколько приспособленцев
				\begin{itemize}
					\item Например, объект ``Range'' может хранить параметры форматирования для всех букв внутри фрагмента
				\end{itemize}
			\end{itemize}
			\item Клиенты не должны инстанцировать приспособленцев сами, иначе трудно обеспечить разделение
			\begin{itemize}
				\item Имеет смысл иметь механизм для удаления неиспользуемых приспособленцев
				\begin{itemize}
					\item Если их может быть много
				\end{itemize}
			\end{itemize}
			\item Приспособленцы немутабельны и Value Objects (с правильно переопределённой операцией сравнения)
			\begin{itemize}
				\item Про hashCode() тоже надо не забыть
			\end{itemize}
		\end{itemize}
	\end{frame}

	\section{Паттерн ``Фабричный метод''}

	\begin{frame}
		\frametitle{``Фабричный метод'' мотивация}
		\framesubtitle{Игра-стратегия}
		\begin{columns}
			\begin{column}{0.5\textwidth}
				\begin{itemize}
					\item Воины
					\begin{itemize}
						\item Мечники
						\item Конница
						\item Лучники
					\end{itemize}
					\item Общее поведение
					\item Общие характеристики
				\end{itemize}
			\end{column}
			\begin{column}{0.5\textwidth}
				\begin{center}
					\includegraphics[width=0.8\textwidth]{warriors.png}
				\end{center}
			\end{column}
		\end{columns}
	\end{frame}

	\begin{frame}[fragile]
		\frametitle{Виртуальный конструктор}
		\begin{columns}
			\begin{column}{0.6\textwidth}
				\begin{footnotesize}
					\begin{minted}{c++}
enum WarriorId { SwordsmanId, ArcherId, HorsemanId };

class Warrior
{
public:  
    Warrior(WarriorId id)
    {
        if (id == SwordsmanId) p = new Swordsman;
        else if (id == ArcherId) p = new Archer;
        else if (id == HorsemanId) p = new Horseman;
        else assert( false);
    }
    virtual void info() { p->info(); }
    virtual ~Warrior() { delete p;  p = 0; }
private:
    Warrior* p;
};
					\end{minted}
				\end{footnotesize}
			\end{column}
			\begin{column}{0.4\textwidth}
				\begin{center}
					\includegraphics[width=0.95\textwidth]{warriorVirtualCtor.png}
				\end{center}
			\end{column}
		\end{columns}
	\end{frame}

	\begin{frame}
		\frametitle{Фабричный метод}
		\begin{columns}
			\begin{column}{0.5\textwidth}
				\begin{itemize}
					\item Базовый класс знает про остальные
					\item switch в createWarrior()
				\end{itemize}
			\end{column}
			\begin{column}{0.5\textwidth}
				\begin{center}
					\includegraphics[width=0.8\textwidth]{warriorFactoryMethod.png}
				\end{center}
			\end{column}
		\end{columns}
	\end{frame}

	\begin{frame}
		\frametitle{Паттерн ``Factory Method''}
		\framesubtitle{Factory Method}
		\begin{center}
			\includegraphics[width=0.8\textwidth]{factoryMethod.png}
		\end{center}
		\begin{itemize}
			\item Применимость:
			\begin{itemize}
				\item классу заранее неизвестно, объекты каких классов ему нужно создавать
				\item объекты, которые создает класс, специфицируются подклассами
				\item класс делегирует свои обязанности одному из нескольких вспомогательных подклассов
			\end{itemize}
		\end{itemize}
	\end{frame}

	\begin{frame}
		\frametitle{Пример, текстовый редактор}
		\begin{center}
			\includegraphics[width=0.9\textwidth]{factoryMethodForTextEditor.png}
		\end{center}
	\end{frame}

	\begin{frame}
		\frametitle{``Фабричный метод'', детали реализации}
		\begin{center}
			\includegraphics[width=0.6\textwidth]{factoryMethod.png}
		\end{center}
		\begin{itemize}
			\item Абстрактный Creator или реализация по умолчанию
			\begin{itemize}
				\item Второй вариант может быть полезен для расширяемости
			\end{itemize}
			\item Параметризованные фабричные методы
			\item Если язык поддерживает инстанциацию по прототипу (JavaScript, Smalltalk), можно хранить порождаемый объект
			\item Creator не может вызывать фабричный метод в конструкторе
			\item Можно сделать шаблонный Creator
		\end{itemize}
	\end{frame}

	\section{Паттерн ``Абстрактная фабрика''}

	\begin{frame}
		\frametitle{``Абстрактная фабрика'', мотивация}
		\begin{itemize}
			\item Хотим поддержать разные стили UI
			\begin{itemize}
				\item Гибкая поддержка в архитектуре
				\item Удобное добавление новых стилей
			\end{itemize}
		\end{itemize}
		\begin{center}
			\includegraphics[width=0.95\textwidth]{widgets.png}
		\end{center}
	\end{frame}

	\begin{frame}
		\frametitle{Создание виджетов}
		\mintinline{c++}|ScrollBar* bar = new RoundedScrollBar;|
		
		\vspace{2mm}
		
		vs
		
		\vspace{2mm}
		
		\mintinline{c++}|ScrollBar* bar = guiFactory->createScrollBar();|
	\end{frame}

	\begin{frame}
		\frametitle{Фабрика виджетов}
		\begin{center}
			\includegraphics[width=0.95\textwidth]{widgetFactory.png}
		\end{center}
	\end{frame}

	\begin{frame}
		\frametitle{Паттерн ``Абстрактная фабрика''}
		\framesubtitle{Abstract Factory}
		\begin{columns}
			\begin{column}{0.4\textwidth}
				\begin{itemize}
					\item Изолирует конкретные классы
					\item Упрощает замену семейств продуктов
					\item Гарантирует сочетаемость продуктов
					\item Поддержать новый вид продуктов непросто
				\end{itemize}
			\end{column}
			\begin{column}{0.6\textwidth}
				\begin{center}
					\includegraphics[width=0.95\textwidth]{abstractFactory.png}
				\end{center}
			\end{column}
		\end{columns}
	\end{frame}

	\begin{frame}
		\frametitle{``Абстрактная фабрика'', применимость}
		\begin{itemize}
			\item Система не должна зависеть от того, как создаются, компонуются и представляются входящие в нее объекты
			\item Система должна конфигурироваться одним из семейств составляющих ее объектов
			\item Взаимосвязанные объекты должны использоваться вместе
			\item Хотите предоставить библиотеку объектов, раскрывая только их интерфейсы, но не реализацию
		\end{itemize}
	\end{frame}

	\begin{frame}
		\frametitle{``Абстрактная фабрика'', детали реализации}
		\begin{columns}
			\begin{column}{0.5\textwidth}
				\begin{itemize}
					\item Хорошо комбинируются с паттерном ``Одиночка''
					\item Если семейств продуктов много, то фабрика может инициализироваться \textit{прототипами}, тогда не надо создавать сотню подклассов
				\end{itemize}
			\end{column}
			\begin{column}{0.5\textwidth}
				\begin{center}
					\includegraphics[width=\textwidth]{abstractFactory.png}
				\end{center}
			\end{column}
		\end{columns}
		\begin{itemize}
			\item Прототип на самом деле может быть классом (например, Class в Java)
			\item Если виды объектов часто меняются, может помочь параметризация метода создания
			\begin{itemize}
				\item Может пострадать типобезопасность
			\end{itemize}
		\end{itemize}
	\end{frame}

	\section{Паттерн ``Одиночка''}

	\begin{frame}
		\frametitle{Паттерн ``Одиночка''}
		\framesubtitle{Singleton}
		\begin{columns}
			\begin{column}{0.6\textwidth}
				\begin{itemize}
					\item Гарантирует, что у класса есть только один экземпляр
					\item Предоставляет глобальный доступ к этому экземпляру
					\item Позволяет использовать подклассы без модификации клиентского кода
				\end{itemize}
			\end{column}
			\begin{column}{0.4\textwidth}
				\begin{center}
					\includegraphics[width=0.6\textwidth]{singleton.png}
				\end{center}
			\end{column}
		\end{columns}
	\end{frame}

	\begin{frame}[fragile]
		\frametitle{``Одиночка'', наивная реализация}
		\begin{minted}{java}
public class Singleton {
    private static Singleton instance;

    private Singleton () {}

    public static Singleton getInstance() {
        if (instance == null) {
            instance = new Singleton();
        }
        return instance;
    }
}
		\end{minted}
	\end{frame}

	\begin{frame}[fragile]
		\frametitle{``Одиночка'', простая многопоточная реализация}
		\begin{minted}{java}
public class Singleton {
    private static Singleton instance = new Singleton();

    private Singleton () {}

    public static Singleton getInstance() {
        return instance;
    }
}
		\end{minted}
	\end{frame}

	\begin{frame}[fragile]
		\frametitle{``Одиночка'', плохая многопоточная реализация}
		\begin{minted}{java}
public class Singleton {
    private static Singleton instance;

    public static synchronized Singleton getInstance() {
        if (instance == null) {
            instance = new Singleton();
        }
        return instance;
    }
}
		\end{minted}
	\end{frame}

	\begin{frame}[fragile]
		\frametitle{Double-checked locking}
		\framesubtitle{Не делайте так}
		\begin{small}
			\begin{minted}{java}
public class Singleton {
    private static volatile Singleton instance;

    public static Singleton getInstance() {
        Singleton localInstance = instance;
        if (localInstance == null) {
            synchronized (Singleton.class) {
                localInstance = instance;
                if (localInstance == null) {
                    instance = localInstance = new Singleton();
                }
            }
        }
        return localInstance;
    }
}
			\end{minted}
		\end{small}
	\end{frame}

	\begin{frame}
		\frametitle{``Одиночка'', критика}
		\begin{itemize}
			\item Добавляет неочевидные зависимости по данным
			\begin{itemize}
				\item По сути, хитрая глобальная переменная
			\end{itemize}
			\item Усложняет тестирование
			\item Нарушает принцип единственности ответственности
			\item Сложно рефакторить, если потребуется несколько экземпляров
		\end{itemize}
	\end{frame}

	\section{Паттерн ``Прототип''}

	\begin{frame}
		\frametitle{``Прототип'', мотивация}
		\begin{center}
			\includegraphics[width=0.9\textwidth]{musicalEditor.png}
		\end{center}
	\end{frame}

	\begin{frame}
		\frametitle{Патерн ``Прототип''}
		\framesubtitle{Prototype}
		\begin{center}
			\includegraphics[width=0.85\textwidth]{prototype.png}
		\end{center}
	\end{frame}
	
	\begin{frame}
		\frametitle{``Прототип'', детали реализации}
		\begin{itemize}
			\item Паттерн интересен только для языков, где мало runtime-информации о типе (C++)
			\item Реестр прототипов, обычно ассоциативное хранилище
			\item Операция Clone
			\begin{itemize}
				\item Глубокое и мелкое копирование
				\item В случае, если могут быть круговые ссылки
				\item Сериализовать/десериализовать объект (но помнить про идентичность)
			\end{itemize}
			\item Инициализация клона
			\begin{itemize}
				\item Передавать параметры в Clone --- плохая идея
			\end{itemize}
		\end{itemize}
	\end{frame}

	\section{Паттерн ``Строитель''}

	\begin{frame}
		\frametitle{``Строитель'', мотивация}
		\framesubtitle{Конвертер текста}
		\begin{center}
			\includegraphics[width=0.8\textwidth]{textConverter.png}
		\end{center}
	\end{frame}

	\begin{frame}
		\frametitle{Патерн ``Строитель''}
		\framesubtitle{Builder}
		\begin{center}
			\includegraphics[width=0.85\textwidth]{builder.png}
		\end{center}
	\end{frame}
	
	\begin{frame}
		\frametitle{``Строитель'' (Builder), детали реализации}
		\begin{center}
			\includegraphics[width=0.7\textwidth]{builder.png}
		\end{center}
		\begin{itemize}
			\item Абстрактные и конкретные строители
			\begin{itemize}
				\item Достаточно общий интерфейс
			\end{itemize}
			\item Общий интерфейс для продуктов не требуется
			\begin{itemize}
				\item Клиент конфигурирует распорядителя конкретным строителем, он же и забирает результат
			\end{itemize}
			\item Пустые методы по умолчанию
		\end{itemize}
	\end{frame}

	\begin{frame}[fragile]
		\frametitle{``Строитель'', примеры}
		\begin{itemize}
			\item StringBuilder
			\item Guava, подсистема работы с графами
			\begin{minted}{java}
MutableNetwork<Webpage, Link> webSnapshot = 
        NetworkBuilder.directed()
    .allowsParallelEdges(true)
    .nodeOrder(ElementOrder.natural())
    .expectedNodeCount(100000)
    .expectedEdgeCount(1000000)
    .build();
			\end{minted}
		\end{itemize}
\end{frame}

\end{document}
