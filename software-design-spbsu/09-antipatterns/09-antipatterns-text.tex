\documentclass[a5paper]{article}
\usepackage[a5paper, top=8mm, bottom=8mm, left=8mm, right=8mm]{geometry}

\usepackage{polyglossia}
\setdefaultlanguage[babelshorthands=true]{russian}

\usepackage{fontspec}
\setmainfont{FreeSerif}
\newfontfamily{\russianfonttt}[Scale=0.7]{DejaVuSansMono}

\usepackage[font=scriptsize]{caption}

\usepackage{amsmath}
\usepackage{amssymb,amsfonts,textcomp}
\usepackage{color}
\usepackage{array}
\usepackage{hhline}
\usepackage{cite}
\usepackage{textcomp}

\usepackage[hang,multiple]{footmisc}
\renewcommand{\footnotelayout}{\raggedright}

\PassOptionsToPackage{hyphens}{url}\usepackage[xetex,linktocpage=true,plainpages=false,pdfpagelabels=false]{hyperref}
\hypersetup{colorlinks=true, linkcolor=blue, citecolor=blue, filecolor=blue, urlcolor=blue, pdftitle=1, pdfauthor=, pdfsubject=, pdfkeywords=}

\newlength\Colsep
\setlength\Colsep{10pt}

\usepackage{tabu}

\usepackage{graphicx}
\usepackage{indentfirst}
\usepackage{multirow}
\usepackage{subfig}
\usepackage{footnote}
\usepackage{minted}

\newcommand{\todo}[1] {
\begin{center}\textcolor{red}{TODO: #1}\end{center}
}

\newcommand{\attribution}[1] {
	\vspace{-5mm}\begin{flushright}\begin{scriptsize}%\textcolor{gray}
	{\textcopyright\, #1}\end{scriptsize}\end{flushright}
}

\sloppy
\pagestyle{plain}

\title{Лекция 9: Антипаттерны}
\author{Юрий Литвинов\\\small{yurii.litvinov@gmail.com}}
\date{}

\begin{document}

\maketitle
\thispagestyle{empty}

\section{Введение}

В этой лекции будет рассказано про то, как делать не надо. В литературе есть термин ``Антипаттерны'', имеющий тот же смысл, что и паттерны, но наоборот: если паттерны --- это часто встречающиеся решения, приводящие к известным преимуществам, то антипаттерны --- это часто встречающиеся решения, приводящие к известным проблемам. Оказывается, что знание антипаттернов для архитектора даже полезнее, чем знание паттернов, поскольку позволяет избежать очень частых и очень дорогостоящих ошибок.

Точно так же, как и паттерны, антипаттерны документируют, чтобы иметь общий словарь (например, не пускаться в пространные рассказы про то, что есть принцип единственности ответственности, что класс, который делает вообще всё --- это плохо, и почему, а просто сказать, что это антипаттерн ``God Object''). Кроме того, описание антипаттерна, помимо описания проблемы и пояснения, почему это плохо, должно по-хорошему содержать и рекомендации по тому, как это исправить. 

Хочется заметить, что антипаттерны часто описывают решения, которые сами по себе неплохи и вполне применяются на практике без всяких проблем, типичный пример --- это ``Busy waiting'', успешно применяющийся в микроконтроллерах. Источником проблем часто является не само решение, а контекст его применения --- например, ``Busy waiting'' может быть очень плохой идеей, если минутами грузит процессор на ноутбуке.

Антипаттерны, так же, как и паттерны, имеют некую классификацию, но не по назначению, а по ``сфере применения'' --- антипаттерны реализации (в том числе, специфичные для конкретного языка или технологии, которые мы в этой лекции затрагивать не будем), архитектурные антипаттерны и организационные антипаттерны (которые мы тоже затрагивать практически не будем, потому что это больше про управление проектами, чем про архитектуру). Мы начнём с антипаттернов реализации и продолжим архитектурными антипаттернами.

\noindent\begin{minipage}{\textwidth}
	\begin{minipage}[c][6cm][c]{\dimexpr0.7\textwidth-0.5\Colsep\relax}
		Дальнейшее изложение будет вестись по книжке AntiPatterns: Refactoring Software, Architectures, and Projects in Crisis by William J. Brown, Raphael C. Malveau, SkipMcCormick, Thomas J. Mowbray, Wiley, 1998, 336pp. Книга столь же старая, что и книга про паттерны, и с момента её издания появилась куча специфичных для языков и технологий антипаттернов, но они нам как раз и не очень интересны. Антипаттерны, касающиеся разработки в целом, за это время нисколько не поменялись. Книга не входит в рекомендованную по этому курсу литературу, потому что там много воды и наиболее содержательные части всё равно будут в этой лекции рассказаны, так что читайте только если особо интересно.
	\end{minipage}\hfill
	\begin{minipage}[c][6cm][c]{\dimexpr0.3\textwidth-0.5\Colsep\relax}
		\includegraphics[width=0.6\textwidth]{bookCover.png}
	\end{minipage}%
\end{minipage}

\section{Семь причин провала проектов}

Авторы книжки про антипаттерны вделяют семь главных причин, по которым эти антипаттерны, собственно, вообще появляются в промышленном коде. Эти причины таковы.

\begin{itemize}
	\item Спешка --- пожалуй, самая главная причина появления плохого кода. Бывает тяжело думать над правильным и архитектурно красивым решением, если завтра релиз, а ещё не поправлен десяток критических багов. Проблема в том, что промышленная разработка почти всегда ведётся в спешке, и на это есть простые экономические причины --- если у программистов есть время делать хорошо и правильно, они могли бы потратить это время на то, чтобы сделать больше фич или сдать больше проектов, поэтому никто, кроме самих программистов, не заинтересован в качестве кода. Заказчику не нужна архитектура, ему нужен работающий продукт. И если даже иногда оно будет вести себя странно, это тоже может быть некритично --- пользователь перезапустит приложение, и всё. Из-за этого растёт технический долг --- вы думаете, что завтра релиз и послезавтра вы перепишете все костыли, что сейчас навставляли, но послезавтра новый релиз через две недели, на вас уже написали задач, с которыми вы чувствуете, что справитесь только недели через три, и цикл повторяется. И получение позитивной обратной связи в виде довольного менеджмента и заказчика только ухудшает ситуацию.
	\item Апатия --- неприменение известных хороших решений, потому что, казалось бы, незачем. Зачем заботиться о переиспользовании, если код всё равно пишется на выброс? Зачем заботиться о производительности, если всё равно потом надо будет оптимизировать код? Зачем думать о расширяемости, если проект всё равно закончится через два месяца? Однако оказывается, что оценки автора кода, касающиеся сроков его жизни, могут отличаться от истины в сотни раз, хороший пример тому --- проблема 2000 (Y2K Problem), когда авторы приложений 80-х хранили год двумя цифрами, будучи уверенными, что уж до 2000-го года их приложения сотню раз заменят на новые.
	\item Недалёкость --- незнание хороших решений. Незнание само по себе не является проблемой, скорее, это обычная ситуация в программной инженерии, одних только библиотек для frontend-разработки появляется по нескольку каждый месяц, и знать всё невозможно. Проблема --- нежелание погуглить. Ну и незнание, что гуглить --- если вы не имеете представления о том, что вообще бывает, то и интернет вам не поможет (для чего, кстати, и нужно современному специалисту университетское образование --- знать, что вообще искать в интернете, и уметь выбрать нужное).
	\item Лень --- движитель прогресса, если её правильно использовать, но если использовать неправильно, то причина появления дурацкого кода, минимальными усилиями решающего поставленную задачу в минимальной её интерпретации.
	\item Жадность --- наоборот, желание сделать больше работы, чем надо (например, из желания продвинуться по служебной лестнице или просто неумения делегировать). Хороший пример --- архитектурная жадность, когда архитектор специфицирует структуру системы с точностью до каждого параметра каждого метода каждого класса. Архитектура выглядит очень детальной, архитектор выглядит очень крутым, но в ходе реализации выясняется, что это всё не имеет отношения к реальной жизни. А если архитектор будет настаивать, то результат получится гораздо хуже, чем если бы излишние подробности доверили программистам.
	\item Неведение --- нежелание понимать причины и смысл чужих решений. Например, этим страдают звёздные программисты, начинающие рефакторить под свой любимый стайлгайд весь существующий код проекта, или архитекторы, которые считают, что они архитекторы и поэтому правы.
	\item Гордость --- нежелание переиспользовать готовые решения. Знаменитый синдром ``Not invented here'' можно отнести именно к гордости. Иногда это неплохо, во избежание привязки проекта к конкретному поставщику конкретной библиотеки или просто из соображений владения кодом и возможности модифицировать его под свои цели. Иногда это очень плохо, поскольку в разы увеличивает трудозатраты при реализации проекта, да и результат получается существенно хуже, чем мог бы быть.
\end{itemize}

\section{Антипаттерны реализации}

Рассмотрение собственно антипаттернов мы начнём с антипаттернов реализации --- частых структур, встречающихся в коде проекта, которые имеют свои корни в неудачной низкоуровневой архитектуре.

\subsection{Circular dependency}

Circular dependency, или ``круговая зависимость'' --- это когда два или больше компонентов зависят друг от друга циклически, например, первый зависит от второго, второй от третьего, а третий от первого. Появляется чаще всего от незнания принципа Dependency Inversion или от неудачных попыток добавить callback-и, чтобы организовать реально двухстороннее общение компонент. Обратите внимание, что компоненты, обменивающиеся данными в обе стороны --- это не Circular dependency, это нормально. Circular dependency --- это круговая зависимость во время компиляции или круговая зависимость в графе вызовов. Особо опасна круговая зависимость заголовочных файлов в C++ --- положим у нас есть файл a.h такого содержания:

\begin{minted}{cpp}
#pragma once

#include "b.h"

class A {
    public:
        B b;
}
\end{minted}

И файл b.h:

\begin{minted}{cpp}
#pragma once

#include "a.h"

class B {
    public:
        A a;
}
\end{minted}

При использовании этих файлов мы неизбежно получим ошибку ``неизвестный идентификатор'' либо на поле типа A, либо на поле типа B. И это может повергнуть в шок начинающего C++-программиста, потому что объявление --- вот оно, include написан, IDE всё подсвечивает и показывает правильно, а компилятор недоволен. Почему --- стражи включения (та самая pragma once) заставляют препроцессор подключать кадждый файл только один раз, при подключении a.h подключится b.h, который захочет подключить a.h, но не сможет, потому что тот файл препроцессор уже видел, ну и в классе B объявление класса A будет не видно (файл подключён, но ``ниже'' по коду).

\end{document}
