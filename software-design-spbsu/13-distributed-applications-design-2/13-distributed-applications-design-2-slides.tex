\documentclass[xetex,mathserif,serif]{beamer}
\usepackage{polyglossia}
\setdefaultlanguage[babelshorthands=true]{russian}
\usepackage{minted}
\usepackage{tabu}

\usepackage{textpos}
\setlength{\TPHorizModule}{1cm}
\setlength{\TPVertModule}{1cm}

\useoutertheme{infolines}

\usepackage{fontspec}
\setmainfont{FreeSans}
\newfontfamily{\russianfonttt}{FreeSans}

\tabulinesep=0.7mm

\title{Проектирование распределённых приложений}
\subtitle{Часть вторая: высокоуровневые вещи}
\author[Юрий Литвинов]{Юрий Литвинов \newline \textcolor{gray}{\small\texttt{yurii.litvinov@gmail.com}}}

\date{14.12.2017}

\begin{document}
	
	\frame{\titlepage}

	\section{RabbitMQ}

	\begin{frame}
		\frametitle{RabbitMQ}
		\begin{itemize}
			\item Сервер и клиенты системы надёжной передачи сообщений
			\begin{itemize}
				\item Сообщение посылается на сервер и хранится там, пока его не заберут
				\item Продвинутые возможности по маршрутизации сообщений
			\end{itemize}
			\item Реализует протокол AMQP (Advanced Message Queuing Protocol), но может использовать и другие протоколы
			\item Сервер написан на Erlang, клиентские библиотеки доступны для практически чего угодно
		\end{itemize}
		\begin{textblock}{3}(8,0)
			\includegraphics[width=\textwidth]{rabbitmqLogo.png}
		\end{textblock}
	\end{frame}

\end{document}
