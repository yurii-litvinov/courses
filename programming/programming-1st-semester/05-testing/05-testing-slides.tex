\documentclass[xetex,mathserif,serif]{beamer}
\usepackage{polyglossia}
\setdefaultlanguage[babelshorthands=true]{russian}
\usepackage{minted}
\usepackage{tabu}

\useoutertheme{infolines}

\usepackage{fontspec}
\setmainfont{FreeSans}
\newfontfamily{\russianfonttt}{FreeSans}

\usepackage{forest}
\usetikzlibrary{arrows}

\setbeamertemplate{blocks}[rounded][shadow=false]
\setbeamercolor*{block title example}{fg=green!50!black,bg=green!20}
\setbeamercolor*{block body example}{fg=black,bg=green!10}

\setbeamercolor*{block title alerted}{fg=red!50!black,bg=red!20}
\setbeamercolor*{block body alerted}{fg=black,bg=red!10}

\tabulinesep=0.7mm

\title{Тестирование и отладка}
\author[Юрий Литвинов]{Юрий Литвинов \newline \textcolor{gray}{\small\texttt{yurii.litvinov@gmail.com}}}

\date{15.09.2020}

\begin{document}
    
    \frame{\titlepage}
    
    \begin{frame}
        \frametitle{Тестирование}
        \begin{itemize}
            \item Любая программа содержит ошибки
            \item Если программа не содержит ошибок, их содержит алгоритм, который реализует эта программа
            \item Если ни программа, ни алгоритм ошибок не содержат, такая программа даром никому не нужна
        \end{itemize}
        \vspace{1cm}
        Тестирование не позволяет доказать отсутствие ошибок, оно позволяет лишь найти ошибки, которые в программе присутствуют
    \end{frame}

    \begin{frame}
        \frametitle{Виды тестирования}
        \begin{itemize}
            \item По уровню тестируемых компонент
            \begin{itemize}
                \item Модульное
                \item Интеграционное
                \item Системное
            \end{itemize}
            \item По целям
            \begin{itemize}
                \item Функциональное
                \item Нагрузочное
                \item Удобства использования
                \item Смоук-тестирование
                \item Регрессионное
                \item Приёмочное тестирование
            \end{itemize}
        \end{itemize}
    \end{frame}

    \begin{frame}
        \frametitle{Тестирование, выполняемое программистами}
        \begin{itemize}
            \item Тестирование типичного сценария работы
            \item Тестирование граничных случаев
            \item Тестирование некорректных входных данных
            \begin{itemize}
                \item Программа должна адекватно себя вести и сообщать об ошибках ввода
            \end{itemize}
            \item Тестирование должно быть по возможности автоматическим
            \begin{itemize}
                \item На самом деле, используются модульные тесты
                \item Можно писать модульные тесты вручную, как функции, возвращающие true/false
            \end{itemize}
        \end{itemize}
    \end{frame}
    
    \begin{frame}[fragile]
        \frametitle{Пример типичного теста}
        \begin{footnotesize}
            \begin{minted}{c}
bool balanceOfParentheses(const char* parentheses) {
    ...
}

bool testCorrectCase() {
    return balanceOfParentheses("()");
}

bool testIncorrectCases() {
    return !balanceOfParentheses("((") && !balanceOfParentheses(")(") ;
}

void main() {
    if (!testCorrectCase() || !testIncorrectCases()) {
        printf("Tests failed\n");
        return;
    }
    printf("Enter string\n");
    ...
}
            \end{minted}
        \end{footnotesize}
    \end{frame}

    \begin{frame}
        \frametitle{Отладка}
        \begin{itemize}
            \item Устойчивое воспроизведение ошибки
            \begin{itemize}
                \item Вместо srand(time(nullptr)) – srand(<какое-то фиксированное значение>)
                \item Ошибка должна воспроизводиться быстро
            \end{itemize}
            \item Локализация ошибки
            \begin{itemize}
                \item Аналитически
                \item Отладка
            \end{itemize}
            \item Отладочная гипотеза
            \begin{itemize}
                \item Похоже на научный подход --- гипотеза, эксперимент, уточнение, эксперимент и т.д.
                \item Тестовый прогон с отладочной печатью
                \item Тестовый прогон под отладчиком
            \end{itemize}
        \end{itemize}
    \end{frame}

    \begin{frame}
        \frametitle{Отладчик}
        \begin{Large}
            \begin{center}
                Демонстрация
            \end{center}
        \end{Large}
    \end{frame}

\end{document}

