\documentclass[xetex,mathserif,serif]{beamer}
\usepackage{polyglossia}
\setdefaultlanguage[babelshorthands=true]{russian}
\usepackage{minted}
\usepackage{tabu}
\usepackage{pgfplots}

\useoutertheme{infolines}

\usepackage{fontspec}
\setmainfont{FreeSans}
\newfontfamily{\russianfonttt}{FreeSans}

\usepackage{textpos}
\setlength{\TPHorizModule}{1cm}
\setlength{\TPVertModule}{1cm}

\usepackage{forest}
\usetikzlibrary{arrows}

\definecolor{links}{HTML}{2A1B81}
\hypersetup{colorlinks,linkcolor=,urlcolor=links}

\tabulinesep=0.7mm

\title{Внутреннее представление данных}
\author[Юрий Литвинов]{Юрий Литвинов \newline \textcolor{gray}{\small\texttt{yurii.litvinov@gmail.com}}}

\date{01.10.2019}

\begin{document}
	
	\frame{\titlepage}
	
	\begin{frame}
		\frametitle{Побитовые операции}
		\begin{columns}
			\begin{column}{0.6\textwidth}
				\begin{itemize}
					\item \& --- побитовое ``И''
					\item | --- побитовое ``ИЛИ''
					\item $\sim$ --- побитовое ``НЕ''
					\item 1 \& 2 == false, но 1 \&\& 2 == true
					\item $<<$, $>>$ --- битовый сдвиг
					\begin{itemize}
						\item int x = 1 $<<$ 3
					\end{itemize}
					\item sizeof --- размер типа в байтах
					\begin{itemize}
						\item int s = sizeof(int) * 8
					\end{itemize}
					\item Обратите внимание, что ВСЁ хранится как набор бит
					\begin{itemize}
						\item ``3'' --- литерал, лишь удобная форма записи 00...0011 в коде
					\end{itemize}
				\end{itemize}
			\end{column}
			\begin{column}{0.4\textwidth}
				Маски
				\vspace{3mm}
				\begin{textblock}{1}(-0.35,0.4)
					\&
				\end{textblock}
				\begin{tabu} {| X[1 l p] | X[1 l p] | X[1 l p] | X[1 l p] | X[1 l p] | X[1 l p] | X[1 l p] | X[1 l p] |}
					\tabucline-
					\everyrow{\tabucline-}
					1 & 1 & 0 & 1 & 1 & 0 & 1 & 0 \\
					0 & 0 & 0 & 0 & 0 & 0 & 0 & 1 \\
					0 & 0 & 0 & 0 & 0 & 0 & 0 & 0 \\
				\end{tabu}
				\vspace{0.5cm}

				\begin{textblock}{1}(-0.35,0.4)
					\&
				\end{textblock}
				\begin{tabu} {| X[1 l p] | X[1 l p] | X[1 l p] | X[1 l p] | X[1 l p] | X[1 l p] | X[1 l p] | X[1 l p] |}
					\tabucline-
					\everyrow{\tabucline-}
					1 & 1 & 0 & 1 & 1 & 0 & 1 & 0 \\
					0 & 0 & 0 & 0 & 0 & 0 & 1 & 0 \\
					0 & 0 & 0 & 0 & 0 & 0 & 1 & 0 \\
				\end{tabu}
			\end{column}
		\end{columns}
	\end{frame}

	\begin{frame}[fragile]
		\frametitle{Работа с масками}
		\begin{footnotesize}
			\begin{minted}{cpp}
char x = 5;

int bit = 0b10000000;
for (int j = 0; j < 8; ++j)
{
    printf((x & bit) ? "1" : "0");
    bit = bit >> 1;
}
			\end{minted}
		\end{footnotesize}
	\end{frame}

	\begin{frame}
		\frametitle{Целые числа}
		\begin{itemize}
			\item Прямой код
			\begin{itemize}
				\item $5$ --- $00000101$, $-5$ --- $10000101$
			\end{itemize}
			\item Обратный код
			\begin{itemize}
				\item $5$ --- $00000101$, $-5$ --- $11111010$
			\end{itemize}
			\item Дополнительный код
			\begin{itemize}
				\item $5$ --- $00000101$, $-5$ --- $11111011$
				\item $-x$ представляется как $2^n - x$, поэтому и дополнительный
				\begin{itemize}
					\item $n$ --- разрядность регистра
				\end{itemize} 
			\end{itemize}
		\end{itemize}
	\end{frame}

	\begin{frame}
		\frametitle{Арифметические действия}
		\begin{itemize}
			\item В обратном коде единица переноса в старшем разряде прибавляется к младшему разряду
			\item В дополнительном коде единица переноса в старшем разряде отбрасывается
		\end{itemize}
	\end{frame}

	\begin{frame}[fragile]
		\frametitle{Формат записи}
		\begin{itemize}
			\item Литералы
			\begin{itemize}
				\item \mintinline{cpp}|int hexadecimal = 0x35FF;|
				\item \mintinline{cpp}|int octal = 03567;|
				\item \mintinline{cpp}|int binary = 0b00100111;| (С++14) 
				\item \mintinline{cpp}|0xFF == 255|
			\end{itemize}
			\item 
			\begin{footnotesize}
				\begin{minted}{cpp}
int x = 239;
unsigned char *b = reinterpret_cast<unsigned char*>(&x);
printf("0x%02X%02X%02X%02X\n", b[0], b[1], b[2], b[3]);
				\end{minted}
			\end{footnotesize}
		\end{itemize}
		\begin{center}
			\includegraphics[width=0.8\textwidth]{little-endian-big-endian.png}
		\end{center}
	\end{frame}

	\begin{frame}
		\frametitle{Вещественные числа}
		\begin{itemize}
			\item IEEE 754 --- международный стандарт
			\item $x = (+-)m * p^q$
			\begin{itemize}
				\item p --- основание системы счисления
				\item q --- порядок числа (целое число)
				\item m --- мантисса числа (правильная p-ичная дробь, у которой первая цифра после запятой не равна 0)
				\begin{itemize}
					\item Часто используют нормализованную запись, $m \in [1, p)$
				\end{itemize}
				\item Например:
				\begin{itemize}
					\item $3,1415926 = 0, 31415926 * 10^1$
					\item $1000=0,1 * 10^4$
					\item $0,123456789 = 0,123456789 * 10^0$
					\item $0,0000107_8 = 0,107_8 * 8^{-4}$
					\item $1000,0001_2 = 0, 10000001_2 * 2^4$
					\item $0 = 0,0 * 10^0$
				\end{itemize}
			\end{itemize}
		\end{itemize}
	\end{frame}

	\begin{frame}
		\frametitle{Внутреннее представление}
		\begin{itemize}
			\item $123.456$
			\item Наиболее точное представление (IEEE 754 Double, 64 бит): $1.23456000000000003069544618484E2$
			\begin{center}
				\includegraphics[width=0.8\textwidth]{internal-representation.png}
			\end{center}
			\item \url{http://www.binaryconvert.com/}
		\end{itemize}
	\end{frame}

	\begin{frame}
		\frametitle{Смещённый порядок}
		\begin{itemize}
			\item $123.456:  q = 10000000101_2???$
			\item Смещённый порядок =  $2^{a - 1} - 1$ + <истинный порядок>
			\begin{itemize}
				\item a --- количество разрядов, отводимых под порядок
				\item Чтобы не хранить знак ещё и порядка числа
			\end{itemize}
			\item $123.456 \approx 1111011.01110100101111 = 1.11101101110100101111 * 2^6$
			\item Смещённый порядок = $2^{10} - 1 + 6 = 1029_{10} = 10000000101_2$
		\end{itemize}
	\end{frame}

	\begin{frame}[fragile]
		\frametitle{Специальные числа}
		\begin{columns}
			\begin{column}{0.6\textwidth}
				\begin{itemize}
					\item Неопределённость (NaN):
					\begin{center}
						\includegraphics[width=0.8\textwidth]{nan.png}
					\end{center}
					\item Бесконечности:
					\begin{center}
						\includegraphics[width=0.8\textwidth]{infinity.png}
					\end{center}
				\end{itemize}
			\end{column}
			\begin{column}{0.4\textwidth}
				\begin{footnotesize}
					\begin{minted}{cpp}
double y = 0.0;
double x = 239.0 / y;
printf("%f", x);
					\end{minted}
				\end{footnotesize}
			\end{column}
		\end{columns}
	\end{frame}

	\begin{frame}
		\frametitle{Строки}
		 Строка как последовательность символов (их кодов) --- таблица символов
			\begin{itemize}
				\item ASCII (American Standard Code for Information Interchange)
				\begin{itemize}
					\item 8 бит на символ (0 -- 255), 0 -- 127 стандартны, 128 -- 255 --- для локальных алфавитов
					\item Кодовые страницы
					\begin{itemize}
						\item cp866
						\item cp1251
						\item koi8-r
						\item ...
					\end{itemize}
				\end{itemize}
				\item Unicode
			\end{itemize}
			Строка как последовательность байт --- кодировка
			\begin{itemize}
				\item UCS-16BE, UCS16-LE, UTF-8
			\end{itemize}
	\end{frame}

	\begin{frame}
		\frametitle{Зачем}
		\begin{itemize}
			\item Локализация --- перевод программы на другой язык (и под другую культуру)
			\item Интернационализация --- сделать так, чтобы программу было можно локализовать
			\item У однобайтовых кодировок некоторые проблемы с иероглифическими языками
			\begin{itemize}
				\item Shift JIS и прочие странные вещи
			\end{itemize}
		\end{itemize}
	\end{frame}

	\begin{frame}
		\frametitle{Юникод}
		\begin{itemize}
			\item UCS, universal character set
			\begin{itemize}
				\item Кодовые позиции --- целые числа (U+0000 – U+007F, …)
				\item Порядка 110 000 кодовых позиций
			\end{itemize}
			\item UTF, Unicode transformation format
			\begin{itemize}
				\item Кодировки --- битовое представление кодов из UCS
			\end{itemize}
			\item UTF-8
			\begin{itemize}
				\item 0x00000000 -- 0x0000007F: 0xxxxxxx
				\item 0x00000080 -- 0x000007FF: 110xxxxx 10xxxxxx
				\item 0x00000800 -- 0x0000FFFF: 1110xxxx 10xxxxxx 10xxxxxx
				\item 0x00010000 -- 0x001FFFFF: 11110xxx 10xxxxxx 10xxxxxx 10xxxxxx
				\item В точности совпадает с ASCII для первых 127 символов
			\end{itemize}
			\item BOM (Byte Order Mark)
			\begin{itemize}
				\item FE FF, FF FE, EF BB BF
			\end{itemize}
		\end{itemize}
	\end{frame}

\end{document}

