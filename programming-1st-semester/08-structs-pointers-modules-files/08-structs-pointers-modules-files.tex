\documentclass[xetex,mathserif,serif]{beamer}
\usepackage{polyglossia}
\setdefaultlanguage[babelshorthands=true]{russian}
\usepackage{minted}
\usepackage{tabu}
\usepackage{pgfplots}
\usepackage{textpos}

\useoutertheme{infolines}

\usepackage{fontspec}
\setmainfont{FreeSans}
\newfontfamily{\russianfonttt}{FreeSans}

\usepackage{forest}
\usetikzlibrary{arrows}

\setbeamertemplate{blocks}[rounded][shadow=false]
\setbeamercolor*{block title example}{fg=green!50!black,bg=green!20}
\setbeamercolor*{block body example}{fg=black,bg=green!10}

\setbeamercolor*{block title alerted}{fg=red!50!black,bg=red!20}
\setbeamercolor*{block body alerted}{fg=black,bg=red!10}

\tabulinesep=0.7mm

\title{Структуры, указатели, модули, файлы}
\author[Юрий Литвинов]{Юрий Литвинов \newline \textcolor{gray}{\small\texttt{yurii.litvinov@gmail.com}}}

\date{05.10.2018}

\begin{document}
	
	\frame{\titlepage}
	
	\begin{frame}
		\frametitle{Структуры}
		\begin{itemize}
			\item Способ группировки родственных по смыслу значений
			\item Структура --- это тип
			\begin{itemize}
				\item В памяти представляется как поля, лежащие друг за другом, возможно, с ``дырками'' (padding)
			\end{itemize}
		\end{itemize}
	\end{frame}

	\begin{frame}[fragile]
		\frametitle{Указатели и ссылки}
		\begin{itemize}
			\item Указатель --- адрес ячейки в памяти
			\item Ссылка --- ``синоним'', просто другое название для ячейки в памяти
			\begin{itemize}
				\item Можно считать, что ссылка --- это указатель, который не надо разыменовывать (и нельзя менять)
			\end{itemize}
			\item Структуры и указатели настолько часто используются вместе, что есть оператор -> (разыменовать указатель на структуру и обратиться к её полю)
			\begin{itemize}
				\item 
				\begin{footnotesize}
					\begin{minted}{cpp}
Point p = new Point { 10, 20 };
cout << p->x << p->y;
					\end{minted}
				\end{footnotesize}
				\item То же самое, что \mintinline{cpp}|(*p).x| и \mintinline{cpp}|(*p).y|
			\end{itemize}
		\end{itemize}
	\end{frame}

	\begin{frame}
		\frametitle{Файлы}
		\begin{itemize}
			\item Последовательность байтов на диске
			\begin{itemize}
				\item Бывают ``сырые'' и ``текстовые''
				\begin{itemize}
					\item Самому файлу всё равно, это лишь способы интерпретации его содержимого
				\end{itemize}
				\item Режимы доступа: r, w, a, r+, w+, a+
				\item Курсор
				\item EOF
			\end{itemize}
			\item Сишные функции
			\begin{itemize}
				\item fopen, fclose, fprintf, fscanf, fseek, ftell, fgets
			\end{itemize}
			\item C++-ные классы
			\begin{itemize}
				\item ifstream
				\begin{itemize}
					\item is\char`_open
					\item $>>$
					\item getline
				\end{itemize}
			\end{itemize}
			\item Файлы надо не забывать закрывать
		\end{itemize}
	\end{frame}

\end{document}

