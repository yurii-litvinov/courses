\documentclass[xetex,mathserif,serif]{beamer}
\usepackage{polyglossia}
\setdefaultlanguage[babelshorthands=true]{russian}
\usepackage{minted}
\usepackage{tabu}
\usepackage{pgfplots}
\usepackage{textpos}
\usepackage{subcaption}
\usepackage{graphicx}
\usepackage[normalem]{ulem}
\usepackage{algorithm2e}
\usepackage{algorithmic}
\usepackage{float}
\useoutertheme{infolines}

\usepackage{fontspec}
\setmainfont{FreeSans}
\newfontfamily{\russianfonttt}{FreeSans}

\usepackage{forest}
\usetikzlibrary{arrows}

\definecolor{links}{HTML}{2A1B81}
\hypersetup{colorlinks,linkcolor=,urlcolor=links}

\newcommand{\attribution}[1] {
	\vspace{-5mm}\begin{flushright}\begin{scriptsize}\textcolor{gray}{\textcopyright\, #1}\end{scriptsize}\end{flushright}
}

\tabulinesep=0.7mm

\title{Обзор парадигм программирования}
\author[Юрий Литвинов]{Юрий Литвинов \newline \textcolor{gray}{\small\texttt{yurii.litvinov@gmail.com}}}

\date{30.11.2018}

\begin{document}
	
	\frame{\titlepage}

	\begin{frame}
		\frametitle{План}
		\begin{enumerate}
			\item Структурное программирование
			\item Объектно-ориентированное программирование
			\item Функциональное программирование
			\item Логическое программирование
			\item Стековое программирование
		\end{enumerate}
	\end{frame}

	\begin{frame}
		\frametitle{Математические модели вычислений}
		\begin{itemize}
			\item Что можно посчитать имея вычислительную машину неограниченной мощности?
			\item Формальные модели вычислений:
			\begin{itemize}
				\item Машина Тьюринга
				\item $\lambda$-исчисление Чёрча
			\end{itemize}
			\item Тезис Чёрча: ``Любая функция, которая может быть вычислена физическим устройством, может быть вычислена машиной Тьюринга.''
		\end{itemize}
	\end{frame}

	\begin{frame}
		\frametitle{Архитектура фон Неймана}
		\begin{center}
			\includegraphics[width=0.8\textwidth]{von-neumann-architecture.png}
		\end{center}
	\end{frame}

	\begin{frame}
		\frametitle{1. Структурное программирование}
		\begin{itemize}
			\item Пришло на смену неструктурированному программированию в начале 70-х
			\item Любая программа может быть представлена как комбинация
			\begin{itemize}
				\item последовательно исполняемых операторов
				\item ветвлений
				\item итераций
			\end{itemize}
			\item Статья Дейкстры ``Go To Statement Considered Harmful'' (1968г)
		\end{itemize}
	\end{frame}

	\begin{frame}
		\frametitle{Языки-представители}
		\begin{itemize}
			\item Алгол
			\item Паскаль
			\item C
			\item Модула-2
			\item Ада
		\end{itemize}
	\end{frame}

	\begin{frame}
		\frametitle{Подробнее: Ада}
		\begin{itemize}
			\item Разработан в начале 80-х по заказу минобороны США
			\item Особенности:
			\begin{itemize}
				\item Строгая типизация
				\item Минимум автоматических преобразований типов
				\item Встроенная поддержка параллелизма
			\end{itemize}
			\item Реализация: GNAT \url{https://www.gnu.org/software/gnat/}
		\end{itemize}
	\end{frame}

	\begin{frame}
		\frametitle{2. Объектно-ориентированное программирование}
		\begin{itemize}
			\item Первый ОО-язык --- Симула-67, были и более ранние разработки
			\item Популярной методология стала только в середине 90-х
			\item Развитие связано с широким распространением графических интерфейсов и компьютерных игр
		\end{itemize}
	\end{frame}

	\begin{frame}
		\frametitle{Основные концепции}
		\begin{itemize}
			\item Программа представляет собой набор объектов
			\item Объекты взаимодействуют путём посылки сообщений по строго определённым интерфейсам
			\item Объекты имеют своё состояние и поведение
			\item Каждый объект является экземпляром некоего класса
		\end{itemize}
	\end{frame}

	\begin{frame}
		\frametitle{Основные концепции (инкапсуляция)}
		\begin{itemize}
			\item Инкапсуляция --- сокрытие реализации от пользователя. Пользователь может взаимодействовать с объектом только через интерфейс.
			\item Позволяет менять реализацию объекта, не модифицируя код, который этот объект использует
		\end{itemize}
	\end{frame}

	\begin{frame}
		\frametitle{Основные концепции (наследование)}
		\begin{itemize}
			\item Наследование позволяет описать новый класс на основе существующего, наследуя его свойства и функциональность
			\item Наследование --- отношение ``является'' между классами, с классом-наследником можно обращаться так же, как с классом-предком
		\end{itemize}
	\end{frame}

	\begin{frame}
		\frametitle{Основные концепции (полиморфизм)}
		\begin{itemize}
			\item Полиморфизм --- классы-потомки могут изменять реализацию методов класса-предка, сохраняя их сигнатуру
			\item Клиенты могут работать с объектами класса-родителя, но вызываться будут методы класса-потомка (позднее связывание)
		\end{itemize}
	\end{frame}

	\begin{frame}[fragile]
		\frametitle{Пример кода}
		\begin{footnotesize}
			\begin{minted}{cpp}
class Animal
{
    public:
        Animal(const string& _name) { name = _name; }
        virtual string talk() = 0;
        void rename(string newName);
    private:
        string name;
};
			\end{minted}
		\end{footnotesize}
	\end{frame}

	\begin{frame}[fragile]
		\frametitle{Пример кода (2)}
		\begin{footnotesize}
			\begin{minted}{cpp}
class Cat : public Animal
{
    public:
        Cat(const string& name) : Animal(name) {}
        string talk() override { return "Meow!"; }
};
class Dog : public Animal
{
    public:
        Dog(const string& name) : Animal(name) {}
        string talk() { return "Arf! Arf!"; }
};
			\end{minted}
		\end{footnotesize}
	\end{frame}

	\begin{frame}
		\frametitle{Языки-представители}
		\begin{itemize}
			\item Java
			\item C++
			\item Object Pascal / Delphi language
			\item Smalltalk
		\end{itemize}
	\end{frame}

	\begin{frame}
		\frametitle{3. Функциональное программирование}
		\begin{itemize}
			\item Вычисления рассматриваются как вычисления значения функций в математическом понимании (без побочных эффектов)
			\item Основано на $\lambda$-исчислении
		\end{itemize}
	\end{frame}

	\begin{frame}
		\frametitle{$\lambda$-исчисление}
		\begin{itemize}
			\item $\lambda$-исчисление основано на функциях
			$\lambda x.2 * x + 1$ --- функция $x \rightarrow 2 * x + 1$
			\item Функции могут принимать функции в качестве параметров и возвращать функции в качестве результата
			\item Функция от $n$ переменных может быть представлена, как функция от одной переменной, возвращающая функцию от $n-1$ переменной (карринг)
		\end{itemize}
	\end{frame}

	\begin{frame}
		\frametitle{Языки-представители}
		\begin{itemize}
			\item Лисп (LIst PRocessing)
			\item ML (OCaml)
			\item Haskell
			\item Erlang
		\end{itemize}
	\end{frame}

	\begin{frame}
		\frametitle{Особенности}
		\begin{itemize}
			\item Программы не имеют состояния и не имеют побочных эффектов
			\item Порядок вычислений не важен
			\item Циклы выражаются через рекурсию
			\item ``Ленивые'' вычисления
			\item Формальные преобразования программ по математическим законам
		\end{itemize}
	\end{frame}

	\begin{frame}
		\frametitle{Что даёт ФП?}
		\begin{itemize}
			\item Тестирование
			\item Отладка
			\item Параллелизм
			\item Горячая замена кода
			\item Машинные доказательства
			\item Оптимизация
			\item Ленивые вычисления
		\end{itemize}
	\end{frame}

	\begin{frame}[fragile]
		\frametitle{Примеры на языке Haskell}
		\begin{itemize}
			\item Факториал:
			\begin{footnotesize}
				\begin{minted}{haskell}
fact :: Integer -> Integer 
fact 0 = 1 
fact n | n > 0 = n * fact (n - 1)
				\end{minted}
			\end{footnotesize}
			\item QSort:
			\begin{footnotesize}
				\begin{minted}{haskell}
sort [] = [] 
sort (pivot:rest) = sort [y | y <- rest, y < pivot] 
                    ++ [pivot]
                    ++ sort [y | y <- rest, y >=pivot]
				\end{minted}
			\end{footnotesize}
		\end{itemize}
	\end{frame}

	\begin{frame}[fragile]
		\frametitle{F\#, мерджсорт}
		\begin{small}
			\begin{minted}{fsharp}
let rec merge l r =
    match (l, r) with
    | ([], r) -> r
    | (l, []) -> l
    | (x::xs, y::ys) -> if (x < y) then x::(merge xs r) else y::(merge l ys)
 
let rec mergesort l = 
    match l with
    | [] -> []
    | x::[] -> l
    | _ -> 
          let (left, right) = List.splitAt (List.length l / 2) l
          let ls = mergesort left
          let rs = mergesort right
          merge ls rs
			\end{minted}
		\end{small}
	\end{frame}

	\begin{frame}[fragile]
		\frametitle{F\#, бесконечная последовательность простых чисел}
		\begin{minted}{fsharp}
let isPrime number =
     seq {2 .. sqrt(double number)}
     |> Seq.exists (fun x -> number % x = 0) 
     |> not

let primeNumbers =
     Seq.initInfinite (fun i -> i + 2)
     |> Seq.filter isPrime
		\end{minted}
	\end{frame}

	\begin{frame}
		\frametitle{Логическое программирование}
		\begin{itemize}
			\item Программа представляет собой набор фактов и правил, система сама строит решение с использованием правил логики
			\begin{itemize}
				\item Использует логику предикатов как математическую формализацию
			\end{itemize}
			\item Создавалось в 60-х для решения задач искусственного интеллекта и экспертных систем
			\begin{itemize}
				\item Автоматическое доказательство теорем
			\end{itemize}
			\item Могут использоваться разные стратегии доказательства
			\begin{itemize}
				\item В общем случае, программа --- это набор фактов и правил + стратегия вывода, которая управляет тем, как новые факты получаются из существующих
				\begin{itemize}
					\item В формальной логике стратегия вывода обычно не важна, для компьютеров это критично
				\end{itemize}
			\end{itemize}
			\item Дедуктивные базы данных --- хранят факты и правила вывода
		\end{itemize}
	\end{frame}

	\begin{frame}
		\frametitle{Пролог}
		\begin{itemize}
			\item Появился в 1972 г. как научная разработка
			\item Реализации:
			\begin{itemize}
				\item SWI-Prolog (\url{http://www.swi-prolog.org/})
				\item Amzi Prolog (\url{http://www.amzi.com/})
				\item Turbo Prolog
			\end{itemize}
			\item Использует метод резолюций – последовательно перебирая правила и факты, пытается подобрать такой набор переменных, которые бы им удовлетворяли
			\begin{itemize}
				\item Пример:
				\begin{itemize}
					\item cat(tom)
					\item ?- cat(tom).

						Yes
					\item ?- cat(X).

						X = tom
				\end{itemize}
			\end{itemize}
		\end{itemize}
	\end{frame}

	\begin{frame}[fragile]
		\frametitle{Пример программы}
		\begin{minted}{prolog}
sibling(X, Y) :- parent_child(Z, X), parent_child(Z, Y).
parent_child(X, Y) :- father_child(X, Y).
parent_child(X, Y) :- mother_child(X, Y).
mother_child(trude, sally).
father_child(tom, sally).
father_child(tom, erica).
father_child(mike, tom).

?- sibling(sally, erica).
Yes
?- father_child(Father, Child).
		\end{minted}
	\end{frame}

	\begin{frame}[fragile]
		\frametitle{Императивное программирование}
		\begin{minted}{prolog}
?- write('Hello world!'), nl.
Hello world!
true.

program_optimized(Prog0, Prog) :-
    optimization_pass_1(Prog0, Prog1),
    optimization_pass_2(Prog1, Prog2),
    optimization_pass_3(Prog2, Prog).
		\end{minted}
	\end{frame}

	\begin{frame}[fragile]
		\frametitle{QSort}
		\begin{minted}{prolog}
quicksort(Xs, Ys) :- quicksort_1(Xs, Ys, []).

quicksort_1([], Ys, Ys).
quicksort_1([X|Xs], Ys, Zs) :-
    partition(Xs, X, Ms, Ns),
    quicksort_1(Ns, Ws, Zs),
    quicksort_1(Ms, Ys, [X|Ws]).
 
partition([K|L], X, M, [K|N]):-
    X < K, !,
    partition(L, X, M, N).
partition([K|L], X, [K|M], N):-
    partition(L, X, M, N).
partition([], _, [], []).
		\end{minted}
	\end{frame}

	\begin{frame}
		\frametitle{Рекурсивное программирование, РЕФАЛ}
		\begin{itemize}
			\item РЕкурсивных Функций АЛгоритмический
			\begin{itemize}
				\item В. Турчин, 1966г.
			\end{itemize}
			\item Ориентирован на символьные вычисления
			\begin{itemize}
				\item ИИ, перевод, манипуляции с формальными системами (лямбда-исчисление, например)
			\end{itemize}
			\item Использует нормальные алгорифмы Маркова в качестве математической формализации
			\item Программа записывается в виде набора функций
			\begin{itemize}
				\item Функция --- упорядоченный набор предложений
				\item Предложение состоит из шаблона и того, на что надо заменить шаблон
				\item Выражения в угловых скобках (активные выражения)
				\item Переменные
			\end{itemize}
			\item Вычисление продолжается, пока в <<поле зрения>> Рефал-машины не окажется выражение без угловых скобок
		\end{itemize}
	\end{frame}

	\begin{frame}[fragile]
		\frametitle{Рефал, пример}
		Hello, world:
		\begin{minted}{text}
$ENTRY Go { = <Hello>;}
Hello {
   = <Prout 'Hello world'>;
}
		\end{minted}
		\vspace{3mm}
		Палиндром:
		\begin{minted}{text}
Palindrom {
    s.1 e.2 s.1 = <Palindrom e.2> ;
    s.1 = True ;
    = True;
    e.1 = False ;
}
		\end{minted}
	\end{frame}

	\begin{frame}
		\frametitle{Стековое программирование}
		\begin{itemize}
			\item Язык Форт (Forth)
			\begin{itemize}
				\item Разработан в 60-х Чарльзом Муром <<для себя>>
				\item Был широко распространён для программирования встроенных систем и задач, естественным образом выражающихся в терминах стеков
				\begin{itemize}
					\item Синтаксический анализ
					\item Анализ естественных языков
				\end{itemize}
			\end{itemize}
		\end{itemize}
	\end{frame}

	\begin{frame}
		\frametitle{Форт, подробнее}
		\begin{itemize}
			\item Основной элемент программы: слово
			\item Форт-система состоит из словаря (набора слов) и стеков --- арифметического и командного (с их помощью производятся вычисления)
			\item Используется обратная польская нотация
		\end{itemize}
	\end{frame}

	\begin{frame}
		\frametitle{Примеры}
		\begin{columns}
			\begin{column}{0.82\textwidth}
				\begin{itemize}
					\item \mintinline{forth}|25 10 * 50 + .|

						Вывод: 300 ok
					\item \mintinline{forth}|: FLOOR5 ( n -- n' )   DUP 6 < IF DROP 5 ELSE 1 -| \\
						\mintinline{forth}|  THEN ;|
					\begin{itemize}
						\item то же самое на C:

						\mintinline{c}|int floor5(int v) { return v < 6 ? 5 : v - 1; }|
					\end{itemize}
					\item более красиво на Форте:

						\mintinline{forth}|: FLOOR5 ( n -- n' ) 1- 5 MAX ;|
					\item \mintinline{forth}|: HELLO  ( -- )  CR ." Hello, world!" ;|
				\end{itemize}
			\end{column}
			\begin{column}{0.18\textwidth}
				\begin{center}
					\includegraphics[width=\textwidth]{forthStack.png}
				\end{center}
			\end{column}
		\end{columns}
	\end{frame}

	\begin{frame}[fragile]
		\frametitle{Форт, пример}
		\begin{minted}{forth}
\ Напечатать знак числа
: .SIGN ( n -- )
   ?DUP 0= IF
     ." НОЛЬ"
   ELSE
     0> IF
     ." ПОЛОЖИТЕЛЬНОЕ ЧИСЛО" ELSE
     ." ОТРИЦАТЕЛЬНОЕ ЧИСЛО" THEN
   THEN
;
		\end{minted}
	\end{frame}

	\begin{frame}
		\frametitle{Реализации}
		\begin{columns}
			\begin{column}{0.8\textwidth}
				\begin{itemize}
					\item SwiftForth
					\begin{itemize}
						\item \url{https://www.forth.com/swiftforth/}
					\end{itemize}
					\item Gforth
					\begin{itemize}
						\item \url{http://www.gnu.org/software/gforth/}
					\end{itemize}
					\item Десятки других реализаций
					\begin{itemize}
						\item \url{http://www.forth.org/commercial.html}
					\end{itemize}
					\item Книжка
					\begin{itemize}
						\item Броуди Л. <<Начальный курс программирования на Форте>>
					\end{itemize}
				\end{itemize}
			\end{column}
			\begin{column}{0.2\textwidth}
				\begin{center}
					\includegraphics[width=\textwidth]{forthBookCover.png}
				\end{center}
			\end{column}
		\end{columns}
	\end{frame}

\end{document}