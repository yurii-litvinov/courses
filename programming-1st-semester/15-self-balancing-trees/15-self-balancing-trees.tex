\documentclass{../../slides-style}

\slidetitle{Самобалансирующиеся деревья}{06.12.2023}

\begin{document}

    \begin{frame}[plain]
        \titlepage
    \end{frame}

    \section{Введение}

    \begin{frame}
        \frametitle{Проблема}
        \begin{itemize}
            \item Если в обычное двоичное дерево поиска вставлять элементы в возрастающем (или убывающем) порядке, оно выродится в список
            \item $n <= 2^{h + 1} - 1$, поэтому $h >= log_2(n + 1) - 1 >= floor(log_2(n))$
        \end{itemize}
        \begin{center}
            \includegraphics[width=0.9\textwidth]{tree-problem.png}
        \end{center}
    \end{frame}

    \begin{frame}
        \frametitle{Балансировка}
        \begin{itemize}
            \item Перестраиваем дерево каждый раз после вставки и удаления, чтобы сохранить высоту дерева возможно меньшей
            \item Основная операция --- поворот
            \begin{itemize}
                \item Сохраняет свойства двоичного дерева поиска
                \item Возможно, уменьшает его общую высоту
            \end{itemize}
            \item Конкретных алгоритмов балансировки очень много
        \end{itemize}
        \begin{center}
            \includegraphics[width=0.5\textwidth]{balancing.png}
        \end{center}
    \end{frame}

    \section{АВЛ-дерево}

    \begin{frame}
        \frametitle{АВЛ-дерево}
        \begin{itemize}
            \item 1962г., Г.М. Адельсон-Вельский и Е.М. Ландис
            \item В каждой вершине хранится разность высот левого и правого поддерева
            \item Вставка и удаление гарантируют, что разность высот будет не больше 1
            \item Теоретически лучшая балансировка из популярных деревьев, но относительно большой оверхэд
        \end{itemize}
        \begin{center}
            \includegraphics[width=0.5\textwidth]{avl-tree.png}
        \end{center}
    \end{frame}

    \begin{frame}
        \frametitle{Балансировка}
        \begin{columns}
            \begin{column}{0.5\textwidth}
                Малое левое вращение
                \begin{center}
                    \includegraphics[width=0.8\textwidth]{small-left-rotation.png}
                \end{center}
            \end{column}
            \begin{column}{0.5\textwidth}
                Малое правое вращение
                \begin{center}
                    \includegraphics[width=0.8\textwidth]{small-right-rotation.png}
                \end{center}
            \end{column}
        \end{columns}
        Проводится в случае, если высота $b >$ высота $L + 1$, и высота $C <=$ высоте $R$
        \begin{itemize}
            \item При повороте важно не забыть обновить значения баланса
            \begin{itemize}
                \item И не запутаться в указателях
            \end{itemize}
        \end{itemize}
    \end{frame}

    \begin{frame}
        \frametitle{Балансировка}
        \begin{columns}
            \begin{column}{0.5\textwidth}
                Большое левое вращение
                \begin{center}
                    \includegraphics[width=0.8\textwidth]{big-left-rotation.png}
                \end{center}
            \end{column}
            \begin{column}{0.5\textwidth}
                Большое правое вращение
                \begin{center}
                    \includegraphics[width=0.8\textwidth]{big-right-rotation.png}
                \end{center}
            \end{column}
        \end{columns}
    \end{frame}

    \begin{frame}
        \frametitle{Пример}
        \framesubtitle{Вставка 1 и 2}
        \begin{center}
            \includegraphics[width=0.4\textwidth]{avl12.png}
        \end{center}
    \end{frame}

    \begin{frame}
        \frametitle{Пример}
        \framesubtitle{Вставка 3}
        \begin{center}
            \includegraphics[width=0.7\textwidth]{avl3.png}
        \end{center}
    \end{frame}

    \begin{frame}
        \frametitle{Пример}
        \framesubtitle{Вставка 4}
        \begin{center}
            \includegraphics[width=0.4\textwidth]{avl4.png}
        \end{center}
    \end{frame}

    \begin{frame}
        \frametitle{Пример}
        \framesubtitle{Вставка 5}
        \begin{center}
            \includegraphics[width=0.9\textwidth]{avl5.png}
        \end{center}
    \end{frame}

    \begin{frame}[fragile]
        \frametitle{Псевдокод}
        \framesubtitle{Малое левое вращение}
        \begin{center}
            \includegraphics[width=0.4\textwidth]{small-left-rotation.png}
        \end{center}
        \begin{minted}{c}
Node* rotateLeft(Node *a)
{
    Node *b = a->right;
    Node *c = b->left;
    b->left = a;
    a->right = c;
    return b;
}
        \end{minted}
    \end{frame}

    \begin{frame}[fragile]
        \frametitle{Псевдокод}
        \framesubtitle{Балансировка}
        \begin{minted}{c}
Node* balance(Node *node) {
    if (node->balance == 2) {
        if (node->right->balance >= 0)
            return rotateLeft(node);
        return bigRotateLeft(node);
    }
    if (node->balance == -2) {
        if (node->left->balance <= 0)
            return rotateRight(node);
        return bigRotateRight(node);
    }
    return node;
}
        \end{minted}
    \end{frame}

    \begin{frame}[fragile]
        \frametitle{Псевдокод}
        \framesubtitle{Вставка}
        \begin{minted}{c}
Node* insert(Node *node, int value) {
    if (node == NULL) {
        Node *newNode = calloc(1, sizeof(Node));
        newNode->value = value;
        return newNode;
    }
    if (value < node->value) {
        node->left = insert(node->left, value);
        --node->balance;
    } else {
        node->right = insert(node->right, value);
        ++node->balance;
    }
    return balance(node);
}
        \end{minted}
    \end{frame}

    \begin{frame}
        \frametitle{Замечания по реализации}
        \begin{itemize}
            \item Балансировка выполняется на обратном проходе рекурсии при вставке и удалении
            \item Принцип ``поменяли-вернули-присвоили''
            \item Не делайте указатель на родителя, запутаетесь
            \item Не храните высоту, храните баланс
            \item Большой поворот --- это на самом деле два маленьких, но эффективнее реализовать его отдельно
            \begin{itemize}
                \item Впрочем, сопровождаемость кода обычно важнее эффективности
            \end{itemize}
            \item Не читайте статью про АВЛ-деревья на Хабре
            \begin{itemize}
                \item То есть прочитать можно, но код там очень небрежный
            \end{itemize}
        \end{itemize}
    \end{frame}

    \section{Красно-чёрные деревья}

    \begin{frame}
        \frametitle{Красно-чёрные деревья}
        \begin{itemize}
            \item 1972г., Р. Байер
            \item Хуже сбалансированы, чем АВЛ-деревья, зато \sout{не придуманы в Советском Союзе} требуют константного количества поворотов на каждую операцию (в отличие от $O(log(n))$ для АВЛ-деревьев)
            \begin{itemize}
                \item Поэтому используются практически во всех стандартных библиотеках
            \end{itemize}
        \end{itemize}
        \begin{columns}
            \begin{column}{0.65\textwidth}
                \begin{itemize}
                    \item В каждой вершине хранится цвет (красный или чёрный)
                    \begin{itemize}
                        \item Корень чёрный
                        \item Все листья чёрные
                        \item Оба потомка красного узла --- чёрные
                        \item Всякий путь от данного узла до любого листового узла, являющегося его потомком, содержит одинаковое число чёрных узлов
                        \begin{itemize}
                            \item Высота поддеревьев не может отличаться более, чем вдвое
                        \end{itemize}
                    \end{itemize}
                \end{itemize}
            \end{column}
            \begin{column}{0.35\textwidth}
                \begin{center}
                    \includegraphics[width=0.95\textwidth]{red-black-tree.png}
                \end{center}
            \end{column}
        \end{columns}
    \end{frame}

    \begin{frame}
        \frametitle{Красно-чёрное дерево, добавление}
        \begin{enumerate}
            \item Добавляем в корень --- ок, красим его в чёрный
            \item Добавляем как сына чёрному узлу --- ок, красим в красный
            \item Если родитель и ``дядя'' красные, перекрашиваем их и добавляем наш узел как красный. Дедушка может нарушить ограничения, так что, возможно, его тоже придётся перекрасить (выполнив перекрашивание рекурсивно до корня)
        \end{enumerate}
        \begin{center}
            \includegraphics[width=0.5\textwidth]{addition-to-red-black-tree.png}
        \end{center}
    \end{frame}

    \begin{frame}
        \frametitle{Красно-чёрное дерево, добавление (2)}
        \begin{enumerate}
            \setcounter{enumi}{3}
            \item Родитель красный, дядя чёрный, узел справа от родителя. Выполняем поворот пары ``родитель-сын''.
            \begin{center}
                \includegraphics[width=0.5\textwidth]{addition-to-red-black-tree2.png}
            \end{center}
            Ограничение ``оба потомка красного узла чёрные'' всё ещё нарушается, но об этом позаботится случай 5.
        \end{enumerate}
    \end{frame}

    \begin{frame}
        \frametitle{Красно-чёрные деревья, добавление (3)}
        \begin{enumerate}
            \setcounter{enumi}{4}
            \item Родитель красный, дядя чёрный, узел слева от родителя. Выполняем поворот относительно пары ``родитель-дедушка'', который и восстанавливает балансировку.
            \begin{center}
                \includegraphics[width=0.5\textwidth]{addition-to-red-black-tree3.png}
            \end{center}
            Опять-таки, надо не забыть перекрасить узлы
        \end{enumerate}
    \end{frame}

    \begin{frame}
        \frametitle{Красно-чёрные деревья, удаление}
        Сначала делаем как обычно --- кладём значение самого большого узла в левом поддереве в удаляемый узел и… надо удалить тот узел, откуда мы взяли значение, но не всё так просто.
        \begin{itemize}
            \item Если он красный, то оба его потомка --- чёрные листы. Удаляем красный узел и ставим на его место лист (они не хранят значений, поэтому не важно, какой)
            \item Если он чёрный, а его единственный нелистовой потомок красный, то ставим потомка на его место и перекрашиваем его в чёрный
            \item Если он чёрный и его потомок чёрный, то его оба потомка листы, но если кого-то просто удалить, то число чёрных узлов в поддереве изменится, так что надо перебалансировать дерево
        \end{itemize}
    \end{frame}

    \begin{frame}
        \frametitle{Красно-чёрные деревья, удаление (2)}
        \begin{enumerate}
            \item Самый простой случай, когда удаляемый узел корень: просто удаляем
            \item У удалённого узла был красный брат: делаем поворот по ребру ``отец-брат''
            \begin{center}
                \includegraphics[width=0.5\textwidth]{deletion-from-red-black-tree.png}
            \end{center}
            Сильно лучше не стало, потому что поддеревья всё ещё имеют разную чёрную высоту, но теперь можно применить правила 4, 5 или 6
        \end{enumerate}
    \end{frame}

    \begin{frame}
        \frametitle{Красно-чёрные деревья, удаление (3)}
        \begin{enumerate}
            \setcounter{enumi}{2}
            \item Если родитель чёрный, брат и его сыновья чёрные: перекрашиваем брата в красный. Поскольку из левого поддерева мы только что удалили один чёрный узел, а в правом поддереве один чёрный узел покрасили в красный, баланс восстановлен. Но только в поддереве, потому как оно стало на 1 чёрный узел короче, надо перебалансировать родителей.
            \begin{center}
                \includegraphics[width=0.5\textwidth]{deletion-from-red-black-tree2.png}
            \end{center}
        \end{enumerate}
    \end{frame}

    \begin{frame}
        \frametitle{Красно-чёрные деревья, удаление (4)}
        \begin{enumerate}
            \setcounter{enumi}{3}
            \item Брат и сыновья брата чёрные, но родитель красный --- просто перекрасить брата нельзя. А вот перекрасить одновременно брата и родителя можно, это восстановит баланс (причём, во всём дереве сразу, потому что его чёрная высота не изменится).
            \begin{center}
                \includegraphics[width=0.5\textwidth]{deletion-from-red-black-tree3.png}
            \end{center}
        \end{enumerate}
    \end{frame}

    \begin{frame}
        \frametitle{Красно-чёрные деревья, удаление (5)}
        \begin{enumerate}
            \setcounter{enumi}{4}
            \item Левый сын брата красный, правый --- чёрный. Выполняем поворот относительно брата и левого сына, одновременно перекрашивая брата и левого сына:
            \begin{center}
                \includegraphics[width=0.5\textwidth]{deletion-from-red-black-tree4.png}
            \end{center}
            Глобально ничего не поменялось, но теперь можно применить случай 6.
        \end{enumerate}
    \end{frame}

    \begin{frame}
        \frametitle{Красно-чёрные деревья, удаление (6)}
        \begin{enumerate}
            \setcounter{enumi}{5}
            \item Брат чёрный, его правый сын красный, левый --- чёрный: выполняем поворот вокруг ребра ``родитель-брат'' и перекрашиваем узлы:
            \begin{center}
                \includegraphics[width=0.5\textwidth]{deletion-from-red-black-tree5.png}
            \end{center}
            Баланс восстановлен (в левом поддереве на один чёрный узел больше), при этом можно доказать, что это всё ещё красно-чёрное дерево.
        \end{enumerate}
    \end{frame}

    \section{Splay-деревья}

    \begin{frame}
        \frametitle{Splay-деревья}
        \begin{itemize}
            \item 1985г., Д. Слитор и Р.А. Тарьян
            \item Продвигает узлы, к которым часто происходит обращение, ближе к корню, поэтому может быть быстрее остальных деревьев
            \item Не хранит дополнительных данных в узлах
            \item Не гарантирует сбалансированности
            \item Не дружит с параллельными алгоритмами
            \item Проще в реализации
        \end{itemize}
    \end{frame}

    \begin{frame}
        \frametitle{Splay-деревья, splaying}
        \begin{itemize}
            \item Zig
            \begin{center}
                \includegraphics[width=0.35\textwidth]{zig.png}
            \end{center}
            \item Zig-zag
            \begin{center}
                \includegraphics[width=0.4\textwidth]{zig-zag.png}
            \end{center}
            \item Zig-zig
            \begin{center}
                \includegraphics[width=0.4\textwidth]{zig-zig.png}
            \end{center}
        \end{itemize}
    \end{frame}

    \begin{frame}
        \frametitle{Splay-деревья, операции}
        \begin{itemize}
            \item Поиск:
            \begin{itemize}
                \item Ищем узел как в обычном двоичном дереве поиска
                \item Выполняем серию splaying-ов до тех пор, пока найденный узел не окажется корнем
            \end{itemize}
            \item Вставка:
            \begin{itemize}
                \item Вставляем узел как обычно в двоичное дерево поиска
                \item Выполняем серию splaying-ов до тех пор, пока вставленный узел не окажется корнем
            \end{itemize}
            \item Удаление:
            \begin{itemize}
                \item Удаляем узел как обычно
                \item Тащим родителя удалённого узла в корень дерева
            \end{itemize}
        \end{itemize}
    \end{frame}

    \begin{frame}
        \frametitle{Пример}
        \framesubtitle{Вставка 1}
        \begin{center}
            \includegraphics[width=0.1\textwidth]{splay1.png}
        \end{center}
    \end{frame}

    \begin{frame}
        \frametitle{Пример}
        \framesubtitle{Вставка 2}
        \begin{center}
            \includegraphics[width=0.4\textwidth]{splay2.png}
        \end{center}
    \end{frame}

    \begin{frame}
        \frametitle{Пример}
        \framesubtitle{Вставка 3}
        \begin{center}
            \includegraphics[width=0.6\textwidth]{splay3.png}
        \end{center}
    \end{frame}

    \begin{frame}
        \frametitle{Пример}
        \framesubtitle{Вставка 4}
        \begin{center}
            \includegraphics[width=0.8\textwidth]{splay4.png}
        \end{center}
    \end{frame}

    \begin{frame}
        \frametitle{Пример}
        \framesubtitle{Поиск 2}
        \begin{center}
            \includegraphics[width=0.8\textwidth]{splaySearch.png}
        \end{center}
    \end{frame}

    \begin{frame}[fragile]
        \frametitle{Псевдокод}
        \framesubtitle{Вставка}
        \begin{minted}{c}
Node* add(Node* node, int value)
{
    if (value < node->value) {
        вставляем как обычно
        return splay(node->left);
    }
    else if (value > node->value) {
        вставляем как обычно
        return splay(node->right);
    }
    return splay(node);
}
        \end{minted}
    \end{frame}

    \begin{frame}[fragile]
        \frametitle{Псевдокод}
        \framesubtitle{Процедура перевешивания узла}
        \begin{minted}{c}
typedef enum Direction
{
    left,
    right
} Direction;

void attach(Node* parent, Node* child, Direction direction)
{
    if (direction == left)
        parent->left = child;
    else
        parent->right = child;
    if (child != NULL)
        child->parent = parent;
}
        \end{minted}
    \end{frame}

    \begin{frame}[fragile]
        \frametitle{Псевдокод}
        \framesubtitle{Zig}
        \begin{columns}
            \begin{column}{0.5\textwidth}
                \begin{minted}{c}
void zig(Node* x)
{
    Node* p = x->parent;
    if (x == p->left) {
        Node* b = x->right;
        attach(x, p, right);
        attach(p, b, left);
    }
    else {
        Node* b = x->left;
        attach(x, p, left);
        attach(p, b, right);
    }
    x->parent = NULL;
}
                \end{minted}
            \end{column}
            \begin{column}{0.5\textwidth}
                \begin{center}
                    \includegraphics[width=\textwidth]{zig.png}
                \end{center}
            \end{column}
        \end{columns}
    \end{frame}

    \begin{frame}
        \frametitle{Замечания по реализации}
        \begin{itemize}
            \item Тут уже нужен родитель
            \begin{itemize}
                \item Можно и без него, на прямом проходе, но концептуально сложнее
            \end{itemize}
            \item Родитель добавляет инвариант (какой?), за ним надо следить
            \item Принцип ``локализации нарушения инварианта''
            \item enum-ы для более человекочитаемого кода
            \begin{itemize}
                \item Вообще, читаемость тут критична
            \end{itemize}
            \item Сильно помогает рисовать картинки
            \item Есть известный вариант реализации со split/merge
        \end{itemize}
    \end{frame}

    \section{Декартовы деревья}

    \begin{frame}
        \frametitle{Декартовы деревья}
        \begin{columns}
            \begin{column}{0.65\textwidth}
                \begin{itemize}
                    \item Бинарное дерево поиска и куча одновременно
                    \begin{itemize}
                        \item Храним ключ и ``приоритет''
                        \item Куча по приоритету
                        \item Приоритет выбирается случайно (!) при добавлении ключа
                    \end{itemize}
                    \item Тоже лишь примерно сбалансировано
                    \item Легко пишется
                    \begin{itemize}
                        \item Поэтому любимо олимпиадниками
                    \end{itemize}
                \end{itemize}
            \end{column}
            \begin{column}{0.35\textwidth}
                \begin{center}
                    \includegraphics[width=0.9\textwidth]{treap.png}
                \end{center}
            \end{column}
        \end{columns}
    \end{frame}

    \begin{frame}
        \frametitle{Декартово дерево и плоскость}
        \begin{center}
            \includegraphics[width=0.9\textwidth]{treap-and-plain.png}
            \attribution{(c) \url{https://habrahabr.ru/post/101818/}}
        \end{center}
    \end{frame}

    \begin{frame}
        \frametitle{Merge}
        \begin{columns}
            \begin{column}{0.3\textwidth}
                \begin{itemize}
                    \item Сливает два декартовых поддерева в одно
                    \item Ключи в левом поддереве должны быть меньше ключей в правом
                \end{itemize}
            \end{column}
            \begin{column}{0.7\textwidth}
                \begin{center}
                    \includegraphics[width=0.9\textwidth]{treap-merge.png}
                \end{center}
            \end{column}
        \end{columns}
        \begin{itemize}
            \item Рекурсивно --- сравниваем приоритеты вершин поддеревьев, если второе меньше, сливаем правое поддерево первого и второе, иначе наоборот
        \end{itemize}
    \end{frame}

    \begin{frame}
        \frametitle{Split}
        \begin{columns}
            \begin{column}{0.4\textwidth}
                \begin{itemize}
                    \item Разделяет декартово дерево на два
                    \item Ключи в левом меньше заданного, ключи в правом больше
                \end{itemize}
            \end{column}
            \begin{column}{0.6\textwidth}
                \begin{center}
                    \includegraphics[width=0.9\textwidth]{treap-split.png}
                \end{center}
            \end{column}
        \end{columns}
        \begin{itemize}
            \item Тоже рекурсивно --- если ключ в корне меньше заданного, добавляем его и его левое поддерево в L, а правое поддерево --- результат split от его корня
        \end{itemize}
    \end{frame}

    \begin{frame}
        \frametitle{Добавление}
        \begin{center}
            \includegraphics[width=0.9\textwidth]{treap-add.png}
        \end{center}
        \begin{itemize}
            \item Делаем split по ключу добавляемого элемента
            \item Делаем merge L и M
            \item Делаем merge того, что получилось, и R
        \end{itemize}
    \end{frame}

    \begin{frame}
        \frametitle{Удаление}
        \begin{center}
            \includegraphics[width=0.9\textwidth]{treap-remove.png}
        \end{center}
        \begin{itemize}
            \item Делаем split по ключу удаляемого элемента
            \item Выкидываем удаляемый элемент
            \item Делаем merge остатков дерева
        \end{itemize}
    \end{frame}

\end{document}