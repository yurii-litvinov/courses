\documentclass{../../slides-style}

\slidetitle{Стайлгайд, процесс компиляции}{13.09.2024}

\begin{document}
    
    \begin{frame}[plain]
        \titlepage
    \end{frame}
    
    \begin{frame}
        \frametitle{Стайлгайд}
        \begin{itemize}
            \item Программы пишутся для людей, а не для компьютера
            \item \enquote{Школьник-стайл} именования переменных (a, b, c1)
            \item Отступы!
            \item Пробелы!
            \item Правила именования
            \begin{itemize}
                \item Переменные со строчной, типы с заглавной
                \item camelCase
            \end{itemize}
            \item Компилироваться без предупреждений
            \item Не должно быть копипаста
            \item Один оператор на одной строке, и побольше фигурных скобок
        \end{itemize}
    \end{frame}

    \begin{frame}
        \frametitle{Стайлгайд}
        \begin{itemize}
            \item По возможности сужайте области видимости переменных
            \item Используйте самые узкие типы из подходящих
            \item Переменные --- это плохо, константы --- хорошо
            \item Глобальные переменные --- это очень, очень плохо
            \item goto --- это вообще ужасно
            \item Одна сущность должна играть одну роль в программе
            \begin{itemize}
                \item Одна функция должна делать одно дело
                \item Одна переменная должна означать что-то одно
            \end{itemize}
            \item Бинарные операторы и ключевые слова выделяются пробелами
            \item Стайлгайд выложен на HwProj, прочитайте и следуйте
        \end{itemize}
    \end{frame}

    \begin{frame}
        \frametitle{Процесс компиляции в С}
        \begin{center}
            \includegraphics[width=0.95\textwidth]{compilation.png}
        \end{center}
    \end{frame}

\end{document}

