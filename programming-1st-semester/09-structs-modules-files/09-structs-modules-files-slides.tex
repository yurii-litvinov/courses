\documentclass[xetex,mathserif,serif]{beamer}
\usepackage{polyglossia}
\setdefaultlanguage[babelshorthands=true]{russian}
\usepackage{minted}
\usepackage{tabu}
\usepackage{pgfplots}
\usepackage{textpos}

\useoutertheme{infolines}

\usepackage{fontspec}
\setmainfont{FreeSans}
\newfontfamily{\russianfonttt}{FreeSans}

\usepackage{forest}
\usetikzlibrary{arrows}

\tabulinesep=0.7mm

\title{Структуры, модули, файлы}
\author[Юрий Литвинов]{Юрий Литвинов \newline \textcolor{gray}{\small\texttt{y.litvinov@spbu.ru}}}

\date{28.09.2021}

\begin{document}
    
    \frame{\titlepage}
    
    \begin{frame}[fragile]
        \frametitle{Структуры}
        \begin{itemize}
            \item Способ группировки родственных по смыслу значений
            \item Структура --- это тип
            \begin{itemize}
                \item В памяти представляется как поля, лежащие друг за другом, возможно, с ``дырками'' (padding)
                \item Объявляется вне функции
            \end{itemize}
            \item Объявление структуры:
            \begin{minted}{c}
struct Point {
    int x;
    int y;
};
            \end{minted}
            \item Использование:
            \begin{minted}{c}
void main() {
    struct Point p;
    p.x = 10;
}
            \end{minted}
        \end{itemize}
    \end{frame}

    \begin{frame}[fragile]
        \frametitle{Структуры (2)}
        \begin{itemize}
            \item Или, чтобы \mintinline{c}{struct} каждый раз не писать:
            \begin{minted}{cpp}
typedef struct {
    int x;
    int y;
} Point;
            \end{minted}
            \begin{itemize}
                \item \mintinline{c}{typedef} --- объявление синонима типа
            \end{itemize}
            \item Использование:
            \begin{minted}{c}
void main() {
    Point p = {10, 20};
    printf("(%d, %d)", p.x, p.y);
}
            \end{minted}
            \item Продвинутая инициализация:
            \begin{minted}{c}
Point p = {.x = 10, .y = 20};
            \end{minted}
        \end{itemize}
    \end{frame}

    \begin{frame}[fragile]
        \frametitle{Указатели и структуры}
        \begin{itemize}
            \item Структуры и указатели настолько часто используются вместе, что есть оператор -> (разыменовать указатель на структуру и обратиться к её полю)
            \begin{itemize}
                \item 
                \begin{footnotesize}
                    \begin{minted}{c}
int main() {
    Point* p = malloc(sizeof(Point));
    if (p == NULL) {
        return -1;
    }
    p->x = 10;
    p->y = 20;
    printf("(%d, %d)", p->x, p->y);
    free(p);

    return 0;
}                    \end{minted}
                \end{footnotesize}
                \item То же самое, что \mintinline{c}|(*p).x| и \mintinline{c}|(*p).y|
            \end{itemize}
        \end{itemize}
    \end{frame}

    \begin{frame}[fragile]
        \frametitle{Операция взятия адреса}
        \begin{itemize}
            \item 
                \begin{minted}{c}
int main() {
    Point p1 = { 10, 20 };
    Point* p = &p1;
    int *test = &(p1.x);
    printf("(%d, %d)\n", p->x, p->y);
    *test = 30;
    printf("(%d, %d)\n", p->x, p->y);
    printf("(%d, %d)\n", p1.x, p1.y);
}
                \end{minted}
        \end{itemize}
    \end{frame}

    \begin{frame}[fragile]
        \frametitle{Структуры и строки}
        \begin{footnotesize}
            \begin{minted}{c}
typedef struct {
    char *name;
    char phone[30];
} PhoneBookEntry;

int main() {
    PhoneBookEntry entry;
    const char* name = "Ivan Ivanov";
    entry.name = malloc(sizeof(char) * (strlen(name) + 1));
    if (entry.name == NULL) {
        return -1;
    }
    strcpy(entry.name, name);
    strcpy(entry.phone, "+7 (911) 123-45-67");
    printf("%s - %s", entry.name, entry.phone);
    free(entry.name);
    return 0;
}
            \end{minted}
        \end{footnotesize}
    \end{frame}

    \begin{frame}
        \frametitle{Полезные операции со строками}
        Модуль string.h/cstring:
        \begin{itemize}
            \item strcpy --- скопировать строку в буфер
            \item strcmp --- сравнить две строки
            \item strcat --- склеить две строки (в буфере должно быть достаточно места!)
            \item strlen --- узнать длину строки
            \item strstr --- найти подстроку в строке
            \item Строки нельзя сравнивать ==
            \item Строки нельзя присваивать =
        \end{itemize}
    \end{frame}

    \begin{frame}[fragile]
        \frametitle{Чтение строки прямо в структуру}
        \begin{minted}{c}
int main() {
    PhoneBookEntry entry;
    entry.name = malloc(sizeof(char) * 30);
    if (entry.name == NULL) {
        return -1;
    }
    scanf("%s", entry.name);
    scanf("%[^\n]", entry.phone);

    printf("%s - %s", entry.name, entry.phone);

    free(entry.name);
    return 0;
}
        \end{minted}
    \end{frame}

    \begin{frame}[fragile]
        \frametitle{Структуры могут указывать сами на себя}
        \begin{footnotesize}
            \begin{minted}{c}
typedef struct ListElement {
    int value;
    struct ListElement *next;
} ListElement;

int main() {
    ListElement* element1 = malloc(sizeof(ListElement));
    element1->value = 1;
    ListElement* element2 = malloc(sizeof(ListElement));
    element2->value = 2;
    element2->next = NULL;
    element1->next = element2;

    printf("%i - %i", element1->value, element1->next->value);

    free(element1);
    free(element2);

    return 0;
}
            \end{minted}
        \end{footnotesize}
    \end{frame}

    \begin{frame}
        \frametitle{Файлы}
        \begin{itemize}
            \item Последовательность байтов на диске
            \begin{itemize}
                \item Бывают ``сырые'' и ``текстовые''
                \begin{itemize}
                    \item Самому файлу всё равно, это лишь способы интерпретации его содержимого
                \end{itemize}
                \item Режимы доступа: r, w, a, r+, w+, a+
                \item Курсор
                \item EOF
            \end{itemize}
            \item Функции для работы с файлами:
            \begin{itemize}
                \item fopen, fclose, fprintf, fscanf, fseek, ftell, fgets
            \end{itemize}
            \item Файлы надо не забывать закрывать
        \end{itemize}
    \end{frame}

    \begin{frame}[fragile]
        \frametitle{Пример, как писать в файл}
        \begin{minted}{cpp}
int main() {
    FILE* out = fopen("ololo.txt", "w");
    if (out == NULL) {
        return -1;
    }
    fwrite("Ololo\n", sizeof(char), 6, out);
    fprintf(out, "%s", "Ololo");
    fclose(out);
    return 0;
}
        \end{minted}
        \begin{itemize}
            \item Файлы нелишне добавлять в проект как <<Файлы ресурсов>>
            \item stdin/stdout --- это тоже файлы
        \end{itemize}
    \end{frame}

    \begin{frame}[fragile]
        \frametitle{Пример, как читать из файла}
        \begin{scriptsize}
            \begin{minted}{cpp}
#include <stdio.h>

int main() {
    FILE *file = fopen("test.txt", "r");
    if (file == NULL) {
        printf("file not found!");
        return 1;
    }
    char *data[100] = {0};
    int linesRead = 0;
    while (!feof(file)) {
        char *buffer = malloc(sizeof(char) * 100);
        const int readBytes = fscanf(file, "%s", buffer);
        if (readBytes < 0) {
            break;
        }
        data[linesRead] = buffer;
        ++linesRead;
    }
    fclose(file);
    ...
}

            \end{minted}
        \end{scriptsize}
    \end{frame}

    \begin{frame}
        \frametitle{Тонкости}
        \begin{itemize}
            \item Чтение строки целиком: \mintinline{cpp}|fscanf(file, "|\%\mintinline{cpp}|[^\n]", buffer);|
            \item Или: \mintinline{cpp}|fgets(buffer, sizeof(buffer), file);|
            \item Working directory
            \begin{itemize}
                \item Свойства проекта -> Отладка -> Рабочая папка
                \item По умолчанию \$(ProjectDir), папка с .vcxproj
            \end{itemize}
        \end{itemize}
    \end{frame}

    \begin{frame}[fragile]
        \frametitle{Модули}
        \begin{itemize}
            \item Способ группировки кода в логически обособленные группы
            \item В C это реализуется с помощью заголовочных файлов и файлов с реализацией
            \begin{itemize}
                \item .h и .c
            \end{itemize}
            \item В отдельный модуль выносятся объявления типов данных и функции, которые делают одно дело
            \begin{itemize}
                \item Например, разные функции сортировки
                \item Или всё для работы с матрицами
            \end{itemize}
            \item В интерфейсную часть модуля выносится только то, что может использовать другой код
            \begin{itemize}
                \item Меньше знаешь --- крепче спишь
            \end{itemize}
            \item Функции, используемые только для реализации, пишутся только в .c-файле
            \begin{itemize}
                \item Например, функция разделения массива для быстрой сортировки или swap
            \end{itemize}
        \end{itemize}
    \end{frame}

    \begin{frame}[fragile]
        \frametitle{Модули}
        \begin{scriptsize}
            Заголовочный файл:
            \begin{minted}{c}
#pragma once

// Комментарий к функции 1
int function1(int x, int y);

// Комментарий к функции 2
void function2();
            \end{minted}
            .c-файл:
            \begin{minted}{c}
#include <имя заголовочного файла.h>

#include <все остальные библиотеки>

int function1(int x, int y)
{
    ...
}

void function2()
{
    ...
}
            \end{minted}
        \end{scriptsize}
    \end{frame}

    \begin{frame}
        \frametitle{Тонкости}
        \begin{itemize}
            \item Реализации функций в .h-файле писать нельзя
            \begin{itemize}
                \item Иначе будет беда, если один .h-ник подключат в два .c-шника
            \end{itemize}
            \item Комментарии обязательны
            \item \#pragma once обязательна
            \item Подключать <<свой>> заголовочный файл в .c обязательно
            \item Файлы .h/.c всегда ходят парами, кроме файла с main
        \end{itemize}
    \end{frame}

\end{document}

