\documentclass{../../slides-style}

\slidetitle{О разработке программных продуктов}{01.11.2023}

\begin{document}
    
    \begin{frame}[plain]
        \titlepage
    \end{frame}
    
    \begin{frame}
        \frametitle{Программа и программный продукт}
        \begin{center}
            \includegraphics[width=0.5\textwidth]{mythical-man-month.png}
        \end{center}
        \begin{center}
            Ф. Брукс, ``Мифический человеко-месяц''
        \end{center}
    \end{frame}

    \begin{frame}
        \frametitle{Жизненный цикл ПО}
        \begin{itemize}
            \item Последовательность стадий
            \begin{itemize}
                \item Состав и последовательность работ
                \item Получаемые результаты
                \item Методы и средства
                \item Роли и ответственности
                \item ...
            \end{itemize}
        \end{itemize}
    \end{frame}

    \begin{frame}
        \frametitle{Фазы жизненного цикла ПО}
        \begin{itemize}
            \item возникновение и исследование идеи
            \item анализ и сбор требований
            \begin{itemize}
                \item пилотный проект
            \end{itemize}
            \item планирование и проектирование
            \item разработка
            \item отладка и тестирование
            \item сдача
            \item сопровождение
            \item вывод из эксплуатации
        \end{itemize}
    \end{frame}

    \begin{frame}
        \frametitle{Водопадная модель}
        \begin{center}
            \includegraphics[width=0.7\textwidth]{waterfall-model.png}
        \end{center}
    \end{frame}

    \begin{frame}
        \frametitle{Спиральная модель}
        \begin{center}
            \includegraphics[width=0.7\textwidth]{spiral-model.png}
        \end{center}
    \end{frame}

    \begin{frame}
        \frametitle{Методологии разработки}
        \begin{itemize}
            \item Модели описывают последовательность фаз и что надо делать на этих фазах, методологии --- как делать
            \item Гибкие методологии
            \begin{itemize}
                \item Agile Manifesto
                \begin{itemize}
                    \item Люди и взаимодействие важнее процессов и инструментов
                    \item Работающий продукт важнее исчерпывающей документации
                    \item Сотрудничество с заказчиком важнее согласования условий контракта
                    \item Готовность к изменениям важнее следования первоначальному плану
                \end{itemize}
                \item SCRUM
                \item XP
            \end{itemize}
        \end{itemize}
    \end{frame}

    \begin{frame}
        \frametitle{Парадигмы программирования}
        \begin{itemize}
            \item Методологии говорят, как делать, но до уровня ``а сейчас писать код'', стиль написания кода --- выбранная парадигма
            \item Структурное программирование
            \item Объектно-ориентированное программирование
            \item Функциональное программирование
            \item Много других
        \end{itemize}
    \end{frame}

    \begin{frame}
        \frametitle{Структурное программирование}
        \begin{itemize}
            \item Дейкстра, 70-е годы XX века
            \item Базовые конструкции
            \begin{itemize}
                \item Последовательное исполнение
                \item Ветвление
                \item Цикл
            \end{itemize}
            \item Произвольная вложенность
            \item Никаких других средств управления последовательностью выполнения
            \item Выделение подпрограмм
            \item Нисходящее проектирование
        \end{itemize}
    \end{frame}

    \begin{frame}
        \frametitle{Нисходящее проектирование}
        \begin{itemize}
            \item Декомпозиция задачи
            \begin{itemize}
                \item Подпрограммы
                \item Модули
            \end{itemize}
            \item Высокоуровневая реализация с помощью ``заглушек''
            \item Постепенная реализация модулей и подпрограмм
            \item Строгое задание входов и выходов
        \end{itemize}
    \end{frame}

    \begin{frame}
        \frametitle{Модули}
        \begin{itemize}
            \item Логически обособленный кусок функциональности
            \begin{itemize}
                \item Интерфейс
                \item Реализация
            \end{itemize}
            \item Особенности
            \begin{itemize}
                \item Четкая декомпозиция
                \item Минимизация
                \item Один модуль --- одна функциональность
                \item Отсутствие побочных эффектов
                \item Независимость от реализации других модулей
                \item Принцип сокрытия данных
            \end{itemize}
        \end{itemize}
    \end{frame}

    \begin{frame}
        \frametitle{Практики}
        \begin{itemize}
            \item Документирование
            \item Моделирование
            \item Тестирование (модульное и регрессионное)
            \item Версионный контроль
            \item Непрерывная интеграция
            \item Code review
        \end{itemize}
    \end{frame}

    \begin{frame}
        \frametitle{Версионный контроль}
        \begin{center}
            \includegraphics[width=0.42\textwidth]{git-flow.png}
            \attribution{https://nvie.com/posts/a-successful-git-branching-model/}
        \end{center}
    \end{frame}

\end{document}