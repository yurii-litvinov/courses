\documentclass{../../slides-style}

\slidetitle{Абстрактные типы данных}{23.10.2024}

\begin{document}
    
    \begin{frame}[plain]
        \titlepage
    \end{frame}
    
    \begin{frame}
        \frametitle{АТД}
        \begin{itemize}
            \item АТД --- некоторая математическая модель и набор операций, определённый в рамках этой модели
            \begin{itemize}
                \item Обобщение понятия \enquote{тип}
            \end{itemize}
            \item Состоит из типа данных и операций, выполняющих над ним преобразования
            \begin{itemize}
                \item Внутреннее устройство типа данных невидимо для остальной программы (принцип сокрытия деталей реализации)
                \item Работа с АТД --- только с помощью связанных с ним функций
                \item Тип данных и операции для работы с ним лежат в одном модуле, так, чтобы все изменения в АТД были локализованы и не затрагивали остальную программу (принцип инкапсуляции)
            \end{itemize}
            \item Дальнейшее обобщение АТД --- классы
        \end{itemize}
    \end{frame}

    \begin{frame}
        \frametitle{Пример --- стек}
        \begin{itemize}
            \item stack.h / stack.c, при этом структура данных описана только в .c-файле, в .h-файле только её предварительное объявление
            \begin{itemize}
                \item Так компилятор может гарантировать сокрытие деталей реализации
                \begin{itemize}
                    \item Всё, что не проверяется автоматически, можно считать не работающим!
                \end{itemize}
                \item Все функции принимают только указатель на структуру, для значения нужно знать размер
            \end{itemize}
            \item Функции:
            \begin{itemize}
                \item createStack()
                \item deleteStack()
                \item push()
                \item pop()
                \item isEmpty()
            \end{itemize}
            \item Внешнему миру вообще всё равно, как стек устроен внутри
            \begin{itemize}
                \item Может быть на массиве
            \end{itemize}
        \end{itemize}
    \end{frame}

    \begin{frame}[fragile]
        \frametitle{Ещё пример --- список}
        \begin{itemize}
            \item Требуется целых два типа --- сам список и позиция внутри списка
            \begin{itemize}
                \item Что-то вроде индекса элемента массива, но может быть устроена хитрее
                \item Позиция должна обеспечивать быструю работу с элементом, на который она указывает
                \item Внешнему миру всё равно, как устроен список и что такое позиция
                \begin{itemize}
                    \item Может быть, список на массивах, а позиция --- число, или список на указателях, а позиция --- указатель на элемент списка (или даже на предыдущий элемент)
                \end{itemize}
            \end{itemize}
            \item Список может хранить разные типы элементов
            \begin{itemize}
                \item typedef --- \enquote{шаблоны для бедных}
                \begin{itemize}
                    \item 
                    \begin{footnotesize}
                        \begin{minted}{cpp}
typedef int Value;
struct ListElement { 
    Value value;
    ListElement *next;
}
                        \end{minted}
                    \end{footnotesize}
                    \item typedef же может использоваться для описания типа позиции
                \end{itemize}
            \end{itemize}
        \end{itemize}
    \end{frame}

    \begin{frame}
        \frametitle{Инвариант}
        \begin{itemize}
            \item Некоторое логическое условие, верное всё время жизни АТД
            \begin{itemize}
                \item Не совсем, внутри функции АТД инвариант может нарушаться
            \end{itemize}
            \item АТД отвечает за поддержание своего инварианта
            \begin{itemize}
                \item Поскольку работа с АТД только через его функции, у внешнего мира нет способа его испортить
            \end{itemize}
            \item Пример --- размер списка
            \begin{itemize}
                \item Можно считать за $O(n)$ каждый раз
                \item Можно хранить как элемент структуры, тогда должен соблюдаться инвариант
            \end{itemize}
            \item Ещё пример --- head и tail у очереди
        \end{itemize}
    \end{frame}

    \begin{frame}
        \frametitle{Пример применения АТД --- сортировка слиянием}
        Если в списке больше одного элемента, делим его на два, вызываем mergesort, получаем два отсортированных списка, которые сливаем в один отсортированный
        \begin{columns}
            \begin{column}{0.6\textwidth}
                \begin{itemize}
                    \item $O(n * log(n))$ в среднем и худшем случае
                    \item Устойчива
                    \item Внешняя (подходит для больших данных, не помещающихся в память)
                    \item \url{https://www.hackerearth.com/practice/algorithms/sorting/merge-sort/visualize/}
                    \item Ей не надо знать внутреннего устройства списка
                \end{itemize}
            \end{column}
            \begin{column}{0.4\textwidth}
                \begin{center}
                    \includegraphics[width=0.95\textwidth]{mergesort.png}
                \end{center}
            \end{column}
        \end{columns}
    \end{frame}

\end{document}