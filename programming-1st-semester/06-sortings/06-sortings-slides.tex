\documentclass{../../slides-style}

\slidetitle{Сортировки}{21.09.2022}

\begin{document}
    
    \begin{frame}[plain]
        \titlepage
    \end{frame}

    \begin{frame}
        \frametitle{Комментарии по домашке}
        \framesubtitle{Коды ошибок}
        \begin{itemize}
            \item Не используйте функцию exit()
            \item Не печатайте на консоль нигде, кроме как в main()
            \begin{itemize}
                \item Или функции, созданной \emph{только} для печати на консоль, вызываемой из main()
            \end{itemize}
            \item Если функция может не выполниться, используйте коды возврата
        \end{itemize}
    \end{frame}

    \begin{frame}[fragile]
        \frametitle{Пример}
        \begin{scriptsize}
            \begin{minted}{c}
int fibonacci(int n, int *result)
{
    if (n <= 0)
        return 1; 
    if (n <= 2) { 
        *result = 1; 
        return 0;
    }
    int previous = 0;
    fibonacci(n - 1, &previous);
    int prePrevious = 0;
    fibonacci(n - 2, &prePrevious);
    *result = previous + prePrevious;
    return 0;
}
...
int result = 0;
const int errorCode = fibonacci(x, &result);
if (errorCode != 0) {
    printf("Всё очень плохо")
} else {
    printf("%d-ое число Фибоначчи равно %d", x, result);
}
            \end{minted}
        \end{scriptsize}
    \end{frame}

    \begin{frame}[fragile]
        \frametitle{Или так}
        \begin{scriptsize}
            \begin{minted}{c}
int fibonacci(int n, int *errorCode)
{
    if (n <= 0) {
        *errorCode = 1;
        return 0;
    } 
    *errorCode = 0;
    if (n <= 2) { 
        return 1;
    }
    return fibonacci(n - 1, errorCode) + fibonacci(n - 2, errorCode);
}
...
int errorCode = 0;
const int result = fibonacci(x, &errorCode);
if (errorCode != 0) {
    printf("Всё очень плохо")
} else {
    printf("%d-ое число Фибоначчи равно %d", x, result);
}
            \end{minted}
        \end{scriptsize}
    \end{frame}

    \begin{frame}[fragile]
        \frametitle{Ещё комментарии}
        Не пишите так:
        \begin{minted}{c}
if (x == 10) 
{
    return true;
}
else
{
    return false;
}
        \end{minted}
        Пишите так:
        \begin{minted}{c}
return x == 10;
        \end{minted}
    \end{frame}

    \begin{frame}[fragile]
        \frametitle{И ещё комментарии}
        \begin{itemize}
            \item Целочисленное и вещественное деление
            \item Предупреждения компилятора
            \item Замер времени в секундах
            \item Выделение памяти
        \end{itemize}
        \begin{minted}{c}
int *array = calloc(size, sizeof(int));
if (array == NULL) 
{
    printf("Всё очень плохо :(")
    return -1;
}
...
free(array);
        \end{minted}
    \end{frame}
    
    \begin{frame}
        \frametitle{Свойства сортировок}
        \begin{itemize}
            \item Работают над любыми контейнерами данных
            \item Есть понятие ``ключ''
            \item Устойчивость --- сохраняется ли взаимное расположение элементов с одинаковым ключом
            \item Естественность --- учёт степени отсортированности исходных данных
            \item Внутренняя сортировка --- работает над данными, целиком помещающимися в память
            \item Внешняя сортировка работает над данными на устройствах с последовательным доступом, которые медленнее, чем память
        \end{itemize}
    \end{frame}

    \begin{frame}
        \frametitle{Сортировка вставкой (insertion sort)}
        \begin{columns}
            \begin{column}{0.4\textwidth}
                \includegraphics[width=0.8\textwidth]{insertion-sort.png}
            \end{column}
            \begin{column}{0.6\textwidth}
                \begin{itemize}
                    \item $O(n^2)$
                    \item Устойчива
                    \item Естественная ($O(n)$ на отсортированном массиве)
                    \item Данные могут приходить постепенно
                    \item Позволяет выбрать наибольшие (или наименьшие) $k$ чисел из входного потока
                \end{itemize}
            \end{column}
        \end{columns}
    \end{frame}

    \begin{frame}
        \frametitle{Сортировка Шелла (Shell sort)}
        \begin{columns}
            \begin{column}{0.4\textwidth}
                \includegraphics[width=0.8\textwidth]{shell-sort.png}
            \end{column}
            \begin{column}{0.6\textwidth}
                \begin{itemize}
                    \item Сортировка вставкой подпоследовательностей в массиве с постепенно убывающим шагом
                    \item Элементы ``быстрее'' встают на свои места
                    \begin{itemize}
                        \item Сортировка вставкой на каждом шаге уменьшает количество инверсий максимум на 1
                    \end{itemize}
                    \item $O(n * log(n)^2)$ при правильном выборе h
                    \item Неустойчива
                    \item Легко пишется и довольно быстра
                    \begin{itemize}
                        \item Не вырождается до квадратичной
                    \end{itemize}
                \end{itemize}
            \end{column}
        \end{columns}
    \end{frame}

    \begin{frame}
        \frametitle{Сортировка выбором (Selection sort)}
        \begin{columns}
            \begin{column}{0.4\textwidth}
                \includegraphics[width=0.8\textwidth]{selection-sort.png}
            \end{column}
            \begin{column}{0.6\textwidth}
                \begin{itemize}
                    \item $O(n^2)$
                    \item Обычно неустойчива ([$2_a$, $2_b$, $1_a$] -> [$1_a$, $2_b$, $2_a$])
                    \item Отсортированность массива ничего не даёт
                    \item Меньше всего операций обмена (меньше операций записи, что иногда позитивно)
                \end{itemize}
            \end{column}
        \end{columns}
    \end{frame}

    \begin{frame}
        \frametitle{Быстрая сортировка (qsort)}
        \begin{columns}
            \begin{column}{0.5\textwidth}
                \includegraphics[width=0.8\textwidth]{qsort.png}
            \end{column}
            \begin{column}{0.5\textwidth}
                \begin{itemize}
                    \item $O(n * log(n))$, вырождается до $O(n^2)$
                    \item Неустойчива
                    \item Требует $O(n * log(n))$ дополнительной памяти
                    \item Самый быстрый на практике алгоритм сортировки, используется в стандартных библиотеках
                    \item Легко пишется (но тяжело отлаживается)
                \end{itemize}
            \end{column}
        \end{columns}
    \end{frame}

    \begin{frame}[fragile]
        \frametitle{Псевдокод}
        \begin{columns}
            \begin{column}{0.5\textwidth}
                \begin{footnotesize}
                    \begin{minted}{pascal}
algorithm quicksort(A, lo, hi) is
    if lo < hi then
        p := partition(A, lo, hi)
        quicksort(A, lo, p – 1)
        quicksort(A, p + 1, hi)
                    \end{minted}
                \end{footnotesize}
            \end{column}
            \begin{column}{0.5\textwidth}
                \begin{footnotesize}
                    \begin{minted}{pascal}
algorithm partition(A, lo, hi) is
    pivot := A[hi]
    i := lo  
    for j := lo to hi – 1 do
        if A[j] ≤ pivot then
            swap A[i] with A[j]
            i := i + 1
    swap A[i] with A[hi]
    return i
                    \end{minted}
                \end{footnotesize}
            \end{column}
        \end{columns}
        \vspace{1cm}
        Нерекурсивная реализация --- через стек, в котором хранятся границы сортируемых кусков массива
    \end{frame}

    \begin{frame}
        \frametitle{Сортировка кучей (пирамидальная, heapsort)}
        \begin{columns}
            \begin{column}{0.6\textwidth}
                \includegraphics[width=0.95\textwidth]{heapsort.png}
            \end{column}
            \begin{column}{0.4\textwidth}
                \begin{itemize}
                    \item $O(n * log(n))$, не вырождается
                    \item Не требует дополнительной памяти
                    \item Неустойчива
                    \item Требует произвольного доступа к памяти
                    \item Сложна в реализации
                \end{itemize}
            \end{column}
        \end{columns}
    \end{frame}

    \begin{frame}
        \frametitle{Двоичный поиск}
        \begin{columns}
            \begin{column}{0.5\textwidth}
                \includegraphics[width=0.8\textwidth]{binary-search.png}
            \end{column}
            \begin{column}{0.5\textwidth}
                \begin{itemize}
                    \item Находит элемент в массиве за $O(log(n))$
                    \item Легко напутать с индексами и уйти в бесконечный цикл
                \end{itemize}
            \end{column}
        \end{columns}
    \end{frame}

\end{document}

