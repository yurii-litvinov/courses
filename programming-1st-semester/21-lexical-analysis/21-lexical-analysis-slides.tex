\documentclass[xetex,mathserif,serif]{beamer}
\usepackage{polyglossia}
\setdefaultlanguage[babelshorthands=true]{russian}
\usepackage{minted}
\usepackage{tabu}
\usepackage{pgfplots}
\usepackage{textpos}
\usepackage{subcaption}
\usepackage{graphicx}
\usepackage[normalem]{ulem}
\usepackage{algorithm2e}
\usepackage{algorithmic}
\usepackage{float}
\usepackage{amsmath}
\useoutertheme{infolines}

\usepackage{fontspec}
\setmainfont{FreeSans}
\newfontfamily{\russianfonttt}{FreeSans}

\usepackage{forest}
\usetikzlibrary{arrows}

\definecolor{links}{HTML}{2A1B81}
\hypersetup{colorlinks,linkcolor=,urlcolor=links}

\newcommand{\attribution}[1] {
    \vspace{-5mm}\begin{flushright}\begin{scriptsize}\textcolor{gray}{\textcopyright\, #1}\end{scriptsize}\end{flushright}
}

\tabulinesep=0.7mm

\title{Автоматы и лексический анализ}
\author[Юрий Литвинов]{Юрий Литвинов \newline \textcolor{gray}{\small\texttt{y.litvinov@spbu.ru}}}

\date{07.12.2021}

\begin{document}
    
    \frame{\titlepage}

    \section{Работа компилятора}

    \begin{frame}
        \frametitle{Напоминание о процессе компиляции кода на C}
        \begin{center}
            \includegraphics[height=0.4\textheight]{compilation.png}
        \end{center}
    \end{frame}

    \begin{frame}
        \frametitle{Фазы компилятора}
        \begin{center}
            \includegraphics[height=0.8\textheight]{compilerPhases.png}
        \end{center}
    \end{frame}

    \begin{frame}
        \frametitle{Лексический анализ}
        \begin{itemize}
            \item Преобразование потока символов в поток токенов
            \item Например, \mintinline{text}|position := initial + rate * 60|
            \begin{itemize}
                \item Идентификатор position
                \item Символ присвоения
                \item Идентификатор initial
                \item Знак сложения
                \item Идентификатор rate
                \item Знак умножения
                \item Число 60
            \end{itemize}
            \item Токен представляется в виде структуры из типа токена и его значения
        \end{itemize}
    \end{frame}

    \section{Формальные языки}

    \begin{frame}
        \frametitle{Формальные языки, определения}
        \begin{itemize}
            \item Алфавит --- произвольное множество. Элементы множества называются \textit{символами} алфавита
            \item Строка (она же \textit{цепочка}) --- конечная последовательность символов из алфавита
            \item Длина строки $s$ (обозначается как $\lvert{s}\rvert$) --- количество символов в строке
            \item Пустая строка ($\epsilon$) --- строка, которая не содержит символов
            \item Конкатенация строк --- две строки, записанные друг за другом
            \item Язык --- множество строк. Например, пустой язык (обозначается $\{\}$), множество всех корректных 
                программ на С, множество всех грамматически корректных предложений русского языка
        \end{itemize}
    \end{frame}

    \begin{frame}
        \frametitle{Операции над языками}
        \framesubtitle{Позволяют собрать из простых более сложные}
        \begin{itemize}
            \item Объединение $L$ и $M$ ($L \cup M$) --- $L \cup M = \{s \mid s \in L \vee s \in M\}$
            \item Конкатенация $L$ и $M$ ($LM$) --- $LM = \{st \mid s \in L \wedge t \in M\}$
            \item Замыкание Клини или \textit{итерация} (она же \textit{звёздочка Клини}) ($L^*$) --- 
                $L^* = \bigcup\limits_{i=0}^{\infty}L^i$, где $L^i$ --- это $LL \ldots L$ $i$ раз
            \begin{itemize}
                \item В том числе, $i = 0$, то есть $\epsilon$ всегда принадлежит $L^*$
                \begin{itemize}
                    \item То есть, пустая строка всегда часть замыкания
                \end{itemize}
            \end{itemize}
            \item Позитивное замыкание $L$ ($L^+$) --- $LL^*$
            \begin{itemize}
                \item Или $L^* = \bigcup\limits_{i=1}^{\infty}L^i$, то есть замыкание без пустой строки
            \end{itemize}
        \end{itemize}
    \end{frame}

    \begin{frame}
        \frametitle{Регулярные выражения}
        \framesubtitle{Формальный язык для записи языков}
        \begin{itemize}
            \item Регулярное выражение $\epsilon$ задаёт язык $\{\epsilon\}$
            \item Регулярное выражение $a$ задаёт язык $\{a\}$ (язык из одного символа)
            \item Пусть $r$ и $s$ --- регулярные выражения, задающие языки $L(r)$ и $L(s)$. Тогда
            \begin{itemize}
                \item Регулярное выражение $(r) \mid (s)$ задаёт объединение, $L(r) \cup L(s)$
                \item Регулярное выражение $(r)(s)$ задаёт конкатенацию, $L(r)L(s)$
            \end{itemize}
            \item Регулярное выражение $(r)^*$ задаёт замыкание Клини, $(L(r))^*$
            \item Регулярное выражение $(r)$ задаёт $L(r)$ (то есть можно ставить скобки)
            \item Регулярное выражение $r?$ задаёт $L(r) \cup \epsilon$
            \item Регулярное выражение $r+$ задаёт $L(r)^+$
            \item Регулярное выражение $[a..z]$ задаёт $L(a) \cup L(b) \cup \ldots \cup L(z)$
        \end{itemize}
    \end{frame}

    \begin{frame}
        \frametitle{Регулярные определения}
        \framesubtitle{Именуем регулярные выражения для удобства записи}
        \begin{itemize}
            \item $letter \rightarrow A | B | ... | Z | a | b | ... | z$
            \item $digit \rightarrow 0 | 1 | ... | 9$
            \item $id \rightarrow letter (letter | digit)*$
            \item $num \rightarrow digit+ (. digit+)? (E(+ | -)? digit+)?$
            \item $relop \rightarrow < | <= | = | <> | > | >=$
        \end{itemize}
    \end{frame}

    \begin{frame}
        \frametitle{Диаграммы переходов}
        $relop \rightarrow < | <= | = | <> | > | >=$
        \begin{center}
            \includegraphics[height=0.5\textheight]{transitionDiagram.png}
        \end{center}
        \begin{itemize}
            \item Двойной кружок --- \textit{принимающее состояние}
            \item Звёздочка --- вернуть последний символ во входной поток
        \end{itemize}
    \end{frame}

    \begin{frame}
        \frametitle{Как это закодить}
        $relop \rightarrow < | <= | = | <> | > | >=$
        \begin{columns}
            \begin{column}{0.5\textwidth}
                \begin{center}
                    \includegraphics[width=0.9\textwidth]{transitionDiagram.png}
                \end{center}
            \end{column}
            \begin{column}{0.5\textwidth}
                \begin{center}
                    \includegraphics[width=0.9\textwidth]{lexerCode.png}
                \end{center}
            \end{column}
        \end{columns}
    \end{frame}

    \section{Автоматы}

    \begin{frame}
        \frametitle{Конечные автоматы}
        \begin{itemize}
            \item Конечный автомат --- $(S, \Sigma, move, s_0, F)$
            \begin{itemize}
                \item $S$ --- множество состояний
                \item $\Sigma$ --- входной алфавит
                \item move --- функция переходов
                \item $s_0$ --- начальное состояние
                \item $F$ --- множество допускающих состояний
            \end{itemize}
            \item Неформально автомат имеет состояние, в зависимости от которого может по-разному реагировать на входные символы (или события)
            \item Применяются не только в лексическом анализе, но и в сетевых протоколах, пользовательских интерфейсах и т.д. и т.п.
            \item Автомат принимает язык, если для любой строки языка после выполнения всех переходов он оказывается в допускающем состоянии
        \end{itemize}
    \end{frame}

    \begin{frame}
        \frametitle{ДКА и НКА}
        \begin{itemize}
            \item Детерминированный конечный автомат (ДКА) --- $move: S \times \Sigma \rightarrow S$
            \item Недетерминированный конечный автомат (НКА) --- $move: 2^S \times \Sigma \cup \epsilon \rightarrow 2^S$
            \begin{itemize}
                \item НКА может находиться в нескольких состояниях одновременно и делать переход одновременно в несколько разных состояний
                \item Или, другая точка зрения --- он не знает, в каком именно состоянии сейчас находится
                \item И есть эпсилон-переходы --- спонтанные переходы, без входного символа
            \end{itemize}
        \end{itemize}
    \end{frame}

    \begin{frame}
        \frametitle{Пример НКА}
        Пусть есть язык, задаваемый регулярным выражением $a a^* \mid b b^*$. Тогда его принимает НКА:
        \begin{center}
            \includegraphics[height=0.6\textheight]{nfa.png}
        \end{center}
    \end{frame}

    \begin{frame}
        \frametitle{Построение НКА по регулярному выражению}
        Теорема: по любому РВ можно построить НКА, принимающий язык РВ. Доказательство:
        \begin{center}
            \includegraphics[height=0.6\textheight]{nfaByRegexp.png}
        \end{center}
    \end{frame}

    \begin{frame}
        \frametitle{Моделирование НКА, таблица переходов}
        \begin{center}
            \includegraphics[width=0.8\textwidth]{abbNfa.png}
        \end{center}
        \begin{center}
            \begin{tabu} {| X[0.9 l p] | X[1 l p] | X[1 l p] |}
                \tabucline-
                Состояние              & a           & b        \\
                \tabucline-
                \everyrow{\tabucline-}
                0                      & $\{0, 1\}$  & $\{0\}$  \\
                1                      & ---         & $\{2\}$  \\
                2                      & ---         & $\{3\}$  \\
            \end{tabu}
        \end{center}
    \end{frame}

    \begin{frame}
        \frametitle{Построение ДКА по НКА}
        Теорема: по любому НКА можно построить ДКА, принимающий в точности тот же язык (то есть НКА и ДКА эквивалентны друг другу в плане выразительности).

        Без доказательства.
    \end{frame}

    \begin{frame}
        \frametitle{Пример}
        \begin{center}
            \includegraphics[width=0.7\textwidth]{abbNfa.png}
        \end{center}
        \begin{center}
            \includegraphics[width=0.7\textwidth]{abbDfa.png}
        \end{center}
    \end{frame}

    \begin{frame}
        \frametitle{Моделирование ДКА}
        \begin{center}
            \includegraphics[width=0.7\textwidth]{abbDfa.png}
        \end{center}
        \begin{center}
            \begin{tabu} {| X[0.9 l p] | X[1 l p] | X[1 l p] |}
                \tabucline-
                Состояние              & a         & b  \\
                \tabucline-
                \everyrow{\tabucline-}
                0                      & 1         & 0  \\
                1                      & 1         & 2  \\
                2                      & 1         & 3  \\
                3                      & 1         & 0  \\
            \end{tabu}
        \end{center}
    \end{frame}

    \begin{frame}[fragile]
        \frametitle{Как это выглядит в коде}
        \begin{itemize}
            \item switch-case вполне вариант
            \item Интерпретировать таблицу состояний
            \begin{itemize}
                \item Лучше, меньше кода и гибче
            \end{itemize}
        \end{itemize}
        \begin{minted}{pascal}
while c <> eof do begin
    s := move(s, c);
    c := nextchar();
end;
return (s in F);
        \end{minted}
    \end{frame}

    \begin{frame}
        \frametitle{Материалы}
        \begin{columns}
            \begin{column}{0.5\textwidth}
                Книжка:

                А. Ахо, Р. Сети, Дж. Ульман, М. Лам. Компиляторы. Принципы, технологии, инструменты.
            \end{column}
            \begin{column}{0.5\textwidth}
                \begin{center}
                    \includegraphics[width=0.38\textwidth]{compilersCover.png}
                \end{center}
            \end{column}
        \end{columns}

        Конструктор автоматов:
        \begin{itemize}
            \item Веб-версия: \url{https://spbu-se.github.io/WebAutomataConstructor}
            \item GitHub: \url{https://github.com/spbu-se/WebAutomataConstructor}
            \item Десктопная версия: \url{https://github.com/spbu-se/DesktopAutomataConstructor}
            \item Экспериментальная десктопная версия: \url{https://github.com/IlyaMuravjov/automaton-constructor/tree/develop}
        \end{itemize}
    \end{frame}


\end{document}