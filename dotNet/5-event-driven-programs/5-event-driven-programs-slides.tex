\documentclass[xetex,mathserif,serif]{beamer}
\usepackage{polyglossia}
\setdefaultlanguage[babelshorthands=true]{russian}
\usepackage{minted}
\usepackage{tabu}

\useoutertheme{infolines}

\usepackage{fontspec}
\setmainfont{FreeSans}
\newfontfamily{\russianfonttt}{FreeSans}

\setbeamertemplate{blocks}[rounded][shadow=false]

\setbeamercolor*{block title alerted}{fg=red!50!black,bg=red!20}
\setbeamercolor*{block body alerted}{fg=black,bg=red!10}

\tabulinesep=1.2mm

\title{Событийно-ориентированное программирование}
\subtitle{Делегаты, события, лямбда-функции}
\author[Юрий Литвинов]{Юрий Литвинов\\\small{\textcolor{gray}{yurii.litvinov@gmail.com}}}
\date{05.10.2017г}

\newcommand{\todo}[1] {
	\begin{center}\textcolor{red}{TODO: #1}\end{center}
}

\begin{document}

	\frame{\titlepage}

	\begin{frame}
		\frametitle{Паттерн ``Наблюдатель''}
		\begin{center}
			\includegraphics[width=0.7\textwidth]{observer.png}
		\end{center}
	\end{frame}

	\begin{frame}[fragile]
		\frametitle{Делегаты}
		\begin{minted}{csharp}
public delegate void Feedback(Int32 value);
		\end{minted}
		\begin{itemize}
			\item Типобезопасный callback
			\item \mintinline{csharp}|delegate| объявляет тип функции-делегата
			\begin{itemize}
				\item На самом деле, автоматически генерируемый класс
				\item Наследник System.MulticastDelegate, который в свою очередь наследник System.Delegate
				\item Методы Invoke, BeginInvoke, EndInvoke
			\end{itemize}
		\end{itemize}
	\end{frame}

	\begin{frame}[fragile]
		\frametitle{Как это устроено}
		\begin{minted}{csharp}
Feedback fbStatic = new Feedback(Program.FeedbackToConsole);
Feedback fbInstance = new Feedback(new Program().FeedbackToFile);
		\end{minted}
		\todo{Картинка}
	\end{frame}

	\begin{frame}[fragile]
		\frametitle{Пример использования}
		\begin{scriptsize}
			\begin{minted}{csharp}
public delegate int HashFunction(string str, int hashSize);

private static class HashFunctions {
    public static int Hash1(string str, int hashSize) {
        return str[0] % hashSize;
    }

    public static int Hash2(string str, int hashSize) {
        int result = 0;
        foreach (var ch in str)
            result += ch;
        return result % hashSize;
    }
}

static void Main(string[] args) {
    var h = new HashFunction(HashFunctions.Hash1);
    var result = h("ololo", 10);
}
			\end{minted}
		\end{scriptsize}
	\end{frame}

	\begin{frame}[fragile]
		\frametitle{Ещё пример, который будем дальше мучить}
		\framesubtitle{Цикл обработки событий}
		\begin{footnotesize}
			\begin{minted}{csharp}
public class EventLoop {
    public void Run() {
        while (true) {
            var key = Console.ReadKey();
            switch (key.Key) {
                case ConsoleKey.LeftArrow:
                    // Сделать что-то по нажатию на "влево"
                    break;
                case ConsoleKey.RightArrow:
                    // Сделать что-то по нажатию на "вправо"
                    break;
            }
        }
    }
}
			\end{minted}
		\end{footnotesize}
	\end{frame}

	\begin{frame}[fragile]
		\frametitle{Цикл с делегатом}
		\begin{footnotesize}
			\begin{minted}{csharp}
public delegate void ArrowHandler();

public class EventLoop {
    public void Run(ArrowHandler left, ArrowHandler right) {
        while (true) {
            var key = Console.ReadKey(true);
            switch (key.Key) {
                case ConsoleKey.LeftArrow:
                    left();
                    break;
                case ConsoleKey.RightArrow:
                    right();
                    break;
            }
        }
    }
}
			\end{minted}
		\end{footnotesize}
	\end{frame}

	\begin{frame}[fragile]
		\frametitle{Обработчики}
		\begin{minted}{csharp}
public class Game
{
    public void OnLeft()
    {
        Console.WriteLine("Going left");
    }

    public void OnRight()
    {
        Console.WriteLine("Going right");
    }
}
		\end{minted}
	\end{frame}

	\begin{frame}[fragile]
		\frametitle{Main}
		\begin{minted}{csharp}
static void Main(string[] args)
{
    var eventLoop = new EventLoop();
    var game = new Game();
    eventLoop.Run(
        new ArrowHandler(game.OnLeft), 
        new ArrowHandler(game.OnRight)
    );
}
		\end{minted}
	\end{frame}

	\begin{frame}[fragile]
		\frametitle{Delegate chaining}
		\begin{minted}{csharp}
public void Register(SomeDelegateType someDelegate)
{
    currentDelegate = Delegate.Combine(currentDelegate, someDelegate);
}
		\end{minted}
		\vspace{3mm}
		Или
		\begin{minted}{csharp}
public void Register(SomeDelegateType someDelegate)
{
    currentDelegate += someDelegate;
}
		\end{minted}
	\end{frame}

	\begin{frame}[fragile]
		\frametitle{Цикл обработки событий с регистрацией (1)}
		\begin{minted}{csharp}
public class EventLoop
{
    private ArrowHandler leftHandler;
    private ArrowHandler rightHandler;

    public void RegisterLeftHandler(ArrowHandler left)
    {
        leftHandler += left;
    }

    public void RegisterRightHandler(ArrowHandler right)
    {
        rightHandler += right;
    }
		\end{minted}
	\end{frame}

	\begin{frame}[fragile]
		\frametitle{Цикл обработки событий с регистрацией (2)}
		\begin{minted}{csharp}
    public void Run() {
        while (true) {
            var key = Console.ReadKey(true);
            switch (key.Key) {
                case ConsoleKey.LeftArrow:
                    if (leftHandler != null)
                        leftHandler();
                    break;
                case ConsoleKey.RightArrow:
                    if (rightHandler != null)
                        rightHandler();
                    break;
            }
        }
    }
}
		\end{minted}
	\end{frame}

	\begin{frame}[fragile]
		\frametitle{Логгер (ещё один наблюдатель)}
		\begin{minted}{csharp}
public class Logger
{
    private List<string> log = new List<string>();

    public void LeftPressed()
    {
        log.Add("left");
    }

    public void RightPressed()
    {
        log.Add("right");
    }
}
		\end{minted}
	\end{frame}

	\begin{frame}[fragile]
		\frametitle{Main}
		\begin{minted}{csharp}
static void Main(string[] args)
{
    var eventLoop = new EventLoop();
    var game = new Game();
    var logger = new Logger();

    eventLoop.RegisterLeftHandler(game.OnLeft);
    eventLoop.RegisterRightHandler(game.OnRight);

    eventLoop.RegisterLeftHandler(logger.LeftPressed);
    eventLoop.RegisterRightHandler(logger.RightPressed);

    eventLoop.Run();
}
		\end{minted}
	\end{frame}

	\begin{frame}[fragile]
		\frametitle{Отписывание от событий}
		\begin{minted}{csharp}
public void UnregisterLeftHandler(ArrowHandler left)
{
    leftHandler = (ArrowHandler) Delegate.Remove(leftHandler , left);
}
		\end{minted}
		\vspace{3mm}
		или
		\begin{minted}{csharp}
public void UnregisterLeftHandler(ArrowHandler left)
{
    leftHandler -= left;
}
		\end{minted}
	\end{frame}

	\begin{frame}[fragile]
		\frametitle{Как примерно выглядит Invoke для цепочки}
		\begin{minted}{csharp}
public int Invoke(int value) {
    int result;
    Delegate[] delegateSet = _invocationList as Delegate[];
    if (delegateSet != null) {
        foreach (Feedback d in delegateSet)
            result = d(value);
    } else {
        result = _methodPtr.Invoke(_target, value);
    }

    return result;
}
		\end{minted}
	\end{frame}

	\begin{frame}[fragile]
		\frametitle{Вызов делегатов из цепочки вручную}
		\begin{small}
			\begin{minted}{csharp}
Delegate[] arrayOfDelegates = leftHandler.GetInvocationList();
foreach (ArrowHandler handler in arrayOfDelegates) {
    try {
        handler.Invoke();
    }
    catch (InvalidOperationException e) {
        Object component = handler.Target;
        Console.WriteLine(
                "Failed to call handler from {1}{2}{0} Error: {3}{0}{0}",
                Environment.NewLine,
                ((component == null) ? "" : component.GetType() + "."),
                handler.GetMethodInfo().Name,
                e.Message);
    }
}
			\end{minted}
		\end{small}
	\end{frame}

	\begin{frame}[fragile]
		\frametitle{Шаблонные типы делегатов}
		\begin{minted}{csharp}
public delegate int HashFunction(string str, int hashSize);
		\end{minted}
		\hspace{2cm}\begin{Large}$\downarrow$\end{Large}
		\begin{minted}{csharp}
Func<string, int, int>
		\end{minted}
		\begin{itemize}
			\item \mintinline{csharp}|Func<T1, T2, ..., TResult>|
			\item \mintinline{csharp}|Action<T1, T2, ...>|
			\item Не дружат с \mintinline{csharp}|ref| и \mintinline{csharp}|out|
		\end{itemize}
	\end{frame}

	\begin{frame}[fragile]
		\frametitle{События}
		\mintinline{csharp}|eventLoop.RegisterLeftHandler(game.OnLeft);| --- много работы

		\mintinline{csharp}|eventLoop.leftHandler += game.OnLeft;| --- очень плохо

		\vspace{3mm}
		Надо (в EventLoop):
		\begin{minted}{csharp}
public event Action LeftHandler;
public event Action RightHandler;
		\end{minted}
		\vspace{3mm}
		и (в Main):
		\begin{minted}{csharp}
eventLoop.LeftHandler += game.OnLeft;
eventLoop.RightHandler += game.OnRight;

eventLoop.LeftHandler += logger.LeftPressed;
eventLoop.RightHandler += logger.RightPressed;
		\end{minted}
	\end{frame}

	\begin{frame}[fragile]
		\frametitle{Анонимные методы}
		\begin{minted}{csharp}
static void Main(string[] args) {
    var eventLoop = new EventLoop();
    var game = new Game();
    eventLoop.LeftHandler += game.OnLeft;
    eventLoop.RightHandler += game.OnRight;
    var log = new List<string>();
    eventLoop.LeftHandler += delegate {
        log.Add("left");
    };
    eventLoop.RightHandler += delegate {
        log.Add("right");
    };
    eventLoop.Run();
}
		\end{minted}
	\end{frame}

	\begin{frame}[fragile]
		\frametitle{Замыкания (1)}
		\begin{minted}{csharp}
static Func<Point, Point> CreateRemapFunction(
        Rectangle rect1, Rectangle rect2)
{
    var xScale = rect2.Width() / rect1.Width();
    var yScale = rect2.Height() / rect1.Height();
    Func<Point, Point> result = delegate(Point point) {
        var returnValue = new Point();
        returnValue.x = point.x * xScale;
        returnValue.y = point.y * yScale;
        return returnValue;
    };
    return result;
}
		\end{minted}
	\end{frame}

	\begin{frame}[fragile]
		\frametitle{Замыкания (2, пример использования)}
		\begin{small}
			\begin{minted}{csharp}
static void Main(string[] args) {
    var rect1 = new Rectangle() {
        topLeft = new Point() { x = 0, y = 0 },
        bottomRight = new Point() { x = 2, y = 2 }
    };
    var rect2 = new Rectangle() {
        topLeft = new Point() { x = 0, y = 0 },
        bottomRight = new Point() { x = 4, y = 6 }
    };

    var remap = CreateRemapFunction(rect1, rect2);
    var point = new Point() { x = 1, y = 1 };
    var transformedPoint = remap(point);
    Console.WriteLine("Transformed point: x = {0}, y = {1}", 
          transformedPoint.x, transformedPoint.y);
}
			\end{minted}
		\end{small}
	\end{frame}

	\begin{frame}[fragile]
		\frametitle{Замыкания, способ прострелить себе ногу}
		\begin{small}
			\begin{minted}{csharp}
delegate void F();

class Program
{
    static void Main(string[] args)
    {
        var delegates = new F[10];
        for (var i = 0; i < 10; ++i)
        {
            delegates[i] = delegate { Console.WriteLine(i); };
        }

        foreach (var f in delegates)
        {
            f();
        }
    }
}
			\end{minted}
		\end{small}
	\end{frame}

	\begin{frame}[fragile]
		\frametitle{Лямбда-выражения}
		\begin{minted}{csharp}
delegate(список параметров)
{
    тело
};
		\end{minted}
		\hspace{2cm}\begin{LARGE}$\downarrow$\end{LARGE}
		\begin{minted}{csharp}
(список параметров) => { тело };
		\end{minted}
	\end{frame}

	\begin{frame}[fragile]
		\frametitle{Примеры}
		\begin{minted}{csharp}
eventLoop.LeftHandler += () => log.Add("left");
eventLoop.RightHandler += () => log.Add("right");

Func<int, int> x2 = x => x * 2;
var n = x2(1);  // n будет 2

List<int> list = new List<int>() { 20, 1, 4, 8, 9, 44 };
List<int> evenNumbers = list.FindAll(i => (i % 2) == 0);
		\end{minted}
	\end{frame}

	\begin{frame}[fragile]
		\frametitle{Каноничное объявление события (1)}
		Наследник EventArgs, содержащий в себе параметры события
		\begin{itemize}
			\item Если параметров нет, то обработчики события всё равно должны принимать EventArgs
			\begin{itemize}
				\item Передавать при вызове имеет смысл EventArgs.Empty
			\end{itemize}
		\end{itemize}

		\vspace{3mm}
		\begin{minted}{csharp}
internal class NewMailEventArgs : EventArgs {
    private readonly string from, to, subject;
    public NewMailEventArgs(string from, string to, string subject) {
        this.from = from; this.to = to; this.subject = subject;
    }
    public string From => from;
    public string To => to;
    public string Subject => subject;
}
		\end{minted}
	\end{frame}

	\begin{frame}[fragile]
		\frametitle{Каноничное объявление события (2)}
		Само событие в наблюдаемом классе
		\begin{itemize}
			\item Инстанциация шаблона EventHandler
			\item 
				\begin{minted}{csharp}
public delegate void EventHandler<TEventArgs>(
        object sender, TEventArgs e);
				\end{minted}
		\end{itemize}

		\vspace{7mm}
		\begin{minted}{csharp}
internal class MailManager {
    public event EventHandler<NewMailEventArgs> NewMail;
    ...
}
		\end{minted}
	\end{frame}

	\begin{frame}[fragile]
		\frametitle{Каноничное объявление события (3)}
		Вспомогательный метод, кидающий событие
		\begin{itemize}
			\item Сюда идёт проверка списка подписчиков на null
			\item Вызов лучше делать потокобезопасным
		\end{itemize}
		\vspace{5mm}
		\begin{minted}{csharp}
internal class MailManager {
    ...
    protected virtual void OnNewMail(NewMailEventArgs e) {
        EventHandler<NewMailEventArgs> temp = 
                Volatile.Read(ref NewMail);
        if (temp != null) 
            temp(this, e);
    }
    ...
}
		\end{minted}
	\end{frame}

	\begin{frame}[fragile]
		\frametitle{Каноничное объявление события (4)}
		Кидание события
		\begin{itemize}
			\item Создаём наследника EventArgs
			\item Вызываем метод, отправляющий событие наблюдателям
		\end{itemize}
		\vspace{5mm}
		\begin{minted}{csharp}
internal class MailManager {
    public void SimulateNewMail(string from, string to, string subject) {
        NewMailEventArgs e = new NewMailEventArgs(from, to, subject);
        OnNewMail(e);
    }
}
		\end{minted}
	\end{frame}

	\begin{frame}[fragile]
		\frametitle{Принимающий событие}
		\begin{minted}{csharp}
internal sealed class Fax {
    public Fax(MailManager mm) {
        mm.NewMail += FaxMsg;
    }

    private void FaxMsg(object sender, NewMailEventArgs e) {
        Console.WriteLine("Faxing mail message:");
        Console.WriteLine(" From={0}, To={1}, Subject={2}",
                e.From, e.To, e.Subject);
    }

    public void Unregister(MailManager mm) {
        mm.NewMail -= FaxMsg;
    }
}
		\end{minted}
	\end{frame}

	\begin{frame}[fragile]
		\frametitle{Ручное управление подписчиками}
		\begin{minted}{csharp}
public event EventHandler<FooEventArgs> Foo {
    add { eventSet.Add(fooEventKey, value); }
    remove { eventSet.Remove(fooEventKey, value); }
}

protected virtual void OnFoo(FooEventArgs e) {
    eventSet.Raise(fooEventKey, this, e);
}
		\end{minted}
		\vspace{7mm}
		eventSet --- рукописный словарь с цепочками обработчиков, не поместившийся в презентацию
	\end{frame}

	\begin{frame}
		\frametitle{Rx.NET}
		\framesubtitle{Reactive Extensions}
		\begin{itemize}
			\item Библиотека, позволяющая работать с событиями как с последовательностями
			\item Зачем:
			\begin{itemize}
				\item Передавать события как параметры, хранить и возвращать как результат
				\item Фильтровать, отображать, соединять друг с другом и т.д.
				\item LINQ
				\item Дружит с многопоточностью
				\item Чтобы было как в F\#-е и даже лучше
			\end{itemize}
			\item Часть библиотеки доступна из коробки в .NET с 4.5, часть --- как пакет в NuGet
		\end{itemize}
	\end{frame}

	\begin{frame}[fragile]
		\frametitle{Основные интерфейсы}
		\begin{minted}{csharp}
public interface IObservable<out T> 
{
    IDisposable Subscribe(IObserver<T> observer); 
}

public interface IObserver<in T> 
{ 
    void OnCompleted();
    void OnError(Exception error);
    void OnNext(T value);
} 
		\end{minted}
	\end{frame}

	\begin{frame}[fragile]
		\frametitle{Пример}
		\begin{minted}{csharp}
static void Main(string[] args)
{
    IObservable<int> source = Observable.Range(1, 10);
    IDisposable subscription = source.Subscribe(
        x => Console.WriteLine($"OnNext: {x}"),
        ex => Console.WriteLine($"OnError: {ex.Message}"),
        () => Console.WriteLine("OnCompleted"));
    Console.WriteLine("Press ENTER to unsubscribe...");
    Console.ReadLine();
    subscription.Dispose();
}
		\end{minted}
	\end{frame}

	\begin{frame}[fragile]
		\frametitle{Пример с таймером}
		\begin{small}
			\begin{minted}{csharp}
Console.WriteLine("Current Time: " + DateTime.Now);

var source = Observable.Timer(
        TimeSpan.FromSeconds(5), 
        TimeSpan.FromSeconds(1))
    .Timestamp();

using (source.Subscribe(x => Console.WriteLine($"{x.Value}: {x.Timestamp}")))
{
    Console.WriteLine("Press any key to unsubscribe");
    Console.ReadKey();
}

Console.WriteLine("Press any key to exit");
Console.ReadKey();
			\end{minted}
		\end{small}
	\end{frame}

	\begin{frame}[fragile]
		\frametitle{``Холодная'' последовательность}
		\begin{footnotesize}
			\begin{minted}{csharp}
IObservable<long> source = Observable.Interval(TimeSpan.FromSeconds(1));   

IDisposable subscription1 = source.Subscribe(
        x => Console.WriteLine($"Observer 1: OnNext: {x}"),
        ex => Console.WriteLine($"Observer 1: OnError: {ex.Message}"),
        () => Console.WriteLine("Observer 1: OnCompleted"));

System.Threading.Threading.Thread.Sleep(3000);

IDisposable subscription2 = source.Subscribe(
        x => Console.WriteLine($"Observer 2: OnNext: {x}"),
        ex => Console.WriteLine($"Observer 2: OnError: {ex.Message}"),
        () => Console.WriteLine("Observer 2: OnCompleted"));

Console.WriteLine("Press any key to unsubscribe");
Console.ReadLine();
subscription1.Dispose();
subscription2.Dispose();
			\end{minted}
		\end{footnotesize}
	\end{frame}

	\begin{frame}[fragile]
		\frametitle{``Горячая'' последовательность}
		\begin{scriptsize}
			\begin{minted}{csharp}
Console.WriteLine("Current Time: " + DateTime.Now);
var source = Observable.Interval(TimeSpan.FromSeconds(1));

IConnectableObservable<long> hot = Observable.Publish(source);

IDisposable subscription1 = hot.Subscribe(
    x => Console.WriteLine($"Observer 1: OnNext: {x}"),
    ex => Console.WriteLine($"Observer 1: OnError: {ex.Message}"),
    () => Console.WriteLine("Observer 1: OnCompleted"));

Console.WriteLine("Current Time after 1st subscription: " + DateTime.Now);
Thread.Sleep(3000);
hot.Connect();
Console.WriteLine("Current Time after Connect: " + DateTime.Now);
Thread.Sleep(3000);
Console.WriteLine("Current Time just before 2nd subscription: " + DateTime.Now);

IDisposable subscription2 = hot.Subscribe(
    x => Console.WriteLine($"Observer 2: OnNext: {x}"),
    ex => Console.WriteLine($"Observer 2: OnError: {ex.Message}"),
    () => Console.WriteLine("Observer 2: OnCompleted"));

Console.ReadKey();
			\end{minted}
		\end{scriptsize}
	\end{frame}

	\begin{frame}[fragile]
		\frametitle{Представление событий как IObservable<T>}
		\begin{minted}{csharp}
var lbl = new Label(); 
var frm = new Form { Controls = { lbl } }; 

IObservable<EventPattern<MouseEventArgs>> move 
        = Observable.FromEventPattern<MouseEventArgs>(
                frm, "MouseMove");

move.Subscribe(evt => { 
        lbl.Text = evt.EventArgs.Location.ToString(); 
        }) ;

Application.Run(frm);
		\end{minted}
	\end{frame}

	\begin{frame}[fragile]
		\frametitle{Операторы над IObservable<T>}
		\begin{minted}{csharp}
var source1 = Observable.Range(1, 3);
var source2 = Observable.Range(1, 3);
source1.Concat(source2)  // Или source1.Merge(source2)
        .Subscribe(Console.WriteLine);

var source3 = Observable.Throw<int>(
        new Exception("An error has occurred."));
var source4 = Observable.Range(1, 3);
source3.OnErrorResumeNext(source4)
        .Subscribe(Console.WriteLine);
		\end{minted}
	\end{frame}

	\begin{frame}[fragile]
		\frametitle{LINQ-синтаксис}
		\begin{small}
			\begin{minted}{csharp}
var seqNum = Observable.Range(1, 5);
var seqString = from n in seqNum
                select new string('*', (int)n);
seqString.Subscribe(str => { Console.WriteLine(str); });
Console.ReadKey();

var frm = new Form();
IObservable<EventPattern<MouseEventArgs>> move 
        = Observable.FromEventPattern<MouseEventArgs>(frm, "MouseMove");

IObservable<System.Drawing.Point> points = from evt in move
                                           select evt.EventArgs.Location;

points.Subscribe(pos => Console.WriteLine("mouse at " + pos));
Application.Run(frm);
			\end{minted}
		\end{small}
	\end{frame}

	\begin{frame}[fragile]
		\frametitle{Или даже вот так}
		\begin{small}
			\begin{minted}{csharp}
var frm = new Form(); 
IObservable<EventPattern<MouseEventArgs>> move 
        = Observable.FromEventPattern<MouseEventArgs>(frm, "MouseMove");

IObservable<System.Drawing.Point> points = 
        from evt in move
        select evt.EventArgs.Location;

var overFirstBisector = 
        from pos in points
        where pos.X == pos.Y 
        select pos;

var moveSub = overFirstBisector.Subscribe(
        pos => Console.WriteLine("mouse at " + pos));

Application.Run(frm);
			\end{minted}
		\end{small}
	\end{frame}

	\begin{frame}[fragile]
		\frametitle{Subject}
		\begin{small}
			\begin{minted}{csharp}
Subject<int> subject = new Subject<int>();
var subscription = subject.Subscribe(
        x => Console.WriteLine("Value published: {0}", x),
        () => Console.WriteLine("Sequence Completed."));

subject.OnNext(1);
subject.OnNext(2);

Console.WriteLine("Press any key to continue");
Console.ReadKey();
subject.OnCompleted();
subscription.Dispose();
			\end{minted}
		\end{small}
	\end{frame}


\end{document}
