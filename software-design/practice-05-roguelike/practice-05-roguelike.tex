\documentclass{../../slides-style}

\slidetitle{Задача: Roguelike}
\author[Юрий Литвинов]{Юрий Литвинов \newline \textcolor{gray}{\small\texttt{yurii.litvinov@gmail.com}}}

\begin{document}
		\maketitle

		\begin{frame}
		\frametitle{Roguelike}
		\begin{itemize}
			\item Жанр компьютерных игр, назван в честь игры Rogue, 1980 года выхода
			\item Характеризуется:
			\begin{itemize}
				\item Простой тайловой или консольной графикой
				\item Активным использованием случайной генерации
				\item Перманентной смертью персонажа и невозможностью загрузить предыдущее сохранение
				\item Чрезвычайно развитым набором игровых правил
				\item Высокой свободой действий персонажа (``игры-песочницы'')
			\end{itemize}
			\item Примеры:
			\begin{itemize}
				\item \url{https://en.wikipedia.org/wiki/NetHack}
				\item \url{https://en.wikipedia.org/wiki/Angband_(video_game)}
				\item \url{https://en.wikipedia.org/wiki/Ancient_Domains_of_Mystery}
			\end{itemize}
		\end{itemize}
	\end{frame}

	\begin{frame}
		\frametitle{Что хочется}
		\begin{itemize}
			\item Персонаж игрока, способный перемещаться по карте, управляемый с клавиатуры
			\begin{itemize}
				\item Карта обычно генерируется, но для некоторых уровней грузится из файла
				\item Характеристики --- здоровье, сила атаки и т.д.
				\item Экспа и уровни персонажа, с ростом уровня повышаются характеристики
			\end{itemize}
			\item Инвентарь персонажа, включающий элементы, влияющие на его характеристики, которые можно надеть и снять
			\item Несколько разных видов мобов, способных перемещаться по карте
			\item Боевая система --- движущиеся объекты, пытающиеся занять одну клетку карты, атакуют друг друга
			\item Все детали --- на ваше усмотрение
		\end{itemize}
	\end{frame}

	\begin{frame}
		\frametitle{Задача}
		\begin{columns}
			\begin{column}{0.5\textwidth}
				Построить модель предметной области Roguelike RPG.

				\begin{itemize}
					\item В командах по 3 человека
					\item Потом надо будет реализовать прототип
					\item Требуется диаграмма компонентов и диаграмма (диаграммы) классов
					\item Это должна быть именно модель предметной области, технические детали пока не интересны
				\end{itemize}
			\end{column}
			\begin{column}{0.5\textwidth}
				\begin{center}
					\includegraphics[width=0.9\textwidth]{roguelike.png}
				\end{center}
			\end{column}
		\end{columns}
	\end{frame}

\end{document}