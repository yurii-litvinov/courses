\documentclass{../../slides-style}

\slidetitle{Ещё примеры архитектур}
\author[Юрий Литвинов]{Юрий Литвинов \newline \textcolor{gray}{\small\texttt{yurii.litvinov@gmail.com}}}

\begin{document}
	
	\begin{frame}[plain]
        \titlepage
    \end{frame}

	\section{Git}

	\begin{frame}
		\frametitle{Краткая история систем контроля версий}
		\begin{itemize}
			\item 1975 -- SCCS (Source Code Control System)
			\begin{itemize}
				\item Дельты
			\end{itemize}
			\item 1982 -- RCS (Revision Control System)
			\begin{itemize}
				\item С открытым исходным кодом (до сих пор поддерживается GNU)
			\end{itemize}
			\item 1986 -- CVS (Concurrent Versioning System)
			\begin{itemize}
				\item Одновременное редактирование, мерджи, ветки, тэги, удалённые репозитории
			\end{itemize}
			\item 2000 -- SVN (Subversion)
			\item 2005 --- Git, Mercurial, Bazaar
			\begin{itemize}
				\item Распределённые
			\end{itemize}
		\end{itemize}
	\end{frame}

	\begin{frame}
		\frametitle{Git\footnote{\tiny{По гл. 10 \url{https://git-scm.com/book} и \url{http://aosabook.org/en/git.html}}}}
		\begin{itemize}
			\item Распределённая VCS
			\item Linus Torvalds, 2005 год, драма с BitKeeper
			\item Architectural drivers
			\begin{itemize}
				\item Распределённая разработка с тысячей коммитеров
				\item Защита от порчи исходников
				\begin{itemize}
					\item Возможность отменить мердж, смерджиться вручную
				\end{itemize}
				\item Высокая скорость работы
			\end{itemize}
		\end{itemize}
	\end{frame}

	\begin{frame}
		\frametitle{Внутреннее устройство Git}
		Структура папки .git:
		\begin{itemize}
			\item HEAD
			\item index
			\item config
			\item description
			\item hooks/
			\item info/
			\item objects/
			\item refs/
			\item ...
		\end{itemize}
	\end{frame}

	\begin{frame}[fragile]
		\frametitle{Объекты}
		Git внутри --- хеш-таблица, отображающая SHA-1-хеш файла в содержимое файла. Пример:
		\begin{minted}{text}
$ git init test
Initialized empty Git repository in /tmp/test/.git/
$ cd test
$ find .git/objects
.git/objects
.git/objects/info
.git/objects/pack

$ echo 'test content' | git hash-object -w --stdin
d670460b4b4aece5915caf5c68d12f560a9fe3e4

$ find .git/objects -type f
.git/objects/d6/70460b4b4aece5915caf5c68d12f560a9fe3e4
		\end{minted}
	\end{frame}

	\begin{frame}[fragile]
		\frametitle{Объекты (2)}
		Как получить сохранённый объект:
		\begin{minted}{text}
$ git cat-file -p d670460b4b4aece5915caf5c68d12f560a9fe3e4
test content
		\end{minted}

		Версионный контроль:
		\begin{minted}{text}
$ echo 'version 1' > test.txt
$ git hash-object -w test.txt
83baae61804e65cc73a7201a7252750c76066a30
$ echo 'version 2' > test.txt
$ git hash-object -w test.txt
1f7a7a472abf3dd9643fd615f6da379c4acb3e3a
$ find .git/objects -type f
.git/objects/1f/7a7a472abf3dd9643fd615f6da379c4acb3e3a
.git/objects/83/baae61804e65cc73a7201a7252750c76066a30
.git/objects/d6/70460b4b4aece5915caf5c68d12f560a9fe3e4
		\end{minted}
	\end{frame}

	\begin{frame}[fragile]
		\frametitle{Объекты (3)}
		Переключение между версиями файла:
		\begin{minted}{text}
$ git cat-file -p 83baae61804e65cc73a7201a7252750c76066a30 \
    > test.txt
$ cat test.txt
version 1

$ git cat-file -p 1f7a7a472abf3dd9643fd615f6da379c4acb3e3a \
    > test.txt
$ cat test.txt
version 2
		\end{minted}
	\end{frame}

	\begin{frame}[fragile]
		\frametitle{Деревья}
		blob (то, что мы видели раньше) хранит только содержимое файла, не хранит даже его имя. Решение проблемы --- tree:
		\begin{scriptsize}
		\begin{minted}{text}
$ git cat-file -p master^{tree}
100644 blob a906cb2a4a904a152e80877d4088654daad0c859      README
100644 blob 8f94139338f9404f26296befa88755fc2598c289      Rakefile
040000 tree 99f1a6d12cb4b6f19c8655fca46c3ecf317074e0      lib
		\end{minted}
		\end{scriptsize}
		\begin{center}
			\includegraphics[width=0.5\textwidth]{gitTreeObject.png}
		\end{center}
	\end{frame}

	\begin{frame}
		\frametitle{Какие ещё виды объектов бывают}
		\begin{center}
			\includegraphics[width=0.7\textwidth]{gitDataStructure.png}
		\end{center}
	\end{frame}

	\begin{frame}[fragile]
		\frametitle{Коммиты}
		tree-объекты могут хранить структуру файлов (как inode в файловой системе), но не хранят метаинформацию типа автора файла и даты создания. Это хранится в commit-объектах:
		\begin{minted}{text}
$ echo 'first commit' | git commit-tree d8329f
fdf4fc3344e67ab068f836878b6c4951e3b15f3d

$ git cat-file -p fdf4fc3
tree d8329fc1cc938780ffdd9f94e0d364e0ea74f579
author Scott Chacon <schacon@gmail.com> 1243040974 -0700
committer Scott Chacon <schacon@gmail.com> 1243040974 -0700

first commit
		\end{minted}
		Ещё коммит хранит список коммитов-родителей
	\end{frame}

	\begin{frame}
		\frametitle{Коммиты, как это выглядит}
		\begin{center}
			\includegraphics[width=0.7\textwidth]{gitCommitObjects.png}
		\end{center}
	\end{frame}

	\begin{frame}[fragile]
		\frametitle{Ссылки}
		Теперь вся информация хранится на диске, но чтобы ей воспользоваться, нужно помнить SHA-1 хеши. На помощь приходят reference-ы. 

		\begin{itemize}
			\item .git/refs
			\item .git/refs/heads
			\item .git/refs/tags
		\end{itemize}

		\begin{minted}{text}
$ echo "1a410efbd13591db07496601ebc7a059dd55cfe9" \
    > .git/refs/heads/master

$ git log --pretty=oneline master
1a410efbd13591db07496601ebc7a059dd55cfe9 third commit
cac0cab538b970a37ea1e769cbbde608743bc96d second commit
fdf4fc3344e67ab068f836878b6c4951e3b15f3d first commit
		\end{minted}
		\begin{itemize}
			\item Команда \verb|git update-ref|
		\end{itemize}
	\end{frame}

	\begin{frame}
		\frametitle{Ссылки, как это выглядит}
		\begin{center}
			\includegraphics[width=0.9\textwidth]{gitRefs.png}
		\end{center}
	\end{frame}

	\begin{frame}[fragile]
		\frametitle{HEAD}
		Теперь не надо помнить хеши, но как переключаться между ветками?

		Текущая ветка хранится в HEAD. HEAD --- символическая ссылка, то есть ссылка на другую ссылку.
		\begin{minted}{text}
$ cat .git/HEAD
ref: refs/heads/master

$ git symbolic-ref HEAD refs/heads/test
$ cat .git/HEAD
ref: refs/heads/test
		\end{minted}
	\end{frame}

	\begin{frame}[fragile]
		\frametitle{Тэги}
		Последний из объектов в Git --- tag. Это просто указатель на коммит.
		\begin{footnotesize}
			\begin{itemize}
				\item Легковесный тэг:
					\begin{minted}{text}
git update-ref refs/tags/v1.0 cac0cab538b970a37ea1e769cbbde608743bc96d
					\end{minted}
					Или просто git tag
				\item Аннотированный тэг:
					\begin{minted}{text}
$ git tag -a v1.1 1a410efbd13591db07496601ebc7a059dd55cfe9 -m 'test tag'

$ git cat-file -p 9585191f37f7b0fb9444f35a9bf50de191beadc2
object 1a410efbd13591db07496601ebc7a059dd55cfe9
type commit
tag v1.1
tagger Scott Chacon <schacon@gmail.com> Sat May 23 16:48:58 2009 -0700

test tag
					\end{minted}
			\end{itemize}
		\end{footnotesize}
	\end{frame}

	\begin{frame}[fragile]
		\frametitle{Packfiles}
		Пока что получалось, что все версии всех файлов в Git хранятся целиком, как они есть. Все они всегда сжимаются zlib, но в целом, если создать репозиторий, добавлять туда файлы, коммитить и т.д., все версии всех файлов будут в нём целиком. На помощь приходят .pack-файлы:
		\begin{footnotesize}
			\begin{minted}{text}
$ git gc
Counting objects: 18, done.
Delta compression using up to 8 threads.
Compressing objects: 100% (14/14), done.
Writing objects: 100% (18/18), done.
Total 18 (delta 3), reused 0 (delta 0)

$ find .git/objects -type f
.git/objects/bd/9dbf5aae1a3862dd1526723246b20206e5fc37
.git/objects/d6/70460b4b4aece5915caf5c68d12f560a9fe3e4
.git/objects/info/packs
.git/objects/pack/pack-978e03944f5c581011e6998cd0e9e30000905586.idx
.git/objects/pack/pack-978e03944f5c581011e6998cd0e9e30000905586.pack
			\end{minted}
		\end{footnotesize}
	\end{frame}

	\begin{frame}
		\frametitle{Как оно устроено}
		\begin{center}
			\includegraphics[width=0.6\textwidth]{gitPackFiles.png}
		\end{center}
	\end{frame}

	\begin{frame}
		\frametitle{Pack-файлы, подробности}
		\begin{itemize}
			\item Упаковка происходит, когда:
			\begin{itemize}
				\item Выполняется git push
				\item Слишком много <<свободных>> объектов (порядка 7000)
				\item Вручную вызвана git gc
			\end{itemize}
			\item Используется дельта-компрессия
			\begin{itemize}
				\item Последняя версия хранится целиком, дельты <<идут назад>>
			\end{itemize}
			\item Можно заглянуть внутрь, git verify-pack
			\item Git может хитро перепаковывать pack-файлы
		\end{itemize}
	\end{frame}

	\begin{frame}[fragile]
		\frametitle{Reflog и восстановление коммитов}
			\begin{minted}{text}
$ git reflog
1a410ef HEAD@{0}: reset: moving to 1a410ef
ab1afef HEAD@{1}: commit: modified repo.rb a bit
484a592 HEAD@{2}: commit: added repo.rb

$ git log -g
commit 1a410efbd13591db07496601ebc7a059dd55cfe9
Reflog: HEAD@{0} (Scott Chacon <schacon@gmail.com>)
Reflog message: updating HEAD
Author: Scott Chacon <schacon@gmail.com>
Date:   Fri May 22 18:22:37 2009 -0700

    third commit
$ git branch recover-branch ab1afef
			\end{minted}
	\end{frame}

	\begin{frame}[fragile]
		\frametitle{Как более капитально прострелить себе ногу}
		\framesubtitle{И что делать}
		\begin{minted}{text}
$ git branch -D recover-branch
$ rm -Rf .git/logs/

$ git fsck --full
Checking object directories: 100% (256/256), done.
Checking objects: 100% (18/18), done.
dangling blob d670460b4b4aece5915caf5c68d12f560a9fe3e4
dangling commit ab1afef80fac8e34258ff41fc1b867c702daa24b
dangling tree aea790b9a58f6cf6f2804eeac9f0abbe9631e4c9
dangling blob 7108f7ecb345ee9d0084193f147cdad4d2998293
		\end{minted}
		Git не удалит даже <<висячие>> объекты несколько месяцев, если его явно не попросить.
	\end{frame}

	\begin{frame}
		\frametitle{Lessons Learned}
		\begin{itemize}
			\item Команды реализовывались как набор шелл-скриптов
			\begin{itemize}
				\item Не портировать под Windows
				\item Сложно интегрировать с IDE
				\item В итоге замедлило внедрение git-а
				\item В итоге побеждено
			\end{itemize}
			\item Большой набор команд (включая plumbing) делает Git тяжёлым для изучения и усложняет сообщения об ошибках
		\end{itemize}
	\end{frame}

	\section{Mercurial}

	\begin{frame}
		\frametitle{Mercurial\footnote{\tiny{По \url{http://aosabook.org}}}}
		\begin{itemize}
			\item Python + C
			\item Распределённая VCS
			\item Architectural drivers
			\begin{itemize}
				\item Масштабные open-source-проекты (ядро Linux)
				\begin{itemize}
					\item Миллионы файлов
					\item Миллионы ревизий
					\item Тысячи пользователей, вносящих изменения параллельно в течение десятилетий
				\end{itemize}
				\item Компрессия хранилища данных
				\item Эффективное получение произвольных ревизий
				\item Эффективное добавление новых ревизий
				\item Работа с историями файлов
			\end{itemize}
		\end{itemize}
	\end{frame}

	\begin{frame}
		\frametitle{Revlog}
		\begin{columns}
			\begin{column}{0.6\textwidth}
				\begin{itemize}
					\item Каждый файл хранится в виде набора ревизий
					\item Ревизии хранятся в виде дельт, иногда снапшоты файла целиком
					\item Каждая ревизия описывается записью с форматом как на рисунке
					\item Отдельно файл с дельтами (данные), отдельно файл с записями (индекс)
					\item Сжатие zlib
				\end{itemize}
			\end{column}
			\begin{column}{0.4\textwidth}
				\begin{center}
					\includegraphics[width=0.8\textwidth]{revlog.png}
				\end{center}
			\end{column}
		\end{columns}
	\end{frame}

	\begin{frame}
		\frametitle{Структура revlog-ов}
		\begin{itemize}
			\item Сhangelog --- метаданные о ревизии + ссылка на манифест
			\item Manifests --- список имён файлов в ревизии + для каждого ссылка на filelog
			\item Filelog --- содержимое файлов ревизии + немного метаданных
			\item Dirstate --- информация о рабочей копии, кеш дерева файлов
			\item Обновление логов в фиксированном порядке, гарантирующее консистентность
			\item Revlog-и хранятся тоже в виде дельт
		\end{itemize}
		\begin{center}
			\includegraphics[width=0.5\textwidth]{mercurialLogStructure.png}
		\end{center}
	\end{frame}

	\begin{frame}[fragile]
		\frametitle{Как это выглядит}
		Changelog:
		\begin{minted}{text}
0a773e3480fe58d62dcc67bd9f7380d6403e26fa
Dirkjan Ochtman <dirkjan@ochtman.nl>
1276097267 -7200
mercurial/discovery.py
discovery: fix description line
		\end{minted}
		\vspace{3mm}
		Manifest:
		\begin{minted}{text}
.hgignore\x006d2dc16e96ab48b2fcca44f7e9f4b8c3289cb701
.hgsigs\x00de81f258b33189c609d299fd605e6c72182d7359
.hgtags\x00b174a4a4813ddd89c1d2f88878e05acc58263efa
CONTRIBUTORS\x007c8afb9501740a450c549b4b1f002c803c45193a
COPYING\x005ac863e17c7035f1d11828d848fb2ca450d89794
		\end{minted}
	\end{frame}

	\begin{frame}
		\frametitle{Ревизии}
		\begin{center}
			\includegraphics[width=\textwidth]{mercurialRevisions.png}
			\attribution{https://www.mercurial-scm.org/wiki/UnderstandingMercurial}
		\end{center}
		\begin{itemize}
			\item Локальный номер ревизии
			\begin{itemize}
				\item Доступ за константное время к узлу в revlog-е
			\end{itemize}
			\item Глобальный SHA-1-хеш ревизии
		\end{itemize}
	\end{frame}

	\begin{frame}
		\frametitle{Ветки}
		\begin{columns}
			\begin{column}{0.6\textwidth}
				\begin{enumerate}
					\item Cоздание ветки через клонирование репозитория
					\item Bookmarks --- объекты-ссылки в духе git
					\item Именованные ветки --- имя ветки в метаданных ревизии
					\item Анонимные ветки
				\end{enumerate}
				\begin{itemize}
					\item Тэги хранятся как версионируемый файл .hgtags в репозитории
				\end{itemize}
			\end{column}
			\begin{column}{0.4\textwidth}
				\includegraphics[width=\textwidth]{mercurialBranches.png}
			\end{column}
		\end{columns}
	\end{frame}

	\begin{frame}
		\frametitle{Статическая структура}
		\begin{columns}
			\begin{column}{0.4\textwidth}
				\begin{itemize}
					\item Один модуль --- один файл
					\item CLI
					\item Одна команда --- одна функция, все в одном файле
					\item Хеш-таблица, отображающая имена команд на функции
					\item Опции, общие наборы опций
				\end{itemize}
			\end{column}
			\begin{column}{0.6\textwidth}
				\includegraphics[width=\textwidth]{mercurialImportGraph.png}
			\end{column}
		\end{columns}
	\end{frame}

	\begin{frame}
		\frametitle{Расширяемость}
		\begin{itemize}
			\item Модули расширения
			\begin{itemize}
				\item Новые команды
				\begin{itemize}
					\item cmdtable, uisetup, reposetup
				\end{itemize}
				\item Обёртки над существующими командами
				\item Обёртки над репозиторием
				\item Обёртки над любой функцией Mercurial
				\item Новые типы репозиториев (например, hgsubversion)
				\item Алиасы
			\end{itemize}
			\item hooks
			\begin{itemize}
				\item Вызов шелл-скрипта
				\item Вызов Python-функции
			\end{itemize}
		\end{itemize}
	\end{frame}

	\begin{frame}
		\frametitle{Lessons Learned}
		\begin{itemize}
			\item Python: и хорошо, и плохо
			\item Намеренно сложно модифицировать changeset после публикации
			\item Revlogs + модель данных -- хорошо и эффективно
			\item Небольшое количество основных команд помогает легче научиться
			\item .hgtags оказался внезапен для пользователей
			\item Люди впервые знакомились с Python, чтобы писать расширения для mercurial, потому что это просто
		\end{itemize}
	\end{frame}

	\section{Wesnoth}

	\begin{frame}
		\frametitle{Battle for Wesnoth\footnote{\tiny{По \url{http://aosabook.org}}}}
		\begin{columns}
			\begin{column}{0.4\textwidth}
				\begin{itemize}
					\item Пошаговая стратегия
					\item Порядка 200000 строк кода на C++
					\item 4 миллиона скачиваний
					\item 9/10 на Steam
					\item 2003 год
				\end{itemize}
			\end{column}
			\begin{column}{0.6\textwidth}
				\includegraphics[width=\textwidth]{wesnoth.png}
				\attribution{https://www.wesnoth.org/}
			\end{column}
		\end{columns}
	\end{frame}

	\begin{frame}
		\frametitle{Architectural Drivers}
		\begin{itemize}
			\item Доступность для новых разработчиков и авторов контента
			\item В ущерб технической красоте
			\item Не nice to have, а условие выживания проекта в контексте широкого open-source сообщества из людей без каких-либо обязательств и разного технического уровня
		\end{itemize}
	\end{frame}

	\begin{frame}
		\frametitle{Высокоуровневая архитектура}
		\begin{columns}
			\begin{column}{0.6\textwidth}
				\begin{itemize}
					\item Wesnoth Markup Language (WML)
					\item Минимизация зависимостей от сторонних библиотек
					\begin{itemize}
						\item SDL Simple Directmedia Layer) для видео и ввода/вывода
						\begin{itemize}
							\item Простота использования и кроссплатформенность
						\end{itemize}
						\item Boost, Pango, zlib, Python, Lua, GNU gettext
					\end{itemize}
				\end{itemize}
			\end{column}
			\begin{column}{0.4\textwidth}
				\includegraphics[width=\textwidth]{wesnothArchitecture.png}
			\end{column}
		\end{columns}
	\end{frame}

	\begin{frame}
		\frametitle{Основные компоненты}
		\begin{itemize}
			\item Парсер и препроцессор WML
			\item Базовый ввод-вывод --- видео, звук, сеть
			\item GUI --- виджеты
			\item Display module --- игровая доска, юниты, анимация и т.д.
			\item ИИ
			\item Поиск пути (плюс утилиты для работы с гексагональной доской)
			\item Генератор карт
			\item Специализированные модули
			\begin{itemize}
				\item Титульный экран
				\item Storyline module --- для проигрывания катсцен
				\item Лобби --- для мультиплеера
				\item ``Play game'' module --- управление основным игровым процессом
			\end{itemize}
			\item Отдельно --- wesnothd и content server
		\end{itemize}
	\end{frame}

	\begin{frame}[fragile]
		\frametitle{Wesnoth Markup Language}
		\begin{minted}{text}
[unit_type]
    id=Elvish Fighter
    name= _ "Elvish Fighter"
    image="units/elves-wood/fighter.png"
    hitpoints=33
    advances_to=Elvish Captain,Elvish Hero
    {LESS_NIMBLE_ELF}
    [attack]
        name=sword
        icon=attacks/sword-elven.png
        range=melee
        damage=5
    [/attack]
[/unit_type]
		\end{minted}
	\end{frame}

	\begin{frame}[fragile]
		\frametitle{Макросы}
		\begin{minted}{c++}
#define GOLD EASY_AMOUNT NORMAL_AMOUNT HARD_AMOUNT
  #ifdef EASY
    gold={EASY_AMOUNT}
  #endif
  #ifdef NORMAL
    gold={NORMAL_AMOUNT}
  #endif
  #ifdef HARD
    gold={HARD_AMOUNT}
  #endif
#enddef
...
{GOLD 50 100 200}
		\end{minted}
	\end{frame}

	\begin{frame}
		\frametitle{Модель данных}
		\begin{itemize}
			\item Всё сливается в один гигантский WML-документ
			\item Перезагружается при смене опций
			\item Всякие хаки на уровне препроцессора, чтобы не грузить вообще всё
			\item Классы unit и unit\_type (архитектурный стиль Knowledge Layer)
			\item Фиксированный набор поддерживаемых движком атрибутов, задаваемых для каждого типа через WML
			\begin{itemize}
				\item Нельзя описывать произвольное поведение через WML, хотели сохранить декларативность
			\end{itemize}
			\item Класс attack\_type
			\item Трейты, инвентарь
		\end{itemize}
	\end{frame}

	\begin{frame}
		\frametitle{Мультиплеер}
		\begin{itemize}
			\item Начальное состояние и команды
			\item Сервер просто пересылает команды между клиентами
			\begin{itemize}
				\item TCP/IP
			\end{itemize}
			\item Replay
			\item Никакой защиты от читов
			\item Версии клиентов
		\end{itemize}
	\end{frame}

	\begin{frame}
		\frametitle{Lessons Learned}
		\begin{itemize}
			\item 250 тысяч строк на WML
			\item Сотни созданных пользователями кампаний
			\item 74 тысячи коммитов, 196 контрибуторов
			\item Сами разработчики смеются над WML
			\item В целом задача обеспечить доступность для модификации очень сложна
		\end{itemize}
	\end{frame}

\end{document}
