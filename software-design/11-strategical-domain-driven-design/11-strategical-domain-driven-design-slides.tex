\documentclass{../../slides-style}

\slidetitleext{Лекция 11: Предметно-ориентированное проектирование, стратегические аспекты}{12.11.2025}{Предметно-ориентированное проектирование}

\begin{document}
    
    \begin{frame}[plain]
        \titlepage
    \end{frame}

    \section{Целостность модели}

    \begin{frame}
        \frametitle{Проблемы DDD в больших системах}
        \begin{itemize}
            \item Несколько команд => несколько видений продукта
            \item Модель предметной области
            \begin{itemize}
                \item Интегрированная --- слишком большие затраты на поддержание целостности, слишком общая модель, чтобы быть полезной
                \item Фрагментированная --- затрудняет переиспользование и интеграцию системы
            \end{itemize}
            \item Опасность ошибок при интеграции и переиспользовании
            \begin{itemize}
                \item Класс ``Платёж'' --- платёж поставщику или платёж клиента
            \end{itemize}
        \end{itemize}
    \end{frame}

    \begin{frame}
        \frametitle{Принципы поддержания целостности модели}
        \begin{itemize}
            \item Ограниченный контекст (Bounded context)
            \item Непрерывная интеграция (Continuous Integration)
            \item Карта контекстов (Context map)
        \end{itemize}
        \begin{center}
            \includegraphics[width=0.7\textwidth]{contextMap.png}
        \end{center}
    \end{frame}

    \begin{frame}
        \frametitle{Пример, границы контекстов}
        \begin{center}
            \includegraphics[width=0.7\textwidth]{contextBoundariesExample.png}
        \end{center}
    \end{frame}

    \begin{frame}
        \frametitle{Типовые ситуации интеграции контекстов}
        \begin{itemize}
            \item Типовые ситуации интеграции контекстов и типовые пути действий в разных ситуациях
            \item Пути действий именованы и документированы (паттерны!)
            \item Чаще всего, выбор паттерна определяется обстоятельствами, не зависящими от команды
            \item Иногда отсутствие какой-либо интеграции --- лучший способ интеграции
        \end{itemize}
    \end{frame}

    \begin{frame}
        \frametitle{Общее ядро}
        \framesubtitle{Shared Kernel}
        \begin{center}
            \includegraphics[width=0.7\textwidth]{sharedKernel.png}
        \end{center}
    \end{frame}

    \begin{frame}
        \frametitle{Заказчик-поставщик}
        \framesubtitle{Customer-Supplier}
        \begin{itemize}
            \item Имеет смысл, когда одна компонента целиком зависит от другой
            \item Может привести к блокированию действий одной или другой команды
            \item Следует явно зафиксировать отношения между командами
            \begin{itemize}
                \item Одна выступает в роли заказчика (одного из заказчиков) --- участвует в планировании, поставляет задачи
                \item Автоматизированные приёмочные тесты
            \end{itemize}
            \item Желательно, чтобы команды находились в одной иерархии управления
        \end{itemize}
    \end{frame}

    \begin{frame}
        \frametitle{Конформист}
        \framesubtitle{Conformist}
        \begin{itemize}
            \item Имеет смысл, когда нет способа повлиять на компонент, от которого полностью зависим
            \begin{itemize}
                \item Legacy-приложение, навязанная сверху технология и т.п.
            \end{itemize}
            \item Просто принимаем модель и миропонимание ``основного'' компонента
            \item Не всегда плохо: чужой код может на самом деле выражать большее понимание предметной области
        \end{itemize}
    \end{frame}

    \begin{frame}
        \frametitle{Предохранительный уровень}
        \framesubtitle{Anticorruption Layer}
        \begin{itemize}
            \item Имеет смысл, когда ``Конформист'' не подходит
            \item Кусок кода (возможно, большой и страшный), отвечающий за трансляцию из одной модели в другую
            \begin{itemize}
                \item Паттерны ``Фасад'' и ``Адаптер'' 
            \end{itemize}
        \end{itemize}
        \begin{center}
            \includegraphics[width=0.9\textwidth]{anticorruptionLayer.png}
        \end{center}
    \end{frame}

    \begin{frame}
        \frametitle{Ещё приёмы}
        \begin{itemize}
            \item \textbf{Отдельное существование (Separate Ways)} --- когда преимущества от интеграции меньше затрат на неё
            \item \textbf{Служба с открытым протоколом (Open Host Service)} --- когда клиентов много
            \item \textbf{Общедоступный язык (Published Language)} --- когда клиентов очень много, общая среда для общения
        \end{itemize}
    \end{frame}

    \begin{frame}
        \frametitle{Итого, шаблоны интеграции}
        \begin{center}
            \includegraphics[width=0.8\textwidth]{integrationPatterns.png}
        \end{center}
    \end{frame}

    \section{Унификация слона}

    \begin{frame}
        \frametitle{Пример: унификация слона}
        \begin{ssmall}
            \begin{columns}
                \begin{column}{0.33\textwidth}
                    Шесть седовласых мудрецов \\
                    Сошлись из разных стран. \\
                    К несчастью, каждый был незряч, \\
                    Зато умом блистал. \\
                    Они исследовать слона \\
                    Явились в Индостан. \\
                    \vspace{5mm}
                    Один погладил бок слона. \\
                    Довольный тем сполна, \\
                    Сказал он: "Истина теперь \\
                    Как божий день видна: \\
                    Предмет, что мы зовем слоном, ­\\
                    Отвесная стена!" \\
                \end{column}
                \begin{column}{0.33\textwidth}
                    А третий хобот в руки взял \\
                    И закричал: "Друзья! \\
                    Гораздо проще наш вопрос, \\
                    Уверен в этом я! \\
                    Сей слон --- живое существо, \\
                    А именно змея!" \\
                    \vspace{5mm}
                    Мудрец четвертый обхватил \\
                    Одну из ног слона \\
                    И важно молвил: "Это ствол, \\
                    Картина мне ясна! \\
                    Слон --- дерево, что зацветет, \\
                    Когда придет весна!" \\
                \end{column}
                \begin{column}{0.33\textwidth}
                    Тем временем шестой из них \\
                    Добрался до хвоста. \\
                    И рассмеялся от того, \\
                    Как истина проста. \\
                    "Ваш слон --- веревка. Если ж нет \\
                    Зашейте мне уста!" \\
                    \vspace{5mm}
                    А как известно, мудрецам \\
                    Присущ упрямый нрав. \\
                    Спор развязав, они дошли \\
                    Едва ль не до расправ. \\
                    Но правды ни один не знал, \\
                    Хотя был в чем-то прав.
                \end{column}
            \end{columns}
        \end{ssmall}
    \end{frame}

    \begin{frame}
        \frametitle{Унификация слона, Separate ways}
        \begin{center}
            \includegraphics[width=0.8\textwidth]{elephantSeparateWays.png}
        \end{center}
    \end{frame}

    \begin{frame}
        \frametitle{Слон, минимальная интеграция}
        \framesubtitle{Anticorruption Layer}
        \begin{center}
            \includegraphics[width=0.9\textwidth]{elephantAnticorruptionLayer.png}
        \end{center}
    \end{frame}

    \begin{frame}
        \frametitle{Слон, слабая интеграция}
        \framesubtitle{Shared Kernel}
        \begin{center}
            \includegraphics[width=0.9\textwidth]{elephantSharedKernel.png}
        \end{center}
    \end{frame}

    \begin{frame}
        \frametitle{Слон, сильная интеграция}
        \framesubtitle{Bounded Context}
        \begin{center}
            \includegraphics[width=0.9\textwidth]{elephantSingleBoundedContext.png}
        \end{center}
    \end{frame}

    \section{Смысловое ядро}

    \begin{frame}
        \frametitle{Дистилляция}
        \begin{itemize}
            \item \textbf{Дистилляция} --- процесс выделения самого существенного в системе и отделения его от вспомогательного кода
            \item \textbf{Смысловое ядро (Core Domain)} --- то, что, собственно, делает систему ценной
            \begin{itemize}
                \item Должно быть минимальным и чётко отделённым от остальных компонентов системы
                \item Опытные программисты не любят им заниматься, с этим надо бороться
                \item Только Core Domain, фактически, составляет know-how
            \end{itemize}
        \end{itemize}
    \end{frame}

    \begin{frame}
        \frametitle{Приёмы дистилляции}
        \begin{itemize}
            \item \textbf{Domain Vision Statement} --- документ (на одну страницу), описывающий смысловое ядро и его полезность
            \item \textbf{Выделенное ядро (Highlighted Core)}
            \begin{itemize}
                \item Дистилляционный документ --- 3-7 страниц текста про то, что составляет смысловое ядро и как его элементы взаимодействуют друг с другом
                \item Flagged Core --- элементы ядра выделены на существующей модели
            \end{itemize}
            \item \textbf{Неспециализированные подобласти (Generic Subdomains)} --- куски кода, неспецифичные для системы
            \item \textbf{Связный механизм (Cohesive Mechanism)} --- куски кода, неспецифичные для предметной области вообще 
            \begin{itemize}
                \item Технические вещи, типа графов
            \end{itemize}
        \end{itemize}
    \end{frame}

    \begin{frame}
        \frametitle{Пример, грузоперевозки}
        \begin{center}
            \includegraphics[width=0.8\textwidth]{shippingRaw.png}
        \end{center}
    \end{frame}

    \begin{frame}
        \begin{center}
            \includegraphics[width=0.55\textwidth]{shippingDistilled.png}
        \end{center}
    \end{frame}

    \begin{frame}
        \begin{center}
            \includegraphics[width=0.6\textwidth]{shippingRepacked.png}
        \end{center}
    \end{frame}

    \begin{frame}
        \frametitle{Абстрактное ядро}
        \begin{center}
            \includegraphics[width=0.3\textwidth]{abstractCore.png}
        \end{center}
        \begin{itemize}
            \item Применяется, когда даже ядро оказывается слишком большим
            \item Состоит из абстрактных классов, которые потом реализуют отдельные модули
        \end{itemize}
    \end{frame}
    
    \section{Крупномасштабная структура}

    \begin{frame}
        \frametitle{Крупномасштабная структура}
        \begin{itemize}
            \item \textbf{Крупномасштабная структура} --- набор общих правил, по которым строится система или группа систем
            \item Должна эволюционировать вместе с моделью и кодом
            \item Не должна быть слишком жёсткой
            \begin{itemize}
                \item Модель ``Архитектор в башне из слоновой кости'' не работает
            \end{itemize}
            \item Лучше какая-то, чем никакой
            \item Небольшие проекты могут прекрасно жить и без всего этого
            \item Самая полезная структура --- общий язык
        \end{itemize}
    \end{frame}

    \begin{frame}
        \frametitle{Метафора системы}
        \begin{itemize}
            \item \textbf{Метафора} определяет то, как в целом понимать систему
            \begin{itemize}
                \item Множества примеров: рабочий стол, firewall и т.д.
            \end{itemize}
            \item Метафора не всегда есть
            \item Метафора может быть опасной
            \begin{itemize}
                \item Метафора тащит за собой лишний смысл
            \end{itemize}
        \end{itemize}
    \end{frame}

    \begin{frame}
        \frametitle{Уровневая структура}
        \framesubtitle{Не должна быть механической}
        \begin{center}
            \includegraphics[width=0.7\textwidth]{meaninglessLayers.png}
        \end{center}
    \end{frame}

    \begin{frame}
        \frametitle{Пример, перевозка грузов}
        \framesubtitle{Исходная модель}
        \begin{center}
            \includegraphics[width=0.9\textwidth]{cargoNonLayered.png}
        \end{center}
    \end{frame}

    \begin{frame}
        \frametitle{Установка пути следования}
        \begin{center}
            \includegraphics[width=0.9\textwidth]{cargoNonLayeredSequence.png}
        \end{center}
    \end{frame}

    \begin{frame}
        \frametitle{Рефакторинг}
        Два уровня:
        \begin{itemize} 
            \item ресурсный --- то, что обеспечивает наши возможности
            \item операционный --- то, как мы пользуемся нашими возможностями
        \end{itemize}
        Двунаправленная связь между \textit{Customer} и \textit{Cargo} мешает
        \begin{center}
            \includegraphics[width=0.9\textwidth]{cargoTwoLayersRefactoring.png}
        \end{center}
    \end{frame}

    \begin{frame}
        \frametitle{Два уровня}
        \begin{center}
            \includegraphics[width=0.9\textwidth]{cargoTwoLayers.png}
        \end{center}
    \end{frame}

    \begin{frame}
        \frametitle{Рефакторинг, выделение уровня принятия решений}
        \begin{center}
            \includegraphics[width=0.9\textwidth]{cargoThirdLayerRefactoring.png}
        \end{center}
    \end{frame}

    \begin{frame}
        \frametitle{Три уровня}
        \begin{center}
            \includegraphics[width=0.8\textwidth]{cargoThreeLayers.png}
        \end{center}
    \end{frame}

    \begin{frame}
        \frametitle{Работа с опасными грузами, первая версия}
        \begin{center}
            \includegraphics[width=0.95\textwidth]{cargoHazMatWrong.png}
        \end{center}
    \end{frame}

    \begin{frame}
        \frametitle{Диаграмма последовательностей}
        \framesubtitle{Работа с опасными грузами, первая версия}
        \begin{center}
            \includegraphics[width=0.85\textwidth]{cargoHazMatWrongSequence.png}
        \end{center}
    \end{frame}

    \begin{frame}
        \frametitle{Работа с опасными грузами, вторая версия}
        \begin{center}
            \includegraphics[width=0.6\textwidth]{cargoHazMatOk.png}
        \end{center}
    \end{frame}
    
    \begin{frame}
        \frametitle{Диаграмма последовательностей}
        \framesubtitle{Работа с опасными грузами, вторая версия}
        \begin{center}
            \includegraphics[width=0.75\textwidth]{cargoHazMatOkSequence.png}
        \end{center}
    \end{frame}

    \begin{frame}
        \frametitle{Типичные уровни в системах автоматизации производства}
        \begin{center}
            \includegraphics[width=0.9\textwidth]{factoryAutomationLayers.png}
        \end{center}
    \end{frame}

    \begin{frame}
        \frametitle{Типичные уровни в финансовых системах}
        \begin{center}
            \includegraphics[width=0.9\textwidth]{accountingLayers.png}
        \end{center}
    \end{frame}

    \begin{frame}
        \frametitle{Другие высокоуровневые структуры}
        \begin{itemize}
            \item \textbf{Уровень знаний (Knowledge level)} использует информацию о типах сущностей, позволяя гибко переконфигурировать систему
            \begin{center}
                \includegraphics[width=0.65\textwidth]{knowledgeLevel.png}
            \end{center}
            \item \textbf{Подключаемые компоненты (Pluggable Component Framework)} --- стиль, описывающий общее ядро и набор взаимозаменяемых плагинов, которыми оно управляет
            \item Разные стили не исключают друг друга!
        \end{itemize}
    \end{frame}

\end{document}
