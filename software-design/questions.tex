\documentclass[a5paper]{article}
\usepackage[a5paper, top=8mm, bottom=8mm, left=8mm, right=8mm]{geometry}

\usepackage{polyglossia}
\setdefaultlanguage[babelshorthands=true]{russian}

\usepackage{fontspec}
\setmainfont{FreeSerif}
\newfontfamily{\russianfonttt}[Scale=0.7]{DejaVuSansMono}

\usepackage{csquotes}

\sloppy
\pagestyle{plain}

\begin{document}

\thispagestyle{empty}

\section*{Вопросы к экзамену \enquote{Проектирование ПО}}

\begin{flushright}\begin{small}Юрий Литвинов\\\small{y.litvinov@spbu.ru}\end{small}\end{flushright}

\begin{enumerate}
    \item Понятие архитектуры, профессия \enquote{Архитектор}.
    \item Архитектурные виды.
    \item Роль архитектуры в жизненном цикле ПО.
    \item Пример архитектуры: Apache Hadoop. Prescriptive и descriptive-архитектура.
    \item Понятие декомпозиции. Модульность, связность, сопряжение, сложность.
    \item Понятия класса и объекта, абстракция, инкапсуляция, наследование. 
    \item Принципы выделения объектов.
    \item Принципы SOLID.
    % \item Закон Деметры. Принципы хорошего объектно-ориентированного кода.
    \item Моделирование, визуальные модели, виды моделей, метафора визуализации.
    \item Язык UML. Проектирование структуры системы, диаграммы классов.
    \item Диаграммы объектов, диаграммы пакетов UML.
    \item Диаграммы компонентов, диаграммы развёртывания UML.
    \item Диаграмма случаев использования UML.
    \item Диаграмма активностей UML.
    \item Диаграммы конечных автоматов UML.
    \item Диаграммы последовательностей UML.
    \item Диаграммы коммуникации UML.
    \item Диаграммы составных структур, коопераций, временные диаграммы.
    \item Диаграммы обзора взаимодействия, диаграммы потоков данных.
    \item Диаграммы IDEF0, характеристик. Feature tree 
    \item Моделирование требований в SysML.
    \item Язык BPMN.
    \item Моделирование данных: диаграммы \enquote{Сущность-связь}.
    \item Концептуальное моделирование, диаграммы ORM.
    \item Сети Петри, их анализ и применение.
    \item Паттерн \enquote{Компоновщик}.
    \item Паттерн \enquote{Декоратор}.
    \item Паттерн \enquote{Стратегия}.
    \item Паттерн \enquote{Адаптер}.
    \item Паттерн \enquote{Заместитель}.
    \item Паттерн \enquote{Фасад}.
    \item Паттерн \enquote{Приспособленец}.
    \item Паттерн \enquote{Мост}.
    \item Паттерн \enquote{Фабричный метод}.
    \item Паттерн \enquote{Абстрактная фабрика}.
    \item Паттерн \enquote{Одиночка}.
    \item Паттерны \enquote{Ленивая инициализация} и \enquote{Пул объектов}.
    \item Паттерн \enquote{Прототип}.
    \item Паттерн \enquote{Строитель}.
    \item Паттерн \enquote{Посредник}.
    \item Паттерн \enquote{Команда}.
    \item Паттерн \enquote{Цепочка ответственности}.
    \item Паттерн \enquote{Наблюдатель}.
    \item Паттерн \enquote{Состояние}.
    \item Паттерн \enquote{Шаблонный метод}.
    \item Паттерн \enquote{Посетитель}.
    \item Паттерн \enquote{Хранитель}.
    \item Паттерн \enquote{Интерпретатор}.
    \item Паттерн \enquote{Итератор}.
    % \item Антипаттерны \enquote{Круговая зависимость}, \enquote{Последовательная связность}, \enquote{Вызов предка}, \enquote{Проблема Йо-Йо}.
    % \item Антипаттерны \enquote{Активное ожидание}, \enquote{Сокрытие ошибки}, \enquote{Магические числа}, \enquote{Магические строки}.
    % \item Антипаттерны \enquote{Божественный объект}, \enquote{Поток лавы}.
    % \item Антипаттерн \enquote{Золотой молоток}.
    % \item Антипаттерны \enquote{Функциональная декомпозиция}, \enquote{Золотой молоток}.
    \item Понятие архитектурного стиля, трёхзвенная архитектура.
    \item Model-View-Controller, Sense-Compute-Control.
    \item Слоистый стиль, \enquote{Клиент-сервер}.
    \item Гексагональная архитектура, луковая архитектура.
    \item Чистая архитектура.
    \item Пакетная обработка, каналы и фильтры, Blackboard.
    \item Событийно-ориентированные стили, Publish-Subscribe.
    % \item Peer-to-peer.
    \item Понятие Domain-Driven Design, единый язык, изоляция предметной области.
    \item DDD, основные структурные элементы модели предметной области.
    \item DDD, паттерн \enquote{Агрегат}.
    \item DDD, паттерны \enquote{Фабрика}, \enquote{Репозиторий}.
    \item Паттерн \enquote{Спецификация}.
	\item Ограниченный контекст, непрерывная интеграция, карта контекстов.
    \item Подходы к интеграции контекстов.
    \item Смысловое ядро, приёмы дистилляции, абстрактное ядро.
    \item Крупномасштабная структура, метафора системы, разбиение по уровням. 
    \item Типичные уровни в производственных и финансовых системах.
    \item Стили \enquote{Уровень знаний}, \enquote{Подключаемые компоненты}.
    \item Понятие распределённой системы, заблуждения при проектировании распределённых систем.
    \item RPC, RMI. Пример: gRPC.
    \item Веб-сервисы, SOAP. WCF.
    \item Очереди сообщений, RabbitMQ, Apache Kafka.
    \item Архитектурные стили распределённых приложений: Big Compute, Big Data.
    \item Web-queue-worker, N-звенная архитектура.
    \item Микросервисная архитектура.
    \item Архитектурный стиль REST.
    % \item Принципы дизайна распределённых приложений: самовосстановление, паттерн Circuit Breaker.
    % \item Принципы дизайна распределённых приложений: самовосстановление, избыточность.
    % \item Принципы дизайна распределённых приложений: минимизация координации, проектирование для обслуживания.
    \item Docker, Docker Compose.
    \item Kubernetes.
    \item Облачная инфраструктура, AWS, Terraform.
\end{enumerate}

\end{document}
