\documentclass[xetex,mathserif,serif]{beamer}
\usepackage{polyglossia}
\setdefaultlanguage[babelshorthands=true]{russian}
\usepackage{minted}
\usepackage{tabu}

\useoutertheme{infolines}

\usepackage{fontspec}
\setmainfont{FreeSans}
\newfontfamily{\russianfonttt}{FreeSans}

\definecolor{links}{HTML}{2A1B81}
\hypersetup{colorlinks,linkcolor=,urlcolor=links}

\setbeamertemplate{blocks}[rounded][shadow=false]

\setbeamercolor*{block title alerted}{fg=red!50!black,bg=red!20}
\setbeamercolor*{block body alerted}{fg=black,bg=red!10}

\tabulinesep=1.2mm

\title{Экосистема open source проектов}
\subtitle{Полезные инструменты и сервисы}
\author[Юрий Литвинов]{Юрий Литвинов\\\small{\textcolor{gray}{yurii.litvinov@gmail.com}}}
\date{06.02.2019г}

\newcommand{\attribution}[1] {
\vspace{-5mm}\begin{flushright}\begin{scriptsize}\textcolor{gray}{\textcopyright\, #1}\end{scriptsize}\end{flushright}
}

\begin{document}

	\frame{\titlepage}

	\section{Домашка}

	\begin{frame}
		\frametitle{Комментарии по домашке}
		\begin{itemize}
			\item Надо закрывать потоки в тестах
			\item Папку .idea можно не выкладывать
			\item Можно инициализировать поля прямо в месте их объявления
			\item Надо ли закрывать стрим в serialize/deserialize?
		\end{itemize}
	\end{frame}

	\section{CI}

	\begin{frame}
		\frametitle{Continuous Integration}
		Непрерывная интеграция --- практика слияния всех изменений по нескольку раз в день, сборки их в известном окружении и запуска юнит-тестов.
		\begin{itemize}
			\item Автоматический билд
			\begin{itemize}
				\item Всё, что нужно для сборки, есть в репозитории, может быть получено на чистую (ну, практически) машину и собрано одной консольной командой
			\end{itemize}
			\item Большое количество юнит-тестов, запускаемых автоматически
			\item Выделенная машина, слушающая репозиторий и выполняющая билд
			\begin{itemize}
				\item Чаще всего каждый билд запускается на заранее настроенной виртуалке
			\end{itemize}
		\end{itemize}
	\end{frame}

	\begin{frame}
		\frametitle{Continuous Integration}
		\begin{itemize}
			\item Извещение всех разработчиков о статусе
			\begin{itemize}
				\item Если билд не прошёл, разработка приостанавливается до его починки
			\end{itemize}
			\item Автоматическое выкладывание
			\item Пока билд не прошёл, задача не считается сделанной
			\begin{itemize}
				\item Короткие билды (<10 мин.)
				\item deployment pipeline
				\begin{itemize}
					\item Отдельная машина для сборки, для коротких тестов, для длинных тестов, для выкладывания
				\end{itemize}
			\end{itemize}
		\end{itemize}
	\end{frame}

	\section{Travis}

	\begin{frame}
		\frametitle{Travis}
		\begin{itemize}
			\item \url{https://travis-ci.org/} --- пример бесплатной для open source-проектов облачной CI-системы
			\item Собирает на чистой виртуальной машине под Ubuntu 12.04 (или Ubuntu 14.04) или OS X (есть экспериментальная поддержка Windows)
			\item Интегрируется с GitHub-ом, Slack-ом, умеет деплоить
			\item Окружение настраивается конфигурационным файлом или ``вручную'' из скрипта сборки (некоторые конфигурации разрешают sudo)
			\item Сборка выполняется либо автоматически, либо указанным скриптом сборки
			\item Build Matrix
			\begin{itemize}
				\item Разные конфигурации сборки, выполняемые на разных виртуальных машинах
			\end{itemize}
		\end{itemize}
	\end{frame}

	\begin{frame}
		\frametitle{Travis, настройка сборки}
		\begin{itemize}
			\item Установить commit hook на гитхабе
			\begin{itemize}
				\item Travis умеет это делать сам, надо залогиниться под своим GitHub-аккаунтом на Travis и выбрать нужный репозиторий в профиле
			\end{itemize}
			\item Добавить .travis.yml в корень репозитория
			\begin{itemize}
				\item Это конфигурационный файл, говорящий, что и как собирать
			\end{itemize}
			\item Закоммитить и запушить, это инициирует процесс сборки
			\begin{itemize}
				\item Коммит, где в комментарии есть подстрока ``[ci skip]'', игнорируется Travis-ом, остальные он собирает
			\end{itemize}
			\item Результаты будут видны у каждого коммита в истории и в пуллреквесте
		\end{itemize}
		\begin{center}
			\includegraphics[width=0.7\textwidth]{travisSuccess.png}
		\end{center}
	\end{frame}

	\begin{frame}[fragile]
		\frametitle{Travis, конфигурационный файл}
		\begin{minted}{yaml}
language: java
		\end{minted}
		--- всё :) Если у вас проект прямо в корне репозитория.

		Жизненный цикл сборки:
		\begin{footnotesize}
			\begin{itemize}
				\item Install apt addons
				\item before\_install
				\item install
				\item before\_script
				\item script
				\item after\_success или after\_failure
				\item \mintinline{text}|[OPTIONAL]| before\_deploy
				\item \mintinline{text}|[OPTIONAL]| deploy
				\item \mintinline{text}|[OPTIONAL]| after\_deploy
				\item after\_script
			\end{itemize}
		\end{footnotesize}
	\end{frame}

	\begin{frame}
		\frametitle{Travis, примеры}
		\begin{itemize}
			\item Веб-приложение из нескольких сервисов на Java:
			\begin{itemize}
				\item \url{https://github.com/qreal/wmp/blob/master/.travis.yml}
			\end{itemize}
			\item Относительно большое десктопное приложение на C++:
			\begin{itemize}
				\item \url{https://github.com/qreal/qreal/blob/master/.travis.yml}
			\end{itemize}
			\item Билд в Docker-окружении (все зависимости носим с собой):
			\begin{itemize}
				\item \url{https://github.com/trikset/trikRuntime/blob/master/.travis.yml}
			\end{itemize}
		\end{itemize}
	\end{frame}

	\begin{frame}[fragile]
		\frametitle{Travis, конфиг для домашки}
		\begin{minted}{yaml}
language: java
os:
  - linux
env:
  - PROJECT_DIR=hw1
  - PROJECT_DIR=hw2
  - PROJECT_DIR=hw3
script: cd $PROJECT_DIR && ./gradlew check
		\end{minted}
	\end{frame}

	\section{AppVeyor}

	\begin{frame}
		\frametitle{AppVeyor}
		\begin{itemize}
			\item CI c поддержкой Windows и Linux, прежде всего для сборки .NET-приложений, но умеет много чего ещё (в т.ч. Java)
			\begin{itemize}
				\item Windows Server 2016 или Windows Server 2012 R2
				\item Ubuntu 16.04.4 LTS или Ubuntu 18.04 LTS
			\end{itemize}
			\item Тоже бесплатен для open source
			\item Конфигурируется через appveyor.yml
			\item Собирает по умолчанию MSBuild
			\begin{itemize}
				\item Можно переубедить
			\end{itemize}
			\item Синтаксис .yml-файла: \url{https://www.appveyor.com/docs/appveyor-yml/}
			\item Пример: \url{https://github.com/qreal/qreal/blob/master/appveyor.yml}
			\begin{itemize}
				\item Собирать двумя CI-серверами один проект не зазорно
			\end{itemize}
		\end{itemize}
	\end{frame}

	\section{CodeCov}

	\begin{frame}
		\frametitle{Анализ тестового покрытия, CodeCov}
		\begin{itemize}
			\item \url{https://codecov.io/}
			\item Визуализатор для функциональности компиляторов или специальных инструментов по слежению за исполнявшимися строчками
			\item Чем больше операторов было исполнено во время тестового прогона, тем меньше вероятность пропустить баг
			\begin{itemize}
				\item 100\% покрытие не гарантирует работоспособность программы
			\end{itemize}
			\item Для Java (Kotlin, Scala) используется библиотека JaCoCo, для C++/gcc --- ключ ``-coverage''
			\item Интегрируется с гитхабом (комментит пуллреквесты информацией о тестовом покрытии)
		\end{itemize}
	\end{frame}

	\begin{frame}[fragile]
		\frametitle{CodeCov, пример конфигурации}
		Travis:
		\begin{minted}{yaml}
language: java

after_success:
  - bash <(curl -s https://codecov.io/bash)
		\end{minted}
		Пример для Android-приложения:
		\begin{footnotesize}
			\begin{itemize}
				\item \url{https://github.com/codecov/example-android/blob/master/.travis.yml}
			\end{itemize}
		\end{footnotesize}
	\end{frame}

	\section{Codacy}

	\begin{frame}
		\frametitle{Статический анализ, Codacy}
		\begin{itemize}
			\item \url{https://www.codacy.com/}
			\item Ищет типичные ошибки: потенциальные баги, стайлгайд, мёртвый код, производительность и т.д.
			\item Поддерживает много языков (в том числе C\#, C++, Java, Kotlin, Python, Scala)
			\item Не требует дополнительных манипуляций с репозиторием
			\item Очень настраиваема
		\end{itemize}
	\end{frame}

	\section{Инструменты планирования}

	\begin{frame}
		\frametitle{Инструменты планирования, Trello}
		\begin{itemize}
			\item \url{https://trello.com/}
			\item Интерактивная доска с карточками, организованными в списки
			\item Карточки легко редактируются и перетаскиваются между списками
			\begin{itemize}
				\item Типичные списки: TODO, In Progress, Done (возможны варианты)
			\end{itemize}
			\item Поддерживает дедлайны, чеклисты, вложения, комментарии, голосования, метки
			\item Легковесный инструмент планирования, подходящий, тем не менее, и для больших проектов
		\end{itemize}
	\end{frame}

	\begin{frame}
		\frametitle{Инструменты планирования, Pivotal Tracker}
		\begin{itemize}
			\item \url{https://www.pivotaltracker.com}
			\item Более ``тяжеловесный'' инструмент, ориентированный на Scrum
			\item Всего три списка
			\begin{itemize}
				\item Icebox --- что было бы неплохо сделать
				\item Backlog --- запланированные задачи
				\item Current --- задачи на текущую итерацию
			\end{itemize}
			\item Задачи можно оценивать, задачи имеют тип и статус
			\begin{itemize}
				\item По оценкам задач и статистике работы команды считается team velocity, позволяющая предсказать линейные сроки
			\end{itemize}
			\item Есть релизы с дедлайнами, метки, epic-и, чеклисты, вложения, комментарии
			\item Умеет считать статистику, рисовать графики (burndown charts)
		\end{itemize}
	\end{frame}

	\section{Средства коммуникации}

	\begin{frame}
		\frametitle{Средства коммуникации, Slack и Gitter}
		\begin{itemize}
			\item Instant messenger-ы, ориентированные на команды и интегрированные со средствами разработки
			\begin{itemize}
				\item Информация о коммитах и пуллреквестах
				\item Статус CI
				\item Другие тулы
			\end{itemize}
			\item Синтаксическая подсветка (markdown), вложения, отображение картинок, ...
			\item Gitter интегрирован с гитхабом и ``более открыт'' (предназначается прежде всего для общения сообщества)
			\item Slack интегрирован с чем угодно, предназначается прежде всего для общения внутри команды
		\end{itemize}
	\end{frame}

	\section{Встроенные средства GitHub}

	\begin{frame}
		\frametitle{GitHub: Issues, Projects, Wiki, Pages}
		\begin{itemize}
			\item GitHub сам многое умеет
			\item Issues --- довольно удобный багтрекер
			\begin{itemize}
				\item Майлстоуны, дедлайны, метки на багах, возможность закрывать баги автоматически (если в сообщении коммита есть ``close'' или ``fix'' и \#<номер бага>)
				\item Пуллреквест тоже считается Issue
			\end{itemize}
			\item Projects --- представляет Issues в виде набора списков, между которыми их можно перетаскивать в духе Trello
			\item Wiki --- викистраницы, куда можно выкладывать полезную информацию о проекте
			\begin{itemize}
				\item Тоже git-репозиторий
			\end{itemize}
			\item Pages --- хостинг для статических сайтов <имя проекта>.github.io
		\end{itemize}
	\end{frame}

	\section{Авторское право}

	\begin{frame}
		\frametitle{Авторское право}
		\begin{itemize}
			\item Open source-кодом можно пользоваться, только если автор явно это разрешил, так что просто код на GitHub --- не совсем open source
			\item Бывают исключительные и личные неимущественные права
			\begin{itemize}
				\item Личные неимущественные права неотчуждаемы
				\item Исключительные права можно передать
				\item Права появляются в момент создания произведения и принадлежат автору
				\begin{itemize}
					\item Если произведение создано по служебному заданию --- работодателю
					\item Знак копирайта служит только для информирования, регистрация прав не требуется
				\end{itemize}
				\item Соавторы владеют произведением в равной степени
			\end{itemize}
			\item Идея не охраняется, охраняется её физическое выражение
		\end{itemize}
	\end{frame}

	\begin{frame}
		\frametitle{Open source-лицензии}
		\begin{itemize}
			\item Лицензия --- способ передачи части прав на произведение
			\item Пример --- ``Do what the **** you want to public license''
			\begin{itemize}
				\item ``Want to'' может включать в себя патентование произведения и подачу в суд на автора за нарушение патента, поэтому обычно лицензии более длинны и унылы
				\item В России и Европе программы не патентуют, в США --- да
			\end{itemize}
			\item Каждый нормальный open source-проект должен иметь лицензию
		\end{itemize}
	\end{frame}

	\begin{frame}
		\frametitle{Open source-лицензии}
		\begin{itemize}
			\item Часто используемые open source-лицензии:
			\begin{itemize}
				\item GPL, LGPL (GPL вирусная, поэтому использовать её, внезапно, плохая практика)
				\item MIT License
				\item Apache License 2.0 (может применяться пофайлово)
				\item BSD License (в разных вариантах)
				\item The Unlicense --- явная передача произведения в Public Domain
				\item Семейство лицензий Creative Commons --- не для софта, но хорошо подходит для ресурсов (картинок, текстов и т.д.)
			\end{itemize}
		\end{itemize}
	\end{frame}

\end{document}
