% Шаблон списка вопросов, который включает в себя вообще весь материал, который может быть в курсе.
% Под каждое конкретное прочтение из него делается список вопросов вырезанием тех, которые не успели пройти.

\documentclass[a5paper]{article}
\usepackage[a5paper, top=8mm, bottom=8mm, left=8mm, right=8mm]{geometry}

\usepackage{polyglossia}
\setdefaultlanguage[babelshorthands=true]{russian}

\usepackage{fontspec}
\setmainfont{FreeSerif}
\newfontfamily{\russianfonttt}[Scale=0.7]{DejaVuSansMono}

\usepackage[font=scriptsize]{caption}

\sloppy
\pagestyle{plain}

\begin{document}

\thispagestyle{empty}

\section*{Вопросы к экзамену <<Разработка программного обеспечения>>}

\begin{flushright}\begin{small}Юрий Литвинов\\\small{y.litvinov@spbu.ru}\end{small}\end{flushright}

\begin{enumerate}
    \item Программная инженерия как область знания
    \item Отличия разработки программного обеспечения от других инженерных областей
    \item Компетенции и профстандарты в области программной инженерии
    \item Понятие жизненного цикла программного обеспечения
    \item Водопадная модель жизненного цикла
    \item Итеративная, спиралевидная модели жизненного цикла
    \item Понятие и виды требований
    \item Требования к требованиям
    \item Работа с требованиями: выявление, анализ, проверка
    \item Навыки и трудовые функции аналитика
    \item Документы, связанные с требованиями
    \item Моделирование требований
    \item Спецификация требований к программному обеспечению (SRS)
    % \item Управление требованиями
    \item Scrum, роли в команде
    \item Scrum, Backlog и спринты
    \item ScrumAnd, ScrumBut, достоинства и недостатки методологии
    \item Управление рисками: идентификация, стратегии противодействия, резерв
    \item Декомпозиция проекта, критерии SMART
    \item Построение графика работ: матрица зависимостей, сетевой график
    \item Оценка графика работ: линейные сроки, прямой и обратный проходы, критический путь
    \item Диаграмма Гантта, оптимизация ресурсов, денежный поток
    \item Отслеживание прогресса проекта, пороги эскалации
    \item Метрики, виды метрики, трудности измерения
    \item Метрики хода разработки
    \item Методика освоенного объёма (Earned Value), основные, производные и прогнозные метрики
    \item Бизнес-метрики
    \item Метрики заинтересованных сторон
    \item Треугольник равновесия проекта
    \item Приёмы балансирования равновесия на уровне проекта
    \item Приёмы балансирования равновесия на уровне бизнес-целей
    \item Понятие и виды тестирования программного обеспечения, пирамида тестирования
    \item Тестирование требований, тестирование архитектуры
    \item Тестовые сценарии
    \item Инструменты тестирования
    \item Отслеживание ошибок, жизненный цикл ошибки
    \item Понятия стоимости разработки, ценности и цены продукта
    \item Модели Fixed Price и Time and Material
    \item Оценка. Типичные ошибки при оценке.
    \item Модели оценки стоимости
    \item Бизнес-модель
\end{enumerate}

\end{document}
