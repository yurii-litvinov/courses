\documentclass{../../slides-style}

\slidetitle{Практика 1: Введение}{19.02.2026}

\begin{document}

    \begin{frame}[plain]
        \titlepage
    \end{frame}

    \section{Введение}

    \begin{frame}
        \frametitle{Что мы будем делать на практике}
        \begin{outline}
            \1 Опробовать знания, полученные из теоретического курса
            \1 Не будем писать код
                \2 Курс очень гуманитарный, но увы, это тоже надо уметь
            \1 Будем \enquote{разрабатывать} воображаемые проекты
                \2 Для этого их придётся придумать
                \2 Можно взять проект с ЧМВ, можно --- учебную практику
            \1 Часть заданий будет прямо на паре, часть --- дома
                \2 Будут командные задачи с дедлайном в неделю, выдаваемые после лекций
                \2 На практике --- разбираем решения, получаем замечания, исправляем их дома
        \end{outline}
    \end{frame}

    \begin{frame}
        \frametitle{Рекомендованная литература}
        \begin{footnotesize}
            \begin{outline}
                \1 \textbf{Роберт Гласс} Программирование и конфликты
                \1 \textbf{Т. Питерс} Основы. Лидерство.
                \1 \textbf{П.Ф. Друкер} Практика менеджмента
                \1 Project Management Body of Knowledge
                \1 \textbf{Т. ДеМарко, Т. Листер} Вальсируя с медведями: управление рисками в проектах по разработке программного обеспечения.
                \1 \textbf{Т. ДеМарко, Т. Листер} Балдеющие от адреналина и зомбированные шаблонами. Паттерны поведения проектных команд.
                \1 \textbf{Т. ДеМарко, Т. Листер} Человеческий фактор. Успешные проекты и команды.
                \1 \textbf{Т. ДеМарко} Deadline. Роман об управлении проектами.
                \1 \textbf{Ф. Брукс} Мифический человеко-месяц
                \1 \textbf{Дж. Рейнвотер} Как пасти котов
                \1 \textbf{Э. Хант} Программист-прагматик. Путь от подмастерья к мастеру
                \1 \textbf{Дж. Спольски} Джоэл о программировании
                \1 \textbf{И. Соммервилл} Инженерия программного обеспечения
                \1 ...
            \end{outline}
        \end{footnotesize}
    \end{frame}

    \begin{frame}
        \frametitle{Что будет в курсе}
        \begin{footnotesize}
            \begin{outline}
                \1 Работа с требованиями
                \1 Практики Agile-методологий (парное программирование, backlog, спринты)
                \1 Декомпозиция и оценка
                \1 Планирование и слежение за ходом проекта
                \1 Тестирование, отслеживание ошибок
                \1 Финансовый план
            \end{outline}
        \end{footnotesize}
    \end{frame}

    \section{Пример проекта}

    \begin{frame}
        \frametitle{Пример проекта}
        Информационный портал студотдела
        \begin{outline}
            \1 Цель проекта:
                \2 автоматизировать выдачу справок об обучении, приём заявлений на повышенную стипендию
            \1 Требования:
                \2 облегчать процесс получения справки
                \2 облегчать процесс подачи студентами заявлений
                \2 снижать нагрузку на работников деканата
                \2 клиентская часть должна корректно работать в Google Chrome, Mozilla Firefox, Opera
                \2 система должна иметь возможность одновременно работать с не менее 100 запросов без существенных потерь производительности
                \2 время обработки одного запроса должно быть не более 5 сек
        \end{outline}
    \end{frame}

    \begin{frame}
        \frametitle{Пример проекта}
        НАСТОлКИ
        \begin{outline}
            \1 Цель проекта:
                \2 Предоставить удобное средство для организации неформальных мероприятий
            \1 Требования:
                \2 Посетителю должна быть предоставлена возможность выбирать мероприятия и регистрироваться на них за плату, сумма которой назначается организатором
                \2 Должна быть предоставлена возможность платно создавать мероприятия, будучи организатором 
                \2 Организатор должен иметь возможность выставить оценку пользователю
                \2 Оценки от организаторов формируют рейтинг посетителей, который будет представлять из себя ценность для них
                \2 Должна быть реализована роль модератора, который занимается проверкой подаваемых жалоб, и за выполняемую работу он получает возможность бесплатно посещать мероприятия
        \end{outline}
    \end{frame}

    \begin{frame}
        \frametitle{Пример проекта}
        Нeraltt
        \begin{outline}
            \1 Цель проекта:
                \2 Инструмент, позволяющий заинтересованным работникам Университета получать уведомления об изменениях в расписании в удобном для них формате
            \1 Требования:
                \2 Регулярная проверка (раз в час) на предмет изменений расписания на timetable
                \2 Определение типов изменений: изменение времени, аудитории, группы, отмена занятий
                \2 Автоматическая генерация письма со всей необходимой информацией при обнаружении изменений
                \2 Автоматическая рассылка писем на рабочую почту преподавателей
                \2 Интерфейс для подписки на изменения расписания конкретных преподавателей для заведующих кафедрами, административных работников
                \2 Подробная техническая документация, чтобы проект меньше страдал от текучки сопровождающих
                \2 Поддержка связи с пользователем через почту или VK
                \2 Регистрация через учетную запись с портала Университета
        \end{outline}
    \end{frame}

    \section{Домашнее задание}

    \begin{frame}
        \frametitle{Домашнее задание}
        \begin{outline}
            \1 Разделиться на команды по 2-3 человека
            \1 Придумать проект, для которого будете писать документацию
                \2 Это может быть ваша нынешняя или прошлая учебная практика, проект для курса ЧМВ или вообще выдуманный с нуля проект
                \2 Он должен быть достаточно содержателен, хотя бы на пару человеколет работы
                \2 Реализовывать его в ходе курса будет не нужно
                \2 Хорошие выдуманные проекты можно будет в следующем году представить как осеннюю практику второго курса или вынести на ЛШ
            \1 Подготовить презентацию на 10 минут с представлением идеи проекта
            \1 Дедлайн --- \textbf{26 февраля}
        \end{outline}
    \end{frame}

\end{document}
